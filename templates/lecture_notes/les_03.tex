%\lesson{3}{wo 09 okt 2019 16:07}{Derivative of a map, submersion}

\section{The derivative of a map}


\begin{definition}[Derivative of a smooth map]
    If $f: M \to  N$ is differentiable, then its derivative at $p$ is
    \[
        (f_{*})_p: T_pM \to  T_{f(p)}N: [\gamma] \mapsto [f  \circ \gamma]
    .\] 
\end{definition}
\begin{prop}
    $(f_*)_p$ is well defined and linear.

\end{prop}
\begin{proof}
    Let  $\phi_M$ be a chart for $M$ near $p$.
    Let  $\phi_N$ be a chart for $N$ near $f(p)$.
    Consider the commutative diagram
    \[
        \begin{tikzcd}[column sep=3cm]
            {[\gamma]}\in   T_p M \arrow[r, "(f_*)_p"] \arrow[d, "~"]& T_{f(p)} N \ni [f \circ \gamma] \arrow[d]\\
            (\phi_M  \circ  \gamma)'(0) \in  \R^m  \arrow[r, dashed, "D_{\phi_M(p)}(\phi_N  \circ  f  \circ  \phi_M ^{-1})"]&  \R^{n} \ni (\phi_N  \circ (f  \circ  \gamma))'(0)
        \end{tikzcd}
    \] 
    As the two vertical maps are linear isomorphisms (as we saw in the proof of Proposition~\ref{prop:tangspace}) and the horizontal map is linear (being the derivative of a smooth map between open subsets of Euclidean space), the composition $(f_*)_p$ is also linear.
\end{proof}

\begin{prop}[Chain rule] If $\smash{M\overset{f}{\to}N\overset{g}{\to}L}$ are smooth maps,
then $\forall p\in M$ we have ${(g\circ f)_*}_p=(g_*)_{f(p)} \circ (f_*)_p \colon T_pM\to T_{(g\circ f)(p)}L$.
%    $\phi:U \to  \phi(U)$ is a diffeomorphism of manifolds, as $U$ is an open set of $M$ and $\phi(U)$ is an open set of $\R^n$.
\end{prop}

%\begin{prop}
%    $\phi:U \to  \phi(U)$ is a diffeomorphism of manifolds, as $U$ is an open set of $M$ and $\phi(U)$ is an open set of $\R^n$.
%\end{prop}

\begin{remark}
Let $p\in M$ and $(U,\phi)$ a chart around $p$. One can show:    $\phi:U \to  \phi(U)$ is a diffeomorphism of manifolds (Notice that since $U$ is an open set of $M$ and $\phi(U)$ is an open set of $\R^n$, both carry manifold structures).
We have a commutative diagram of isomorphisms.
\[
    \begin{tikzcd}[column sep=-1em]
    & {[\gamma] \in T_pM} \arrow[dl] \arrow[dr, "{(\phi_*)_p}"]&\\
    {(\phi \circ  \gamma)'(0) \in  \R^{m}} &\cong& T_{\phi(p)} (\phi(U))
\end{tikzcd}
\] 
In other words, the map $T_p M \to  \R^{m}$ from the proof of Proposition \ref{prop:tangspace}, under the identification given in the previous remark, is $(\phi_*)_p$.
In particular
\[
\left.\frac{\partial}{\partial x^i} \right|_p \in T_pM
\] 
and $i$-th  standard basis vector of  $T_{\phi(p)} \phi(U) \cong \R^m$ correspond under $(\phi_*)_p$.


\begin{figure}[H]
    \centering
    \incfig{figure-1}
    \caption{The $i$-th  standard basis vector and $\left.\frac{\partial}{\partial x^i}\right|_p$ correspond under $(\phi_*)_p$.}
    \label{fig:figure-1}
\end{figure}
\end{remark}

\section{The regular level set theorem}
\begin{remark}
    Let $f: U \to  V$ be a diffeomorphism between open subsets of $\R^n$.
    Then $\forall q \in U$, $ D_q f: \R^n \to  \R^n $ is an isomorphism, because its inverse is $D_{f(q)} f^{-1}$.
\end{remark}
Conversely,
\begin{lemma}[Inverse function theorem in $\R^n$]
    Let $U \subset \R^n$ open, $f: U \to  \R^n$ smooth s.t.\ $D_qf:\R^n \to  \R^n$ is an isomorphism for some $q \in U$.
    Then there exists a neighbourhood $V \subset U$ of $q$ such that $ f|_V : V \to  f(V) $ is a diffeomorphism.
\end{lemma}
\begin{corollary}[Inverse function theorem for manifolds]
    Let $f: M \to  N$ be a smooth map $p \in M$, such that $f_*(p): T_pM \to  T_{f(p)}N$ is an isomorphism.
    Then there exists a neighbourhood $W$ of $p$ s.t.\ the map $f|_W: W \to  f(W)$ is a diffeomorphism.
\end{corollary}
Idea of the proof: this is a local statement, and by means of charts, locally every manifolds can be identified with an open subset of $\R^n$.

\begin{eg}
 Consider   $f\colon \R \to S^1\subset \mathbb C, t \mapsto e^{2 \pi it}$.
    This is not a diffeomorphism.
    But $f_*(t)$ is an isomorphism for all $t\in \R$. So locally $f$ restricts to a diffeomorphism onto it image. 
\end{eg}

\filbreak
Given $k \ge  n$, we denote
\[
    \pi: \R^n \times \R^{k-n} \to  \R^n, (v, w) \mapsto  v
.\] 
\begin{lemma}[Submersion theorem in $\R^n$]
    Let $U$ be a neighbourhood of the origin $0$ in $\R^n \times \R^{k-n}$ and $f: U \to \R^n$ smooth such that
    \[
        (D_0  f)|_{\R^n \times \{0\} }: \R^{n}\times \{0\}  \to  \R^n
    \] 
    is an isomorphism.
  %  This map is in particular surjective.
    Then there exists a diffeomorphism $\tau$ between neighbourhoods in $\R^n \times \R^{k-n}$ such that $f  \circ \tau^{-1} = \pi$.
\end{lemma}

This theorem states that precomposing $f$ with a diffeomorphism, we can arrange that it becomes the projection on the first components.
\begin{figure}[ht]
    \centering
    \incfig{submersion-theorem}
    \caption{The submersion theorem. The dotted lines denote the preimages of points of $\R^n$ under $f$ and $\pi$.}
    \label{fig:submersion-theorem}
\end{figure}

\filbreak

\begin{proof}
    Consider $\tau:= (f_1, \ldots f_n, x_{n+1}, \ldots, x_k): U \to  \R^n \times \R^{k-m}$, which we can write concisely as $(f, \text{Id}_{\R^{k-n}})$. Consider its derivative at the origin
    \[
    D_0 \tau = \begin{pmatrix}
        (D_0 f)|_{\R^n \times \{0\} } & * \\
    	 0 & \text{Id}_{\R^{k-n}}
    \end{pmatrix}
    .\] 
    This matrix is invertible, because it is an upper block matrix, $(D_0 f)|_{\R^n \times \{0\} } $ is invertible, and $\text{Id}_{\R^{k-n}}$ is invertible.
        This means that $\tau$ is a diffeomorphism near $0$, by the inverse function theorem.
    We have $ \pi  \circ \tau = f$.
\end{proof}
\begin{definition}[Regular value]
    Given a smooth map $f: M \to  N$, a point $c\in N$ is a regular value iff $\forall p \in f^{-1}(c)$, $(f_*)_p : T_p M \to  T_c N$ is \emph{surjective}.
\end{definition}

\begin{theorem}
    Let $f: M^{k} \to  N^{n}$ be a smooth, and let $c \in N$ be a regular value s.t.\ $f^{-1}(c) \neq \O$.
    Then 
    \begin{itemize}
        \item $f^{-1}(c)$ is a submanifold of $M$ with dimension $k - n$
        \item $\forall  p \in  f^{-1}(c): T_p(f^{-1}(c)) = \text{Ker}(f_*(p))$
    \end{itemize}
    \end{theorem}
    
The first item states that the codimension is preserved when taking inverse images.    
\begin{eg}
Consider the ``height function'' $f: S^2 \to \R: (x, y, z) \mapsto z$.
    Note that $-1$ and $1$ are not regular values. 
    %Indeed, the derivative of $f$ vanishes at the Northpole and Southpole.
\end{eg}

\begin{figure}[ht]
    \centering
    \incfig{example-submersion}
    \caption{The ``height function'' $f: S^2 \to \R$. The derivative of $f$ vanishes at the Northpole and Southpole.}
    \label{fig:example-submersion}
\end{figure}





\begin{proof}
    Fix $p \in f ^{-1}(c)$.
    Take charts $(U, \phi_M)$ near $p$,
    and  $(V, \phi_N)$ near  $f(p)=c$, chosen so that  $\phi_N(c)=0$.
    We know that $D_{\phi(p)}(\phi_N  \circ  f  \circ  \phi_M^{-1}) : \R^k \to  \R^n$. is surjective (because $c$ is a regular value). We can assume that
    \[
        D_{\phi(p)}(\phi_N  \circ  f  \circ  \phi_M^{-1})\Big|_{\R^{n} \times \{0\} }
    \] 
    is an isomorphism. (When we restrict a surjective linear map to a subspace
     transverse to the kernel, it becomes an isomorphism. If $\R^{n} \times \{0\}$ is not transverse to the kernel of $D_{\phi(p)}(\phi_N  \circ  f  \circ  \phi_M^{-1}) $, we can change the chart $\phi_N$ by composing it with e.g.\ a rotation in $\R^n$.)

    By the last lemma, there exists a diffeomorphism $\tau$ such that $ (\phi_N  \circ  f  \circ  \phi_M^{-1})\circ  \tau^{-1} = \pi$, which can be rewritten as
    \[
        \phi_N  \circ  f  \circ  (\tau  \circ  \phi_M)^{-1} = \pi
    .\] 

The situation is summarized by the following diagram:
    \[
        \begin{tikzcd}
            U \arrow[r, "f"] \arrow[d, "\tau  \circ  \phi_M"]& V \ni c \arrow[d, "\phi_N"] \\
            \text{open $\subset \R^k$} \arrow[r, dashed, "\pi"]& \text{open $\subset \R^n \ni 0$}
        \end{tikzcd}
    .\] 
%    Note that we chose the chart such that $c\mapsto 0$.

    Hence $\tau  \circ  \phi_M$ is a chart of $M$ adapted to $f^{-1}(c)$.

    \hr

    For the second part,
    % we have to show that 
    %$T_p(f^{-1}(c)) = \text{Ker}((f_*)_p)$.
    let $p \in  f^{-1}(c)$, and take a path $\gamma: (-\epsilon, \epsilon) \to  f^{-1}(c)$ with $\phi(0) = p$.
    Then $$(f_*)_p [\gamma] = [f  \circ  \gamma] = 0 \in T_c N$$ as $f  \circ  \gamma \equiv c$.
    This shows that $T_p (f^{-1}(c)) \subset \text{Ker}(f_*)_p$.
    Both vector spaces have the same dimension, which gives the equality.
\end{proof}

\begin{figure}[ht]
    \centering
    \incfig{proofsubmersion}
    \caption{The chart $\tau  \circ  \phi_M$  straightens out
  $f^{-1}(c)$ (intersected with the domain of the chart).}
    \label{fig:example-submersion}
\end{figure}
