%\lesson{10}{do 21 nov 2019 14:02}{De Rham Cohomology}

\setcounter{chapter}{6}
\chapter{De Rham Cohomology}
\section{Basic definitions}

Let $M$ be a manifold of dimension $m$.


\begin{definition}[Closed  forms]
    $\omega \in \Omega^{k}(M)$ is called closed iff $d \omega = 0$.
\end{definition}
\begin{definition}[Exact forms]
    $\omega \in \Omega^{k}(M)$ is called exact iff $d \alpha = \omega$ for some $\alpha\in \Omega^{k-1}(M)$.
\end{definition}

We have
\[
    \begin{tikzcd}
        0 \arrow[r, ""] 
        &\Omega^{0}(M) \arrow[r, "d_0"] &
        \Omega^{1}(M) \arrow[r, "d_1"]
        &\cdots \arrow[r, "d_{m-1}"]
        &\Omega ^{m}(M) \arrow[r, ""]
        &0
    \end{tikzcd}
\]
with $d_k  \circ  d_{k-1} = 0$,
so in particular  the image of $d_{k-1}$ is included in the kernel of $d_k$.
This is an example of a cochain complex.
\emph{Co}chain because $d$ increases the degree.

\begin{figure}[H]
    \centering
    \incfig{rham-cohomology}
    \caption{
        Visualization of the de Rham cochain complex.
        Exact $k$-forms lay per definition in the image of $d_{k-1}$, and closed $k$-forms in the kernel of $d_k$.
        As $d^2 = 0$, the exact $k$-forms form a subspace of the closed $k$-forms.
    }
    \label{fig:rham-cohomology}
\end{figure}

\begin{definition}[De Rham cohomology]
    The $k$-th de Rham cohomology group of $M$ is 
    \[
        H^{k}(M) := H^{k}_{dR}(M) := \frac{\Ker d_k}{\im d_{k-1}} = \frac{\text{closed $k$-forms}}{\text{exact $k$-forms}}
    .\] 
    %as a quotient of vector spaces.
    If $\omega \in \Omega^{k}(M)$ is closed, we denote by $[\omega] \in  H^{k}(M)$ its class.
\end{definition}

\begin{figure}[H]
    \centering
    \incfig{rham-cohomology-definition}
    \caption{Definition of the De Rham cohomology. }
    \label{fig:rham-cohomology-definition}
\end{figure}
\begin{remark}
    Each $H^{k}(M)$ is a vector space.
    Only when $0 \le k \le m$, $H^{k}(M)$ can be different from zero.
    If $M$ is compact, one can show that $H^{k}(M)$ is finite dimensional for all $k$.
    This is not trivial, because the dimension of $\Ker d_k$ is normally $\infty$-dimensional.
 \end{remark}

\begin{prop}
    If $M$ is connected, then $H^{0}(M) \cong \R$.
\end{prop}
\begin{proof}
    For all $f \in \Omega^{0}(M)$, $f$ is closed if $df = 0$, so $f$ is constant ($M$ is connected and $f$ is smooth).
    $f$ is exact  iff it is zero, since $\Omega^{-1}(M) = \{0\}$.
\end{proof}
\begin{remark}
    If not connected, we get $\R^{k}$, with $k$ being the number of connected components.
\end{remark}
\begin{theorem}
    If $M^{m}$ is \emph{compact}, connected and orientable, then $H^{m}(M)\cong \R$.
\end{theorem}
\begin{proof}
    Forms of `top degree' are always closed.
    The idea is that we construct a map from the   $m$-forms to $\R$ and then show that the kernel are exactly the exact $m$-forms.
    Consider
    \begin{align*}
       I: \Omega^{m}(M)  &\longrightarrow \R \\
       \omega & \longmapsto \int_M \omega
    .\end{align*}
    Then $I$ is surjective: since $M$ is oriented, there exists a volume form, $\Omega$, and $\int_M \Omega \neq 0$. We now argue that $\ker(I)=\{\text{exact $m$-forms}\}$.
    
    
    
    Consider an exact $m$-form $\omega=d \alpha$. Then by Stokes' theorem
    \[
     \int_M d \alpha =  \int_{\partial M} \alpha = 0
    .\]
   Idea for the other inclusion: Assume $\int_M \omega = 0$, then we need to show that  $\omega = d\alpha$ for some $\alpha$.
    Reduce to forms with  support (necessarily compact) contained in a chart.
    Then this reduces to the following:
    if $f$ is a function on $\R$ with compact support and $\int_\R f(x) dx = 0$, then $F(x) = \int_{-\infty}^{x} f(t) dt $ is a primitive of $f$ with compact support  (because when $x$ is really small, this integral is zero, and when $x$ is large enough, $F(x) = \int_{\R} f(x) dx = 0$).

This shows that 
$$
    \frac{\Omega^{m}(M) }{\Ker (I)} = H^{m}(M).$$ But this quotient is 
isomorphic to the image of $I$, which is $\R$.\end{proof}
%\begin{remark}
%    Where did we use that fact that $M$ is connected?
%    Look at $H^k(M_1 \sqcup M_2) = H^{k}(M_1) \oplus H^{k}(M_2)$.
%
%    Consider two copies of a manifold.
%    Let $\omega \in \Omega^{m}(M \sqcup M) $ with $[\omega] \neq 0$.
%    Then
%    \[
%        I(\omega|_{M_1} + (-\omega)|_{M_2}) = \int_M \omega - \int_M \omega = 0
%    .\] 
%\end{remark}

\section{Homotopic maps and cohomology}
\begin{lemma}
    Let $f: N \to  M$ be a smooth map.
    Then
    \begin{itemize}
        \item If $\omega \in \Omega(M) $ is closed, then $f^* \omega$ is also closed.
        \item If $\omega \in \Omega(M) $ is exact, then $f^* \omega$ is also exact.
    \end{itemize}
\end{lemma}
\begin{proof}
    We only prove the first part.
    $d(f^* \omega) = f^* d \omega = f^* 0 = 0$.
\end{proof}

Given $f: N \to  M$, consider the pullback of differential forms $f^*: \Omega^{k}(M) \to \Omega^{k}(N)$.
This restricts to a map between closed forms, and also between exact forms, so it induces a map on the level of cohomologies:
\begin{align*}
    H(f): H^{k}(M)  &\longrightarrow H^{k}(N)  \\
    [\omega] &\longmapsto [f^* \omega].
\end{align*}
If $f$ is a diffeomorphism,
then it's clear that $H(f)$ is an isomorphism.
But there are more maps that induces an isomorphism  in cohomology!

\begin{definition}[Smooth homotopy]
    Let $f, g: N \to M$ smooth maps.
    A \emph{smooth homotopy} between $f$ and $g$ is a smooth map $h: N \times [0, 1] \to  M$ such that $h|_{N\times \{0\}} = f$ and $h_{N\times \{1\}} = g$ .
\end{definition}
In other words,  $h$ is a smooth family of maps that interpolate between $f$ and $g$.
%\begin{remark}
%Smooth homotopy defines an equivalence relation.
%\end{remark}
\filbreak
One can prove:
\begin{prop}
    If $f, g: N \to  M$ are smoothly homotopic, then
    $f^*$ and $g^*$ are cochain homotopic, i.e. there exists a linear map \[K: \Omega^{\bullet}(M)  \to  \Omega^{\bullet - 1}(M),\]
    such that $d  \circ  K + K  \circ  d = f^* - g^*$.
    \[
        \begin{tikzcd}
            \Omega^{k-1}(M) \arrow[r, "d"] & \Omega^{k}(M) \arrow[d, "f^*- g^*"] \arrow[dl, "K"]\arrow[r, "d"] & \Omega^{k+1}(M) \arrow[dl, "K"]\\
            \Omega^{k-1}(N) \arrow[r, "d"] & \Omega^{k}(N)  \arrow[r, "d"] & \Omega^{k+1}(N) \\
        \end{tikzcd}
    \]
\end{prop}
%\begin{proof}
%    See Symplectic Geometry.
%\end{proof}
\begin{theorem}
    If $f, g$ are smoothly homotopic, then $H(f) = H(g)$.
\end{theorem}
\begin{proof}
    Let $K = \Omega^{\bullet}(M)  \to  \Omega^{\bullet - 1}(M) $ a cochain homotopy between $f^*$ and $g^*$.
    Let $\omega \in \Omega^{k}(M) $ be closed.
    Then     
    $$        f^*\omega - g^*\omega = d ( K\omega)  + K  ( \underbrace{d \omega}_{=0})$$     is exact. Hence $ [f^*\omega] =[g^*\omega]$.
\end{proof}

\begin{corollary}
    Let $M$ and $N$ be homotopy equivalent, i.e.\ $\exists \alpha:N \to  M$ and $\beta: M \to N$ such that $\alpha  \circ  \beta \simeq 1_{N}$ and $\beta  \circ  \alpha \simeq 1_{M}$, where $\simeq$ means smoothly homotopic.
    Then $H(\alpha): H(M) \to  H(N)$ is an isomorphism with inverse $H(\beta)$.
\end{corollary}
\begin{proof}
    $H(\alpha)  \circ  H(\beta)= H(\beta  \circ \alpha) = H(1_{N})$, because of the previous theorem, and $H(1_N) = 1_{H(N)}$.    You can do the same for the reverse order.
\end{proof}


\begin{remark}[Poincaré Lemma]
    Let $U \subset \R^n$ be an open subset which is star shaped\footnote{$\exists p$ such that  for all $u \in U$ the line segment between $u$ and $p$ lies in $U$.}.

    Then $U$ is homotopy equivalent to a point $\{p\}$,
    via the inclusion  $ \alpha: \{p\}  \to U $  and the constant map
    $ \beta: U \to \{p\}$. 
    Hence, \[
        H^{k}(U) \cong H^{k}(\{p\}) \cong
        \begin{cases}
            \R &\text{if $k = 0$}\\
            0 &\text{if $k > 0$}
        \end{cases}
    \]
    Notice that $\R^n$ is star-shaped, and every point of a manifold has a star-shaped neighborhood.
\end{remark}

\begin{eg}
    If $E \to  M$ is a vector bundle, then it is homotopic equivalent to $M$,
    as we can shrink the fibers (which are vector spaces) to the zero section. So the cohomology of $M$ is the same as that of $E$: $H(E) \cong H(M)$.   
\end{eg}

\begin{remark}
    Given manifolds $M, N$, if $H(M) \neq H(N)$, then the manifold are not homotopic equivalent.
\end{remark}

\section{The Mayer--Vietoris theorem}

\begin{theorem}
    Suppose a manifold admits a cover by 2 open sets $U, V$, i.e $M = U \cup V$.
    Then
    \[
        \begin{tikzcd}
        & \cdots & H^{k-1}(U \cap V) \arrow[dll, in=70, out=-90, looseness=0.4]\\
            H^{k}(M) \arrow[r, "A_k"] & H^{k}(U)  \oplus H^{k}(V)  \arrow[r, "B_k"] & H^{k}(U \cap V) \arrow[dll, "\delta_k", in=70, out=-90, looseness=0.4, swap]\\
            H^{k+1}(M) \arrow[r, ""] &\cdots &\\
 %           H^{k+1}(M) \arrow[r, ""] & H^{k+1}(U)  \oplus H^{k+1}(V)  \arrow[r, ""] & H^{k+1}(U \cap V)\\
        \end{tikzcd}
    \]
    is a long exact sequence, i.e. the image of any arrow is the kernel of the next one. Here, 
    \begin{align*}
        A_k[\omega] &:= [\omega|_U] \oplus [\omega|_V]\\
        B_k(\alpha \oplus \beta) &:= \beta|_{U \cap V} - \alpha|_{U \cap V}
    .\end{align*}
\end{theorem}
\begin{remark}
  The proof uses the smooth maps
    \[
    U \cap  V \xrightarrow[i_{U \cap  V, U}]{i_{U \cap  V, V}} U \sqcup V \to  M
    .\] 

\end{remark} 
\begin{eg}
    Let's consider the sphere $S^{m}$ with $m \ge 2$.
    Write $$S^{m} = U \cup  V$$ where $U=S^{m}\setminus\{\text{North pole}\}$, $V=S^{m}\setminus\{\text{South pole}\}$.
    Notice that
    \begin{itemize}
\item $U \simeq  \{p\}$,
\item $V \simeq  \{p\}$,
\item $U\cap V \simeq  S^{m-1}$.
\end{itemize}
    Applying the theorem and doing some diagram-chasing, we get
    \[
        H^{k}(S^{m}) \cong 
        \begin{cases}
            \R & \text{if $k = 0, m$}\\
            0 & \text{otherwise}.
        \end{cases}
    \] 
 %   This implies that all $k$-forms on the sphere are exact.
\end{eg}
