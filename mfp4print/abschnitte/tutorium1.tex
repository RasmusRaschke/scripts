\section{Tutorium 10.04.25}
\label{sec:10_04_25}

\subsection{Organisatorisches}
\begin{itemize}
\item Festlegung des allgemeinen Tutoriumtermins. Ort: Voraussichtlich wieder Geomatikum, wenn Raum verfügbar.
\item Fahrplan MfP4: Differentialgeometrie, Funktionentheorie, Funktionalanalysis
\item Wünsche für das Tutorium? Vergleich letzte Semester, konkrete Vorstellungen, etc.
\item MathNet-Wolke, Moodle, Homepage, Telegram-Gruppe
\end{itemize}

\subsection{Literaturempfehlungen}
Zur Differentialgeometrie kennt ihr bereits meine Empfehlungen von letztem Semester.
Zur Funktionentheorie:
\begin{itemize}
\item Jänich, \href{https://link.springer.com/book/10.1007/3-540-35015-2}{Funktionentheorie}: Sehr angenehm zu lesendes Buch, vor allem, wenn man vielleicht auch an gut geschriebenen Büchern und Anekdoten Spaß hat. 
\item Salamon, \href{https://link.springer.com/book/10.1007/978-3-0348-0169-0}{Funktionentheorie}: Das Buch basiert auf Ahlfors, Complex Analysis, einem beachtlich guten Standardwerk im englischen Raum. Etwas trockener als Jänich, eher im typischen Mathe-Stil.
\item Needham, \href{https://umv.science.upjs.sk/hutnik/NeedhamVCA.pdf}{Visual Complex Analysis}: Das (englischsprachige) Buch kann man am besten nebenher lesen, wenn man sich für die geometrische Intuition hinter der Funktionentheorie interessiert.
\end{itemize}
Zur Funktionalanalysis:
Hier ist das Skript kaum zu schlagen, da die Themenwahl wirklich sehr speziell auf das Notwendigste für die Physik gelegt wurde. Die meisten Mathebücher würden die Konzepte aus dem zweiten Semesterteil auf zwei ganze Semester aufteilen. Und dazu kommt noch mehr Maßtheorie\dots. Nichtsdestotrotz, ein bisschen was fällt mir ein:
\begin{itemize}
\item Kaballo, \href{https://link.springer.com/book/10.1007/978-3-662-54748-9}{Grundkurs Funktionalanalysis}: Das Buch ist aus einem MfP-ähnlichen Modul entstanden und daher einigermaßen geeignet. Allgemein ist das Buch relativ nett geschrieben.
\item Hall, \href{https://link.springer.com/book/10.1007/978-1-4614-7116-5}{Quantum Theory for Mathematicians}: Hier wird die Funktionalanalysis aus Sicht der Physik mit mittelmäßigem mathematischen Anspruch betrachtet. Mathematisch streng veranlagte Leute können mit dem Buch oft nicht so viel anfangen, manchen gereicht es aber auch zur Rettung. Einfach mal reinlesen :)
\end{itemize}
Im Allgemeinen ist MfP4 meiner Erfahrung nach, vor allem wenn noch Differentialgeometrie abgefragt wird, sehr gut machbar, vor allem im Vergleich zu den sehr dichten MfP2- und MfP3-Modulen. Erfahrungsgemäß mögen viele die Funktionentheorie sehr und können damit gut umgehen. Funktionalanalysis fällt hingegen meist schwer, da das Thema vielen weniger intuitiv scheint. Im besten Fall findet man aber an beidem Gefallen, im schlimmsten Fall kommt man mit DiffGeo und Funktionentheorie sicher durch die Klausur.


