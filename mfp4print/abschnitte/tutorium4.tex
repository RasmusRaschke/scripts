\section{Tutorium 08.05.25}
\label{sec:08_04_25}

Nachdem ihr nun voll in die Funktionentheorie eingestiegen seid, wollen wir uns damit etwas näher beschäftigen:
 
\subsection{Funktionentheorie ist \textit{nicht} reelle Analysis in zwei Variablen}
Wenn wir nun zur Analysis auf $\C$ übergehen, ist man leicht verleitet, den $\R-$\red{Vektorraum}-Isomorphismus $\C \cong \R^2$ so zu interpretieren, dass Funktionentheorie doch eigentlich nichts anderes sei als Analysis im $\R^2$. Das ist aber ein fataler Fehler! Der Schlüssel liegt darin, dass auf $\C$ eine andere Struktur, die Multiplikation mit komplexen Zahlen, definiert ist, die über die Vektorraumstruktur des $\R^2$ hinausgeht.\\
In der reellen Analysis war die Erkenntnis, dass sich differenzierbare Funktionen durch ihr Differential annähern lassen, fundamental. Ist $f: \R^2 \to \R^2$ eine beliebige, reell-differenzierbare Funktion, so ist ihr Differential am Punkt $p \in  \R^2$ eine reell-lineare Abbildung
\begin{equation}
\begin{split}
df_p: \R^2 \to \R^2\\
x \mapsto df_p (x).
\end{split}
\end{equation} 
Mit der Wahl einer Basis des $\R^2$ lässt sich das Differential als Matrix
\begin{equation}
Df_p := \left. \mat{\frac{\partial f_1}{\partial x_1}, \frac{\partial f_1}{\partial x_2}}{\frac{\partial f_2}{\partial x_1}, \frac{\partial f_2}{\partial x_2}}\right|_p 
\end{equation}
darstellen, genannt \textit{Funktionalmatrix}. Dies ist also einfach eine Matrix der Form $\mat{\alpha, \beta}{\gamma,\delta}$ mit vier reellen Einträgen.\\
Wie sieht das im Komplexen aus? Ist ein komplex-lineares Differential auch einfach eine beliebige $2 \times 2$-Matrix mit reellen Einträgen? Fangen wir mal mit der komplexen Linearität an:
\begin{definition}{Komplex-Lineare Abbildung}{komplexlinear}
Eine Abbildung $A: \C \to \C$ heißt \textbf{komplex-linear}, falls für alle $z_1,z_2,\lambda, \mu \in \C$ gilt:
\begin{equation}
A(\lambda z_1+\mu z_2) = \lambda A(z_1)+ \mu A(z_2).
\end{equation}
\end{definition} 
Wir fordern für eine Funktion $f: \C \to C$ nun, dass das Differential $df_p: \C \to \C$ nicht bloß linear, sondern komplex-linear ist. Bereits erwähnt habe ich, dass Addition nichts Neues ist. Diese funktioniert auf $\R^2$ und $\C$ völlig analog, nämlich komponentenweise. Neu ist hingegen die Multiplikation mit einer komplexen Zahl, aber was bedeutet das für die darstellende Matrix für das Differential? Wenn wir nur Multiplikation mit komplexen Zahlen betrachten wollen und uns nicht für Addition interessieren, liegt es nahe, erst einmal nur $\C$ mit Multiplikation zu betrachten. Wir gehen es noch langsamer an und schauen, was passiert, wenn wir nur komplexe Zahlen mit Radius $r=1$ zulassen, also die Einheitskreislinie $\S^1 \sub \C$ anschauen. Praktischerweise ist $(\S^1, \cdot)$ eine Gruppe, die einer uns bekannten Gruppe sehr ähnlich sieht. Sind $z_1 = \exp(i\phi_1)$ und $z_2 = \exp(i \phi_2)$ zwei komplexe, normierte Zahlen, so ist ihr Produkt gegeben durch
\begin{equation}
z_1z_2 = \exp(i(\phi_1+\phi_2)).
\end{equation}
Dass sich die Winkel addieren, ist der entscheidende Hinweis für den gesuchten Isomorphismus:
\begin{satz}{Multiplikation ist Rotation}{komplexmat}
Es existiert ein Gruppen-Isomorphismus $(\S^1, \cdot) \cong \so{2}$ zwischen dem Einheitskreis
\begin{equation}
\S^1 := \{x+iy \in \C \mid x^2+y^2=1\} \sub \C
\end{equation}
mit komplexer Multiplikation und der speziellen orthogonalen Matrizengruppe.
\end{satz}
\begin{beweis}
Aus MfP2 wissen wir, dass sich $\so{2}$-Matrizen parametrisieren lassen als
\begin{equation}
\mat{\cos \phi, -\sin \phi}{\sin \phi, \cos \phi} =: R(\phi)
\end{equation}
mit $\phi \in (-\pi, \pi]$. Darauf aufbauend ist der Isomorphismus mit der Polarform schnell konstruiert: Wir behaupten, dass 
\begin{equation}
\begin{split}
\psi: (\S^1, \cdot) &\to \so{2}\\
\exp(i\phi) &\mapsto \mat{\cos \phi, -\sin \phi}{\sin \phi, \cos \phi}
\end{split}
\end{equation}
passt. Dabei haben wir den Einheitskreis in der komplexen Ebene mit komplexen Zahlen des Radius $r=1$ identifiziert. Also fehlen nur noch die Eigenschaften:
\begin{enumerate}[({I}1)]
\item Gruppenhomomorphismus: Seien $\exp(i\phi_1), \exp(i \phi_2) \in \S^1$. Dann gilt:
\begin{align*}
\psi \left( \exp(i\phi_1)\right) \cdot \psi \left(\exp(i \phi_2)\right) &=  \mat{\cos \phi_1 , -\sin \phi_1}{\sin \phi_1, \cos \phi_1}\mat{\cos \phi_2, -\sin \phi_2}{\sin \phi_2, \cos \phi_2}\\
&= \mat{\cos \phi_1 \cos \phi_2 - \sin \phi_1 \sin \phi_2 , -(\sin \phi_2 \cos \phi_1 + \sin \phi_1 \cos \phi_2)}{\sin \phi_1 \cos \phi_2 + \sin \phi_2 \cos \phi_1, -\sin \phi_1 \sin \phi_2 + \cos \phi_1 \cos \phi_2}\\
&= \mat{\cos(\phi_1+\phi_2) , -\sin(\phi_1+\phi_2)}{\sin(\phi_1 + \phi_2), \cos(\phi_1+\phi_2)}=\psi \left( \exp(i\phi_1+\phi_2)\right)\\
&=\psi(\exp(i\phi_1)\exp(i\phi_2)).
\end{align*}
Benutzt haben wir dabei lediglich die beiden gängigen Additionstheoreme.
\item Bijektiv: Das ist tatsächlich trivial, man kann die Inverse von $\psi$ Dank der Parametrisierung von $\so{2}$ direkt ablesen.
\end{enumerate}
\end{beweis}
Das ist doch mal ein Ergebnis. Jetzt wissen wir, dass Multiplikation mit komplexen Einheiten äquivalent zu Rotationen in der Ebene sind. Aber was passiert, wenn man beliebige komplexe Zahlen und nicht nur normierte zulässt? Das Verhalten kennen wir schon aus der linearen Algebra: Multiplikation mit einem reellen Skalar bewirkt eine Streckung oder Stauchung.\\
Komplex-lineare Abbildungen werden also von Matrizen der Form
\begin{equation}
\mat{r \cos \phi, - r \sin \phi}{r \sin \phi, r \cos \phi} =: \mat{\alpha, -\beta}{\beta, \alpha}
\end{equation}
repräsentiert. Dies ist auf $\C$ gerade die Multiplikation mit der komplexen Zahl $z=\alpha+i \beta \neq 0$ und eine starke Einschränkung gegenüber dem reellen Fall! Betrachten wir z.B. die Multiplikation mit $z=i \in \C$, dann entspricht dies der linearen Abbildung
\begin{equation}
\mat{0, -1}{1, 0},
\end{equation}
also der Drehung um $90^\circ$ gegen den Uhrzeigersinn.\\
Eine komplex-differenzierbare Funktion muss also nicht bloß lokal wie eine reell-lineare Abbildung, sondern sogar wie eine Drehskalierung aussehen. Diese reichere Struktur wird viele, sehr schöne Phänomene in der Funktionentheorie zur Folge haben.\\
Da wir nun so weit sind, bietet sich noch ein Vergleich der Funktionalmatrix mit der neuen Differentialmatrix an: Fassen wir $f: \C \to \C$ auf als $f: \R^2 \to \R^2$ mit $f(x,y) = f_1(x,y)+if_2(x,y)$ und vergleichen, so erhalten wir
\begin{align*}
\alpha &= \frac{\partial f_1}{\partial x} = \frac{\partial f_2}{\partial y}\\
\beta  &= \frac{\partial f_2}{\partial x} = -\frac{\partial f_1}{\partial y}.
\end{align*}
Überraschung (oder auch nicht), das sind die \textbf{Cauchy-Riemannschen Differentialgleichungen}. Diese entsprechen also der Anforderung, dass unsere Funktion komplex-differenzierbar ist.

