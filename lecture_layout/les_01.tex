%\lesson{1}{wo 25 sep 2019 15:00}{Introduction}
%\begin{itemize}
%    \item Marco Zambon. TA: R. Storm
%    \item Literature: Tu, Lee
%    \item Exercises: Mondays. Look at exercises in advance. (part of exam is based on them)
%    \item Take home sheets: $2$, two weeks to solve them. 6/20. (solving together is allowed.) \emph{Write up by hand; no typing}
%    \item Exam: 16/20.
%        \begin{itemize}
%            \item 2 problems (closed book) (inspired by homework problems)
%            \item 2 questions (closed book)
%        \end{itemize}
%        Oral defence of this and one TH.
%    \item Time: 16:10 -- 18:00
%\end{itemize}

\chapter{Differentiable manifolds}
\setcounter{section}{-1}
\section{Topological spaces}

\begin{definition}[Topological space]
    A topological space is $(X, \sigma)$ where  $X$ is a set and $\sigma$
  a family of subsets of $X$, called  open sets, such that:
    \begin{itemize}
        \item $\O, X \in \sigma$ 
        \item $\bigcup_{i \in I} U_i \in \sigma$ whenever $ U_i \in \sigma$ for all $i$
        \item $\bigcap_{i<n} U_i \in \sigma$ whenever $ U_i \in \sigma$ for all $i$
    \end{itemize}
\end{definition}

Let $(X, \sigma)$ be a topological space.
\begin{definition}[Open neighbourhood]
    An open subset that contains $p \in X$ is called a (open) neighbourhood of $p$.
\end{definition}
\begin{definition}[Subspace topology]
    If $Y \subset X$ then $(Y, \sigma_Y)$ is a topological space, where
    \[
    \sigma_Y = \{U \cap Y  \mid  U \in \sigma\} 
    .\]  We call $\sigma_Y$ the subspace topology.
\end{definition}
\begin{eg}
Endowing $\mathbb{R}^2$ with the Euclidean topology,
the subspace topology on    $\mathbb R \times \{0\}\subset \mathbb{R}^2$ is also the
 Euclidean topology.
\end{eg}

\begin{definition}[Quotient topology]
    Let $\sim $ be an equivalence relation on $X$.
    Consider $\pi: X \to  X / \sim $.
    Then $X / \sim $ is a topological space, where the open sets are by definition sets such that $\pi^{-1}(U)$ is open in $X$.
\end{definition}
\begin{definition}[Continuous functions]
    A function $f: X_1 \to  X_2$ is called continuous iff $\forall  U \in \sigma_2: f ^{-1}(U) \in \sigma_1$.
\end{definition}
\begin{definition}
    A topological space is called Hausdorff iff
    $\forall x, y \in X$, there exist neighbourhoods $U$ of $x$, $V$ of $y$ such that $U \cap V = \O$.
\end{definition}
\begin{eg}
%[Not Hausdorff]
 Endow $\mathbb R^2 \setminus \{0\}$   with the equivalence relation given by the thick lines and the two half lines in the following figure. That is:
    \[
        (x, y) \sim (x', y') \iff \begin{cases}
            x = x' & \text{if $x \neq 0$}\\
            y y' > 0 & \text{if $x = 0$.}
        \end{cases}
    \] 
  Then the quotient topology on $(\mathbb R^2 \setminus \{0\}) / \sim $ is not Hausdorff.
\end{eg}
\begin{figure}[ht]
    \centering
    \incfig{not-hausdorff-example}
    \caption{Example of a topology which is not Hausdorff}
    \label{fig:not-hausdorff-example}
\end{figure}
\begin{definition}[Basis for a topology]
    A basis for a topology is $S \subset \sigma$ such that every open set of $X$ is a union of elements of $S$.
\end{definition}

\begin{definition}[C2]
    A space $(X, \sigma)$ is second countable if there exists a countable basis.
\end{definition}
\begin{eg}
    $\mathbb R^{n}$ is second countable. 
   Indeed $\{ B_{\frac{1}{m}}(x)  \mid  x \in \mathbb Q^{n}, m \in \N\} $ is a countable basis for the topology. Here $B_{r}(x)$ is the open ball with radius $r$ around $x$.
\end{eg}
\section{Differentiable manifolds}
\begin{definition}[Topological manifold]
    A topological manifold $M$ of dimension~$m$ is a second countable, Hausdorff topological space which is locally homeomorphic to $\mathbb R^{m}$.
\end{definition}
\begin{remark}
`Locally homeomorphic to $\mathbb R^{m}$' means that $\forall p \in M, $ there exists a neighborhood $U$ of $p$ and a homeomorphism $\phi: U \to  V \underset{\text{open}}{\subset} \mathbb R^{m}$. Recall that homeomorphism means: bijective map that is continuous in \emph{both} directions.   
\end{remark}

\begin{definition}[Chart]
    The pair $(U, \phi)$ is called a chart.
\end{definition}

\begin{remark}\leavevmode 
    \begin{itemize}
        \item Any subset of a Hausdorff space is Hausdorff
        \item Any subset of a C2 space is C2.
    \end{itemize}
\end{remark}



\begin{definition}[Compatibility]
    Two charts $(U_1, \phi_1)$ and $(U_2, \phi_2)$ are called smoothly compatible if 
    \[
        \phi_2 \circ(\phi_1)^{-1}  |_{\phi_1(U_1 \cap  U_2)}: \phi_1(U_1 \cap  U_2) \to  \phi_2(U_1 \cap U_2)
    \] 
    is a diffeomorphism (i.e. bijective, differentiable and inverse differentiable, where ``differentiable'' means that all partial derivatives exist).
\end{definition}

\begin{figure}[ht]
    \centering
    \incfig{compatible}
    \caption{Compatible charts}
    \label{fig:compatible}
\end{figure}

\begin{definition}[Smooth atlas]
    A smooth atlas for $M$ is called a collection of charts $\{(U_\alpha, \phi_\alpha)\}$ such that $\bigcup_{\alpha} U_\alpha = M$ and any two charts are smoothly compatible.
\end{definition}
\begin{definition}[Maximal smooth atlas]
    A smooth atlas $\mathcal A$ is maximal if: whenever $\mathcal B$ is a smooth atlas and $\mathcal A \subset \mathcal B$ then $\mathcal B = \mathcal A$.
\end{definition}
\begin{definition}[Differentiable manifold ]
    A differentiable manifold (also called a smooth manifold) is a topological manifold $M$ together with a maximal smooth atlas.
\end{definition}
\begin{remark}
    Given a smooth atlas $\mathcal A$ on a topological manifold  $M$, there exists a unique maximal smooth atlas containing it, namely
    \[
        \{(V, \psi)  \mid  (V, \psi) \text{ is smoothly compatible with all charts of $\mathcal A$}\} 
    .\] 
\end{remark}

\begin{eg}
    Let $U \subset \R^n$. It's a smooth manifold: an atlas is $\{(U, \text{Id})\}$. Take the maximal smooth atlas containing it.
\end{eg}
\begin{eg}
    Let $S^{n} := \{\mathbf{x} \in \mathbb R^{n+1}  \mid \|\mathbf{x}\| = 1\}$.
    The sphere $S^{n}$ with the subspace topology is Hausdorff and C2, simply because $\mathbb R^{n+1} $ is.
    Two charts are given by the stereographic projections from the Northpole $N$ and Southpole $S$:
    \begin{align*}
        \phi_N: S^{n} \setminus \{N\} \to \mathbb R^{n}: (x_1, \ldots, x_{n +1}) &\mapsto \frac{(x_1, \ldots, x_n)}{1 - x_{n+1}}\\
        \phi_S: S^{n} \setminus \{S\} \to \mathbb R^{n}: (x_1, \ldots, x_{n+1}) &\mapsto \frac{(x_1, \ldots, x_n)}{1 + x_{n+1}}
    .\end{align*}

    Now, $\phi_N$ and  $\phi_S$ are homomorphisms.
    Furthermore, $\|\phi_N(p)\|\cdot \|\phi_S(p)\| = 1$, which allows us to calculate the inverse of $\phi_{N}$.
    Hence
    \[
        (\phi_S \circ \phi_N ^{-1})|_{\phi_N(S^n\setminus\{N,S\})}: \R^n \setminus \{0\} \to  \R^n \setminus \{0\}: y \mapsto \frac{y}{\|y\|^2}
    ,\] 
    so $\phi_N$ and  $\phi_S$ are smoothly compatible.
    Take the maximal smooth atlas containing  $\phi_N$ and  $\phi_S$.
\end{eg}
\begin{remark}
    We could have started with other points $P,Q\in S^n$ instead of $N, S$. The smooth atlases $\{\phi_P,\phi_Q\}$ and $\{\phi_N,\phi_S\}$ would be different, but they define    the same maximal smooth atlas.
\end{remark}

\section{Differentiable map}
Let $M$ be a smooth manifold.

\begin{definition}[Smooth function]
    A function $f: M \to  \mathbb R$ is differentiable (or smooth) at $p \in M$ iff $\exists$ a chart $(U, \phi)$ around $p$ such that $f \circ \phi^{-1}$ is differentiable in $\phi(p)$.
\end{definition}


\begin{figure}[H]
    \centering
    \incfig{smooth-function-to-r}
    \caption{Smooth function from $M$ to $\mathbb R$}
    \label{fig:smooth-function-to-r}
\end{figure}

\begin{remark}
    If $f \circ \phi^{-1}$ is differentiable at $p$ for a chart $(U, \phi)$, then $f \circ \psi^{-1}$ is also differentiable at $p$, for any other chart $(V,\psi)$ (in the maximal atlas of $M$).
\end{remark}
\begin{proof}
    We want to argue that $f \circ \psi^{-1}$ is smooth.
    \[
        f \circ \psi^{-1} = \underbrace{(f \circ \phi^{-1})}_{C^{\infty}} \circ \underbrace{(\phi \circ \psi^{-1})}_{C^{\infty}}
    .\] 
\end{proof}
\begin{notation}
    We write $C^{\infty}(M)$ to denote all smooth functions  $M \to  \mathbb R$.
\end{notation}
\begin{definition}[Smooth map]

    $f: M \to  N$ is differentiable at $p \in M$ iff
    \begin{itemize}
        \item it is continuous
        \item there exists charts $(U_M,\phi_M)$ around $p$ and $(U_N,\phi_N)$  around $f(p)$ such that 
            $\phi_N \circ f \circ \phi_M^{-1}$ is differentiable at $\phi_M(p)$
    \end{itemize}

\end{definition}
\begin{remark} The map $\phi_N \circ f \circ \phi_M^{-1}$  is defined on 
$\phi_M(U_M\cap f^{-1}(U_N))$.
The continuity of $f$ ensures that this is an open neighborhood of $\phi_M(p)$ in $\mathbb{R}^m$, hence it makes sense to talk about the differentiability of the above map at 
$\phi_M(p)$.
\end{remark}

\begin{figure}[ht]
    \centering
    \incfig{smooth-function-from-m-to-n}
    \caption{Smooth function from $M$ to $N$}
    \label{fig:smooth-function-from-m-to-n}
\end{figure}

\begin{remark}
A map $f$ being a differentiable and a homeomorphism does not imply that $f$ is a diffeomorphism (which also include the differentiability of the inverse).
For instance, $f\colon \mathbb R\to \mathbb R, x\mapsto x^3$ is not a diffeomorphism, because the inverse $x\mapsto \sqrt[3]{x}$ is not differentiable at zero.
\end{remark}
