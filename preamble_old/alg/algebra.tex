\documentclass[10pt]{extarticle}

\usepackage[utf8]{inputenc}
\usepackage[T1]{fontenc}
\usepackage{textcomp}

\usepackage{url}

% \usepackage{hyperref}
% \hypersetup{
%     colorlinks,
%     linkcolor={black},
%     citecolor={black},
%     urlcolor={blue!80!black}
% }

\usepackage{graphicx}
\usepackage{float}
\usepackage[usenames,dvipsnames]{xcolor}

% \usepackage{cmbright}

\usepackage{amsmath, amsfonts, mathtools, amsthm, amssymb}
\usepackage{mathrsfs}
\usepackage{cancel}

% horizontal rule
\newcommand\hr{
    \noindent\rule[0.5ex]{\linewidth}{0.5pt}
}

\usepackage{tikz}
\usepackage{tikz-cd}

% theorems
\usepackage{thmtools}
\usepackage[framemethod=TikZ]{mdframed}
\mdfsetup{skipabove=1em,skipbelow=0em, innertopmargin=5pt, innerbottommargin=6pt}

\theoremstyle{definition}

\makeatletter

\declaretheoremstyle[headfont=\bfseries\sffamily, bodyfont=\normalfont, mdframed={ nobreak } ]{thmgreenbox}
\declaretheoremstyle[headfont=\bfseries\sffamily, bodyfont=\normalfont, mdframed={ nobreak } ]{thmredbox}
\declaretheoremstyle[headfont=\bfseries\sffamily, bodyfont=\normalfont]{thmbluebox}
\declaretheoremstyle[headfont=\bfseries\sffamily, bodyfont=\normalfont]{thmblueline}
\declaretheoremstyle[headfont=\bfseries\sffamily, bodyfont=\normalfont, numbered=no, mdframed={ rightline=false, topline=false, bottomline=false, }, qed=\qedsymbol ]{thmproofbox}
\declaretheoremstyle[headfont=\bfseries\sffamily, bodyfont=\normalfont, numbered=no, mdframed={ nobreak, rightline=false, topline=false, bottomline=false } ]{thmexplanationbox}


\declaretheorem[numberwithin=chapter, style=thmgreenbox, name=Definition]{definition}
\declaretheorem[sibling=definition, style=thmredbox, name=Corollary]{corollary}
\declaretheorem[sibling=definition, style=thmredbox, name=Proposition]{prop}
\declaretheorem[sibling=definition, style=thmredbox, name=Theorem]{theorem}
\declaretheorem[sibling=definition, style=thmredbox, name=Lemma]{lemma}



\declaretheorem[numbered=no, style=thmexplanationbox, name=Proof]{explanation}
\declaretheorem[numbered=no, style=thmproofbox, name=Proof]{replacementproof}
\declaretheorem[style=thmbluebox,  numbered=no, name=Exercise]{ex}
\declaretheorem[style=thmbluebox,  numbered=no, name=Example]{eg}
\declaretheorem[style=thmblueline, numbered=no, name=Remark]{remark}
\declaretheorem[style=thmblueline, numbered=no, name=Note]{note}

\renewenvironment{proof}[1][\proofname]{\begin{replacementproof}}{\end{replacementproof}}

\AtEndEnvironment{eg}{\null\hfill$\diamond$}%

\newtheorem*{uovt}{UOVT}
\newtheorem*{notation}{Notation}
\newtheorem*{previouslyseen}{As previously seen}
\newtheorem*{problem}{Problem}
\newtheorem*{observe}{Observe}
\newtheorem*{property}{Property}
\newtheorem*{intuition}{Intuition}


\usepackage{etoolbox}
\AtEndEnvironment{vb}{\null\hfill$\diamond$}%
\AtEndEnvironment{intermezzo}{\null\hfill$\diamond$}%




% http://tex.stackexchange.com/questions/22119/how-can-i-change-the-spacing-before-theorems-with-amsthm
% \def\thm@space@setup{%
%   \thm@preskip=\parskip \thm@postskip=0pt
% }

\usepackage{xifthen}

\def\testdateparts#1{\dateparts#1\relax}
\def\dateparts#1 #2 #3 #4 #5\relax{
    \marginpar{\small\textsf{\mbox{#1 #2 #3 #5}}}
}

\def\@lesson{}%
\newcommand{\lesson}[3]{
    \ifthenelse{\isempty{#3}}{%
        \def\@lesson{Lecture #1}%
    }{%
        \def\@lesson{Lecture #1: #3}%
    }%
    \subsection*{\@lesson}
    \testdateparts{#2}
}

% fancy headers
\usepackage{fancyhdr}
\pagestyle{fancy}

% \fancyhead[LE,RO]{Gilles Castel}
\fancyhead[RO,LE]{\@lesson}
\fancyhead[RE,LO]{}
\fancyfoot[LE,RO]{\thepage}
\fancyfoot[C]{\leftmark}
\renewcommand{\headrulewidth}{0pt}

\makeatother

% figure support (https://castel.dev/post/lecture-notes-2)
\usepackage{import}
\usepackage{xifthen}
\pdfminorversion=7
\usepackage{pdfpages}
\usepackage{transparent}
\newcommand{\incfig}[1]{%
    \def\svgwidth{\columnwidth}
    \import{./figures/}{#1.pdf_tex}
}

% %http://tex.stackexchange.com/questions/76273/multiple-pdfs-with-page-group-included-in-a-single-page-warning
\pdfsuppresswarningpagegroup=1

\author{Gilles Castel}

\makeindex

\title{Algebra (Bachelor)}
\author{zur Vorlesung von Prof. Dr. Tobias Dyckerhoff}
\date{\today} % Replace with \today to show the current date

\begin{document}

\maketitle

\definecolor{tcol_CNT1}{HTML}{72E094} % First color for Contents
\definecolor{tcol_CNT2}{HTML}{24E2D6} % Second color for Contents
\definecolor{tcol_CNV1}{HTML}{8E44AD} % First color for Conventions
\definecolor{tcol_CNV2}{HTML}{A10B49} % First color for Conventions

\begin{tcolorbox}[enhanced,
    title=Inhaltsverzeichnis,
    fonttitle=\fontsize{20}{24}\sffamily\bfseries\selectfont,
    coltitle=black,
    fontupper=\sffamily,
    interior style={left color=tcol_CNT1!80,right color=tcol_CNT2!80},
    frame style={left color=tcol_CNT1!60!black,right color=tcol_CNT2!60!black},
    attach boxed title to top center={yshift=10pt},
    boxed title style={frame hidden,
        interior style={left color=tcol_CNT1,right color=tcol_CNT2},
        frame style={left color=tcol_CNT1!60!black,right color=tcol_CNT2!60!black},
        height=24pt,bean arc,drop fuzzy shadow
    },
    top=2mm,bottom=2mm,left=2mm,right=2mm,
    before skip=20mm,after skip=20mm,
    drop fuzzy shadow,breakable]
%
\makeatletter
\@starttoc{toc}
\makeatother
\end{tcolorbox}

\begin{tcolorbox}[enhanced,
    frame hidden,
    title=Konventionen,
    fonttitle=\large\sffamily\bfseries\selectfont,
    interior code={
        \shade[top color=tcol_CNV2!50,bottom color=white] ([yshift=2mm]interior.north west) arc(-180:-90:2mm)--(interior.north east)--(interior.south east)--(interior.south west)--cycle;
        },
    overlay={
        \draw[tcol_CNV1!50!black,line width=0.5mm] ([xshift=2mm]frame.north west)--(frame.north east);
    },
    boxrule=0pt,left=2pt,right=2pt,
    sharp corners=north,
    attach boxed title to top left,
    boxed title style={interior hidden,
    left=1mm,right=1mm,
    frame code={
        \path[draw=tcol_CNV1!50!black,line width=0.5mm,fill=tcol_CNV1,rounded corners=2mm] ([xshift=2mm]frame.south east)--(frame.south east)--(frame.north east)--([xshift=0.25mm]frame.north west)--([xshift=0.25mm]frame.south west)--cycle;}
    },
    top=2mm,bottom=2mm,left=2mm,right=2mm,
    before skip=10mm,after skip=10mm]
%
\begin{itemize}
\item Wir schreiben für einen Körper $\K$ kurz $\K^\ast := \K \exc \{0\}$.
\item Real- und Imaginärteil werden mit $\Re(\cdot)$ respektive $\Im(\cdot)$ bezeichnet, das Bild einer Abbildung $f$ hingegen mit $\im (f)$.
\item Echte Teilmengen tragen das Symbol $\subset$, allgemeine Teilmengen das Symbol $\sub$.
\end{itemize}
\end{tcolorbox}
Dies ist ein inoffizielles Skript zur Vorlesung Algebra bei Prof. Dr. Tobias Dyckerhoff im Wintersemester 24/25. Fehler und Verbesserungsvorschläge immer gerne an \url{rasmus.raschke@uni-hamburg.de}.
\newpage 
\sloppy
\section{Gruppen und Symmetrie}
\label{gruppentheorie}

\begin{bemerkung}
Wir möchten Gruppentheorie zunächst motivieren: Man betrachte einen Tetraeder. Um dessen Symmetrien zu erfassen, könnten wir z.B. schauen, welche Bewegungen diesen in sich selbst überführen. Es gibt vier Rotationsachsen, die eine Ecke und eine Fläche durchdringen und bei Rotation um $120^\circ$ den Tetraeder in sich selbst überführen. Weiterhin gibt es drei $180^\circ$-Rotationsachsen mittig durch gegenüberliegende Kanten. Auch die Identität lässt den Tetraeder unverändert. Also gibt es $1+4 \cdot 2 + 3 = 12$ Symmetrien. Gruppen bieten eine Möglichkeit, solche Symmetrien und deren Verkettungen zu erfassen und zu untersuchen.
\end{bemerkung}

\subsection{Grundbegriffe}
\label{subsec:grundbegriffe}

\begin{definition}{Gruppe}{gruppe}
Eine \textbf{Gruppe} ist ein Paar $(G, \circ)$, bestehend aus einer Menge\footnote{im ZFC-Axiomensystem} $G$ und einer Abbildung
\begin{align}
\circ: G \times G &\to G\\
(g,h) &\mapsto g \circ h
\end{align} 
mit folgenden Eigenschaften:
\begin{enumerate}[({G}1)]
\item Für alle $g_1,g_2,g_3 \in G$ gilt das Assoziativgesetz: $(g_1 \circ g_2) \circ g_3 = g_1 \circ (g_2 \circ g_3)$.
\item Es gibt ein Element $e \in G$, sodass gilt:
\begin{enumerate}[({2}a)]
\item Für jedes $g \in G$ gilt $e \circ g = g$.
\item Für jedes $g \in G$ existiert ein $g' \in G$ mit $g' \circ g = e$.
\end{enumerate}
\end{enumerate}
Die Abbildung $\circ$ heißt \textbf{Verknüpfung}, ein Element $e \in G$ mit den Eigenschaften aus (2G) heißt \textbf{neutrales Element}, und ein Element $g' \in G$ zu gegebenem $g \in G$ mit Eigenschaft (2b) heißt \textbf{Inverses} von $g$.
\end{definition}

\begin{übung}
Sei $(G, \circ)$ eine Gruppe. Dann gelte:
\begin{enumerate}
\item Das neutrale Element $e \in G$ ist eindeutig bestimmt, außerdem gelte $ \forall g \in G : g \circ e = g$.
\item Zu gegebenem $g \in G$ ist das Inverse $g' \in G$ eindeutig bestimmt und erfüllt zudem $g \circ g' = e$.
\item Für $n \geq 3$ hängt das Produkt von Gruppenelementen $g_1, g_2, \dots, g_n$ nicht von der Klammerung ab.
\end{enumerate}
\end{übung}

\begin{lösung}
Zuerst zeigen wir Kommutativität des Inversen. Sei $g \in G$, dann gilt:
\begin{align}
g \circ g^{-1} &= (e \circ g) \circ g^{-1} = \left( \left( \left( g^{-1}\right)^{-1} \circ g^{-1} \right) \circ g \right) \circ g^{-1} = \left(  \left( g^{-1}\right)^{-1} \circ \left( g^{-1} \circ g \right)\right) \circ g^{-1}\\ 
&= \left( g^{-1}\right)^{-1} \circ \left( e  \circ g^{-1} \right) = \left( g^{-1}\right)^{-1} \circ g^{-1} = e = g^{-1} \circ g,
\end{align}
also stimmen Links- und Rechtsinverses in Gruppen überein.
Die Kommutativität des neutralen Elements folgt damit direkt aus:
\begin{equation}
g \circ e = g \circ (g^{-1} \circ g) = (g \circ g^{-1}) \circ g = (g^{-1} \circ g) \circ g = e \circ g,
\end{equation}
womit auch Links-Einselement und Rechts-Einselement übereinstimmen.
Für die Eindeutigkeit des Inversen seien $g^{-1}, g'^{-1} \in G$ zwei Inverse von $g \in G$. Dann gilt:
\begin{equation}
g^{-1} = g^{-1} \circ e = g^{-1} \circ (g'^{-1} \circ g) = g^{-1} \circ (g \circ g'^{-1}) = (g^{-1} \circ g) \circ g'^{-1} = e \circ g'^{-1} = g'^{-1}.
\end{equation}
Weiterhin seien $e,e' \in G$ zwei Einselemente. Da $e = e \circ e' = e' \circ e = e$ gilt, ist das neutrale Element eindeutig. \qed
\end{lösung}

\begin{beispiele}
Wir geben einige Beispiele für Gruppen:
\begin{enumerate}
\item Die Gruppe $(\Z, +)$ der ganzen Zahlen $\Z$ mit der Addition $+$.
\item Für einen Körper $\K$ existiert die additive Gruppe $(\K, +)$ und die multiplikative Gruppe $(\K \exc \{0\}, \cdot)$.
\item Für jede Menge $M$ existiert die \textbf{symmetrische Gruppe} $(\mathfrak{S}_M, \circ)$, wobei $\mathfrak{S}_M$ die Menge der bijektiven Selbstabbildungen von $M$ und $\circ$ die Komposition ist. Für $n \geq 1$ vereinbaren wir $\mathfrak{S}_n := \mathfrak{S}_{\{1,2,\dots,n\}}$. Wir vereinbaren als Konvention die \textbf{Zykelschreibweise}. In $\mathfrak{S}_3$ beispielsweise ist ein Zykel
\begin{align}
\sigma: \{1,2,3\} &\to \{1,2,3\}\\
1 &\mapsto 2 \\
2 &\mapsto 1 \\
3 &\mapsto 3,
\end{align}
auch darstellbar als
\begin{equation}
\mat{1,2,3}{2,1,3}
\end{equation}
oder einfacher als $(12)$.
\item Für $n \geq 1$ und einen Körper $\K$ ist die \textbf{allgemeine lineare Gruppe} $(\text{GL}(n,\K), \circ)$ definiert, wobei 
\begin{equation}
\text{GL}(n, \K) := \left\{ A \in \K^{n\times n} \, | \, \det A \neq 0 \right\}
\end{equation}
die Menge der invertierbaren $n \times n$-Matrizen mit Einträgen in $\K$ ist. Typische Beispiele für Körper sind $\K = \Q, \R, \C, \F_q$ mit $q = p^n$, $p$ prim.\\
ÜA: $| \text{GL}(n, \F_q)| = ?$.
\end{enumerate}
\end{beispiele}
\begin{bemerkung}
Um den alltäglichen Gebrauch von Gruppen zu vereinfachen, machen wir folgende Vereinbarungen:
\begin{enumerate}
\item Wir bezeichnen $(G, \circ)$ üblicherweise einfach mit $G$ und lassen $\circ$ implizit.
\item Für $g,h \in G$ schreiben wir $gh = g \circ h$, für $e \in G$ schreiben wir $1$ und für $g'$ schlicht $g^{-1}$.
\item Gilt $g \circ h = h \circ g$ für alle $g,h \in G$, so heißt $G$ \textbf{abelsch}. In diesem Fall wird die Verknüpfung oft mit $+$, das neutrale Element mit $0$ und das inverse Element mit $-g$ bezeichnet.
\item Gemäß obiger ÜA zur Klammerung schreiben wir einfach $g_1 g_2 \cdots g_n \in G$ ohne Klammerung.
\end{enumerate}
\end{bemerkung}
\begin{definition}{Ordnung}{ordnung}
Für eine Gruppe $G$ bezeichnen wir die Kardinalität \begin{equation}
|G| \in \N \cup \{+\infty \}
\end{equation}
als \textbf{Ordnung} von $G$.
\end{definition}

\subsection{Untergruppen}
\label{subsec:untergruppen}
\begin{definition}{Untergruppe}{untergruppe}
Sei $(G, \circ)$ eine Gruppe. Eine Teilmenge $H \sub G$ heißt \textbf{Untergruppe}, falls gilt:
\begin{enumerate}[({U}1)]
\item $H \neq \emptyset$
\item Abgeschlossenheit: Für alle $a, b \in H$ gilt $ab^{-1} \in H$.
\end{enumerate}
Wir verwenden dann die Notation $H \leq G$, um Untergruppen zu kennzeichnen.
\end{definition}

\begin{bemerkung} Übungsaufgabe:
Sei $G$ eine Gruppe und $H \leq G$ eine Untergruppe. Dann gilt:
\begin{enumerate}
\item Aus Eigenschaft 1:Da $H \neq \emptyset$, existiert ein $a \in H$.
\item Aus Eigenschaft 2: $a \cdot a^{-1} = e \in H$.
\item Aus Eigenschaft 2: Für jedes $a \in H$ gilt $a^{-1} = e \cdot a^{-1} \in H$.
\item Aus Eigenschaft 2: Für jedes $a,b \in H$ gilt $ab = a \cdot (b^{-1})^{-1} \in H$.
\end{enumerate}
Also: $H \sub G$ ist eine Untergruppe genau dann, wenn folgende alternativen Eigenschaften gelten:
\begin{enumerate}[1.{$^\ast $}]
\item $e_G \in H$
\item Für alle $a,b \in H$ muss $a \cdot b \in H$ gelten.
\item Für alle $a \in H$ ist $a^{-1} \in H$.
\end{enumerate}
Die andere Richtung der Äquivalenz ist trivial. Daraus folgt auch, dass $(H, \circ|_{H \times H})$ mit der auf $H$ eingeschränkten Verknüpfung $\circ|_{H \times H}$ eine Gruppe ist. 
\end{bemerkung}
\begin{beispiele}
Einige Beispiele für Untergruppen sind:
\begin{enumerate}
\item $(G, \circ) = (\R, +)$ hat $(\Z, +)$ als Untergruppe mit $\Z \sub \R$.
\item Sei $n \geq 1$ und $\K$ ein Körper. Die \textbf{spezielle lineare Gruppe}
\begin{equation}
\text{SL} (n, \K) := \left\{ A \in \text{GL}(n, \K) \, | \, \det A = 1\right\} \leq \text{GL}(n, \K)
\end{equation}
ist eine Untergruppe von $\text{GL}(n,\K)$.
\item Für $n \geq$ und einen Körper $\K$ ist die \textbf{orthogonale Gruppe}
\begin{equation}
\text{O}(n, \K) := \left\{ A \in \text{GL}(n, \K) \, | \, A^TA = I_n \right\} \leq \text{GL}(n, \K)
\end{equation}
definiert, die auch eine Untergruppe von $\text{GL}(n, \K)$ ist.
\item Seien $H_1, H_2 \leq G$ Untergruppen. Dann ist $H_1 \cap H_2 \leq G$ auch eine Untergruppe. So kann z.B. die \textbf{spezielle orthogonale Gruppe}
\begin{equation}
\text{SO}(n, \K) := \text{O}(n, \K) \cap \text{SL}(n, \K)
\end{equation}
als Untergruppe von $\text{GL}(n, \K)$ konstruiert werden.
\item Etwas allgemeiner: Für jede Familie $\{H_i\}_{i\in I}$ von Untergruppen $H_i \leq G$ gilt, dass
\begin{equation}
\bigcap_{i \in I} H_i \leq G
\end{equation}
wieder eine Untergruppe ist.
\end{enumerate}
\end{beispiele}
\begin{definition}{Erzeugte Untergruppe}{}
Sei $G$ eine Gruppe und $M \sub G$ eine beliebige Teilmenge. Dann heißt die \textbf{Untergruppe}
\begin{equation}
\langle M \rangle := \bigcup_{M \sub H \leq G} H \leq G
\end{equation}
die \textbf{von} $M$ \textbf{erzeugte Untergruppe} von $G$. Falls $M = \{g\} \leq G$ eine einelementige Menge ist, schreiben wir \begin{equation}
\langle g \rangle := \langle \{g\} \rangle \leq G.
\end{equation}
\end{definition}
\begin{definition}{Ordnung eines Elements}{ordnungelement}
Sei $G$ eine Gruppe und $g \in G$ ein Element. Dann heißt die Kardinalität 
\begin{equation}
\text{ord}(g) := | \langle g \rangle | \in \N \cup \{\infty \}
\end{equation}
die \textbf{Ordnung von} $g$.
\end{definition}
\begin{satz}{Charakterisierung von einelementigen Untergruppen}{charakterisierungeinelementig}
Sei $G$ eine Gruppe und $g \in G$ ein Element.
\begin{enumerate}
\item Falls $\ord (g) < \infty$, dann gilt \begin{equation}
\ord (g) = \min \{k \geq 1 | g^k= 1\}
\end{equation}
und \begin{equation}
\langle g \rangle = \{1, g, g^2, \dots, g^{n-1} \},
\end{equation}
wobei $n := \ord (g)$.
\item Falls $\ord (g) = \infty$, dann gilt
\begin{equation}
\langle g \rangle = \{g^i | i \in \Z \} = \{ \dots, g^{-2}, g^{-1}, 1, g^1, g^2, \dots \},
\end{equation}
wobei die Potenzen $g^i$, $i \in \Z$ paarweise verschiedene Elemente in $G$ sind.
\end{enumerate}
\end{satz}
\begin{beweis}
Zunächst gilt für beliebiges $g \in G$ das Folgende:
\begin{equation}
\langle g \rangle = \{ \dots,g^{-2},g^{-1},1, g, g^2, \dots \} = \{g^i | i \in \Z\},
\end{equation}
wobei die Potenzen im Allgemeinen nicht notwendigerweise paarweise verschieden sind.
Dies folgt, da, damit $\langle g \rangle$ eine Untergruppe sein kann, zunächst das neutrale Element $1 = g^0$ und $g$ selbst enthalten sein muss. Dann muss aber auch die Selbstverknüpfung und das Inverse (sowie dessen Selbstverknüpfungen) enthalten sein.
\begin{enumerate}
\item Sei $\ord (g) < \infty$. Dann gibt es insbesondere $i,j \in \Z$ mit $i \neq j$ und $g^i = g^j$. O.B.d.A. sei $i > j$. Dann ist also $k = i-j \geq 1$ eine natürliche Zahl, für die gilt: $g^k = 1$. Nach dem Wohlordnungssatz existiert eine \textit{kleinste} natürliche Zahl $n \geq 1$, für die gilt: $g^n =1$. Sei nun $m \in \Z$. Dann gibt es eindeutig bestimmte Zahlen $a \in \Z$ und $0 \leq r < n$, sodass \begin{equation}
m = an +r.
\end{equation}
Damit folgt
\begin{equation}
g^m = g^{an+r} = (\underbrace{g^n}_{=1})^a \cdot g^r = g^r.
\end{equation}
Dies impliziert $\langle g \rangle = \{1,g,g^2,\dots, g^{n-1}\}$, da $r$ der Rest ist, der bei der Division von $n$ durch $m$ bleibt. Die möglichen Reste für gegebenes $n$ legen also die Elemente von $G$ fest.\\
Wir müssen noch zeigen, dass $1, g, \dots, g^{n-1}$ paarweise verschieden sind. Dies folgt allerdings direkt aus der Tatsache, dass $n$ minimal ist.
\item Das obige Argument zeigt per Kontraposition auch 2., denn wenn die Potenzen $g^i$, $i \in \Z$ nicht paarweise verschieden sind, dann zeigt obiges Argument, dass $\langle g \rangle = \{1,g,\dots, g^{n-1}\}$ für $n \in \N$, was ein Widerspruch zur Annahme $\ord (g) = \infty$ ist.
\end{enumerate}
\end{beweis}
\begin{definition}{zyklische Gruppe}{zyklisch}
Sei $G$ eine Gruppe. Existiert ein $g\in G$, sodass sich jedes $h \in G$ als $g^n = h$ für ein $n \in \Z$ schreiben lässt, heißt $G$ \textbf{zyklisch der Ordnung} $\ord (g)$. Das Element $g$ heißt \textbf{Erzeuger von} $G$.
\end{definition}
\subsection{Homomorphismen}
\label{subsec:homomorphismen}
\begin{definition}{Homomorphismus}{homomorphismus}
Seien $G$ und $G'$ Gruppen. Eine Abbildung 
\begin{equation}
\phi: G \to G'
\end{equation}
heißt \textbf{(Gruppen-)Homomorphismus}, falls gilt:
\begin{enumerate}[({H}1)]
\item Für alle $g,h \in G$ gilt 
\begin{equation}
\phi(gh) = \phi(g) \cdot \phi(h).
\end{equation}
Die Menge der Homomorphismen von $G$ nach $G'$ wird mit $\Hom (G, G')$ bezeichnet.
\end{enumerate}
\end{definition}
\begin{bemerkung}
Jeder Homomorphismus erfüllt außerdem folgende Eigenschaften, die aus Definition \ref{homomorphismus} folgen:
\begin{enumerate}[({H}1)]
\setcounter{enumi}{1}
\item $\phi (1_G) = 1_{G'}$
\item Für alle $g \in G$ gilt $\phi (g^{-1}) = \phi (g)^{-1}$.
\end{enumerate}
Das sieht man schnell, da $\phi(1) = \phi(1g) = \phi(1) \phi(g)$ gilt, also $\phi(1) = 1'$ sein muss. Weiterhin gilt $1' = \phi(1) = \phi(gg^{-1}) = \phi(g) \phi(g^{-1})$, Linksmultiplikation mit $\phi^{-1}(g)$ liefert (H3).
\end{bemerkung}
\begin{beispiele}
\begin{enumerate}
\item Die \textbf{Einbettung} $\phi: H \hookrightarrow G$ einer Untergruppe $H \leq G$ ist ein Homomorphismus.
\item Die \textbf{Determinantenabbildung}
\begin{equation}
\det: \text{GL} (n, \K) \to (\K \exc \{0\}, \cdot)
\end{equation}
ist ein Homomorphismus.
\item Für $n \geq 1$ und einen Körper $\K$ ist die Permutationsabbildung
\begin{align}
P: \mathfrak{S}_n &\to \text{GL}(n, \K) \\
\sigma &\mapsto P_\sigma,
\end{align}
mit der \textbf{Permutation}
\begin{equation}
(P_\sigma)_{ij} := \begin{cases} 1 \, \text{falls} \, i = \sigma(j)\\0 \, \text{sonst} \end{cases}
\end{equation}
ein Homomorphismus. \textit{Der Beweis sei dem Leser überlassen.}
Für $\sigma = (123) \in \mathfrak{S}_3$ gilt z.B. 
\begin{equation}
P_\sigma = \mat{0,0,1}{1,0,0}{0,1,0}.
\end{equation}
\item Sei $G$ eine Gruppe und $g \in G$. Dann ist 
\begin{align}
\gamma_g: G &\to G \\
h &\mapsto ghg^{-1}
\end{align}
ein Homomorphismus, genannt \textbf{Konjugation mit} $g$.
\item Sei $G$ eine Gruppe und $g \in G$. Dann ist 
\begin{align}
\Z &\to G \\
i &\mapsto g^i
\end{align}
ein Homomorphismus von $(\Z, +)$ nach $(G, \circ)$.
\end{enumerate}
\end{beispiele}
\begin{definition}{Isomorphismus}{isomorphismus}
Sei $\phi$ ein Gruppenhomomorphismus, der zusätzlich bijektiv ist. Dann heißt $\phi$ \textbf{Isomorphismus}. Zwei Gruppen $G$ und $G'$ heißen \textbf{isomorph}, in Zeichen $G \cong G'$, falls es einen Isomorphismus zwischen ihnen gibt.
\end{definition}
\begin{bemerkung}
Anschaulich bedeutet das, dass zwei isomorphe Gruppen identisch bis auf Umbenennung ihrer Elemente sind.
\end{bemerkung}
\begin{beispiele}
\begin{enumerate}
\item Die Permutationsabbildung $P$ induziert einen Isomorphismus
\begin{align}
P: \mathfrak{S}_n &\to P(n, \K)\\
\sigma &\mapsto P_\sigma
\end{align}
zwischen der symmetrischen Gruppe und der Untergruppe der Permuationsmatrizen. Letztere sind Matrizen, die in jeder Zeile und Spalte \textit{genau eine} $1$ und sonst $0$ haben. \textit{Der Beweis sei dem Leser überlassen.}
\item Die \textbf{Exponentialfunktion}
\begin{equation}
\exp: (\R, +) \to (\R_{> 0}, \cdot)
\end{equation}
und ihre Umkehrfunktion, gegeben durch den \textbf{Logarithmus}
\begin{equation}
\ln: (\R_{>0}, \cdot) \to (\R, +),
\end{equation}
bilden einen Isomorphismus, also gilt $(\R, +) \cong (\R_{>0}, \cdot)$.
\end{enumerate}
\end{beispiele}
\begin{definition}{Bild und Kern}{bildkern}
Sei $\phi: G \to G'$ ein Gruppenhomomorphismus. Dann heißt die Teilmenge
\begin{equation}
\im (\phi) := \{g' \in G' | \exists g \in G: \phi(g) = g'\} \leq G',
\end{equation}
das \textbf{Bild von} $\phi$ und die Teilmenge
\begin{equation}
\ker (\phi) := \{g \in G|\phi(g) = 1_{G'} \} \leq G,
\end{equation}
der \textbf{Kern von} $\phi$.
\end{definition}
\begin{satz}{Bild und Kern sind Untergruppen}{bildkernuntergruppe}
Sei $\phi: G \to G'$ ein Gruppenhomomorphismus. Dann sind $\im (\phi) \leq G'$ und $\ker (\phi) \leq G$ Untergruppen der jeweiligen Gruppen $G$ und $G'$.
\end{satz}
\begin{beweis}
Nachrechnen mittels (H1), (H2) und (H3), exemplarisch für den Kern gezeigt:
\begin{enumerate}
\item (U1) ist erfüllt, da $1_G \in \ker (\phi)$ wegen (H2) gilt.
\item (U2) kann nachgerechnet werden. Seien dafür $g,h \in \ker (\phi)$:
\begin{equation}
\phi(gh^{-1}) =^{\text{(H1)}} \phi(g) \cdot \phi(h^{-1}) =^{\text{(H3)}} \phi(g) \cdot \phi(h)^{-1} = 1_{G'},
\end{equation}
also $gh^{-1} \in \ker (\phi)$.
\end{enumerate}
\end{beweis}
\begin{satz}{}{}
Für einen Homomorphismus $\phi: G \to G'$ sind folgende Aussagen äquivalent:
\begin{enumerate}[(i)]
\item $\phi$ ist injektiv.
\item $\ker (\phi) = \{1\}$
\end{enumerate}
\end{satz}
\begin{beweis}
(i) $\implies$ (ii) ist offensichtlich. Wir zeigen noch (ii) $\implies$ (i):
Sei also $\ker (\phi) = \{1\}$ und $g,h \in G$ mit $\phi(g) = \phi(h)$. Dann gilt $\phi(gh^{-1})=\phi(g)\phi(h)^{-1} =1$, also ist $gh^{-1} \in \ker (\phi) = \{1\}$ und damit $g = h$.
\end{beweis}
\begin{definition}{Links- und Rechtsnebenklassen}{linksrechtsnebenklasse}
Sei $G$ eine Gruppe und $H \leq G$ eine Untergruppen. Dann ist die \textbf{Linksnebenklasse von} $H$ \textbf{bezüglich} $g \in G$ als 
\begin{equation}
gH := \{gh \, | \, h \in H\}
\end{equation}
und die \textbf{Rechtsnebenklasse von} $H$ \textbf{bezüglich} $g \in G$ als 
\begin{equation}
Hg := \{hg \, | \, h \in H \}
\end{equation}
definiert.
\end{definition}
\begin{satz}{Nebenklassen sind Äquivalenzklassen}{nebenklassenäquivalenzrel}
Sei $G$ eine Gruppe und $H \leq G$ eine Untergruppe. Dann gilt:
\begin{enumerate}
\item Die Linksnebenklassen sind die Äquivalenzklassen bezüglich der Äquivalenzrelation
\begin{equation}
a \sim_L b :\iff b^{-1}a \in H
\end{equation}
auf $G$.
\item Die Rechtsnebenklassen sind die Äquivalenzklassen bezüglich der analogen Äquivalenzrelation
\begin{equation}
a \sim_R b : \iff ab^{-1} \in H.
\end{equation}
\end{enumerate}
\end{satz}
\begin{übung}
Beweis des Satzes.
\end{übung}
\begin{lösung}
Zunächst ist zu zeigen, dass tatsächlich eine Äquivalenzrelation definiert wird.\\
\begin{enumerate}[(a)]
\item Reflexivität: Sei $a \in G$. Dann gilt $a^{-1}a = 1 \in H$, also ist $a \sim_L a$.
\item Symmetrie: Seien $a,b \in G$ mit $a \sim_L b$. Dann gilt $a^{-1} b = h$ für ein $h \in H$. Daraus folgt:
\begin{equation}
a = bh \iff ah^{-1} =b \iff a^{-1}b = h^{-1} \in H,
\end{equation}
also ist auch $b \sim_L a$, da $H$ abgeschlossen unter Inversenbildung ist.
\item Transitivität: Seien $a,b,c \in G$ mit $a \sim_L b$ und $b \sim_L c$. Dann gilt $b^{-1}a = h \in H$ und $c^{-1} b = h' \in H$. Also folgt $H \ni h'h = c^{-1}bb^{-1}a = c^{-1} a$ und damit die Behauptung.
\end{enumerate}
Ist nun $g \in G$ und $h \in H$, so besteht die Äquivalenzklasse von $g$ unter $\sim_L$ aus allen Elementen der Form $ah$ mit $a \in G$, $h \in H$. Die Vereinigung aller Äquivalenzklassen muss also per Konstruktion ganz $gH$ sein. Der Beweis für $\sim_R$ ist dual dazu. \qed
\end{lösung}

Damit bezeichnen wir die Menge der Linksnebenklassen von $H$ mit $\quotient{G}{H} = \quotient{G}{\sim_L}$ und die der Rechtsnebenklassen mit $\invquotient{G}{H} = \quotient{G}{\sim_R}$.
\begin{definition}{Index}{index}
Die Kardinalität
\begin{equation}
(G : H) := \left| \quotient{G}{H} \right| \in \N \cup \{\infty\}
\end{equation}
heißt \textbf{Index von} $H$ \textbf{in} $G$.
\end{definition}
Man beachte, dass $\left| \quotient{H}{G} \right| = \left| \invquotient{H}{G} \right|$ gilt, da die Abbildung \textcolor{red}{Tafel nicht hochgeschoben...}.
\begin{theorem}{Satz von Lagrange}{satzvonlagrange}
Sei $G$ eine Gruppe und $H \leq G$ eine Untergruppe. Dann gilt 
\begin{equation}
|G| = (G : H) \cdot |H|.
\end{equation}
Ist $|G| < \infty$, so gilt insbesonders
\begin{equation}
(G : H) = \frac{|G|}{|H|} = \left| \quotient{G}{H} \right|.
\end{equation}
\end{theorem}
\begin{beweis}
Dies ist ein direktes Korollar von Satz \ref{nebenklassenäquivalenzrel}: Als Äquivalenzklassen bzgl. einer Äquivalenzrelation bilden die Linksklassen eine Partition von $G$, also 
\begin{equation}
G = \bigsqcup_{gH \in \quotient{G}{H}} gH.
\end{equation}
Es gilt zudem für alle $g \in G$, dass $|gH| = |H|$, da Linksmultiplikation mit $g$, definiert durch 
\begin{align}
G &\to G\\
x &\mapsto gx,
\end{align}
bijektiv ist, also eine Bijektion $H \to gH$ induziert. Insbesondere gilt für jedes $g \in G$, dass $\ord (g) \mid |G|$.
\end{beweis}
\begin{definition}{Normalteiler}{normalteiler}
Eine Untergruppe $N \leq G$ heißt \textbf{normal} oder \textbf{Normalteiler}, falls für alle $g \in G$ 
\begin{equation}
gN = Ng
\end{equation}
gilt. Wir schreiben dafür $N \trianglelefteq G$.
\end{definition}
\begin{bemerkung}
Eine Untergruppe $N \leq G$ ist normal genau dann, wenn für alle $g \in G$ und $n \in N$ gilt:
\begin{equation}
gng^{-1} \in N,
\end{equation}
also $N$ abgeschlossen unter Konjugation mit beliebigen Elementen aus $G$ ist.
\end{bemerkung}
\begin{satz}{}{}
Sei $\phi: G \to G'$ ein Gruppenhomomorphismus. Dann gilt $\ker (\phi) \trianglelefteq G$.
\end{satz}
\begin{beweis}
Sei $g \in G$ und $x \in \ker (\phi)$, also $\phi (x) = 1$. Dann gilt auch 
\begin{equation}
\phi (gxg^{-1}) = \phi(g)\underbrace{\phi(x)}_{=1}\phi(g^{-1}) = \phi(g)\phi(g)^{-1} = 1.
\end{equation}
\end{beweis}
\begin{beispiele}
Wir betrachteten einige Beispiele für Normalteiler:
\begin{enumerate}
\item Sei $n \geq 1$ und $\K$ ein Körper. Für 
\begin{equation}
\det: \text{GL}(n,\K) \to \K^\ast
\end{equation}
gilt
\begin{equation}
\ker (\det) = \text{SL}(n, \K) \trianglelefteq \text{GL}(n,\K).
\end{equation}
\item Betrachte für $n \geq 1$ die Komposition
\begin{center}
\begin{tikzcd}
    \mathfrak{S}_n \arrow[r,"P"] & P(n, \Q) \arrow[r, "\det"] & \{+1, -1\}.
\end{tikzcd}
\end{center}
Also ist 
\begin{equation}
A_n := \ker (\sgn) \trianglelefteq \mathfrak{S}_n
\end{equation}
normal. $A_n$ heißt \textbf{alternierende Gruppe}.
\end{enumerate}
\end{beispiele}
\begin{satz}{Gruppenstuktur auf Nebenklassen}{nebenklassengruppen}
Sei $G$ eine Gruppe und $N \trianglelefteq G$. Dann gilt:
\begin{enumerate}
\item Auf der Menge $\quotient{G}{N}$ von Nebenklassen von $N$ existiert eine Gruppenstruktur mit Verknüpfung
\begin{align}
\quotient{G}{N} \times \quotient{G}{N} &\to \quotient{G}{N} \\
(aN, bN) &\mapsto abN.
\end{align}
\item Die Quotientenabbildung
\begin{align}
\pi: G &\to \quotient{G}{N}\\
a &\mapsto aN
\end{align}
ist ein Gruppenhomomorphismus mit $\ker (\pi) = N$.
\end{enumerate}
\end{satz}
\begin{beweis}
\begin{enumerate}
\item Zunächst muss die Wohldefiniertheit der Verknüpfung bewiesen werden. Seien $\tila \in aN$ und $\tilb \in bN$ Vertreter der Nebenklassen $aN$ und $bN$ $(\iff \tila N = aN)$. Dann existieren $m,n \in N$ mit $\tila = am$ und $\tilb = bn$. Nun gilt
\begin{equation}
\tila \cdot \tilb = am \circ bn = ab \circ \underbrace{b^{-1} mb}_{N \trianglelefteq G \implies \in N} \circ n \in N,
\end{equation}
also ist der Ausdruck wohldefiniert.
\begin{enumerate}[({G}1)]
\item Seien $aN, bN, cN \in \quotient{G}{N}$. Dann gilt
\begin{equation}
(aN \cdot bN) \cdot cN =^{\text{(G1) für G}} (ab)cN = a(bc)N = aN(bN \cdot cN).
\end{equation}
\item Neutrales Element: $1 \cdot N = N$
\item Inverses Element: $(aN)^{-1} = a^{-1} N$
\end{enumerate}
\item Es gilt 
\begin{equation}
\pi (ab)  = (ab)N = (aN)(bN)=\pi(a)\pi(b)
\end{equation}
nach Definition von $\pi$, also ist $\pi$ ein Homomorphismus. Darüber hinaus gilt
\begin{equation}
a \in \ker (\pi) \iff \pi(a) = 1_{\quotient{G}{H}} = N \iff aN = N \iff a \in N,
\end{equation}
also gilt $\ker (\pi) = N$.
\end{enumerate}
\end{beweis}
\begin{satz}{Homomorphiesatz (erster Isomorphiesatz)}{homomorphiesatz}
Sei $\phi: G \to G'$ ein Gruppenhomomorphismus. Dann induziert $\phi$ einen Isomorphismus
\begin{align}
\overline{\phi} : \quotient{G}{\ker (\phi)} &\to \im (\phi)\\
g \ker (\phi) &\mapsto \phi(g).
\end{align}
\end{satz}
\begin{beweis}
Zunächst ist Wohldefiniertheit zu zeigen. Für $\tilg \in gN$, also $\tilg = gn$ für $n \in \ker (\phi)$, gilt:
\begin{equation}
\phi( \tilg) = \phi(gn) = \phi(g)\underbrace{\phi(n)}_{=1} = \phi(g),
\end{equation}
also ist die Abbildung wohldefiniert.\\
Die Surjektivität von $\overline{\phi}$ ist trivial. Wir wissen, dass $\overline{\phi}$ genau dann injektiv ist, wenn $\ker (\overline{\phi}) = \{1_{\quotient{G}{\ker (\phi) }}\} = \ker (\phi)$. Wir rechnen nach:
\begin{equation}
g \ker (\phi) \in \ker (\overline{\phi}) \iff \phi(g) = 1_{G'} \iff g \in \ker (\phi) \iff g \ker (\phi) = \ker (\phi)
\end{equation}
\end{beweis}
\begin{beispiel}
Wir können die Vorzeichenfunktion 
\begin{equation}
\sgn: \mathfrak{S}_n \to \{\pm 1\}
\end{equation}
betrachten, dann ist $\ker{\sgn} = A_n \trianglelefteq \Sf_n$, also erhalten wir einen Isomorphismus 
\begin{equation}
\quotient{\Sf_n}{A_n} \to \{ \pm 1\}.
\end{equation}
Insbesondere gilt $\Sf_n : A_n = 2$.
\end{beispiel}
\begin{korollar}{Korollar aus Satz \ref{homomorphiesatz}}{korollarhomosatz}
Sei $\phi: G \to G'$ ein Gruppenhomomorphismus. Dann lässt sich $\phi$ schreiben als
\begin{equation}
\phi = \iota \circ \overline{\phi} \circ \pi,
\end{equation}
wobei:
\begin{enumerate}
\item $\pi:G \to \quotient{G}{\ker (\phi)}$ der surjektive Quotientenkern ist.
\item $\overline{\phi}: \quotient{G}{\ker (\phi)} \to \im (\phi)$ der Isomorphismus aus \ref{homomorphiesatz} ist.
\item $\iota: \im (\phi) \hookrightarrow G'$ die injektive Einbettung von $\im (\phi) \leq G'$ ist.
\end{enumerate}
Das ist äquivalent dazu, dass folgendes Diagramm kommutiert:
\begin{center}
\begin{tikzcd}
    G \arrow[r,"\phi"] \arrow[d, twoheadrightarrow, "\pi"] & G' \\
    \quotient{G}{\ker (\phi)} \arrow[r, "\cong", "\overline{\phi}"'] & \im (\phi) \arrow[u, hook, "\iota"]
\end{tikzcd}
\end{center}
Ausgedrückt in Elementen:
\begin{center}
\begin{tikzcd}
    g \arrow[r, mapsto] \arrow[d, mapsto] & \phi(g) \\
    g \ker (\phi) \arrow[r, mapsto] & \phi (g) \arrow[u, mapsto]
\end{tikzcd}
\end{center}
\end{korollar}
\begin{beispiele}
\begin{enumerate}
\item Für $n \geq 1$ und einen Körper $\K$ induziert der Homomorphismus
\begin{equation}
\det: \text{GL}(n,\K) \to \K^\ast
\end{equation}
einen Isomorphismus
\begin{equation}
\quotient{\text{GL}(n,\K)}{\text{SL}(n, \K)} \to \K^\ast.
\end{equation}
\item Ein weiterer induzierter Isomorphismus ist
\begin{equation}
\overline{\sgn}: \quotient{\Sf_n}{A_n} \to \{\pm 1\}.
\end{equation}
\item Sei $G$ eine Gruppe mit $g \in G$. Betrachte den Homomorphismus
\begin{equation}
\phi: (\Z, +) \to G, i \mapsto g^i.
\end{equation}
\begin{enumerate}[(a)]
\item Falls $\ord (g) = \infty$, gilt $\ker (\phi) = \{0\}$ und $\phi$ induziert einen Isomorphismus
\begin{center}
\begin{tikzcd}
    \Z \arrow[r,"\pi", "\cong"'] \arrow[rr, "\phi", bend right] & \quotient{\Z}{\{0\}} \arrow[r, "\overline{\phi}", "\cong"'] & \langle g \rangle 
\end{tikzcd}
\end{center}
\item Falls $\ord (g) =N < \infty$, dann gilt
\begin{equation}
\ker (\phi) = N \cdot \Z
\end{equation}
und $\phi$ induziert einen Isomorphismus
\begin{center}
\begin{tikzcd}
\overline{\phi}: \quotient{\Z}{N\Z} \arrow[r, "\cong"]& \langle g \rangle.
\end{tikzcd}
\end{center}
\end{enumerate}
\end{enumerate}
\end{beispiele}
\subsection{Gruppenwirkung}
\label{subsec:wirkung}
\begin{definition}{Gruppenoperation}{operation}
Eine \textbf{Operation} oder \textbf{Wirkung} einer Gruppe $G$ auf einer Menge $M$ ist eine Abbildung
\begin{align}
G \times M &\to M \\
(g, x) &\mapsto g . x,
\end{align}
sodass gilt:
\begin{enumerate}[({O}1)]
\item Für alle $g,h \in G$ und $x \in M$ gilt: $g.(h.x) = (g \cdot h).x$.
\item Für alle $x \in M$ gilt: $1.x=x$.
\end{enumerate}
Dann sagen wir, dass $G$ auf $M$ \textbf{operiert} und schreiben $G \acts M$.
\end{definition}
\begin{beispiele}
\begin{enumerate}
\item Jede Gruppe $G$ operiert auf sich selbst via
\begin{enumerate}[(a)]
\item \textbf{Linkstranslation}: $G \times G \to G$, $(g,h) \mapsto gh$ und
\item \textbf{Rechtstranslation}: $G \times G \to G$, $(g,h) \mapsto hg^{-1}$, aber auch durch
\item \textbf{Konjugation}: $G \times G \to G$, $(g,h) \mapsto ghg^{-1}$.
\end{enumerate}
\item Für jede Menge $M$ operiert die symmetrische Gruppe $\Sf_M$ auf $M$ via
\begin{equation}
\begin{split}
\Sf_M \times M &\to M \\
(\sigma, x) &\mapsto \sigma(x).
\end{split}
\end{equation}
\item Für $n\geq 1$ und einen Körper $\K$ operiert die Gruppe $\text{GL}(n,\K)$ auf $\K^n$ via 
\begin{equation}
\begin{split}
\text{GL}(n,\K) \times \K^n &\to \K^n\\
(A,v) &\mapsto Av.
\end{split}
\end{equation}
\end{enumerate}
\end{beispiele}
\begin{definition}{Äquivarianz}{aequivarianz}
Für Operationen $G \acts M$ und $G \acts N$ heißt eine Abbildung (von Mengen) $f: M \to N$ $G$-\textbf{äquivariant}, falls für alle $g \in G$ und $x \in M$ gilt: 
\begin{equation}
f(g.x) = g.f(x).
\end{equation}
\end{definition}
\begin{definition}{Bahnen}{bahnen}
Sei $G \acts M$ eine Operation von $G$ auf $M$. Die Relation
\begin{equation}
x \sim_G g : \iff \exists g \in G: g.x = g
\end{equation}
definiert eine Äquivalenzrelation auf $M$. Die Äquivalenzklassen sind die Mengen der Form
\begin{equation}
G.x := \{g.x | g \in G\}
\end{equation}
für $x \in M$, die \textbf{Bahnen} von $x$ unter $G \acts M$ genannt werden. Die Quotientenmenge 
\begin{equation}
\invquotient{M}{G} := \quotient{M}{\sim_G}
\end{equation}
heißt \textbf{Bahnenraum von} $G \acts M$.
\end{definition}
\begin{beweis}
\textit{Das Nachweisen der Relationseigenschaften der Äquivalenzrelation ist dem Leser überlassen.}
\end{beweis}
\begin{beispiel}
Betrachte die Rotationsgruppe
\begin{equation}
G = \text{SO}(2, \R) := \text{SL}(2,\R) \cap O(2, \R) = \left\{ \left. \mat{\cos \phi, - \sin \phi}{\sin \phi, \cos \phi} \, \right| \, \phi \in \quotient{\R}{2\pi \Z}   \right\} \leq \text{GL}(2,\R).
\end{equation}
Wir erhalten Operationen 
\begin{equation}
\begin{split}
\text{SO}(2,\R) \times \R^2 &\to \R^2\\
(A,v) &\mapsto Av,
\end{split}
\end{equation}
deren Bahnen konzentrische Kreise im $\R^2$ sind. Dadurch wird eine Partition von $\R^2$ erreicht.
\end{beispiel}
\begin{definition}{Stabilisator, Fixpunkte und Transitivität}{stabilisator}
Sei $G \acts M$ eine Operation. 
\begin{enumerate}[(i)]
\item Für $x \in M$ heißt die Untergruppe
\begin{equation}
G_x := \{g \in G | g.x =x\} \leq G
\end{equation}
der \textbf{Stabilisator von} $x$.
\item Ein Punkt $x \in M$ heißt \textbf{Fixpunkt von} $G \acts M$, falls $G_x = G$. Die Menge aller Fixpunkte wird mit 
\begin{equation}
M^G \sub M
\end{equation}
bezeichnet.
\item Die Operation $G \acts M$ heißt \textbf{transitiv}, falls für jedes $x \in M$ gilt, dass $G.x = M$ ist, also genau eine Bahn existiert.
\end{enumerate}
\end{definition}
\begin{beispiel}
Bleiben wir bei vorigem Beispiel, so hat ein Vektor $v \neq (0,0)$ nur die Identität $\id$ als Stabilisator. Der Nullvektor wird hingegen von ganz $\text{SO}(2,\R)$ stabilisiert. Es scheint einen Zusammenhang zwischen der Größe des Stabilisators und der Bahn zu geben.
\end{beispiel}
\begin{satz}{Bahnformel}{bahnformel}
Sei $G\acts M$ eine Operation auf $M$ und $x \in M$. Dann definiert 
\begin{equation}
\begin{split}
\quotient{G}{G_x} &\to G.x\\
gG_x &\mapsto  g.x
\end{split}
\end{equation}
eine bijektive, $G$-äquivariante Abbildung, wobei $G \acts G.x$ durch Einschränkung von $G\acts M$ gegeben ist. Insbesondere gilt die \textbf{Bahnformel}
\begin{equation}
|G.x| = (G : G_x).
\end{equation}
Eine Wirkung $G \acts \quotient{G}{G_x} = \{gG_x|g \in G\}$ erhält man durch $g' . gG_x:= g'g.G_x$.
\end{satz}
\begin{beweis}
Die Abbildung ist wohldefiniert: Sei $g \in G$ und $h \in G_x$, dann gilt
\begin{equation}
(gh).x = g.(h.x) = g.x.
\end{equation}
Weiterhin ist die Abbilung injektiv, denn falls $g_1.x = g_2.x$, so ist $(g_1^{-1}).g_2.x = x$, also ist $(g_1^{-1}).g_2 \in G_x$, also $g_1G_x = g_2G_x$. Surjektivität ist per Konstruktion durch Einschränkung auf die Bahn gegeben.
\end{beweis}

\section{Ringe}
\label{sec:ringe}
\subsection{Ringe, Ideale und Homomorphismen}
\label{subsec:ringeidealehomo}
\begin{definition}{Ring}{ring}
Ein \textbf{Ring} ist ein Tripel $(R, +, \cdot)$, bestehend aus einer Menge $R$ und Abbildungen
\begin{equation}
+: R \times R \to R
\end{equation}
und
\begin{equation}
\cdot: R \times R \to R,
\end{equation}
sodass gilt:
\begin{enumerate}[({R}1)]
\item Das Paar $(R,+)$ ist eine abelsche Gruppe.
\item Für $a,b \in R$ gilt:
\begin{equation}
\begin{split}
a \cdot (b \cdot c) = (a \cdot b) \cdot c \\
(a+b) \cdot c = a\cdot c + b \cdot c \\
a \cdot (b+c) = a \cdot b + a \cdot c
\end{split}
\end{equation}
\item Es existiert ein Einselement $1 \in R$, sodass für alle $a \in R$ gilt:
\begin{equation}
1 \cdot a = a \cdot 1 = a.
\end{equation}
\end{enumerate}
Gilt zusätzlich 
\begin{equation}
a \cdot b = b \cdot a,
\end{equation}
dann heißt der Ring \textbf{kommutativ}.
\end{definition}
\begin{bemerkung}
Manche Autoren definieren Ringe, ohne ein Einselement zu fordern. Dann werden Ringe mit Einselement als \textbf{unitale Ringe} bezeichnet.
\end{bemerkung}
\begin{beispiele}
\begin{enumerate}
\item Der \textbf{Nullring} $\{0\}$ ist ein Ring, da wir insbesondere nicht fordern, dass $0 \neq 1$ gelten muss. Jedoch ist der Nullring (bis auf Umbenennung der Elemente) der einzige Ring mit dieser Eigenschaft.
\item $(\Z, +, \cdot)$ bildet einen kommutativen Ring.
\item Jeder Körper bildet einen kommutativen Ring.
\item Für $\alpha \in \C$ ist die Menge 
\begin{equation}
\Z [\alpha] := \left\{ \sum_{k=0}^w \lambda_k \alpha^k \, | \, k \in \N, \lambda_k \in \Z \right\} \sub \C
\end{equation}
abgeschlossen unter Addition und Multiplikation. Die Ringaxiome werden von $(\C, +, \cdot)$ vererbt, also bildet $\Z[\alpha]$ einen kommutativen Ring, den \textbf{Polynomring über} $\Z$. Besonders wichtig für die Zahlentheorie is der Fall, wenn $\alpha$ eine \textit{ganze algebraische Zahl} ist, also $\alpha$ als Nullstelle eines monischen Polynoms
\begin{equation}
x^n + a_{n-1}x^{n-1} + \cdots + a_1x + a_0
\end{equation}
mit Koeffizienten $a_i \in \Z$ auftritt. Zum Beispiel ist die imaginäre Zahl $i \in \C$ als Nullstelle des Polynoms $x^2+1$ eine ganze algebraische Zahl. Der Ring
\begin{equation}
\Z[i] = \{ a+bi | a,b \in \Z \}
\end{equation}
heißt \textbf{Gaußscher Zahlenring}.
\item Für einen Ring $R$ bildet die Menge $R[x]$ der Polynome mit Koeffizienten in $R$ einen Ring, genannt \textbf{Polynomring über} $R$.
\item Für jeden Körper $\K$ bildet die Menge $M(n,\K)$ der $\K$-wertigen $n \times n$-Matrizen einen Ring mit elementweiser Addition und Matrixmultiplikation.
\item Für Ringe $R$ und $S$ ist das Produkt $R \times S$ wieder ein Ring mit komponentenweisen Verknüpfungen. 
\end{enumerate}
\end{beispiele}
\begin{definition}{Schiefkörper}{schiefkoerper}
Sei $(R,+,\cdot)$ ein Ring. Existiert zusätzlich für alle $a \in R$ ein $a^{-1} \in R$ mit 
\begin{equation}
a \cdot a^{-1} = a^{-1} a = 1,
\end{equation}
so heißt $(R,+,\cdot)$ \textbf{Schiefkörper}.
\end{definition}
\begin{beispiel}
Die \textbf{Quaternionen} $\H$ bilden einen nicht-kommutativen Ring. Da jedes $q \in \H \exc \{0\}$ allerdings ein multiplikatives Inverses besitzt, ist $\H$ ein Schiefkörper.
\end{beispiel}
\begin{bemerkung}
Historisch wurde der Begriff des Rings und zugehörige Definitionen wie Ideale und Moduln in der algebraischen Zahlentheorie des 19. Jahrhunderts eingeführt und entwickelt (\textit{Kummer}, \textit{Noether}, \textit{Dedekind}, \textit{Hilbert}, \dots).\\
Ziel der Einführung der Ringstruktur ist die Verallgemeinerung des Primzahlbegriffs und der Primfaktorzerlegung für ganze algebraische Zahlen. Zum Beispiel zerfällt die Primzahl $2$ in ein Produkt
\begin{equation}
2 = (1+i)(1-i),
\end{equation}
welches in $\Z[i]$ die neue Primfaktorzerlegung von $2$ wird. Die Tatsache, dass $\Z[i]$ noch immer eindeutige Primfaktorzerlegung besitzt, hat direkte zahlentheoretische Konsequenzen. Ein Beispiel dafür ist der sehr elegante Beweis des folgenden Satzes.
\end{bemerkung}
\begin{satz}{Quadratsumme}{quadratsumme}
Eine ganze Zahl $n \in \Z$ ist genau dann eine Summe $a^2+b^2$ von Quadraten mit $a,b \in \Z$, falls gilt: Jeder Primfaktor $p \mid n$ mit $ p \equiv_4 3$ kommt mit gerader Vielfachheit vor. 
\end{satz}
Umgekehrt gibt es algebraische Zahlenringe ohne eindeutige Primfaktorzerlegung, z.B. $\Z [\sqrt{-5}]$:
\begin{equation}
6 = 2 \cdot 3 = (1-\sqrt{-5})(1+\sqrt{-5}).
\end{equation}
\textbf{Ab jetzt vereinbaren wir, dass alle vorkommenden Ringe kommutativ sind.}
\begin{definition}{Ideal}{ideal}
Sei $R$ ein Ring. Eine nicht-leere Teilmenge $\cI \sub R$ heißt \textbf{Ideal}, falls gilt:
\begin{enumerate}[({I}1)]
\item Für alle $x, y \in \cI$ gilt: $x+y \in \cI$.
\item Für alle $r \in R$ und $x \in \cI$ gilt: $r \cdot x \in \cI$.
\end{enumerate}
\end{definition}
\begin{beispiel}Hauptideale\\
Sei $R$ ein Ring und $x \in R$. Dann bildet 
\begin{equation}
(x) := Rx = \{rx | r \in R \} \sub R
\end{equation}
ein Ideal, das von $x$ erzeugte \textbf{Hauptideal}. Allgemeiner ist für jede Teilmenge $M \sub R$ 
\begin{equation}
(M) := \bigcap_{M \sub \cI \sub R} \cI \sub R
\end{equation}
das von $M$ \textbf{erzeugte Ideal}.
\end{beispiel}
\begin{bemerkung}
Jeder vom Nullring verschiedene Ring $R$ besitzt mindestens zwei unterschiedliche Ideale, nämlich $\{0\}$ und $R$ selbst.
\end{bemerkung}
\begin{satz}{Ring $=$ Körper?}{ringkörper}
Ein Ring $R$ ist genau dann ein Körper, wenn $R$ genau zwei verschiedene Ideale besitzt.
\end{satz}
\begin{beweis}
Sei $\K$ ein Körper und $\{0\} \subset \cI \sub \K$ ein Ideal. Wähle $0 \neq x \in \cI$. Dann existiert ein $r \in R$ mit $rx = 1 \in \cI$. Also gilt für alle $s \in \K$: $s \cdot 1 = s \in \cI$, und somit $\cI = \K$.\\
Angenommen, $R$ hat genau zwei Ideale. Sei $0 \neq x \in R$, dann ist $\{0\} \subset (x) \sub R$ ein Ideal. Daraus folgt aber, dass $(x)=R \ni 1$, also existiert ein $r \in R$ mit $rx = 1$.
\end{beweis}
\begin{definition}{Ringhomomorphismus}{ringhomomorphismen}
Eine Abbildung $\phi: R \to S$ zwischen Ringen $R$ und $S$ heißt \textbf{(Ring-)Homomorphismus}, falls gilt:
\begin{enumerate}[({H}1)]
\item Für alle $r_1, r_2 \in R$ gilt:
\begin{align}
\phi(r_1+r_2) &= \phi(r_1)+\phi(r_2)\\
\phi(r_1 \cdot r_2) &= \phi(r_1) \cdot \phi(r_2).
\end{align}
\item Für $1 \in R$, $1' \in S$ gilt:
\begin{equation}
\phi(1) = 1'.
\end{equation}
\end{enumerate}
\end{definition}
\begin{beispiel}
Sei $R$ ein Ring, $R[x]$ der zugehörige Polynomring über $R$ und $\alpha \in R$. Dann ist die \textbf{Auswertungsabbildung}
\begin{equation}
\begin{split}
\text{ev}_\alpha: R[x] &\to R\\
f(x) &\mapsto f(\alpha)
\end{split}
\end{equation}
ein Homomorphismus, genannt \textbf{Einsetzungshomomorphismus}.
\end{beispiel}
\section{Faktorisierung in Polynomringen}
\label{sec:galois}

\begin{beispiele}
Sei $\K$ ein Körper. Dann ist $\K[x]$ euklidisch, also insbesondere ein Hauptidealring. Damit ist $\K[x]$ faktoriell, $\K[x]^\times = \K \exc \{0\} \cdots$. Was sind die irreduziblen Polynome?
\begin{enumerate}
\item Sei $\K = \C$. Dann gilt der Fundamentalsatz der Algebra, sodass jedes Polynom $f(x) \in \C[x]$ in Linearfaktoren zerfällt:
\begin{equation}
f(x) = c \prod_{i=1}^n (x-\lambda_i).
\end{equation}
Die irreduziblen Polynome sind also $(x-\lambda_i)$ mit $\lambda_i \in \C$.
\item Für $\K = \R$ haben wir:
\begin{itemize}
\item $(x-\lambda)$, $\lambda \in \R$ ist irreduzibel.
\item $x^2+ax+b$ mit $a,b \in \R$ ist irreduzibel, falls für die Diskriminante $a^2-4b<0$ gilt. Ist $\deg(f)\geq 3$, ist $f$ nicht irreduzibel, da in $\C[x]$ gilt:
\begin{equation}
f(x) = r(x-\lambda_1)(x-\lambda_2) \cdots (x-\lambda_n)= r(x-\overline{\lambda}_1) \cdots (x -\overline{\lambda}_n)
\end{equation}
da für $f \in \R[x]$ die Konjugation der Identität entspricht. Also folgt $\lambda_i = \overline{\lambda}_1$. Ist $\lambda_1 = \overline{\lambda}_1$, hat $f$ eine reelle Nullstelle $\lambda = \lambda_1$, also $f=(x-\lambda)g$. Ist hingegen $\lambda_i = \overline{lambda}_i =: \lambda$ mit $i \neq 1$, so ist 
\begin{equation}
f = (x-\lambda)(x-\overline{\lambda})g = \underbrace{(x^2 - (\lambda + \overline{\lambda})x + \lambda \overline{\lambda})}_{\in \R[x]}g.
\end{equation} 
\end{itemize}
\item $\K = \Q$ wollen wir sodann näher untersuchen.
\item $\K = \F_2$. Ist $x^2++x+1$ irreduzibel? Ausprobieren aller Elemente von $\F_2$ zeigt, dass das Polynom irreduzibel ist. Daraus folgt auch, dass das Polynom in $\Z[x]$ irreduzibel ist! Auf dieser Idee wollen wir fortan aufbauen.
\end{enumerate}
\end{beispiele}
Unser Ziel ist nun, den Zusammenhang der Irreduzibilität in $\Z[x]$ mit der in $\Q[x]$ zu verstehen.
\begin{definition}{Primitivität}{primitiv}
Wir nennen ein Polynom 
\begin{equation}
f(x) = a_nx^n+a_{n-1}x^{n-1} + \cdots + a_0 \in \Z[x]
\end{equation}
\textbf{primitiv}, falls:
\begin{enumerate}[(i)]
\item $a_n > 0$ 
\item $\text{ggT} (a_0, a_1, \dots, a_n) = \{\pm 1\}$, also $(a_0, a_1, \dots, a_n) = (1)$.
\end{enumerate}
\end{definition}
\section{Appendix}
\label{sec:appendix}
\subsection{Lösungen ausgewählter Übungsaufgaben}
\label{subsec:solutions}
\begin{lösung} Zu Aufgabe \ref{exc:gruppenchrakterisierung}.\\
Zuerst zeigen wir Kommutativität des Inversen. Sei $g \in G$, dann gilt:
\begin{align}
g \circ g^{-1} &= (e \circ g) \circ g^{-1} = \left( \left( \left( g^{-1}\right)^{-1} \circ g^{-1} \right) \circ g \right) \circ g^{-1} = \left(  \left( g^{-1}\right)^{-1} \circ \left( g^{-1} \circ g \right)\right) \circ g^{-1}\\ 
&= \left( g^{-1}\right)^{-1} \circ \left( e  \circ g^{-1} \right) = \left( g^{-1}\right)^{-1} \circ g^{-1} = e = g^{-1} \circ g,
\end{align}
also stimmen Links- und Rechtsinverses in Gruppen überein.
Die Kommutativität des neutralen Elements folgt damit direkt aus:
\begin{equation}
g \circ e = g \circ (g^{-1} \circ g) = (g \circ g^{-1}) \circ g = (g^{-1} \circ g) \circ g = e \circ g,
\end{equation}
womit auch Links-Einselement und Rechts-Einselement übereinstimmen.
Für die Eindeutigkeit des Inversen seien $g^{-1}, g'^{-1} \in G$ zwei Inverse von $g \in G$. Dann gilt:
\begin{equation}
g^{-1} = g^{-1} \circ e = g^{-1} \circ (g'^{-1} \circ g) = g^{-1} \circ (g \circ g'^{-1}) = (g^{-1} \circ g) \circ g'^{-1} = e \circ g'^{-1} = g'^{-1}.
\end{equation}
Weiterhin seien $e,e' \in G$ zwei Einselemente. Da $e = e \circ e' = e' \circ e = e$ gilt, ist das neutrale Element eindeutig. \qed
\end{lösung}
\begin{lösung} Zu Aufgabe \ref{exc:untergruppencharakterisierung}.\\
$(\Leftarrow)$: Wegen (U1) ist $H \neq \emptyset$. Seien $a,b \in H$. Wegen (U3) ist dann auch $b^{-1} \in H$ und wegen (U2) auch $ab^{-1} \in H$.\\
$(\Rightarrow)$: Aus $H \neq \emptyset$ und $\forall a,b \in H: ab^{-1} \in H$ folgt mit $b=a$ auch $1 \in H$. Mit $a=1$ folgt für alle $b \in H$ auch $b^{-1} \in H$. Also ist mit $a,b \in H$ auch $b^{-1} \in H$. Dann ist aber $a(b^{-1})^{-1} = ab \in H$. \qed
\end{lösung}
\begin{lösung} Zu Aufgabe \ref{exc:inverseistisomorphismus}.
\begin{itemize}
\item Homomorphismus: Seien $g_1', g_2' \in G'$. Dann finden wir Urbilder $\varphi(g_1)=g_1$ und $\varphi(g_2)=g_2$. Für diese gilt mit den Homomorphieeigenschaften von $\varphi$:
\begin{equation}
\psi(g_1'g_2') = \psi(\phi(g_1)\varphi(g_2))=\psi(\varphi(g_1g_2)) = g_1g_2=\psi(\varphi(g_1))\psi(\varphi(g_2))=\psi(g_1')\psi(g_2')
\end{equation}
\item Injektion: Seien $g_1',g_2' \in G'$ mit $g_1' \neq g_2'$. Wir finden eindeutige Urbilder $\varphi(g_1)=g_1'$ und $\varphi(g_2)=g_2'$. Dann gilt $g_1' \neq g_2'$ genau dann, wenn $\varphi(g_1) \neq \varphi(g_2)$ wegen der Injektivität von $\varphi$. Genau dann ist aber 
\begin{equation}
\psi(g_1) = \psi(\varphi(g_1)) = g_1' \neq g_2' = \psi(\varphi(g_2)) = \psi(g_2').
\end{equation}
\item Surjektion: Sei $g \in G$ und $g':= \varphi(g)$. Dann ist $g= \psi(\varphi(g))=\psi(g')$.
\end{itemize}
Also ist $\psi$ ein bijektiver Homomorphismus und damit ein Isomorphismus. Ein simpleres Argument wäre die Tatsache, dass beidseitige Inverse sein äquivalent zu Bijektivität ist. \qed
\end{lösung}
\printindex
\end{document}
