\section{Gruppen und Symmetrie}
\label{gruppentheorie}

\begin{bemerkung}
Wir möchten Gruppentheorie zunächst motivieren: Man betrachte einen Tetraeder. Um dessen Symmetrien zu erfassen, könnten wir z.B. schauen, welche Bewegungen diesen in sich selbst überführen. Es gibt vier Rotationsachsen, die eine Ecke und eine Fläche durchdringen und bei Rotation um $120^\circ$ den Tetraeder in sich selbst überführen. Weiterhin gibt es drei $180^\circ$-Rotationsachsen mittig durch gegenüberliegende Kanten. Auch die Identität lässt den Tetraeder unverändert. Also gibt es $1+4 \cdot 2 + 3 = 12$ Symmetrien. Gruppen bieten eine Möglichkeit, solche Symmetrien und deren Verkettungen zu erfassen und zu untersuchen.
\end{bemerkung}

\subsection{Grundbegriffe}
\label{subsec:grundbegriffe}

\begin{definition}{Gruppe}{gruppe}
Eine \textbf{Gruppe} ist ein Paar $(G, \circ)$, bestehend aus einer Menge\footnote{im ZFC-Axiomensystem} $G$ und einer Abbildung
\begin{align}
\circ: G \times G &\to G\\
(g,h) &\mapsto g \circ h
\end{align} 
mit folgenden Eigenschaften:
\begin{enumerate}
\item Für alle $g_1,g_2,g_3 \in G$ gilt das Assoziativgesetz: $(g_1 \circ g_2) \circ g_3 = g_1 \circ (g_2 \circ g_3)$.
\item Es gibt ein Element $e \in G$, sodass gilt:
\begin{itemize}
\item Für jedes $g \in G$ gilt $e \circ g = g$.
\item Für jedes $g \in G$ existiert ein $g' \in G$ mit $g' \circ g = e$.
\end{itemize}
\end{enumerate}
Die Abbildung $\circ$ heißt \textbf{Verknüpfung}, ein Element $e \in G$ mit den Eigenschaften aus 2. heißt \textbf{neutrales Element}, und ein Element $g' \in G$ zu gegebenem $g \in G$ mit Eigenschaft 2b heißt \textbf{Inverses} von $g$.
\end{definition}

\begin{bemerkung}Übungsaufgabe:\\
Sei $(G, \circ)$ eine Gruppe. Dann gelte:
\begin{enumerate}
\item Das neutrale Element $e \in G$ ist eindeutig bestimmt, außerdem gelte $ \forall g \in G : g \circ e = g$.
\item Zu gegebenem $g \in G$ ist das Inverse $g' \in G$ eindeutig bestimmt und erfüllt zudem $g \circ g' = e$.
\item Für $n \geq 3$ hängt das Produkt von Gruppenelementen $g_1, g_2, \dots, g_n$ nicht von der Klammerung ab.
\end{enumerate}
\end{bemerkung}

\begin{beispiele}
Wir geben einige Beispiele für Gruppen:
\begin{enumerate}
\item Die Gruppe $(\Z, +)$ der ganzen Zahlen $\Z$ mit der Addition $+$.
\item Für einen Körper $\K$ existiert die additive Gruppe $(\K, +)$ und die multiplikative Gruppe $(\K \exc \{0\}, \cdot)$.
\item Für jede Menge $M$ existiert die \textbf{symmetrische Gruppe} $(\mathfrak{S}_M, \circ)$, wobei $\mathfrak{S}_M$ die Menge der bijektiven Selbstabbildungen von $M$ und $\circ$ die Komposition ist. Für $n \geq 1$ vereinbaren wir $\mathfrak{S}_n := \mathfrak{S}_{\{1,2,\dots,n\}}$. Wir vereinbaren als Konvention die \textbf{Zykelschreibweise}. Z.B. in $\mathfrak{S}_3$ existieren Zykel
\begin{align}
\sigma: \{1,2,3\} &\to \{1,2,3\}\\
1 &\mapsto 2 \\
2 &\mapsto 1 \\
3 &\mapsto 3,
\end{align}
auch darstellbar als
\begin{equation}
\mat{1,2,3}{2,1,3}
\end{equation}
oder $(12)$.
\item Für $n \geq 1$ und einen Körper $\K$ ist die \textbf{allgemeine lineare Gruppe} $\text{GL}(n,\K), \circ)$ definiert, wobei 
\begin{equation}
\text{GL}(n, \K) := \left\{ A \in \K^{n\times n} \, | \, \det A \neq 0 \right\}
\end{equation}
Die Menge der invertierbaren $n \times n$-Matrizen mit Einträgen in $\K$ ist. Typische Beispiele für Körper sind $\K = \Q, \R, \C, \F_q$ mit $q = p^n$, $p$ prim.\\
ÜA: $| \text{GL}(n, \F_q)| = ?$.
\end{enumerate}
\end{beispiele}
\begin{bemerkung}
Um den alltäglichen Gebrauch von Gruppen zu vereinfachen, machen wir folgende Vereinbarungen:
\begin{enumerate}
\item Wir bezeichnen $(G, \circ)$ üblicherweise einfach mit $G$ und lassen $\circ$ implizit.
\item Für $g,h \in G$ schreiben wir $gh = g \circ h$, für $e \in G$ schreiben wir $1$ und für $g'$ schlicht $g^{-1}$.
\item Gilt $g \circ h = h \circ g$ für alle $g,h \in G$, so heißt $G$ \textbf{abelsch}. In diesem Fall wird die Verknüpfung oft mit $+$, das neutrale Element mit $0$ und das inverse Element mit $-g$ bezeichnet.
\item Gemäß obiger ÜA zur Klammerung schreiben wir einfach $g_1 g_2 \cdots g_n \in G$ ohne Klammerung.
\item Für eine Gruppe $G$ bezeichnen wir die Kardinalität \begin{equation}
|G| \in \N \cup \{+\infty \}
\end{equation}
als \textbf{Ordnung} von $G$.
\end{enumerate}
\end{bemerkung}

\subsection{Untergruppen}
\label{subsec:untergruppen}
\begin{definition}{Untergruppe}{untergruppe}
Sei $(G, \circ)$ eine Gruppe. Eine Teilmenge $H \sub G$ heißt \textbf{Untergruppe}, falls gilt:
\begin{enumerate}
\item $H \neq \emptyset$
\item Abgeschlossenheit: Für alle $a, b \in H$ gilt $ab^{-1} \in H$.
\end{enumerate}
Wir verwenden dann die Notation $H \leq G$, um Untergruppen zu kennzeichnen.
\end{definition}

\begin{bemerkung} Übungsaufgabe:
Sei $G$ eine Gruppe und $H \leq G$ eine Untergruppe. Dann gilt:
\begin{enumerate}
\item Aus Eigenschaft 1:Da $H \neq \emptyset$, existiert ein $a \in H$.
\item Aus Eigenschaft 2: $a \cdot a^{-1} = e \in H$.
\item Aus Eigenschaft 2: Für jedes $a \in H$ gilt $a^{-1} = e \cdot a^{-1} \in H$.
\item Aus Eigenschaft 2: Für jedes $a,b \in H$ gilt $ab = a \cdot (b^{-1})^{-1} \in H$.
\end{enumerate}
Also: $H \sub G$ ist eine Untergruppe genau dann, wenn folgende alternativen Eigenschaften gelten:
\begin{enumerate}[1.{$^\ast $}]
\item $e_G \in H$
\item Für alle $a,b \in H$ muss $a \cdot b \in H$ gelten.
\item Für alle $a \in H$ ist $a^{-1} \in H$.
\end{enumerate}
Die andere Richtung der Äquivalenz ist trivial. Daraus folgt auch, dass $(H, \circ|_{H \times H})$ mit von $G$ eingeschränkter Verknüpfung $\circ|_{H \times H}$ ist eine Gruppe. 
\end{bemerkung}
\begin{beispiele}
Einige Beispiele für Untergruppen sind:
\begin{enumerate}
\item $(G, \circ) = (\R, +)$ hat $(\Z, +)$ als Untergruppe mit $\Z \sub \R$.
\item Sei $n \geq 1$ und $\K$ ein Körper. Die \textbf{spezielle lineare Gruppe}
\begin{equation}
\text{SL} (n, \K) := \left\{ A \in \text{GL}(n, \K) \, | \, \det A = 1\right\} \leq \text{GL}(n, \K)
\end{equation}
ist eine Untergruppe von $\text{GL}(n,\K)$.
\item Für $n \geq$ und einen Körper $\K$ ist die \textbf{orthogonale Gruppe}
\begin{equation}
\text{O}(n, \K) := \left\{ A \in \text{GL}(n, \K) \, | \, A^TA = I_n \right\} \leq \text{GL}(n, \K)
\end{equation}
definiert, die auch eine Untergruppe von $\text{GL}(n, \K)$ ist.
\item Seien $H_1, H_2 \leq G$ Untergruppen. Dann ist $H_1 \cap H_2 \leq G$ auch eine Untergruppe. So kann z.B. die \textbf{spezielle orthogonale Gruppe}
\begin{equation}
\text{SO}(n, \K) := \text{O}(n, \K) \cap \text{SL}(n, \K)
\end{equation}
als Untergruppe von $\text{GL}(n, \K)$ konstruiert werden.
\item Etwas allgemeiner: Für jede Familie $\{H_i\}_{i\in I}$ von Untergruppen $H_i \leq G$ gilt:
\begin{equation}
\bigcap_{i \in I} H_i \leq G
\end{equation}
ist wieder eine Untergruppe.
\end{enumerate}
\end{beispiele}
\begin{definition}{Erzeugte Untergruppe}{}
Sei $G$ eine Gruppe und $M \sub G$ eine beliebige Teilmenge. Dann heißt die \textbf{Untergruppe}
\begin{equation}
\langle M \rangle := \bigcup_{H \leq G, M \sub H} H \leq G
\end{equation}
die \textbf{von} $M$ \textbf{erzeugte Untergruppe} von $G$.
\end{definition}