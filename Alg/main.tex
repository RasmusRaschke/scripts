\documentclass[10pt]{extarticle}

\usepackage[german]{babel}
\usepackage{graphicx}
\usepackage{framed}
\usepackage[normalem]{ulem}
\usepackage{indentfirst}
\usepackage{amsmath,amsthm,amssymb,amsfonts}
\usepackage{mathtools} % Wraparound amsmath. For fancy math typesetting
\usepackage[nointegrals]{wasysym} % nointegrals prevents wasysym from overwriting integral symbols from LaTeX and amsmath
\usepackage{bbm} % For extended bold and blackboard bold characters
\usepackage[italicdiff]{physics} % italicdiff causes derivatives to be rendered with italic d's instead of upright d's
\usepackage[T1]{fontenc}
\usepackage{xparse}
\usepackage{xstring}
\usepackage{float}
\usepackage{stmaryrd}
\usepackage{enumerate}
\usepackage{grffile}
\usepackage{cancel}
\usepackage{hyperref}
\hypersetup{
    colorlinks=true,
    linkcolor=blue,
    filecolor=magenta,      
    urlcolor=cyan,
    pdftitle={Overleaf Example},
    pdfpagemode=FullScreen,
    }

\urlstyle{same}
\numberwithin{equation}{subsection}
\usepackage{pifont} % For unusual symbols
\usepackage{mathdots} % For unusual combinations of dots
\usepackage{wrapfig}
\usepackage{lmodern,mathrsfs}
\usepackage[inline,shortlabels]{enumitem}
\setlist{topsep=2pt,itemsep=2pt,parsep=0pt,partopsep=0pt}
\usepackage[table,dvipsnames]{xcolor}
\usepackage[utf8]{inputenc}
\usepackage{csquotes} % Must be loaded AFTER inputenc
\usepackage[a4paper,top=0.5in,bottom=0.2in,left=0.5in,right=0.5in,footskip=0.3in,includefoot]{geometry}
\usepackage[most]{tcolorbox}
\usepackage{tikz,tikz-3dplot,tikz-cd,tkz-tab,tkz-euclide,pgf,pgfplots}
\usetikzlibrary{babel}
\pgfplotsset{compat=newest}
% \usepackage{comment} % For commenting large blocks of text and math efficiently
% \usepackage{fancyvrb} % For custom verbatim environments
\usepackage{multicol}
\usepackage[bottom,multiple]{footmisc} % Ensures footnotes are at the bottom of the page, and separates footnotes by a comma if they are adjacent
\usepackage[backend=bibtex,style=numeric]{biblatex}
\renewcommand*{\finalnamedelim}{\addcomma\addspace} % Forces authors' names to be separated by comma, instead of "and"
\addbibresource{bibliography}
\usepackage[nameinlink]{cleveref} % nameinlink ensures that the entire element is clickable in the pdf, not just the number

\newcommand{\remind}[1]{\textcolor{red}{\textbf{#1}}} % To remind me of unfinished work to fix later
\newcommand{\hide}[1]{} % To hide large blocks of code without using % symbols
\newcommand{\hodge}{{\star}}
% Same as \href, but the text appears in typewriter font and in a custom color
\newcommand{\Href}[3][red!50!black]{\href{#2}{\textcolor{#1}{\texttt{#3}}}}

\newcommand{\ep}{\varepsilon}
\newcommand{\vp}{\varphi}
\newcommand{\lam}{\lambda}
\newcommand{\Lam}{\Lambda}
\DeclareDocumentCommand\ip{ l m }{\braces#1{\langle}{\rangle}{#2}} % Inner product ⟨x,y⟩ (but only one argument is taken, so \ip{x,y} renders as ⟨x,y⟩)
\DeclareDocumentCommand\floor{ l m }{\braces#1{\lfloor}{\rfloor}{#2}} % Floor function ⌊x⌋
\DeclareDocumentCommand\ceil{ l m }{\braces#1{\lceil}{\rceil}{#2}} % Ceiling function ⌈x⌉

% Shortcuts for blackboard bold letters, e.g. \A outputs \mathbb{A}
\def\do#1{\csdef{#1}{\mathbb{#1}}}
\docsvlist{A,B,C,D,E,F,G,I,J,K,M,N,Q,R,T,U,V,W,X,Y,Z}
% \H is already defined as a 1-argument command, it places a double acute accent (hungarumlaut) on a character, e.g. \H{o} yields ő
% \L is already defined as the uppercase Ł (L with stroke)
% \O is already defined as the uppercase Ø (O with stroke)
% \P is already defined as the pilcrow ¶ (paragraph mark)
% \S is already defined as the section sign §

% Shortcuts for calligraphic letters, e.g. \As outputs \mathcal{A}
\def\do#1{\csdef{#1s}{\mathcal{#1}}}
\docsvlist{A,B,C,D,E,F,G,H,I,J,K,L,M,N,O,P,Q,R,S,T,U,V,W,X,Y,Z}

% Shortcuts for letters with a bar on top, e.g. \Abar outputs \overline{A}
\def\do#1{\csdef{#1bar}{\overline{#1}}}
\docsvlist{a,b,c,d,e,f,g,i,j,k,l,m,n,o,p,q,r,s,t,u,v,w,x,y,z,A,B,C,D,E,F,G,H,I,J,K,L,M,N,O,P,Q,R,S,T,U,V,W,X,Y,Z}
% \hbar is already defined as the symbol ℏ (reduced Planck constant)

% Shortcuts for boldface letters, e.g. \Ab outputs \textbf{A}
\def\do#1{\csdef{#1b}{\textbf{#1}}}
\docsvlist{a,b,c,d,e,f,g,h,i,j,k,l,m,n,o,q,r,t,u,w,x,y,z,A,B,C,D,E,F,G,H,I,J,K,L,M,N,O,P,Q,R,S,T,U,V,W,X,Y,Z}
% \pb is already defined (by the physics package) as a 2-argument command, denoting the anticommutator or Poisson bracket, e.g. \pb{A,B} yields {A,B}
% \sb is already defined in the LaTeX kernel. This is a fundamental LaTeX command, DO NOT overwrite it!
% \vb is already defined (by the physics package) as a 1-argument command, for boldface text, e.g. \vb{A} yields \textbf{A}

% Shortcuts for letters with a tilde on top, e.g. \Atil outputs \widetilde{A}
\def\do#1{\csdef{#1til}{\widetilde{#1}}}
\docsvlist{a,b,c,d,e,f,g,h,i,j,k,l,m,n,o,p,q,r,s,t,u,v,w,x,y,z,A,B,C,D,E,F,G,H,I,J,K,L,M,N,O,P,Q,R,S,T,U,V,W,X,Y,Z}

\def\do#1{\csdef{#1f}{\mathfrak{#1}}}
\docsvlist{A,B,C,D,E,F,G,H,I,J,K,L,M,N,O,P,Q,R,S,T,U,V,W,X,Y,Z}

\newcommand{\quotient}[2]{{\raisebox{.0em}{$#1$}/\raisebox{-.2em}{$#2$}}}
\newcommand{\tm}{^{\mathsf{T}}}     % Transpose
\newcommand{\hm}{^{\mathsf{H}}}     % Conjugate transpose (Hermitian conjugate)
\newcommand{\itm}{^{-\mathsf{T}}}   % Inverse transpose
\newcommand{\ihm}{^{-\mathsf{H}}}   % Inverse conjugate transpose (Inverse Hermitian conjugate)
\newcommand{\ex}{\textbf{e}_x}
\newcommand{\ey}{\textbf{e}_y}
\newcommand{\ez}{\textbf{e}_z}
\newcommand{\Aint}{A^\circ}
\newcommand{\Bint}{B^\circ}
\newcommand{\limk}{\lim_{k\to\infty}}
\newcommand{\limm}{\lim_{m\to\infty}}
\newcommand{\limn}{\lim_{n\to\infty}}
\newcommand{\limx}[1][a]{\lim_{x\to#1}}
\newcommand{\limz}[1][{z_0}]{\lim_{z\to#1}}
\newcommand{\liminfm}{\liminf_{m\to\infty}}
\newcommand{\limsupm}{\limsup_{m\to\infty}}
\newcommand{\liminfn}{\liminf_{n\to\infty}}
\newcommand{\limsupn}{\limsup_{n\to\infty}}
\newcommand{\sumkn}{\sum_{k=1}^n}
\newcommand{\sumk}[1][1]{\sum_{k=#1}^\infty}
\newcommand{\summ}[1][1]{\sum_{m=#1}^\infty}
\newcommand{\sumn}[1][1]{\sum_{n=#1}^\infty}
\newcommand{\emp}{\varnothing}
\newcommand{\exc}{\backslash}
\newcommand{\sub}{\subseteq}
\newcommand{\sups}{\supseteq}
\newcommand{\capp}{\bigcap}
\newcommand{\cupp}{\bigcup}
\newcommand{\kupp}{\bigsqcup}
\newcommand{\cappkn}{\bigcap_{k=1}^n}
\newcommand{\cuppkn}{\bigcup_{k=1}^n}
\newcommand{\kuppkn}{\bigsqcup_{k=1}^n}
\newcommand{\cappk}[1][1]{\bigcap_{k=#1}^\infty}
\newcommand{\cuppk}[1][1]{\bigcup_{k=#1}^\infty}
\newcommand{\cappm}[1][1]{\bigcap_{m=#1}^\infty}
\newcommand{\cuppm}[1][1]{\bigcup_{m=#1}^\infty}
\newcommand{\cappn}[1][1]{\bigcap_{n=#1}^\infty}
\newcommand{\cuppn}[1][1]{\bigcup_{n=#1}^\infty}
\newcommand{\kuppk}[1][1]{\bigsqcup_{k=#1}^\infty}
\newcommand{\kuppm}[1][1]{\bigsqcup_{m=#1}^\infty}
\newcommand{\kuppn}[1][1]{\bigsqcup_{n=#1}^\infty}
\newcommand{\cappa}{\bigcap_{\alpha\in I}}
\newcommand{\cuppa}{\bigcup_{\alpha\in I}}
\newcommand{\kuppa}{\bigsqcup_{\alpha\in I}}
\newcommand{\dx}{\,dx}
\newcommand{\dy}{\,dy}
\newcommand{\dt}{\,dt}
\newcommand{\dmu}{\,d\mu}
\newcommand{\dnu}{\,d\nu}
\DeclareMathOperator{\glb}{\text{glb}}
\DeclareMathOperator{\lub}{\text{lub}}
\newcommand{\xh}{\widehat{x}}
\newcommand{\yh}{\widehat{y}}
\newcommand{\zh}{\widehat{z}}
\newcommand{\<}{\langle}
\newcommand{\diff}{\mathcal{D}}
\newcommand{\sph}{\mathbb{S}}
\renewcommand{\>}{\rangle}
\newcommand{\graph}{\text{graph}}
\newcommand{\id}{\text{id}}
\newcommand{\iprod}{\mathbin{\lrcorner}}
\newcommand{\diffm}{\text{Diff}}
\DeclareMathOperator{\Ric}{Ric}
\DeclareMathOperator{\ric}{ric}
\newcommand{\so}[1]{\text{SO}(#1)}
\newcommand{\ggt}{\text{ggT}}
\newcommand{\der}[1]{\text{Der}_{#1}}
\newcommand{\gl}[2]{\text{GL}(#1, #2)}
\newcommand{\inj}{\text{inj}}
%Define behavior of the command with one parameter
\newcommand{\cinfa}[1]{\text{C}^\infty (#1)}
%Define behavior of the command with two parameter
\newcommand{\cinfb}[2]{\text{C}^\infty (#1, #2)}
\newcommand{\fracpart}[1]{\frac{\partial}{\partial #1_i}}
\NewDocumentCommand\cinf{ m g }{
  \IfNoValueTF{#2}{\cinfa{#1}}{\cinfb{#1}{#2}}
}
\newcommand{\supp}{$\text{supp}\ $}
\newcommand{\sgn}{\text{sgn}}
\newcommand{\diag}{\text{diag}}
% Shortcuts for inverse hyperbolic functions (and other operators with the same structure)
\def\do#1{\csdef{#1}{\trigbraces{\operatorname{#1}}}}
\docsvlist{
    asinh,acosh,atanh,acoth,asech,acsch,
    arsinh,arcosh,artanh,arcoth,arsech,arcsch,
    arcsinh,arccosh,arctanh,arccoth,arcsech,arccsch,
    sen,tg,cth,senh,tgh,ctgh,
    Re,Im,arg,Arg,im,ker
}

% \spn has to be defined separately as the syntax "spn" is different from the output "span"
% \span is already defined in the LaTeX kernel. This is a fundamental LaTeX command, DO NOT overwrite it!
\newcommand{\spn}{\trigbraces{\operatorname{span}}}

\makeatletter
% Redefining the commands \iff (given by LaTeX), \implies and \impliedby (given by amsmath)
% Math mode is automatically enforced, starred version makes the arrows shorter
\renewcommand{\impliedby}{\@ifstar{\ensuremath{\Longleftarrow}}{\ensuremath{\Leftarrow}}} % Corresponding Unicode character: U+21D0 ⇐
\renewcommand{\implies}{\@ifstar{\ensuremath{\Longrightarrow}}{\ensuremath{\Rightarrow}}} % Corresponding Unicode character: U+21D2 ⇒
\renewcommand{\iff}{\@ifstar{\ensuremath{\Longleftrightarrow}}{\ensuremath{\Leftrightarrow}}} % Corresponding Unicode character: U+21D4 ⇔
\makeatother

\newtheoremstyle{mystyle}{}{}{}{}{\sffamily\bfseries}{.}{ }{}
\makeatletter
\newenvironment{beweis}[1][\proofname] {\par\pushQED{\qed}{\normalfont\sffamily\bfseries\topsep6\p@\@plus6\p@\relax #1\@addpunct{.} }}{\popQED\endtrivlist\@endpefalse}
\makeatother
\theoremstyle{mystyle}{\newtheorem*{bemerkung}{Bemerkung}}
\theoremstyle{mystyle}{\newtheorem*{bemerkungen}{Bemerkungen}}
\theoremstyle{mystyle}{\newtheorem*{beispiel}{Beispiel}}
\theoremstyle{mystyle}{\newtheorem*{beispiele}{Beispiele}}
\theoremstyle{definition}{\newtheorem*{übung}{Übung}}

% Warning environment
\newtheoremstyle{warn}{}{}{}{}{\normalfont}{}{ }{}
\theoremstyle{warn}
\newtheorem*{warning}{\warningsign{0.2}\relax}

% Symbol for the warning environment, designed to be easily scalable
\newcommand{\warningsign}[1]{
    \tikz[scale=#1,every node/.style={transform shape}]{
        \draw[-,line width={#1*0.8mm},red,fill=yellow,rounded corners={#1*2.5mm}] (0,0)--(1,{-sqrt(3)})--(-1,{-sqrt(3)})--cycle;
        \node at (0,-1) {\fontsize{48}{60}\selectfont\bfseries!};
}}

% verbbox environment, for showing verbatim text next to code output (for package documentation and user learning purposes)
\NewTCBListing{verbbox}{ !O{} }{boxrule=1pt,sidebyside,skin=bicolor,colback=gray!10,colbacklower=white,valign=center,top=2pt,bottom=2pt,left=2pt,right=2pt,#1} % Last argument allows more tcolorbox options to be added




\makeatletter
% \fsize stores the current font size but is expandable (and can be called later without using \makeatletter and \makeatother)
\def\fsize{\dimexpr\f@size pt\relax}
\makeatother

\makeatletter
% Adapted from https://tex.stackexchange.com/a/19700
\def\my@vector #1,#2\@eolst{
    \ifx\relax#2\relax
        #1
    \else
        #1\my@delim
        \my@vector #2\@eolst
    \fi}
\newcommand\vcstring[2][\\]{% Converts comma-separated string to #1-separated string
    \def\my@delim{#1}
        \my@vector #2,\relax\noexpand\@eolst}
\newcommand\cvc[2][p]{% Converts comma-separated string to column vector, optional argument defines matrix brackets
    \def\my@delim{\\}
        \begin{#1matrix} % Empty argument also possible
            \my@vector #2,\relax\noexpand\@eolst
        \end{#1matrix}}
\newcommand\rvc[2][p]{% Converts comma-separated string to row vector, optional argument defines matrix brackets
    \def\my@delim{&}
        \begin{#1matrix} % Empty argument also possible
            \my@vector #2,\relax\noexpand\@eolst
        \end{#1matrix}}
% Matrix environment with variable number of arguments. Adapted from https://davidyat.es/2016/07/27/writing-a-latex-macro-that-takes-a-variable-number-of-arguments/
\newcommand{\mat}[2][p]{
    \def\matrixenvironment{#1matrix} % Specifying the matrix brackets, this has to be done beforehand as '#1' changes under \passtonextarg
    \def\my@delim{&}
        \begin{\matrixenvironment} % Begin matrix environment
            \my@vector #2,\relax\noexpand\@eolst
            \@ifnextchar\bgroup{\passtonextarg}{\end{\matrixenvironment}}% % Pass to next argument (if any), otherwise end matrix environment
}
\newcommand{\passtonextarg}[1]{\\ \my@vector #1,\relax\noexpand\@eolst
    \@ifnextchar\bgroup{\passtonextarg}{\end{\matrixenvironment}}% Passing to next argument
}
\makeatother

\definecolor{tcol_DEF}{HTML}{E40125} % Color for Definition
\definecolor{tcol_PRP}{HTML}{EB8407} % Color for Proposition
\definecolor{tcol_LEM}{HTML}{05C4D9} % Color for Lemma
\definecolor{tcol_THM}{HTML}{1346E4} % Color for Theorem
\definecolor{tcol_COR}{HTML}{7904C2} % Color for Corollary
\definecolor{tcol_REM}{HTML}{18B640} % Color for Remark
\definecolor{tcol_PRF}{HTML}{5A76B2} % Color for Proof
\definecolor{tcol_EXA}{HTML}{21340A} % Color for Example

\tcbset{
tbox_DEF_style/.style={enhanced jigsaw,
    colback=tcol_DEF!10,colframe=tcol_DEF!80!black,,
    fonttitle=\sffamily\bfseries,
    separator sign=.,label separator={},
    sharp corners,top=2pt,bottom=2pt,left=2pt,right=2pt,
    before skip=10pt,after skip=10pt,breakable
},
tbox_PRP_style/.style={enhanced jigsaw,
    colback=tcol_PRP!10,colframe=tcol_PRP!80!black,
    fonttitle=\sffamily\bfseries,
    attach boxed title to top left={yshift=-\tcboxedtitleheight},
    boxed title style={
        boxrule=0pt,boxsep=2.5pt,
        colback=tcol_PRP!80!black,colframe=tcol_PRP!80!black,
        sharp corners=uphill
    },
    separator sign=.,label separator={},
    top=\tcboxedtitleheight,bottom=2pt,left=2pt,right=2pt,
    before skip=10pt,after skip=10pt,drop fuzzy shadow,breakable
},
tbox_THM_style/.style={enhanced jigsaw,
    colback=tcol_THM!10,colframe=tcol_THM!80!black,
    fonttitle=\sffamily\bfseries,coltitle=black,
    attach boxed title to top left={xshift=10pt,yshift=-\tcboxedtitleheight/2},
    boxed title style={
        colback=tcol_THM!10,colframe=tcol_THM!80!black,height=16pt,bean arc
    },
    separator sign=.,label separator={},
    sharp corners,top=6pt,bottom=2pt,left=2pt,right=2pt,
    before skip=10pt,after skip=10pt,breakable
},
tbox_LEM_style/.style={enhanced jigsaw,
    colback=tcol_LEM!10,colframe=tcol_LEM!80!black,
    boxrule=0pt,
    fonttitle=\sffamily\bfseries,
    attach boxed title to top left={yshift=-\tcboxedtitleheight},
    boxed title style={
        boxrule=0pt,boxsep=2pt,
        colback=tcol_LEM!80!black,colframe=tcol_LEM!80!black,
        interior code={\fill[tcol_LEM!80!black] (interior.north west)--(interior.south west)--([xshift=-2mm]interior.south east)--([xshift=2mm]interior.north east)--cycle;
    }},
    separator sign=.,label separator={},
    frame hidden,borderline north={1pt}{0pt}{tcol_LEM!80!black},
    before upper={\hspace{\tcboxedtitlewidth}},
    sharp corners,top=2pt,bottom=2pt,left=5pt,right=5pt,
    before skip=10pt,after skip=10pt,breakable
},
tbox_COR_style/.style={enhanced jigsaw,
    colback=tcol_COR!10,colframe=tcol_COR!80!black,
    boxrule=0pt,
    fonttitle=\sffamily\bfseries,coltitle=black,
    separator sign={},label separator={},
    description font=\normalfont\sffamily,
    description delimiters={(}{)},
    attach title to upper,after title={.\ },
    frame hidden,borderline west={2pt}{0pt}{tcol_COR},
    sharp corners,top=2pt,bottom=2pt,left=5pt,right=5pt,
    before skip=10pt,after skip=10pt,breakable
},
}

\newtcbtheorem[number within=subsection,
    crefname={\color{tcol_DEF!50!black} definition}{\color{tcol_DEF!50!black} definitionen},
    Crefname={\color{tcol_DEF!50!black} Definition}{\color{tcol_DEF!50!black} Definitionen}
    ]{definition}{Definition}{tbox_DEF_style}{}
\newtcbtheorem[use counter from=definition,
    crefname={\color{tcol_PRP!50!black} satz}{\color{tcol_PRP!50!black} sätze},
    Crefname={\color{tcol_PRP!50!black} Satz}{\color{tcol_PRP!50!black} Sätze}
    ]{satz}{Satz}{tbox_PRP_style}{}
\newtcbtheorem[use counter from=definition,
    crefname={\color{tcol_THM!50!black} theorem}{\color{tcol_THM!50!black} theoreme},
    Crefname={\color{tcol_THM!50!black} Theorem}{\color{tcol_THM!50!black} Theoreme}
    ]{theorem}{Theorem}{tbox_THM_style}{}
\newtcbtheorem[use counter from=definition,
    crefname={\color{tcol_LEM!50!black} lemma}{\color{tcol_LEM!50!black} lemmata},
    Crefname={\color{tcol_LEM!50!black} Lemma}{\color{tcol_LEM!50!black} Lemmata}
    ]{lemma}{Lemma}{tbox_LEM_style}{}
\newtcbtheorem[use counter from=definition,
    crefname={\color{tcol_COR!50!black} korollar}{\color{tcol_COR!50!black} korollare},
    Crefname={\color{tcol_COR!50!black} Korollar}{\color{tcol_COR!50!black} Korollare}
    ]{korollar}{Korollar}{tbox_COR_style}{}

\makeatletter
\@namedef{tcolorboxshape@filingbox@ul}#1#2#3{
    (frame.south west)--(title.north west)--([xshift=-\dimexpr#1\relax]title.north east) to[out=0,in=180] ([xshift=\dimexpr#2\relax,yshift=\dimexpr#3\relax]title.south east)--(frame.north east)--(frame.south east)--cycle
}
\@namedef{tcolorboxshape@filingbox@uc}#1#2#3{
    (frame.south west)--(frame.north west)--([xshift=-\dimexpr#2\relax,yshift=\dimexpr#3\relax]title.south west) to[out=0,in=180] ([xshift=\dimexpr#1\relax]title.north west)--([xshift=-\dimexpr#1\relax]title.north east) to[out=0,in=180] ([xshift=\dimexpr#2\relax,yshift=\dimexpr#3\relax]title.south east)--(frame.north east)--(frame.south east)--cycle
}
\@namedef{tcolorboxshape@filingbox@ur}#1#2#3{
    (frame.south east)--(title.north east)--([xshift=\dimexpr#1\relax]title.north west) to[out=180,in=0] ([xshift=-\dimexpr#2\relax,yshift=\dimexpr#3\relax]title.south west)--(frame.north west)--(frame.south west)--cycle
}
\@namedef{tcolorboxshape@filingbox@dl}#1#2#3{
    (frame.north west)--(title.south west)--([xshift=-\dimexpr#1\relax]title.south east) to[out=0,in=180] ([xshift=\dimexpr#2\relax,yshift=-\dimexpr#3\relax]title.north east)--(frame.south east)--(frame.north east)--cycle
}
\@namedef{tcolorboxshape@filingbox@dc}#1#2#3{
    (frame.north west)--(frame.south west)--([xshift=-\dimexpr#2\relax,yshift=-\dimexpr#3\relax]title.north west) to[out=0,in=180] ([xshift=\dimexpr#1\relax]title.south west)--([xshift=-\dimexpr#1\relax]title.south east) to[out=0,in=180] ([xshift=\dimexpr#2\relax,yshift=-\dimexpr#3\relax]title.north east)--(frame.south east)--(frame.north east)--cycle
}
\@namedef{tcolorboxshape@filingbox@dr}#1#2#3{
    (frame.north east)--(title.south east)--([xshift=\dimexpr#1\relax]title.south west) to[out=180,in=0] ([xshift=-\dimexpr#2\relax,yshift=-\dimexpr#3\relax]title.north west)--(frame.south west)--(frame.north west)--cycle
}
\@namedef{tcolorboxshape@railingbox@ul}#1#2#3{
    (frame.south west)--(title.north west)--([xshift=-\dimexpr#1\relax]title.north east)--([xshift=\dimexpr#2\relax,yshift=\dimexpr#3\relax]title.south east)--(frame.north east)--(frame.south east)--cycle
}
\@namedef{tcolorboxshape@railingbox@uc}#1#2#3{
    (frame.south west)--(frame.north west)--([xshift=-\dimexpr#2\relax,yshift=\dimexpr#3\relax]title.south west)--([xshift=\dimexpr#1\relax]title.north west)--([xshift=-\dimexpr#1\relax]title.north east)--([xshift=\dimexpr#2\relax,yshift=\dimexpr#3\relax]title.south east)--(frame.north east)--(frame.south east)--cycle
}
\@namedef{tcolorboxshape@railingbox@ur}#1#2#3{
    (frame.south east)--(title.north east)--([xshift=\dimexpr#1\relax]title.north west)--([xshift=-\dimexpr#2\relax,yshift=\dimexpr#3\relax]title.south west)--(frame.north west)--(frame.south west)--cycle
}
\@namedef{tcolorboxshape@railingbox@dl}#1#2#3{
    (frame.north west)--(title.south west)--([xshift=-\dimexpr#1\relax]title.south east)--([xshift=\dimexpr#2\relax,yshift=-\dimexpr#3\relax]title.north east)--(frame.south east)--(frame.north east)--cycle
}
\@namedef{tcolorboxshape@railingbox@dc}#1#2#3{
    (frame.north west)--(frame.south west)--([xshift=-\dimexpr#2\relax,yshift=-\dimexpr#3\relax]title.north west)--([xshift=\dimexpr#1\relax]title.south west)--([xshift=-\dimexpr#1\relax]title.south east)--([xshift=\dimexpr#2\relax,yshift=-\dimexpr#3\relax]title.north east)--(frame.south east)--(frame.north east)--cycle
}
\@namedef{tcolorboxshape@railingbox@dr}#1#2#3{
    (frame.north east)--(title.south east)--([xshift=\dimexpr#1\relax]title.south west)--([xshift=-\dimexpr#2\relax,yshift=-\dimexpr#3\relax]title.north west)--(frame.south west)--(frame.north west)--cycle
}
\newcommand{\TColorBoxShape}[2]{\expandafter\ifx\csname tcolorboxshape@#1@#2\endcsname\relax
\expandafter\@gobble\else
\csname tcolorboxshape@#1@#2\expandafter\endcsname
\fi}
\makeatother

\tcbset{ % Styles for filingbox, railingbox and flagbox environments
% Adapted from https://tex.stackexchange.com/questions/587912/tcolorbox-custom-title-box-style
filingstyle/ul/.style 2 args={
    attach boxed title to top left={yshift=-2mm},
    boxed title style={empty,top=0mm,bottom=1mm,left=1mm,right=0mm},
    interior code={
        \path[fill=#1,rounded corners] \TColorBoxShape{filingbox}{ul}{9pt}{18pt}{6pt};
    },
    frame code={
        \path[draw=#2,line width=0.5mm,rounded corners] \TColorBoxShape{filingbox}{ul}{9pt}{18pt}{6pt};
    }},
filingstyle/uc/.style 2 args={
    attach boxed title to top center={yshift=-2mm},
    boxed title style={empty,top=0mm,bottom=1mm,left=0mm,right=0mm},
    interior code={
        \path[fill=#1,rounded corners] \TColorBoxShape{filingbox}{uc}{9pt}{18pt}{6pt};
    },
    frame code={
        \path[draw=#2,line width=0.5mm,rounded corners] \TColorBoxShape{filingbox}{uc}{9pt}{18pt}{6pt};
    }},
filingstyle/ur/.style 2 args={
    attach boxed title to top right={yshift=-2mm},
    boxed title style={empty,top=0mm,bottom=1mm,left=0mm,right=1mm},
    interior code={
        \path[fill=#1,rounded corners] \TColorBoxShape{filingbox}{ur}{9pt}{18pt}{6pt};
    },
    frame code={
        \path[draw=#2,line width=0.5mm,rounded corners] \TColorBoxShape{filingbox}{ur}{9pt}{18pt}{6pt};
    }},
filingstyle/dl/.style 2 args={
    attach boxed title to bottom left={yshift=2mm},
    boxed title style={empty,top=1mm,bottom=0mm,left=1mm,right=0mm},
    interior code={
        \path[fill=#1,rounded corners] \TColorBoxShape{filingbox}{dl}{9pt}{18pt}{6pt};
    },
    frame code={
        \path[draw=#2,line width=0.5mm,rounded corners] \TColorBoxShape{filingbox}{dl}{9pt}{18pt}{6pt};
    }},
filingstyle/dc/.style 2 args={
    attach boxed title to bottom center={yshift=2mm},
    boxed title style={empty,top=1mm,bottom=0mm,left=0mm,right=0mm},
    interior code={
        \path[fill=#1,rounded corners] \TColorBoxShape{filingbox}{dc}{9pt}{18pt}{6pt};
    },
    frame code={
        \path[draw=#2,line width=0.5mm,rounded corners] \TColorBoxShape{filingbox}{dc}{9pt}{18pt}{6pt};
    }},
filingstyle/dr/.style 2 args={
    attach boxed title to bottom right={yshift=2mm},
    boxed title style={empty,top=1mm,bottom=0mm,left=0mm,right=1mm},
    interior code={
        \path[fill=#1,rounded corners] \TColorBoxShape{filingbox}{dr}{9pt}{18pt}{6pt};
    },
    frame code={
        \path[draw=#2,line width=0.5mm,rounded corners] \TColorBoxShape{filingbox}{dr}{9pt}{18pt}{6pt};
    }},
railingstyle/ul/.style 2 args={
    attach boxed title to top left={yshift=-2mm},
    boxed title style={empty,top=0mm,bottom=1mm,left=1mm,right=0mm},
    interior code={
        \path[fill=#1] \TColorBoxShape{railingbox}{ul}{3pt}{12pt}{6pt};
    },
    frame code={
        \path[draw=#2,line width=0.5mm] \TColorBoxShape{railingbox}{ul}{3pt}{12pt}{6pt};
    }},
railingstyle/uc/.style 2 args={
    attach boxed title to top center={yshift=-2mm},
    boxed title style={empty,top=0mm,bottom=1mm,left=0mm,right=0mm},
    interior code={
        \path[fill=#1] \TColorBoxShape{railingbox}{uc}{3pt}{12pt}{6pt};
    },
    frame code={
        \path[draw=#2,line width=0.5mm] \TColorBoxShape{railingbox}{uc}{3pt}{12pt}{6pt};
    }},
railingstyle/ur/.style 2 args={
    attach boxed title to top right={yshift=-2mm},
    boxed title style={empty,top=0mm,bottom=1mm,left=0mm,right=1mm},
    interior code={
        \path[fill=#1] \TColorBoxShape{railingbox}{ur}{3pt}{12pt}{6pt};
    },
    frame code={
        \path[draw=#2,line width=0.5mm] \TColorBoxShape{railingbox}{ur}{3pt}{12pt}{6pt};
    }},
railingstyle/dl/.style 2 args={
    attach boxed title to bottom left={yshift=2mm},
    boxed title style={empty,top=1mm,bottom=0mm,left=1mm,right=0mm},
    interior code={
        \path[fill=#1] \TColorBoxShape{railingbox}{dl}{3pt}{12pt}{6pt};
    },
    frame code={
        \path[draw=#2,line width=0.5mm] \TColorBoxShape{railingbox}{dl}{3pt}{12pt}{6pt};
    }},
railingstyle/dc/.style 2 args={
    attach boxed title to bottom center={yshift=2mm},
    boxed title style={empty,top=1mm,bottom=0mm,left=0mm,right=0mm},
    interior code={
        \path[fill=#1] \TColorBoxShape{railingbox}{dc}{3pt}{12pt}{6pt};
    },
    frame code={
        \path[draw=#2,line width=0.5mm] \TColorBoxShape{railingbox}{dc}{3pt}{12pt}{6pt};
    }},
railingstyle/dr/.style 2 args={
    attach boxed title to bottom right={yshift=2mm},
    boxed title style={empty,top=1mm,bottom=0mm,left=0mm,right=1mm},
    interior code={
        \path[fill=#1] \TColorBoxShape{railingbox}{dr}{3pt}{12pt}{6pt};
    },
    frame code={
        \path[draw=#2,line width=0.5mm] \TColorBoxShape{railingbox}{dr}{3pt}{12pt}{6pt};
    }},
flagstyle/ul/.style 2 args={
    interior hidden,frame hidden,colbacktitle=#1,
    borderline west={1pt}{0pt}{#2},
    attach boxed title to top left={yshift=-8pt,yshifttext=-8pt},
    boxed title style={boxsep=3pt,boxrule=1pt,colframe=#2,sharp corners,left=4pt,right=4pt},
    bottom=0mm
    },
flagstyle/ur/.style 2 args={
    interior hidden,frame hidden,colbacktitle=#1,
    borderline east={1pt}{0pt}{#2},
    attach boxed title to top right={yshift=-8pt,yshifttext=-8pt},
    boxed title style={boxsep=3pt,boxrule=1pt,colframe=#2,sharp corners,left=4pt,right=4pt},
    bottom=0mm
    },
flagstyle/dl/.style 2 args={
    interior hidden,frame hidden,colbacktitle=#1,
    borderline west={1pt}{0pt}{#2},
    attach boxed title to bottom left={yshift=8pt,yshifttext=8pt},
    boxed title style={boxsep=3pt,boxrule=1pt,colframe=#2,sharp corners,left=4pt,right=4pt},
    top=0mm
    },
flagstyle/dr/.style 2 args={
    interior hidden,frame hidden,colbacktitle=#1,
    borderline east={1pt}{0pt}{#2},
    attach boxed title to bottom right={yshift=8pt,yshifttext=8pt},
    boxed title style={boxsep=3pt,boxrule=1pt,colframe=#2,sharp corners,left=4pt,right=4pt},
    top=0mm
    }
}

% Box in the shape of a filing divider, position of tab can be ul (up left), uc (up center), ur (up right), dl (down left), dc (down center) or dr (down right). Default is ul (upper left)
\NewTColorBox{filingbox}{ D(){ul} O{black} m O{} }{enhanced,
    top=1mm,bottom=1mm,left=1mm,right=1mm,
    title={#3},
    fonttitle=\sffamily\bfseries,
    coltitle=black,
    filingstyle/#1={#2!10}{#2},
    #4
}

% Box in the shape of a railing bar, position of tab can be ul (up left), uc (up center), ur (up right), dl (down left), dc (down center) or dr (down right). Default is ul (upper left)
\NewTColorBox{railingbox}{ D(){ul} O{black} m O{} }{enhanced,
    top=1mm,bottom=1mm,left=1mm,right=1mm,
    title={#3},
    fonttitle=\sffamily\bfseries,
    coltitle=black,
    railingstyle/#1={#2!10}{#2},
    #4
}

% Box in the shape of a flag, position of tab can be ul (up left), ur (up right), dl (down left) or dr (down right). Default is ul (upper left)
\NewTColorBox{flagbox}{ D(){ul} O{black} m O{} }{enhanced,breakable,
    top=1mm,bottom=1mm,left=1mm,right=1mm,
    title={#3},
    fonttitle=\sffamily\bfseries,
    coltitle=black,
    flagstyle/#1={#2!10}{#2},
    #4
}

\makeatletter
\newcommand*{\CreateSmartLargeOperator}[2]{
% Adapted from https://tex.stackexchange.com/questions/61598/new-command-with-cases-conditionals-if-thens/61600
    % Plain operator (no customization)
    \csdef{LargeOperator@#1@}{\csdef{LargeOperator@#1@Symbol}{\csuse{#1}}}
    % Operator with limits above and below symbol
    \csdef{LargeOperator@#1@l}{\csdef{LargeOperator@#1@Symbol}{\csuse{#1}\limits}}
    % Operato with limits beside symbol
    \csdef{LargeOperator@#1@n}{\csdef{LargeOperator@#1@Symbol}{\csuse{#1}\nolimits}}
    % Inline style operator
    \csdef{LargeOperator@#1@i}{\csdef{LargeOperator@#1@Symbol}{\textstyle\csuse{#1}}}
    % Display style operator
    \csdef{LargeOperator@#1@d}{\csdef{LargeOperator@#1@Symbol}{\displaystyle\csuse{#1}}}
    % Inline style operator with limits above and below symbol
    \csdef{LargeOperator@#1@il}{\csdef{LargeOperator@#1@Symbol}{\textstyle\csuse{#1}\limits}}
    % Inline style operator with limits beside symbol
    \csdef{LargeOperator@#1@in}{\csdef{LargeOperator@#1@Symbol}{\textstyle\csuse{#1}\nolimits}}
    % Display style operator with limits above and below symbol
    \csdef{LargeOperator@#1@dl}{\csdef{LargeOperator@#1@Symbol}{\displaystyle\csuse{#1}\limits}}
    % Display style operator with limits beside symbol
    \csdef{LargeOperator@#1@dn}{\csdef{LargeOperator@#1@Symbol}{\displaystyle\csuse{#1}\nolimits}}

% NOTE: In the command below, ##1 denotes the operator. It is NOT to be used as an argument!
\def\LargeOperatorSpecs@i##1,##2,##3,##4,##5,##6,##7\@nil{
% If no arguments, operate over n from 1 to infinity
    \ifx$##2$\csuse{LargeOperator@##1@Symbol}_{n=1}^{\infty}\else
    % If one argument, operate over n from ##2 to infinity
        \ifx$##3$\csuse{LargeOperator@##1@Symbol}_{n=##2}^{\infty}\else
        % If two arguments, operate over n from ##2 to ##3
            \ifx$##4$\csuse{LargeOperator@##1@Symbol}_{n=##2}^{##3}\else
            % If three arguments, operate over ##2 from ##3 to ##4
                \ifx$##5$\csuse{LargeOperator@##1@Symbol}_{##2=##3}^{##4}\else
                % If four arguments, operate over ##2 and ##3 from ##4 to ##5
                    \ifx$##6$\csuse{LargeOperator@##1@Symbol}_{##2,##3=##4}^{##5}\else
                    % If five arguments, operate over ##2, ##3 and ##4 from ##5 to ##6
                        \csuse{LargeOperator@##1@Symbol}_{##2,##3,##4=##5}^{##6}
                    \fi
                \fi
            \fi
        \fi
    \fi
}

% Flexible "smart" large operator macro with comma-separated arguments and optional argument for formatting. Default is over n from 1 to infinity. Adapted from https://tex.stackexchange.com/a/15722
\expandafter\DeclareDocumentCommand\csname#2\endcsname{ O{} m }{ % New operator macro
\bgroup % Group created to keep operator style (e.g. \limits) local
    \expandafter\ifx\csname LargeOperator@#1@##1\endcsname\relax
    \expandafter\@gobble\else
    \csname LargeOperator@#1@##1\expandafter\endcsname
    \fi
    \expandafter\LargeOperatorSpecs@i#1,##2,,,,,\@nil% % #1 stands in for the first "argument" of \LargeOperatorSpecs@i (the operator), the actual arguments are from ##2 onward
\egroup}
}
\makeatother

% Create the smart large operator #2 based on the large operator #1. For example, \CreateSmartLargeOperator{sum}{Sum} will define \Sum as the smart large operator based on \sum
% Equivalent Unicode characters are given here (but they are NOT the same as the operators)
\CreateSmartLargeOperator{sum}{Sum}             % Large: U+2211 ∑ (no small version)
\CreateSmartLargeOperator{prod}{Prod}           % Small: U+2293 ⊓, Large: U+220F ∏
\CreateSmartLargeOperator{coprod}{Coprod}       % Small: U+2294 ⊔, Large: U+2210 ∐
\CreateSmartLargeOperator{bigcap}{Capp}         % Small: U+2229 ∩, Large: U+22C2 ⋂
\CreateSmartLargeOperator{bigcup}{Cupp}         % Small: U+222A ∪, Large: U+22C3 ⋃
\CreateSmartLargeOperator{bigsqcup}{Kupp}       % Small: U+2294 ⊔, Large: U+2210 ∐
\CreateSmartLargeOperator{bigodot}{Odot}        % Small: U+2299 ⊙ (no large version)
\CreateSmartLargeOperator{bigoplus}{Oplus}      % Small: U+2295 ⊕ (no large version)
\CreateSmartLargeOperator{bigotimes}{Otimes}    % Small: U+2297 ⊗ (no large version)
\CreateSmartLargeOperator{biguplus}{Uplus}      % Small: U+228E ⊎ (no large version)
\CreateSmartLargeOperator{bigwedge}{Wedge}      % Small: U+2227 ∧, Large: U+22C0 ⋀
\CreateSmartLargeOperator{bigvee}{Vee}          % Small: U+2228 ∨, Large: U+22C1 ⋁

\tcolorboxenvironment{beweis}{boxrule=0pt,boxsep=0pt,blanker,
    borderline west={2pt}{0pt}{tcol_PRF},left=8pt,right=8pt,sharp corners,
    before skip=10pt,after skip=10pt,breakable
}
\tcolorboxenvironment{anmerkung}{boxrule=0pt,boxsep=0pt,blanker,
    borderline west={2pt}{0pt}{tcol_REM},left=8pt,right=8pt,
    before skip=10pt,after skip=10pt,breakable
}
\tcolorboxenvironment{anmerkungen}{boxrule=0pt,boxsep=0pt,blanker,
    borderline west={2pt}{0pt}{tcol_REM},left=8pt,right=8pt,
    before skip=10pt,after skip=10pt,breakable
}
\tcolorboxenvironment{beispiel}{boxrule=0pt,boxsep=0pt,blanker,
    borderline west={2pt}{0pt}{tcol_EXA},left=8pt,right=8pt,sharp corners,
    before skip=10pt,after skip=10pt,breakable
}
\tcolorboxenvironment{beispiele}{boxrule=0pt,boxsep=0pt,blanker,
    borderline west={2pt}{0pt}{tcol_EXA},left=8pt,right=8pt,sharp corners,
    before skip=10pt,after skip=10pt,breakable
}

% align and align* environments with inline size
\newenvironment{talign}{\let\displaystyle\textstyle\align}{\endalign}
\newenvironment{talign*}{\let\displaystyle\textstyle\csname align*\endcsname}{\endalign}

\usepackage[explicit]{titlesec}
% Setting the format for sections, subsections and subsubsections
\titleformat{\section}{\fontsize{24}{30}\sffamily\bfseries}{\thesection}{20pt}{#1}
\titleformat{\subsection}{\fontsize{16}{18}\sffamily\bfseries}{\thesubsection}{12pt}{#1}
\titleformat{\subsubsection}{\fontsize{10}{12}\sffamily\large\bfseries}{\thesubsubsection}{8pt}{#1}
% Setting the spacing for sections, subsections and subsubsections
% First argument is the left indent, second argument is the spacing above, third argument is the spacing below
\titlespacing*{\section}{0pt}{5pt}{5pt}
\titlespacing*{\subsection}{0pt}{5pt}{5pt}
\titlespacing*{\subsubsection}{0pt}{5pt}{5pt}

\newcommand{\Disp}{\displaystyle}
\newcommand{\qe}{\hfill\(\bigtriangledown\)}
\DeclareMathAlphabet\mathbfcal{OMS}{cmsy}{b}{n}
\setlength{\parindent}{0.2in}
\setlength{\parskip}{0pt}
\setlength{\columnseprule}{0pt}

\makeatletter
% Modify spacing above and below display equations
\g@addto@macro\normalsize{
    \setlength\abovedisplayskip{3pt}
    \setlength\belowdisplayskip{3pt}
    \setlength\abovedisplayshortskip{0pt}
    \setlength\belowdisplayshortskip{0pt}
}
\makeatother

\makeatletter
% Redefining the title block
\renewcommand\maketitle{
\null % \vspace does not work with nothing above it, so \null is added
\vspace{5mm}
\begingroup % Creating a group to ensure col_stripes is only defined locally, i.e. only for the title
\definecolor{col_stripes}{HTML}{1B0982} % Color of the stripes above and below the title components
    \begin{tcolorbox}[enhanced,blanker,
    borderline horizontal={2pt}{0pt}{col_stripes},
    borderline horizontal={1pt}{-3.5pt}{col_stripes},
    borderline horizontal={2pt}{-8pt}{col_stripes},
    fontupper=\fontfamily{bch},
    halign=flush center,top=10mm,bottom=10mm,after skip=20mm,
    ]
        {\fontsize{24}{28}\bfseries\selectfont\@title}\\
            \vspace{6mm}
        {\fontsize{20}{24}\selectfont\@author}\\
            \vspace{6mm}
        {\fontsize{16}{20}\selectfont\@date}
    \end{tcolorbox}
\endgroup}
% Adapted from https://tex.stackexchange.com/questions/483953/how-to-add-new-macros-like-author-without-editing-latex-ltx?noredirect=1&lq=1
\makeatother

\title{Algebra (Bachelor)}
\author{zur Vorlesung von Prof. Dr. Tobias Dyckerhoff}
\date{\today} % Replace with \today to show the current date

\begin{document}

\maketitle

\definecolor{tcol_CNT1}{HTML}{72E094} % First color for Contents
\definecolor{tcol_CNT2}{HTML}{24E2D6} % Second color for Contents
\definecolor{tcol_CNV1}{HTML}{8E44AD} % First color for Conventions
\definecolor{tcol_CNV2}{HTML}{A10B49} % First color for Conventions

\begin{tcolorbox}[enhanced,
    title=Inhaltsverzeichnis,
    fonttitle=\fontsize{20}{24}\sffamily\bfseries\selectfont,
    coltitle=black,
    fontupper=\sffamily,
    interior style={left color=tcol_CNT1!80,right color=tcol_CNT2!80},
    frame style={left color=tcol_CNT1!60!black,right color=tcol_CNT2!60!black},
    attach boxed title to top center={yshift=10pt},
    boxed title style={frame hidden,
        interior style={left color=tcol_CNT1,right color=tcol_CNT2},
        frame style={left color=tcol_CNT1!60!black,right color=tcol_CNT2!60!black},
        height=24pt,bean arc,drop fuzzy shadow
    },
    top=2mm,bottom=2mm,left=2mm,right=2mm,
    before skip=20mm,after skip=20mm,
    drop fuzzy shadow,breakable]
%
\makeatletter
\@starttoc{toc}
\makeatother
\end{tcolorbox}

\begin{tcolorbox}[enhanced,
    frame hidden,
    title=Konventionen,
    fonttitle=\large\sffamily\bfseries\selectfont,
    interior code={
        \shade[top color=tcol_CNV2!50,bottom color=white] ([yshift=2mm]interior.north west) arc(-180:-90:2mm)--(interior.north east)--(interior.south east)--(interior.south west)--cycle;
        },
    overlay={
        \draw[tcol_CNV1!50!black,line width=0.5mm] ([xshift=2mm]frame.north west)--(frame.north east);
    },
    boxrule=0pt,left=2pt,right=2pt,
    sharp corners=north,
    attach boxed title to top left,
    boxed title style={interior hidden,
    left=1mm,right=1mm,
    frame code={
        \path[draw=tcol_CNV1!50!black,line width=0.5mm,fill=tcol_CNV1,rounded corners=2mm] ([xshift=2mm]frame.south east)--(frame.south east)--(frame.north east)--([xshift=0.25mm]frame.north west)--([xshift=0.25mm]frame.south west)--cycle;}
    },
    top=2mm,bottom=2mm,left=2mm,right=2mm,
    before skip=10mm,after skip=10mm]
%
\begin{itemize}
\item TBD
\end{itemize}
\end{tcolorbox}
Dies ist ein inoffizielles Skript zur Vorlesung Algebra bei Prof. Dr. Tobias Dyckerhoff im Wintersemester 24/25. Fehler und Verbesserungsvorschläge immer gerne an \url{rasmus.raschke@uni-hamburg.de}.
\newpage
\sloppy
\section{Gruppen und Symmetrie}
\label{gruppentheorie}

\begin{bemerkung}
Wir möchten Gruppentheorie zunächst motivieren: Man betrachte einen Tetraeder. Um dessen Symmetrien zu erfassen, könnten wir z.B. schauen, welche Bewegungen diesen in sich selbst überführen. Es gibt vier Rotationsachsen, die eine Ecke und eine Fläche durchdringen und bei Rotation um $120^\circ$ den Tetraeder in sich selbst überführen. Weiterhin gibt es drei $180^\circ$-Rotationsachsen mittig durch gegenüberliegende Kanten. Auch die Identität lässt den Tetraeder unverändert. Also gibt es $1+4 \cdot 2 + 3 = 12$ Symmetrien. Gruppen bieten eine Möglichkeit, solche Symmetrien und deren Verkettungen zu erfassen und zu untersuchen.
\end{bemerkung}

\subsection{Grundbegriffe}
\label{subsec:grundbegriffe}

\begin{definition}{Gruppe}{gruppe}
Eine \textbf{Gruppe} ist ein Paar $(G, \circ)$, bestehend aus einer Menge\footnote{im ZFC-Axiomensystem} $G$ und einer Abbildung
\begin{align}
\circ: G \times G &\to G\\
(g,h) &\mapsto g \circ h
\end{align} 
mit folgenden Eigenschaften:
\begin{enumerate}[({G}1)]
\item Für alle $g_1,g_2,g_3 \in G$ gilt das Assoziativgesetz: $(g_1 \circ g_2) \circ g_3 = g_1 \circ (g_2 \circ g_3)$.
\item Es gibt ein Element $e \in G$, sodass gilt:
\begin{enumerate}[({2}a)]
\item Für jedes $g \in G$ gilt $e \circ g = g$.
\item Für jedes $g \in G$ existiert ein $g' \in G$ mit $g' \circ g = e$.
\end{enumerate}
\end{enumerate}
Die Abbildung $\circ$ heißt \textbf{Verknüpfung}, ein Element $e \in G$ mit den Eigenschaften aus (2G) heißt \textbf{neutrales Element}, und ein Element $g' \in G$ zu gegebenem $g \in G$ mit Eigenschaft (2b) heißt \textbf{Inverses} von $g$.
\end{definition}

\begin{übung}
Sei $(G, \circ)$ eine Gruppe. Dann gelte:
\begin{enumerate}
\item Das neutrale Element $e \in G$ ist eindeutig bestimmt, außerdem gelte $ \forall g \in G : g \circ e = g$.
\item Zu gegebenem $g \in G$ ist das Inverse $g' \in G$ eindeutig bestimmt und erfüllt zudem $g \circ g' = e$.
\item Für $n \geq 3$ hängt das Produkt von Gruppenelementen $g_1, g_2, \dots, g_n$ nicht von der Klammerung ab.
\end{enumerate}
\end{übung}

\begin{lösung}
Zuerst zeigen wir Kommutativität des Inversen. Sei $g \in G$, dann gilt:
\begin{align}
g \circ g^{-1} &= (e \circ g) \circ g^{-1} = \left( \left( \left( g^{-1}\right)^{-1} \circ g^{-1} \right) \circ g \right) \circ g^{-1} = \left(  \left( g^{-1}\right)^{-1} \circ \left( g^{-1} \circ g \right)\right) \circ g^{-1}\\ 
&= \left( g^{-1}\right)^{-1} \circ \left( e  \circ g^{-1} \right) = \left( g^{-1}\right)^{-1} \circ g^{-1} = e = g^{-1} \circ g,
\end{align}
also stimmen Links- und Rechtsinverses in Gruppen überein.
Die Kommutativität des neutralen Elements folgt damit direkt aus:
\begin{equation}
g \circ e = g \circ (g^{-1} \circ g) = (g \circ g^{-1}) \circ g = (g^{-1} \circ g) \circ g = e \circ g,
\end{equation}
womit auch Links-Einselement und Rechts-Einselement übereinstimmen.
Für die Eindeutigkeit des Inversen seien $g^{-1}, g'^{-1} \in G$ zwei Inverse von $g \in G$. Dann gilt:
\begin{equation}
g^{-1} = g^{-1} \circ e = g^{-1} \circ (g'^{-1} \circ g) = g^{-1} \circ (g \circ g'^{-1}) = (g^{-1} \circ g) \circ g'^{-1} = e \circ g'^{-1} = g'^{-1}.
\end{equation}
Weiterhin seien $e,e' \in G$ zwei Einselemente. Da $e = e \circ e' = e' \circ e = e$ gilt, ist das neutrale Element eindeutig. \qed
\end{lösung}

\begin{beispiele}
Wir geben einige Beispiele für Gruppen:
\begin{enumerate}
\item Die Gruppe $(\Z, +)$ der ganzen Zahlen $\Z$ mit der Addition $+$.
\item Für einen Körper $\K$ existiert die additive Gruppe $(\K, +)$ und die multiplikative Gruppe $(\K \exc \{0\}, \cdot)$.
\item Für jede Menge $M$ existiert die \textbf{symmetrische Gruppe} $(\mathfrak{S}_M, \circ)$, wobei $\mathfrak{S}_M$ die Menge der bijektiven Selbstabbildungen von $M$ und $\circ$ die Komposition ist. Für $n \geq 1$ vereinbaren wir $\mathfrak{S}_n := \mathfrak{S}_{\{1,2,\dots,n\}}$. Wir vereinbaren als Konvention die \textbf{Zykelschreibweise}. In $\mathfrak{S}_3$ beispielsweise ist ein Zykel
\begin{align}
\sigma: \{1,2,3\} &\to \{1,2,3\}\\
1 &\mapsto 2 \\
2 &\mapsto 1 \\
3 &\mapsto 3,
\end{align}
auch darstellbar als
\begin{equation}
\mat{1,2,3}{2,1,3}
\end{equation}
oder einfacher als $(12)$.
\item Für $n \geq 1$ und einen Körper $\K$ ist die \textbf{allgemeine lineare Gruppe} $(\text{GL}(n,\K), \circ)$ definiert, wobei 
\begin{equation}
\text{GL}(n, \K) := \left\{ A \in \K^{n\times n} \, | \, \det A \neq 0 \right\}
\end{equation}
die Menge der invertierbaren $n \times n$-Matrizen mit Einträgen in $\K$ ist. Typische Beispiele für Körper sind $\K = \Q, \R, \C, \F_q$ mit $q = p^n$, $p$ prim.\\
ÜA: $| \text{GL}(n, \F_q)| = ?$.
\end{enumerate}
\end{beispiele}
\begin{bemerkung}
Um den alltäglichen Gebrauch von Gruppen zu vereinfachen, machen wir folgende Vereinbarungen:
\begin{enumerate}
\item Wir bezeichnen $(G, \circ)$ üblicherweise einfach mit $G$ und lassen $\circ$ implizit.
\item Für $g,h \in G$ schreiben wir $gh = g \circ h$, für $e \in G$ schreiben wir $1$ und für $g'$ schlicht $g^{-1}$.
\item Gilt $g \circ h = h \circ g$ für alle $g,h \in G$, so heißt $G$ \textbf{abelsch}. In diesem Fall wird die Verknüpfung oft mit $+$, das neutrale Element mit $0$ und das inverse Element mit $-g$ bezeichnet.
\item Gemäß obiger ÜA zur Klammerung schreiben wir einfach $g_1 g_2 \cdots g_n \in G$ ohne Klammerung.
\end{enumerate}
\end{bemerkung}
\begin{definition}{Ordnung}{ordnung}
Für eine Gruppe $G$ bezeichnen wir die Kardinalität \begin{equation}
|G| \in \N \cup \{+\infty \}
\end{equation}
als \textbf{Ordnung} von $G$.
\end{definition}

\subsection{Untergruppen}
\label{subsec:untergruppen}
\begin{definition}{Untergruppe}{untergruppe}
Sei $(G, \circ)$ eine Gruppe. Eine Teilmenge $H \sub G$ heißt \textbf{Untergruppe}, falls gilt:
\begin{enumerate}[({U}1)]
\item $H \neq \emptyset$
\item Abgeschlossenheit: Für alle $a, b \in H$ gilt $ab^{-1} \in H$.
\end{enumerate}
Wir verwenden dann die Notation $H \leq G$, um Untergruppen zu kennzeichnen.
\end{definition}

\begin{bemerkung} Übungsaufgabe:
Sei $G$ eine Gruppe und $H \leq G$ eine Untergruppe. Dann gilt:
\begin{enumerate}
\item Aus Eigenschaft 1:Da $H \neq \emptyset$, existiert ein $a \in H$.
\item Aus Eigenschaft 2: $a \cdot a^{-1} = e \in H$.
\item Aus Eigenschaft 2: Für jedes $a \in H$ gilt $a^{-1} = e \cdot a^{-1} \in H$.
\item Aus Eigenschaft 2: Für jedes $a,b \in H$ gilt $ab = a \cdot (b^{-1})^{-1} \in H$.
\end{enumerate}
Also: $H \sub G$ ist eine Untergruppe genau dann, wenn folgende alternativen Eigenschaften gelten:
\begin{enumerate}[1.{$^\ast $}]
\item $e_G \in H$
\item Für alle $a,b \in H$ muss $a \cdot b \in H$ gelten.
\item Für alle $a \in H$ ist $a^{-1} \in H$.
\end{enumerate}
Die andere Richtung der Äquivalenz ist trivial. Daraus folgt auch, dass $(H, \circ|_{H \times H})$ mit der auf $H$ eingeschränkten Verknüpfung $\circ|_{H \times H}$ eine Gruppe ist. 
\end{bemerkung}
\begin{beispiele}
Einige Beispiele für Untergruppen sind:
\begin{enumerate}
\item $(G, \circ) = (\R, +)$ hat $(\Z, +)$ als Untergruppe mit $\Z \sub \R$.
\item Sei $n \geq 1$ und $\K$ ein Körper. Die \textbf{spezielle lineare Gruppe}
\begin{equation}
\text{SL} (n, \K) := \left\{ A \in \text{GL}(n, \K) \, | \, \det A = 1\right\} \leq \text{GL}(n, \K)
\end{equation}
ist eine Untergruppe von $\text{GL}(n,\K)$.
\item Für $n \geq$ und einen Körper $\K$ ist die \textbf{orthogonale Gruppe}
\begin{equation}
\text{O}(n, \K) := \left\{ A \in \text{GL}(n, \K) \, | \, A^TA = I_n \right\} \leq \text{GL}(n, \K)
\end{equation}
definiert, die auch eine Untergruppe von $\text{GL}(n, \K)$ ist.
\item Seien $H_1, H_2 \leq G$ Untergruppen. Dann ist $H_1 \cap H_2 \leq G$ auch eine Untergruppe. So kann z.B. die \textbf{spezielle orthogonale Gruppe}
\begin{equation}
\text{SO}(n, \K) := \text{O}(n, \K) \cap \text{SL}(n, \K)
\end{equation}
als Untergruppe von $\text{GL}(n, \K)$ konstruiert werden.
\item Etwas allgemeiner: Für jede Familie $\{H_i\}_{i\in I}$ von Untergruppen $H_i \leq G$ gilt, dass
\begin{equation}
\bigcap_{i \in I} H_i \leq G
\end{equation}
wieder eine Untergruppe ist.
\end{enumerate}
\end{beispiele}
\begin{definition}{Erzeugte Untergruppe}{}
Sei $G$ eine Gruppe und $M \sub G$ eine beliebige Teilmenge. Dann heißt die \textbf{Untergruppe}
\begin{equation}
\langle M \rangle := \bigcup_{M \sub H \leq G} H \leq G
\end{equation}
die \textbf{von} $M$ \textbf{erzeugte Untergruppe} von $G$. Falls $M = \{g\} \leq G$ eine einelementige Menge ist, schreiben wir \begin{equation}
\langle g \rangle := \langle \{g\} \rangle \leq G.
\end{equation}
\end{definition}
\begin{definition}{Ordnung eines Elements}{ordnungelement}
Sei $G$ eine Gruppe und $g \in G$ ein Element. Dann heißt die Kardinalität 
\begin{equation}
\text{ord}(g) := | \langle g \rangle | \in \N \cup \{\infty \}
\end{equation}
die \textbf{Ordnung von} $g$.
\end{definition}
\begin{satz}{Charakterisierung von einelementigen Untergruppen}{charakterisierungeinelementig}
Sei $G$ eine Gruppe und $g \in G$ ein Element.
\begin{enumerate}
\item Falls $\ord (g) < \infty$, dann gilt \begin{equation}
\ord (g) = \min \{k \geq 1 | g^k= 1\}
\end{equation}
und \begin{equation}
\langle g \rangle = \{1, g, g^2, \dots, g^{n-1} \},
\end{equation}
wobei $n := \ord (g)$.
\item Falls $\ord (g) = \infty$, dann gilt
\begin{equation}
\langle g \rangle = \{g^i | i \in \Z \} = \{ \dots, g^{-2}, g^{-1}, 1, g^1, g^2, \dots \},
\end{equation}
wobei die Potenzen $g^i$, $i \in \Z$ paarweise verschiedene Elemente in $G$ sind.
\end{enumerate}
\end{satz}
\begin{beweis}
Zunächst gilt für beliebiges $g \in G$ das Folgende:
\begin{equation}
\langle g \rangle = \{ \dots,g^{-2},g^{-1},1, g, g^2, \dots \} = \{g^i | i \in \Z\},
\end{equation}
wobei die Potenzen im Allgemeinen nicht notwendigerweise paarweise verschieden sind.
Dies folgt, da, damit $\langle g \rangle$ eine Untergruppe sein kann, zunächst das neutrale Element $1 = g^0$ und $g$ selbst enthalten sein muss. Dann muss aber auch die Selbstverknüpfung und das Inverse (sowie dessen Selbstverknüpfungen) enthalten sein.
\begin{enumerate}
\item Sei $\ord (g) < \infty$. Dann gibt es insbesondere $i,j \in \Z$ mit $i \neq j$ und $g^i = g^j$. O.B.d.A. sei $i > j$. Dann ist also $k = i-j \geq 1$ eine natürliche Zahl, für die gilt: $g^k = 1$. Nach dem Wohlordnungssatz existiert eine \textit{kleinste} natürliche Zahl $n \geq 1$, für die gilt: $g^n =1$. Sei nun $m \in \Z$. Dann gibt es eindeutig bestimmte Zahlen $a \in \Z$ und $0 \leq r < n$, sodass \begin{equation}
m = an +r.
\end{equation}
Damit folgt
\begin{equation}
g^m = g^{an+r} = (\underbrace{g^n}_{=1})^a \cdot g^r = g^r.
\end{equation}
Dies impliziert $\langle g \rangle = \{1,g,g^2,\dots, g^{n-1}\}$, da $r$ der Rest ist, der bei der Division von $n$ durch $m$ bleibt. Die möglichen Reste für gegebenes $n$ legen also die Elemente von $G$ fest.\\
Wir müssen noch zeigen, dass $1, g, \dots, g^{n-1}$ paarweise verschieden sind. Dies folgt allerdings direkt aus der Tatsache, dass $n$ minimal ist.
\item Das obige Argument zeigt per Kontraposition auch 2., denn wenn die Potenzen $g^i$, $i \in \Z$ nicht paarweise verschieden sind, dann zeigt obiges Argument, dass $\langle g \rangle = \{1,g,\dots, g^{n-1}\}$ für $n \in \N$, was ein Widerspruch zur Annahme $\ord (g) = \infty$ ist.
\end{enumerate}
\end{beweis}
\begin{definition}{zyklische Gruppe}{zyklisch}
Sei $G$ eine Gruppe. Existiert ein $g\in G$, sodass sich jedes $h \in G$ als $g^n = h$ für ein $n \in \Z$ schreiben lässt, heißt $G$ \textbf{zyklisch der Ordnung} $\ord (g)$. Das Element $g$ heißt \textbf{Erzeuger von} $G$.
\end{definition}
\subsection{Homomorphismen}
\label{subsec:homomorphismen}
\begin{definition}{Homomorphismus}{homomorphismus}
Seien $G$ und $G'$ Gruppen. Eine Abbildung 
\begin{equation}
\phi: G \to G'
\end{equation}
heißt \textbf{(Gruppen-)Homomorphismus}, falls gilt:
\begin{enumerate}[({H}1)]
\item Für alle $g,h \in G$ gilt 
\begin{equation}
\phi(gh) = \phi(g) \cdot \phi(h).
\end{equation}
Die Menge der Homomorphismen von $G$ nach $G'$ wird mit $\Hom (G, G')$ bezeichnet.
\end{enumerate}
\end{definition}
\begin{bemerkung}
Jeder Homomorphismus erfüllt außerdem folgende Eigenschaften, die aus Definition \ref{homomorphismus} folgen:
\begin{enumerate}[({H}1)]
\setcounter{enumi}{1}
\item $\phi (1_G) = 1_{G'}$
\item Für alle $g \in G$ gilt $\phi (g^{-1}) = \phi (g)^{-1}$.
\end{enumerate}
Das sieht man schnell, da $\phi(1) = \phi(1g) = \phi(1) \phi(g)$ gilt, also $\phi(1) = 1'$ sein muss. Weiterhin gilt $1' = \phi(1) = \phi(gg^{-1}) = \phi(g) \phi(g^{-1})$, Linksmultiplikation mit $\phi^{-1}(g)$ liefert (H3).
\end{bemerkung}
\begin{beispiele}
\begin{enumerate}
\item Die \textbf{Einbettung} $\phi: H \hookrightarrow G$ einer Untergruppe $H \leq G$ ist ein Homomorphismus.
\item Die \textbf{Determinantenabbildung}
\begin{equation}
\det: \text{GL} (n, \K) \to (\K \exc \{0\}, \cdot)
\end{equation}
ist ein Homomorphismus.
\item Für $n \geq 1$ und einen Körper $\K$ ist die Permutationsabbildung
\begin{align}
P: \mathfrak{S}_n &\to \text{GL}(n, \K) \\
\sigma &\mapsto P_\sigma,
\end{align}
mit der \textbf{Permutation}
\begin{equation}
(P_\sigma)_{ij} := \begin{cases} 1 \, \text{falls} \, i = \sigma(j)\\0 \, \text{sonst} \end{cases}
\end{equation}
ein Homomorphismus. \textit{Der Beweis sei dem Leser überlassen.}
Für $\sigma = (123) \in \mathfrak{S}_3$ gilt z.B. 
\begin{equation}
P_\sigma = \mat{0,0,1}{1,0,0}{0,1,0}.
\end{equation}
\item Sei $G$ eine Gruppe und $g \in G$. Dann ist 
\begin{align}
\gamma_g: G &\to G \\
h &\mapsto ghg^{-1}
\end{align}
ein Homomorphismus, genannt \textbf{Konjugation mit} $g$.
\item Sei $G$ eine Gruppe und $g \in G$. Dann ist 
\begin{align}
\Z &\to G \\
i &\mapsto g^i
\end{align}
ein Homomorphismus von $(\Z, +)$ nach $(G, \circ)$.
\end{enumerate}
\end{beispiele}
\begin{definition}{Isomorphismus}{isomorphismus}
Sei $\phi$ ein Gruppenhomomorphismus, der zusätzlich bijektiv ist. Dann heißt $\phi$ \textbf{Isomorphismus}. Zwei Gruppen $G$ und $G'$ heißen \textbf{isomorph}, in Zeichen $G \cong G'$, falls es einen Isomorphismus zwischen ihnen gibt.
\end{definition}
\begin{bemerkung}
Anschaulich bedeutet das, dass zwei isomorphe Gruppen identisch bis auf Umbenennung ihrer Elemente sind.
\end{bemerkung}
\begin{beispiele}
\begin{enumerate}
\item Die Permutationsabbildung $P$ induziert einen Isomorphismus
\begin{align}
P: \mathfrak{S}_n &\to P(n, \K)\\
\sigma &\mapsto P_\sigma
\end{align}
zwischen der symmetrischen Gruppe und der Untergruppe der Permuationsmatrizen. Letztere sind Matrizen, die in jeder Zeile und Spalte \textit{genau eine} $1$ und sonst $0$ haben. \textit{Der Beweis sei dem Leser überlassen.}
\item Die \textbf{Exponentialfunktion}
\begin{equation}
\exp: (\R, +) \to (\R_{> 0}, \cdot)
\end{equation}
und ihre Umkehrfunktion, gegeben durch den \textbf{Logarithmus}
\begin{equation}
\ln: (\R_{>0}, \cdot) \to (\R, +),
\end{equation}
bilden einen Isomorphismus, also gilt $(\R, +) \cong (\R_{>0}, \cdot)$.
\end{enumerate}
\end{beispiele}
\begin{definition}{Bild und Kern}{bildkern}
Sei $\phi: G \to G'$ ein Gruppenhomomorphismus. Dann heißt die Teilmenge
\begin{equation}
\im (\phi) := \{g' \in G' | \exists g \in G: \phi(g) = g'\} \leq G',
\end{equation}
das \textbf{Bild von} $\phi$ und die Teilmenge
\begin{equation}
\ker (\phi) := \{g \in G|\phi(g) = 1_{G'} \} \leq G,
\end{equation}
der \textbf{Kern von} $\phi$.
\end{definition}
\begin{satz}{Bild und Kern sind Untergruppen}{bildkernuntergruppe}
Sei $\phi: G \to G'$ ein Gruppenhomomorphismus. Dann sind $\im (\phi) \leq G'$ und $\ker (\phi) \leq G$ Untergruppen der jeweiligen Gruppen $G$ und $G'$.
\end{satz}
\begin{beweis}
Nachrechnen mittels (H1), (H2) und (H3), exemplarisch für den Kern gezeigt:
\begin{enumerate}
\item (U1) ist erfüllt, da $1_G \in \ker (\phi)$ wegen (H2) gilt.
\item (U2) kann nachgerechnet werden. Seien dafür $g,h \in \ker (\phi)$:
\begin{equation}
\phi(gh^{-1}) =^{\text{(H1)}} \phi(g) \cdot \phi(h^{-1}) =^{\text{(H3)}} \phi(g) \cdot \phi(h)^{-1} = 1_{G'},
\end{equation}
also $gh^{-1} \in \ker (\phi)$.
\end{enumerate}
\end{beweis}
\begin{satz}{}{}
Für einen Homomorphismus $\phi: G \to G'$ sind folgende Aussagen äquivalent:
\begin{enumerate}[(i)]
\item $\phi$ ist injektiv.
\item $\ker (\phi) = \{1\}$
\end{enumerate}
\end{satz}
\begin{beweis}
(i) $\implies$ (ii) ist offensichtlich. Wir zeigen noch (ii) $\implies$ (i):
Sei also $\ker (\phi) = \{1\}$ und $g,h \in G$ mit $\phi(g) = \phi(h)$. Dann gilt $\phi(gh^{-1})=\phi(g)\phi(h)^{-1} =1$, also ist $gh^{-1} \in \ker (\phi) = \{1\}$ und damit $g = h$.
\end{beweis}
\begin{definition}{Links- und Rechtsnebenklassen}{linksrechtsnebenklasse}
Sei $G$ eine Gruppe und $H \leq G$ eine Untergruppen. Dann ist die \textbf{Linksnebenklasse von} $H$ \textbf{bezüglich} $g \in G$ als 
\begin{equation}
gH := \{gh \, | \, h \in H\}
\end{equation}
und die \textbf{Rechtsnebenklasse von} $H$ \textbf{bezüglich} $g \in G$ als 
\begin{equation}
Hg := \{hg \, | \, h \in H \}
\end{equation}
definiert.
\end{definition}
\begin{satz}{Nebenklassen sind Äquivalenzklassen}{nebenklassenäquivalenzrel}
Sei $G$ eine Gruppe und $H \leq G$ eine Untergruppe. Dann gilt:
\begin{enumerate}
\item Die Linksnebenklassen sind die Äquivalenzklassen bezüglich der Äquivalenzrelation
\begin{equation}
a \sim_L b :\iff b^{-1}a \in H
\end{equation}
auf $G$.
\item Die Rechtsnebenklassen sind die Äquivalenzklassen bezüglich der analogen Äquivalenzrelation
\begin{equation}
a \sim_R b : \iff ab^{-1} \in H.
\end{equation}
\end{enumerate}
\end{satz}
\begin{übung}
Beweis des Satzes.
\end{übung}
\begin{lösung}
Zunächst ist zu zeigen, dass tatsächlich eine Äquivalenzrelation definiert wird.\\
\begin{enumerate}[(a)]
\item Reflexivität: Sei $a \in G$. Dann gilt $a^{-1}a = 1 \in H$, also ist $a \sim_L a$.
\item Symmetrie: Seien $a,b \in G$ mit $a \sim_L b$. Dann gilt $a^{-1} b = h$ für ein $h \in H$. Daraus folgt:
\begin{equation}
a = bh \iff ah^{-1} =b \iff a^{-1}b = h^{-1} \in H,
\end{equation}
also ist auch $b \sim_L a$, da $H$ abgeschlossen unter Inversenbildung ist.
\item Transitivität: Seien $a,b,c \in G$ mit $a \sim_L b$ und $b \sim_L c$. Dann gilt $b^{-1}a = h \in H$ und $c^{-1} b = h' \in H$. Also folgt $H \ni h'h = c^{-1}bb^{-1}a = c^{-1} a$ und damit die Behauptung.
\end{enumerate}
Ist nun $g \in G$ und $h \in H$, so besteht die Äquivalenzklasse von $g$ unter $\sim_L$ aus allen Elementen der Form $ah$ mit $a \in G$, $h \in H$. Die Vereinigung aller Äquivalenzklassen muss also per Konstruktion ganz $gH$ sein. Der Beweis für $\sim_R$ ist dual dazu. \qed
\end{lösung}

Damit bezeichnen wir die Menge der Linksnebenklassen von $H$ mit $\quotient{G}{H} = \quotient{G}{\sim_L}$ und die der Rechtsnebenklassen mit $\invquotient{G}{H} = \quotient{G}{\sim_R}$.
\begin{definition}{Index}{index}
Die Kardinalität
\begin{equation}
(G : H) := \left| \quotient{G}{H} \right| \in \N \cup \{\infty\}
\end{equation}
heißt \textbf{Index von} $H$ \textbf{in} $G$.
\end{definition}
Man beachte, dass $\left| \quotient{H}{G} \right| = \left| \invquotient{H}{G} \right|$ gilt, da die Abbildung \textcolor{red}{Tafel nicht hochgeschoben...}.
\begin{theorem}{Satz von Lagrange}{satzvonlagrange}
Sei $G$ eine Gruppe und $H \leq G$ eine Untergruppe. Dann gilt 
\begin{equation}
|G| = (G : H) \cdot |H|.
\end{equation}
Ist $|G| < \infty$, so gilt insbesonders
\begin{equation}
(G : H) = \frac{|G|}{|H|} = \left| \quotient{G}{H} \right|.
\end{equation}
\end{theorem}
\begin{beweis}
Dies ist ein direktes Korollar von Satz \ref{nebenklassenäquivalenzrel}: Als Äquivalenzklassen bzgl. einer Äquivalenzrelation bilden die Linksklassen eine Partition von $G$, also 
\begin{equation}
G = \bigsqcup_{gH \in \quotient{G}{H}} gH.
\end{equation}
Es gilt zudem für alle $g \in G$, dass $|gH| = |H|$, da Linksmultiplikation mit $g$, definiert durch 
\begin{align}
G &\to G\\
x &\mapsto gx,
\end{align}
bijektiv ist, also eine Bijektion $H \to gH$ induziert. Insbesondere gilt für jedes $g \in G$, dass $\ord (g) \mid |G|$.
\end{beweis}
\begin{definition}{Normalteiler}{normalteiler}
Eine Untergruppe $N \leq G$ heißt \textbf{normal} oder \textbf{Normalteiler}, falls für alle $g \in G$ 
\begin{equation}
gN = Ng
\end{equation}
gilt. Wir schreiben dafür $N \trianglelefteq G$.
\end{definition}
\begin{bemerkung}
Eine Untergruppe $N \leq G$ ist normal genau dann, wenn für alle $g \in G$ und $n \in N$ gilt:
\begin{equation}
gng^{-1} \in N,
\end{equation}
also $N$ abgeschlossen unter Konjugation mit beliebigen Elementen aus $G$ ist.
\end{bemerkung}
\begin{satz}{}{}
Sei $\phi: G \to G'$ ein Gruppenhomomorphismus. Dann gilt $\ker (\phi) \trianglelefteq G$.
\end{satz}
\begin{beweis}
Sei $g \in G$ und $x \in \ker (\phi)$, also $\phi (x) = 1$. Dann gilt auch 
\begin{equation}
\phi (gxg^{-1}) = \phi(g)\underbrace{\phi(x)}_{=1}\phi(g^{-1}) = \phi(g)\phi(g)^{-1} = 1.
\end{equation}
\end{beweis}
\begin{beispiele}
Wir betrachteten einige Beispiele für Normalteiler:
\begin{enumerate}
\item Sei $n \geq 1$ und $\K$ ein Körper. Für 
\begin{equation}
\det: \text{GL}(n,\K) \to \K^\ast
\end{equation}
gilt
\begin{equation}
\ker (\det) = \text{SL}(n, \K) \trianglelefteq \text{GL}(n,\K).
\end{equation}
\item Betrachte für $n \geq 1$ die Komposition
\begin{center}
\begin{tikzcd}
    \mathfrak{S}_n \arrow[r,"P"] & P(n, \Q) \arrow[r, "\det"] & \{+1, -1\}.
\end{tikzcd}
\end{center}
Also ist 
\begin{equation}
A_n := \ker (\sgn) \trianglelefteq \mathfrak{S}_n
\end{equation}
normal. $A_n$ heißt \textbf{alternierende Gruppe}.
\end{enumerate}
\end{beispiele}
\begin{satz}{Gruppenstuktur auf Nebenklassen}{nebenklassengruppen}
Sei $G$ eine Gruppe und $N \trianglelefteq G$. Dann gilt:
\begin{enumerate}
\item Auf der Menge $\quotient{G}{N}$ von Nebenklassen von $N$ existiert eine Gruppenstruktur mit Verknüpfung
\begin{align}
\quotient{G}{N} \times \quotient{G}{N} &\to \quotient{G}{N} \\
(aN, bN) &\mapsto abN.
\end{align}
\item Die Quotientenabbildung
\begin{align}
\pi: G &\to \quotient{G}{N}\\
a &\mapsto aN
\end{align}
ist ein Gruppenhomomorphismus mit $\ker (\pi) = N$.
\end{enumerate}
\end{satz}
\begin{beweis}
\begin{enumerate}
\item Zunächst muss die Wohldefiniertheit der Verknüpfung bewiesen werden. Seien $\tila \in aN$ und $\tilb \in bN$ Vertreter der Nebenklassen $aN$ und $bN$ $(\iff \tila N = aN)$. Dann existieren $m,n \in N$ mit $\tila = am$ und $\tilb = bn$. Nun gilt
\begin{equation}
\tila \cdot \tilb = am \circ bn = ab \circ \underbrace{b^{-1} mb}_{N \trianglelefteq G \implies \in N} \circ n \in N,
\end{equation}
also ist der Ausdruck wohldefiniert.
\begin{enumerate}[({G}1)]
\item Seien $aN, bN, cN \in \quotient{G}{N}$. Dann gilt
\begin{equation}
(aN \cdot bN) \cdot cN =^{\text{(G1) für G}} (ab)cN = a(bc)N = aN(bN \cdot cN).
\end{equation}
\item Neutrales Element: $1 \cdot N = N$
\item Inverses Element: $(aN)^{-1} = a^{-1} N$
\end{enumerate}
\item Es gilt 
\begin{equation}
\pi (ab)  = (ab)N = (aN)(bN)=\pi(a)\pi(b)
\end{equation}
nach Definition von $\pi$, also ist $\pi$ ein Homomorphismus. Darüber hinaus gilt
\begin{equation}
a \in \ker (\pi) \iff \pi(a) = 1_{\quotient{G}{H}} = N \iff aN = N \iff a \in N,
\end{equation}
also gilt $\ker (\pi) = N$.
\end{enumerate}
\end{beweis}
\begin{satz}{Homomorphiesatz (erster Isomorphiesatz)}{homomorphiesatz}
Sei $\phi: G \to G'$ ein Gruppenhomomorphismus. Dann induziert $\phi$ einen Isomorphismus
\begin{align}
\overline{\phi} : \quotient{G}{\ker (\phi)} &\to \im (\phi)\\
g \ker (\phi) &\mapsto \phi(g).
\end{align}
\end{satz}
\begin{beweis}
Zunächst ist Wohldefiniertheit zu zeigen. Für $\tilg \in gN$, also $\tilg = gn$ für $n \in \ker (\phi)$, gilt:
\begin{equation}
\phi( \tilg) = \phi(gn) = \phi(g)\underbrace{\phi(n)}_{=1} = \phi(g),
\end{equation}
also ist die Abbildung wohldefiniert.\\
Die Surjektivität von $\overline{\phi}$ ist trivial. Wir wissen, dass $\overline{\phi}$ genau dann injektiv ist, wenn $\ker (\overline{\phi}) = \{1_{\quotient{G}{\ker (\phi) }}\} = \ker (\phi)$. Wir rechnen nach:
\begin{equation}
g \ker (\phi) \in \ker (\overline{\phi}) \iff \phi(g) = 1_{G'} \iff g \in \ker (\phi) \iff g \ker (\phi) = \ker (\phi)
\end{equation}
\end{beweis}
\begin{beispiel}
Wir können die Vorzeichenfunktion 
\begin{equation}
\sgn: \mathfrak{S}_n \to \{\pm 1\}
\end{equation}
betrachten, dann ist $\ker{\sgn} = A_n \trianglelefteq \Sf_n$, also erhalten wir einen Isomorphismus 
\begin{equation}
\quotient{\Sf_n}{A_n} \to \{ \pm 1\}.
\end{equation}
Insbesondere gilt $\Sf_n : A_n = 2$.
\end{beispiel}
\begin{korollar}{Korollar aus Satz \ref{homomorphiesatz}}{korollarhomosatz}
Sei $\phi: G \to G'$ ein Gruppenhomomorphismus. Dann lässt sich $\phi$ schreiben als
\begin{equation}
\phi = \iota \circ \overline{\phi} \circ \pi,
\end{equation}
wobei:
\begin{enumerate}
\item $\pi:G \to \quotient{G}{\ker (\phi)}$ der surjektive Quotientenkern ist.
\item $\overline{\phi}: \quotient{G}{\ker (\phi)} \to \im (\phi)$ der Isomorphismus aus \ref{homomorphiesatz} ist.
\item $\iota: \im (\phi) \hookrightarrow G'$ die injektive Einbettung von $\im (\phi) \leq G'$ ist.
\end{enumerate}
Das ist äquivalent dazu, dass folgendes Diagramm kommutiert:
\begin{center}
\begin{tikzcd}
    G \arrow[r,"\phi"] \arrow[d, twoheadrightarrow, "\pi"] & G' \\
    \quotient{G}{\ker (\phi)} \arrow[r, "\cong", "\overline{\phi}"'] & \im (\phi) \arrow[u, hook, "\iota"]
\end{tikzcd}
\end{center}
Ausgedrückt in Elementen:
\begin{center}
\begin{tikzcd}
    g \arrow[r, mapsto] \arrow[d, mapsto] & \phi(g) \\
    g \ker (\phi) \arrow[r, mapsto] & \phi (g) \arrow[u, mapsto]
\end{tikzcd}
\end{center}
\end{korollar}
\begin{beispiele}
\begin{enumerate}
\item Für $n \geq 1$ und einen Körper $\K$ induziert der Homomorphismus
\begin{equation}
\det: \text{GL}(n,\K) \to \K^\ast
\end{equation}
einen Isomorphismus
\begin{equation}
\quotient{\text{GL}(n,\K)}{\text{SL}(n, \K)} \to \K^\ast.
\end{equation}
\item Ein weiterer induzierter Isomorphismus ist
\begin{equation}
\overline{\sgn}: \quotient{\Sf_n}{A_n} \to \{\pm 1\}.
\end{equation}
\item Sei $G$ eine Gruppe mit $g \in G$. Betrachte den Homomorphismus
\begin{equation}
\phi: (\Z, +) \to G, i \mapsto g^i.
\end{equation}
\begin{enumerate}[(a)]
\item Falls $\ord (g) = \infty$, gilt $\ker (\phi) = \{0\}$ und $\phi$ induziert einen Isomorphismus
\begin{center}
\begin{tikzcd}
    \Z \arrow[r,"\pi", "\cong"'] \arrow[rr, "\phi", bend right] & \quotient{\Z}{\{0\}} \arrow[r, "\overline{\phi}", "\cong"'] & \langle g \rangle 
\end{tikzcd}
\end{center}
\item Falls $\ord (g) =N < \infty$, dann gilt
\begin{equation}
\ker (\phi) = N \cdot \Z
\end{equation}
und $\phi$ induziert einen Isomorphismus
\begin{center}
\begin{tikzcd}
\overline{\phi}: \quotient{\Z}{N\Z} \arrow[r, "\cong"]& \langle g \rangle.
\end{tikzcd}
\end{center}
\end{enumerate}
\end{enumerate}
\end{beispiele}
\subsection{Gruppenwirkung}
\label{subsec:wirkung}
\begin{definition}{Gruppenoperation}{operation}
Eine \textbf{Operation} oder \textbf{Wirkung} einer Gruppe $G$ auf einer Menge $M$ ist eine Abbildung
\begin{align}
G \times M &\to M \\
(g, x) &\mapsto g . x,
\end{align}
sodass gilt:
\begin{enumerate}[({O}1)]
\item Für alle $g,h \in G$ und $x \in M$ gilt: $g.(h.x) = (g \cdot h).x$.
\item Für alle $x \in M$ gilt: $1.x=x$.
\end{enumerate}
Dann sagen wir, dass $G$ auf $M$ \textbf{operiert} und schreiben $G \acts M$.
\end{definition}
\begin{beispiele}
\begin{enumerate}
\item Jede Gruppe $G$ operiert auf sich selbst via
\begin{enumerate}[(a)]
\item \textbf{Linkstranslation}: $G \times G \to G$, $(g,h) \mapsto gh$ und
\item \textbf{Rechtstranslation}: $G \times G \to G$, $(g,h) \mapsto hg^{-1}$, aber auch durch
\item \textbf{Konjugation}: $G \times G \to G$, $(g,h) \mapsto ghg^{-1}$.
\end{enumerate}
\item Für jede Menge $M$ operiert die symmetrische Gruppe $\Sf_M$ auf $M$ via
\begin{equation}
\begin{split}
\Sf_M \times M &\to M \\
(\sigma, x) &\mapsto \sigma(x).
\end{split}
\end{equation}
\item Für $n\geq 1$ und einen Körper $\K$ operiert die Gruppe $\text{GL}(n,\K)$ auf $\K^n$ via 
\begin{equation}
\begin{split}
\text{GL}(n,\K) \times \K^n &\to \K^n\\
(A,v) &\mapsto Av.
\end{split}
\end{equation}
\end{enumerate}
\end{beispiele}
\begin{definition}{Äquivarianz}{aequivarianz}
Für Operationen $G \acts M$ und $G \acts N$ heißt eine Abbildung (von Mengen) $f: M \to N$ $G$-\textbf{äquivariant}, falls für alle $g \in G$ und $x \in M$ gilt: 
\begin{equation}
f(g.x) = g.f(x).
\end{equation}
\end{definition}
\begin{definition}{Bahnen}{bahnen}
Sei $G \acts M$ eine Operation von $G$ auf $M$. Die Relation
\begin{equation}
x \sim_G g : \iff \exists g \in G: g.x = g
\end{equation}
definiert eine Äquivalenzrelation auf $M$. Die Äquivalenzklassen sind die Mengen der Form
\begin{equation}
G.x := \{g.x | g \in G\}
\end{equation}
für $x \in M$, die \textbf{Bahnen} von $x$ unter $G \acts M$ genannt werden. Die Quotientenmenge 
\begin{equation}
\invquotient{M}{G} := \quotient{M}{\sim_G}
\end{equation}
heißt \textbf{Bahnenraum von} $G \acts M$.
\end{definition}
\begin{beweis}
\textit{Das Nachweisen der Relationseigenschaften der Äquivalenzrelation ist dem Leser überlassen.}
\end{beweis}
\begin{beispiel}
Betrachte die Rotationsgruppe
\begin{equation}
G = \text{SO}(2, \R) := \text{SL}(2,\R) \cap O(2, \R) = \left\{ \left. \mat{\cos \phi, - \sin \phi}{\sin \phi, \cos \phi} \, \right| \, \phi \in \quotient{\R}{2\pi \Z}   \right\} \leq \text{GL}(2,\R).
\end{equation}
Wir erhalten Operationen 
\begin{equation}
\begin{split}
\text{SO}(2,\R) \times \R^2 &\to \R^2\\
(A,v) &\mapsto Av,
\end{split}
\end{equation}
deren Bahnen konzentrische Kreise im $\R^2$ sind. Dadurch wird eine Partition von $\R^2$ erreicht.
\end{beispiel}
\begin{definition}{Stabilisator, Fixpunkte und Transitivität}{stabilisator}
Sei $G \acts M$ eine Operation. 
\begin{enumerate}[(i)]
\item Für $x \in M$ heißt die Untergruppe
\begin{equation}
G_x := \{g \in G | g.x =x\} \leq G
\end{equation}
der \textbf{Stabilisator von} $x$.
\item Ein Punkt $x \in M$ heißt \textbf{Fixpunkt von} $G \acts M$, falls $G_x = G$. Die Menge aller Fixpunkte wird mit 
\begin{equation}
M^G \sub M
\end{equation}
bezeichnet.
\item Die Operation $G \acts M$ heißt \textbf{transitiv}, falls für jedes $x \in M$ gilt, dass $G.x = M$ ist, also genau eine Bahn existiert.
\end{enumerate}
\end{definition}
\begin{beispiel}
Bleiben wir bei vorigem Beispiel, so hat ein Vektor $v \neq (0,0)$ nur die Identität $\id$ als Stabilisator. Der Nullvektor wird hingegen von ganz $\text{SO}(2,\R)$ stabilisiert. Es scheint einen Zusammenhang zwischen der Größe des Stabilisators und der Bahn zu geben.
\end{beispiel}
\begin{satz}{Bahnformel}{bahnformel}
Sei $G\acts M$ eine Operation auf $M$ und $x \in M$. Dann definiert 
\begin{equation}
\begin{split}
\quotient{G}{G_x} &\to G.x\\
gG_x &\mapsto  g.x
\end{split}
\end{equation}
eine bijektive, $G$-äquivariante Abbildung, wobei $G \acts G.x$ durch Einschränkung von $G\acts M$ gegeben ist. Insbesondere gilt die \textbf{Bahnformel}
\begin{equation}
|G.x| = (G : G_x).
\end{equation}
Eine Wirkung $G \acts \quotient{G}{G_x} = \{gG_x|g \in G\}$ erhält man durch $g' . gG_x:= g'g.G_x$.
\end{satz}
\begin{beweis}
Die Abbildung ist wohldefiniert: Sei $g \in G$ und $h \in G_x$, dann gilt
\begin{equation}
(gh).x = g.(h.x) = g.x.
\end{equation}
Weiterhin ist die Abbilung injektiv, denn falls $g_1.x = g_2.x$, so ist $(g_1^{-1}).g_2.x = x$, also ist $(g_1^{-1}).g_2 \in G_x$, also $g_1G_x = g_2G_x$. Surjektivität ist per Konstruktion durch Einschränkung auf die Bahn gegeben.
\end{beweis}

\end{document}
