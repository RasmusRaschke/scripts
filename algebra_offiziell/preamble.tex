%-----------Packages--------------------------------------------------

\usepackage[T1]{fontenc}
\usepackage[utf8]{inputenc}

%\geometry{showframe}% for debugging purposes -- displays the margins
\usepackage[margin=1in]{geometry}

% AMS packages
\usepackage{amsmath,amssymb,amsthm, amsfonts}

% Improved dots
\usepackage{mathdots}

% Factor 
\usepackage{faktor}

% relsize
\usepackage{relsize}

% Hyperlinks
\usepackage{xcolor}
\usepackage{hyperref}
\definecolor{winered}{rgb}{0.5,0,0}
\hypersetup{
    pdfborder = {0 0 0}, 
    colorlinks=true,
    linkcolor=winered,
    urlcolor={winered},
    filecolor={winered},
}

% Flexible bibliography
\usepackage{natbib}
\bibliographystyle{plain}
\renewcommand\bibname{Literturverzeichnis}

% Tikz
\usepackage{tikz,tikz-3dplot,tikz-cd,tkz-tab,tkz-euclide,pgf,pgfplots}

% enumerate improvements
\usepackage{enumitem}

% Language and date
\usepackage[german]{babel}
\usepackage{datetime}

% The following package makes prettier tables.  
\usepackage{booktabs}

% The units package provides nice, non-stacked fractions and better spacing
% for units.
\usepackage{units}

% The fancyvrb package lets us customize the formatting of verbatim
% environments.  We use a slightly smaller font.
\usepackage{fancyvrb}
\fvset{fontsize=\normalsize}

% Small sections of multiple columns
\usepackage{multicol}

% Euscript font
\usepackage{euscript}

% diff ops
\usepackage[thinc]{esdiff}

%-----------Environments----------------------------------------------

% Thm Environments
\numberwithin{equation}{section}
\newtheorem{thm}[equation]{Satz}
\newtheorem{lem}[equation]{Lemma}
\newtheorem{prop}[equation]{Proposition}
\newtheorem{cor}[equation]{Korollar}

% Def Environments
\theoremstyle{definition}
\newtheorem{exa}[equation]{Beispiel}
\newtheorem{exas}[equation]{Beispiele}
\newtheorem{rem}[equation]{Bemerkung}
\newtheorem{defi}[equation]{Definition}
\newtheorem{prob}[equation]{Aufgabe}
\newtheorem{term}[equation]{Terminologie}
\newtheorem{konv}[equation]{Konvention}


%------------Macros---------------------------------------------------

% matrices

\def\mat#1{\begin{pmatrix} #1 \end{pmatrix}}
\def\sk#1{\langle #1 \rangle}

% faktors

\makeatletter
\DeclareRobustCommand*{\mfaktor}[3][]
{
   { \mathpalette{\mfaktor@impl@}{{#1}{#2}{#3}} }
}
\newcommand*{\mfaktor@impl@}[2]{\mfaktor@impl#1#2}
\newcommand*{\mfaktor@impl}[4]{
   \settoheight{\faktor@zaehlerhoehe}{\ensuremath{#1#2{#3}}}%
   \settoheight{\faktor@nennerhoehe}{\ensuremath{#1#2{#4}}}%
      \raisebox{-0.5\faktor@zaehlerhoehe}{\ensuremath{#1#2{#3}}}%
      \mkern-4mu\diagdown\mkern-5mu%
      \raisebox{0.5\faktor@nennerhoehe}{\ensuremath{#1#2{#4}}}%
}
\makeatother

% Operators
\newcommand{\no}{\trianglelefteq}
\newcommand{\equ}{\Leftrightarrow}
\def\lra{\longrightarrow}
\def\hra{\hookrightarrow}
\def\llra{\longleftrightarrow}

\def\on{\operatorname}

\newcommand{\Kern}{\on{Kern}}
\newcommand{\Ann}{\on{Ann}}
\newcommand{\Bild}{\on{Bild}}
\newcommand{\Konv}{\on{Konv}}

\newcommand{\charac}{\on{char}}
\newcommand{\grad}{\on{grad}}
\newcommand{\Hom}{\on{Hom}}
\newcommand{\Abb}{\on{Abb}}
\newcommand{\End}{\on{End}}
\newcommand{\Aut}{\on{Aut}}
\newcommand{\GL}{\on{GL}}
\newcommand{\SL}{\on{SL}}
\newcommand{\SO}{\on{SO}}
\newcommand{\Sym}{\on{Sym}}
\renewcommand{\O}{\on{O}}
\newcommand{\Z}{\on{Z}}
\newcommand{\C}{\on{C}}
\newcommand{\N}{\on{N}}

\newcommand{\eU}{\EuScript{U}}
\newcommand{\eZ}{\EuScript{Z}}
\newcommand{\eE}{\EuScript{E}}

\newcommand{\id}{\on{id}}

\newcommand{\ggT}{\on{ggT}}
\newcommand{\LM}{\on{LM}}
\newcommand{\ord}{\on{ord}}
\renewcommand{\dim}{\on{dim}}
\newcommand{\sign}{\on{sign}}

% Define an "actson" symbol
\newcommand{\rotate}[1]{\rotatebox[origin=c]{270}{\ensuremath{#1}}}
\newcommand{\rotater}[1]{\rotatebox[origin=c]{90}{\ensuremath{#1}}}
\newcommand{\actson}{\ \rotate{\mathlarger{\mathlarger{\circlearrowleft}}}\ }
\newcommand{\actsonright}{\ \rotater{\mathlarger{\mathlarger{\circlearrowleft}}}\ }

% Standard BBs/cals
\newcommand{\RR}{\mathbb{R}}
\newcommand{\ZZ}{\mathbb{Z}}
\newcommand{\CC}{\mathbb{C}}
\newcommand{\FF}{\mathbb{F}}
\newcommand{\HH}{\mathbb{H}}
\newcommand{\QQ}{\mathbb{Q}}
\def\NN{\mathbb{N}}

% Swap varphi and phi
\let\temp\phi
\let\phi\varphi
\let\varphi\temp

\let\subset\subseteq

% bibliography management
\makeatletter
\renewcommand\bibsection%
{
  \subsection*{\refname
    \@mkboth{\MakeUppercase{\refname}}{\MakeUppercase{\refname}}}
}
\makeatother

% These commands are used to pretty-print LaTeX command

%TOC pretty
\usepackage{tocloft}
\usepackage{titletoc}
\renewcommand{\cftchapfont}{\scshape\bfseries}
\renewcommand{\cftpartfont}{\scshape\Large}

\newlength\mylenprt

\renewcommand{\cftpartpresnum}{\partname\hspace{10pt}}

\setlength{\cftpartindent}{0cm}
\setlength{\cftchapindent}{0cm}




% Lists

\setlist[enumerate,1]{label=(\arabic{*})}
\setlist[enumerate,2]{label=(\alph{*})}
\setlist[enumerate,3]{label=(\roman{*})}

%%% The following additions are from me %%%
\usepackage{makeidx}
\newcommand*{\red}{\textcolor{red}} %for annotations
\newcommand*{\blue}{\textcolor{blue}} %for additions
\newcommand{\Spec}{\on{Spec}} %for ring spectra


