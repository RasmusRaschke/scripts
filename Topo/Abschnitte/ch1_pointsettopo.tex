\section{Mengentheoretische Topologie}
\label{PST}
\subsection{Metrische Räume}
\label{subsec:metrischeraeume}

\begin{definition}{Metrischer Raum}{metrischerraum}
Ein \textbf{metrischer Raum} ist ein Paar $(X, d)$, bestehend aus einer Menge $X$ und einer Abstandsfunktion \begin{equation}
d: X \times X \to \R,
\end{equation}
genannt \textbf{Metrik}, die die folgenden Axiome erfüllt:
\begin{enumerate}[({M}1)]
\item \textit{Positivität}: $\forall x,y \in X: d(x,y)>0$
\item \textit{Symmetrie}: $\forall x,y \in X: d(x,y)=d(y,x)$
\item \textit{Dreiecksungleichung}: $\forall x,y,z \in X: d(x,z) \leq d(x,y) + d(y,z)$
\end{enumerate}
\end{definition}

\begin{beispiele}
\begin{enumerate}
\item Im $\R^n$ ist die \textbf{Standardmetrik} oder \textbf{euklidische Metrik} für $x,y \in \R$ gegeben durch 
\begin{equation}
d_2(x,y) := \sqrt{\sum_{i=1}^n (x_i-y_i)^2}.
\end{equation}
\item Auf $(\R^n, d_n)$ ist eine Metrik durch \begin{equation}
d_n(x,y) := \sum_{i=1}^n |x_i - y_i|
\end{equation}
gegeben.
\item Die \textbf{Maximumsnorm} $(\R^n, d_\infty)$ ist gegeben durch 
\begin{equation}
d_\infty (x,y) = \max_{i \in \{1,\dots, n\}} |x_i - y_i|.
\end{equation}
\item Eine weitere Metrik auf $\R^n$ ist gegeben durch 
\begin{equation}
d_{\sqrt{\cdot}} (x,y) = \sqrt{d_2(x,y)}.
\end{equation}
Diese Metrik kommt nicht von einer Norm.
\item Die \textbf{diskrete Metrik} auf einer beliebigen Menge $X$ ist gegeben durch 
\begin{equation}
d(x,y) := \begin{cases} 0, \, x = y \\ 1, \, x \neq y \end{cases}.
\end{equation}
\item Auf $X= \mathcal{C} ([0,1], \R)$ ist für $f,g \in X$ durch das Integral eine Metrik 
\begin{equation}
d(f,g) := \int_0^1 |f(x) - g(x)| dx
\end{equation}
definiert.
\end{enumerate}
\end{beispiele}
\begin{bemerkungen}
\begin{enumerate}
\item Wenn $(X,d)$ ein metrischer Raum ist, so ist $Y \sub X$ als $(Y,d|_{Y \times Y})$ auch ein metrischer Raum.
\item Wenn $(X_1, d_1)$ und $(X_2, d_2)$ metrische Räume sind, so ist $(X_1 \times X_2, d_1 \times d_2)$ wieder ein metrischer Raum.
\item Vorsicht: Für eine Familie $(X_i, d_i)_{i \in I}$ ist der Sachverhalt komplizierter.
\end{enumerate}
\end{bemerkungen}
\begin{definition}{$\epsilon$-Ball}{ball}
Sei $(X,d)$ ein metrischer Raum, $x \in X$ und $\epsilon > 0$. Dann ist der $\epsilon$\textbf{-Ball} mit $x$ im Zentrum definiert als 
\begin{equation}
B_\epsilon (x) := \{ y \in X \, | \, d(x,y) < \epsilon \}.
\end{equation}
\end{definition}
\begin{definition}{Umgebung}{umgebung}
Sei $(X,d)$ ein metrischer Raum. Eine Menge $U \sub X$ heißt \textbf{Umgebung} von $x \in X$, falls ein $\epsilon > 0$ mit $B_\epsilon (x) \sub U$ existiert.
\end{definition}
\begin{definition}{Offen und Abgeschlossen}{offenabgeschlossen}
Sei $(X,d)$ ein metrischer Raum. Eine Teilmenge $O \sub X$ heißt \textbf{offen}, falls für alle $x \in O$ ein $\epsilon > 0$ existiert, sodass $B_\epsilon (x) \sub O$ gilt. $O$ ist also eine Umgebung all seiner Elemente.\\
Eine Menge $A \sub X$ heißt \textbf{abgeschlossen}, falls $X \exc A$ offen ist.
\end{definition}
\begin{bemerkungen}
\begin{enumerate}
\item Sei $\epsilon > 0$ und $(X,d)$ ein metrischer Raum. Dann ist $B_\epsilon (x) \sub X$ offen und eine Umgebung von $x$.
\item ÜA: Sei $(X,d)$ ein metrischer Raum und $x \in X$. Dann ist $\{x\}$ abgeschlossen.
\end{enumerate}
\end{bemerkungen}
\begin{satz}{Umgebungseigenschaften metrischer Räume}{umgebungseigenschaften}
Sei $(X,d)$ ein metrischer Raum. Dann gilt:
\begin{enumerate}[({U}1)]
\item Jede Umgebung von $x \in X$ enthält $x$ und $X$ ist eine Umgebung von $x$.
\item Ist $U \sub X$ eine Umgebung von $X$ und $U \sub V \sub X$, so ist $V$ auch eine Umgebung von $x$.
\item Wenn $U_1$ und $U_2$ Umgebungen von $x$ sind, so auch $U_1 \cap U_2$.
\item Ist $U \sub X$ eine Umgebung von $x$, so existiert eine weitere Teilmenge $V \sub X$, sodass $U$ eine Umgebung von allen $y \in V$ ist.
\end{enumerate}
\end{satz}
\begin{beweis}
\begin{enumerate}
\item Trivial.
\item Trivial.
\item Nach Voraussetzung existiert für $x \in U_1 \cap U_2$ ein $\epsilon_1 > 0$, sodass $B_{\epsilon_1} (x) \sub U_1$ und ein $\epsilon_2 > 0$, sodass $B_{\epsilon_2} (x) \sub U_2$. Definiere $\epsilon := \min (\epsilon_1, \epsilon_2)$. Dann gilt $B_\epsilon (X) \sub U_1$ und $B_\epsilon (x) \sub U_2$, also $B_\epsilon (x) \sub U_1 \cap U_2$.
\item Nach Voraussetzung existiert ein $\epsilon > 0$ mit $B_\epsilon (x) \sub U$. Dann ist die Behauptung durch $V:=B_\epsilon(x)$ erfüllt.
\end{enumerate}
\end{beweis}
\begin{satz}{Eigenschaften offener Mengen}{eigenschaftoffen}
Sei $(X,d)$ ein metrischer Raum. Dann gilt:
\begin{enumerate}
\item $\emptyset$ und $X$ sind offen.
\item Sind $O_1, O_2 \sub X$ offen, so auch $O_1 \cap O_2$.
\item Ist $(O_i)_{i \in I}$ eine Familie offener Teilmengen $O_i \sub X$, so ist $\cup_i O_i$ auch offen.
\end{enumerate}
\end{satz}
\begin{beweis}
\begin{enumerate}
\item Trivial.
\item Mit $\min (\epsilon_1, \epsilon_2)$ analog zum obigen Beweis.
\item Sei $x \in \cup_i O_i$. Dann existiert ein $i \in I$ mit $x \in O_i$, sodass ein $\epsilon > 0$ existiert mit $B_\epsilon(x) \sub O_i \sub \cup_i O_i$.
\end{enumerate}
\end{beweis}
\begin{satz}{Eigenschaften abgeschlossener Mengen}{eigenschaftenabgeschlossen}
Sei $(X,d)$ ein metrischer Raum. Dann gilt:
\begin{enumerate}[({A}1)]
\item $\emptyset$ und $X$ sind abgeschlossen.
\item Wenn $A_1, A_2 \sub X$ abgeschlossene Teilmengen sind, so ist auch $A_1 \cup A_2$ abgeschlossen.
\item Seien $(A_i)_{i \in I}$ abgeschlossene Teilmengen von $X$. Dann ist $\cup_i A_i$ wieder abgeschlossen.
\end{enumerate}
\end{satz}
\begin{beweis}
\begin{enumerate}
\item Da $\emptyset = X \exc X$ und $X = X \exc \emptyset$ gilt, sind $X$ und $\emptyset$ gemäß Satz \ref{eigenschaftoffen} offen.
\item Sei $A_1 = X \exc O_1$ und $A_2 = X \exc O_2$ mit $O_1, O_2 \sub X$ offen. Gemäß Satz \ref{eigenschaftoffen} (2.) folgt 
\begin{equation}
X \exc (A_1 \cup A_2) = (X \exc A_1) \cap (X \exc A_2) = O_1 \cap O_2,
\end{equation}
wobei $O_1 \cap O_2$ wieder offen ist.
\item Wir betrachten offene Teilmengen $O_i := X \exc A_i$. Gemäß Satz \ref{eigenschaftoffen} ist 
\begin{equation}
X \exc \bigcap_{i\in I} A_i = \bigcup_{i \in I} X \exc A_i = \bigcup_{i \in I} O_i 
\end{equation}
offen.
\end{enumerate}
\end{beweis}
\begin{definition}{stetige Abbildung}{metstetabb}
Seien $(X,d)$ und $(Y,d')$ metrische Räume und $f: X \to Y$. Dann heißt $f$ \textbf{stetig in} $x_0 \in X$, wenn für alle $\epsilon > 0$ ein $\delta > 0$ existiert, sodass 
\begin{equation}
d(x_0,x) < \delta \implies d'(f(x_0), f(x)) < \epsilon
\end{equation}
gilt. $f$ heißt \textbf{stetig}, falls dies für alle $x_0 \in X$ erfüllt ist.
\end{definition}

\begin{satz}{Äquivalente Formulierung der Stetigkeit}{umgebungstetig}
Seien $(X,d)$ und $(Y,d')$ metrische Räume und $f: X \to Y$. Dann sind äquivalent:
\begin{enumerate}
\item $f$ ist stetig.
\item $V$ ist Umgebung von $f(x)$ $\implies$ $f^{-1}(V)$ ist eine Umgebung von $x$.
\item $O \in Y$ ist offen $\implies$ $f^{-1}(O)$ ist offen in $X$.
\item $A \in Y$ ist abgeschlossen $\implies$ $f^{-1}(A)$ ist abgeschlossen in $X$.
\end{enumerate}
\end{satz}
\begin{beweis}
(1. $\implies$ 2.): Sei $V$ eine Umgebung von $f(x)$. Per Definition existiert ein $\epsilon > 0$, sodass $f(x) \in B_\epsilon(f(x)) \sub V$ gilt. Gemäß Annahme existiert ein $\delta > 0$ mit $f(B_\delta(x))\sub B_\epsilon(f(x))$. Daraus folgt, dass 
\begin{equation}
B_\delta (x) \sub f^{-1}f(B_\delta(x)) \sub f^{-1}(B_\epsilon(f(x))) \sub f^{-1}(V).
\end{equation}
Also ist $f^{-1}(V)$ eine Umgebung von $x$.\\
(2. $\implies$ 3.): $O$ ist Umgebung all seiner Elemente.\\
(3. $\implies$ 4.): $Y \exc A$ ist offen in $Y$, d.h. $f^{-1}(Y \exc A) = f^{-1}(Y) \exc f^{-1}(A) = X \exc f^{-1}(A)$ ist offen in $X$. Also ist $f^{-1}(A)$ abgeschlossen.\\
(4. $\implies$ 1.): $Y \exc B_\epsilon(f(x))$ ist abgeschlossen impliziert, dass $f^{-1} (Y \exc B_\epsilon(f(x)))$ auch abgeschlossen ist. Damit folgt, dass $X \exc f^{-1}(B_\epsilon(f(x)))$ abgeschlossen und damit $f^{-1}(B_\epsilon(f(x)))$ offen ist. Für $x \in f^{-1}(B_\epsilon(f(x)))$ existiert ein $\delta > 0$ mit $B_\delta(x) \sub f^{-1}(B_\epsilon(f(x)))$, also auch $f(B_\delta(x)) \sub B_\epsilon(f(x))$.
\end{beweis}

\begin{definition}{Äquivalenz von Metriken}{aequivalentmetrik}
Seien $d_1$ und $d_2$ Metriken auf einer Menge $X$.
\begin{enumerate}
\item Gibt es $\alpha, \beta > 0$ mit
\begin{equation}
\alpha d_1(x,y) \leq d_2(x,y) \leq \beta d_1(x,y)
\end{equation}
für alle $x,y \in X$, so heißen $d_1$ und $d_2$ \textbf{stark äquivalent}.
\item $d_1$ und $d_2$ heißen \textbf{äquivalent}, falls es für jedes $x \in X$ und alle $\epsilon > 0$ ein $\delta > 0$ gibt, sodass
\begin{enumerate}[(i)]
\item $d_1(x,y) < \delta \implies d_2(x,y) < \epsilon$
\item $d_2(x,y) < \delta \implies d_1(x,y) < \epsilon$
\end{enumerate}
gilt.
\end{enumerate}
\end{definition}
\begin{bemerkungen}
\begin{enumerate}
\item Genau dann, wenn $d_1$ und $d_2$ äquivalente Metriken sind, sind $\id_X: (X, d_1) \to (X, d_2)$ und $id_X: (X,d_2) \to (X, d_1)$ stetig.
\item $d_1$ und $d_2$ stark äquivalent impliziert, dass $\id_X$ gleichmäßig stetig ist.
\item Äquivalente Metriken ergeben die gleichen offenen (und abgeschlossenen) Mengen.
\end{enumerate}
\begin{beispiele}
\begin{enumerate}
\item Die $d_1$-, $d_2$- und $d_\infty$-Metrik auf dem $\R^n$ sind stark äquivalent.
\item Sei $d_0(x,y) = |x^3 - y^3|$ und $d_2(x,y)$ die euklidische Metrik. Die Identität $\id_\R (\R, d_2) \to (\R, d_0)$ ist stetig, aber nicht gleichmäßig stetig.
\item Sei $X$ eine beliebige Menge mit einer beliebigen Metrik $d$. Dann ist $d$ äquivalent zu 
\begin{equation}
d'(x,y) := \frac{d(x,y)}{1+d(x,y)} < 1
\end{equation}
für alle $x,y \in X$. Also ist \textit{jede Metrik äquivalent zu einer beschränkten Metrik}.
\end{enumerate}
\end{beispiele}
\end{bemerkungen}
\subsection{Topologische Räume}
\label{subsec:topologischeraeume}
Der Begriff des topologischen Raums wird durch Abstraktion der Eigenschaften offener Mengen und stetiger Abbildungen in metrischen Räumen konstruiert.
\begin{definition}{Topologischer Raum}{topologischerraum}
Ein \textbf{topologischer Raum} ist ein Paar $(X, \Tc)$, bestehend aus einer Menge $X$ und einer Familie $\Tc$ von Teilmengen von $X$, sodass folgende Axiome erfüllt sind:
\begin{enumerate}[({O}1)]
\item $\emptyset, X \in \Tc$
\item $O_1, O_2 \in \Tc \implies O_1 \cap O_2 \in \Tc$
\item Für eine Familie $(O_i)_{i \in I}$ mit $O_i \in \Tc$ für alle $i \in I$ folgt $\cup_{i \in I} O_i \in \Tc$.
\end{enumerate}
$\Tc$ heißt \textbf{Topologie} auf $X$ und alle $O \in \Tc$ heißen \textbf{offene Mengen}.
\end{definition}
\begin{bemerkung}
Äquivalent dazu ist: Eine Topologie $\Tc \sub \Pc(X)$ ist abgeschlossen unter endlichen Schnitten und beliebigen Vereinigungen. 
\end{bemerkung}
\begin{beispiele}
\begin{enumerate}
\item Metrische Räume $(X,d)$ sind auch topologische Räume mit offenen Mengen gegeben durch $d$.
\item Auf einer beliebigen Menge $X$ kann die \textbf{diskrete Topologie} $\Tc := \Pc(X)$ definiert werden, in der alle Teilmengen von $X$ offen sind.
\item Auch kann auf beliebigem $X$ die \textbf{indiskrete Topologie} oder \textbf{Klumpentopologie} durch $\Tc := \{\emptyset, X\}$ definiert werden.
\item Auf beliebigem $X$ existiert die \textbf{koendliche Topologie}: $O\sub X$ ist offen genau dann, wenn $X \exc O$ endlich ist oder $O = \emptyset$ gilt.
\end{enumerate}
\end{beispiele}
\begin{definition}{Topologische Grundbegriffe}{topgrundbegriffe}
Sei $(X, \Tc)$ ein topologischer Raum.
\begin{enumerate}
\item $A \sub X$ heißt \textbf{abgeschlossen}, falls $X \exc A \in \Tc$.
\item Sei $x \in U \sub X$. Dann heißt $U$ \textbf{Umgebung von} $x$, falls ein $O \in \Tc$ existiert, sodass $x \in O \sub U$ gilt.
\item Die Menge aller Umgebungen von $X$ wird mit $\Uf (x)$ bezeichnet und heißt \textbf{Umgebungssystem von} $x$. 
\item Ein Punkt $x \in X$ heißt \textbf{Berührpunkt} von $B \sub X$, falls für alle $U \sub \Uf(x)$ gilt: $U \cap B \neq \emptyset$.
\item Die \textbf{abgeschlossene Hülle} von $B \sub X$ ist definiert als 
\begin{equation}
\overline{B} := \bigcap_{B \sub X, C \, \text{abg.}} C.
\end{equation}
\item Ein Punkt $x \in X$ heißt \textbf{innerer Punkt} von $B \sub X$, falls es ein $U \in \Uf(x)$ gibt, sodass $x \in U \sub B$ gilt.
\item Für $B \sub X$ ist 
\begin{equation}
\mathring{B} := \bigcup_{O \sub B, O \, \text{offen}}
\end{equation}
der \textbf{offene Kern} von $B$.
\item Der \textbf{Rand von} $A \sub X$ ist definiert als
\begin{equation}
\partial A:= \{x \in X|\forall U \in \Uf(x): U \cap A \neq \emptyset \neq U \cap (X \exc A)\}.
\end{equation}
\end{enumerate}
\end{definition}
\begin{satz}{Eigenschaften bestimmter Mengen}{eigenschaftmengen}
Sei $(X, \Tc)$ ein top. Raum. Dann gilt:
\begin{enumerate}
\item Die abgeschlossenen Mengen von $X$ erfüllen (A1)-(A3).
\item Die Umgebungen erfüllen (U1)-(U4).
\end{enumerate}
\end{satz}
\begin{bemerkung}
Eine Topologie kann äquivalent durch die Auflistung abgeschlossener Mengen definiert werden, wenn diese (A1)-(A3) erfüllen.
\end{bemerkung}
\begin{definition}{Stetigkeit in top. Räumen}{stetigtopo}
Seien $(X, \Tc)$ und $(Y, \Tc')$ top. Räume und $f: X \to Y$.
\begin{enumerate}[(i)]
\item $f$ heißt \textbf{stetig in} $x \in X$, wenn für alle $U \in \Uf(f(x))$ auch $f^{-1}(U) \in \Uf(x)$ gilt.
\item $f$ heißt \textbf{stetig}, falls für alle $O \in \Tc'$ gilt, dass $f^{-1} \in \Tc$.
\end{enumerate}
\end{definition}
\begin{satz}{Eigenschaften von Abschluss und Innerem}{abschlussundinneres}
Sei $(X,\Tc)$ ein top. Raum mit $A \in \Tc$. Dann gilt
\begin{enumerate}
\item \begin{enumerate}
\item $\overline{A}$ ist abgeschlossen und $A \sub \overline{A}$.
\item $A = \overline{A}$ gilt genau dann, wenn $A$ abgeschlossen ist.
\item $\overline{A}$ besteht aus der Menge der Berührpunkte von $A$.
\end{enumerate}
\item \begin{enumerate}
\item $\mathring{B}$ ist offen und $\mathring{B} \sub B$.
\item $B = \mathring{B}$ genau dann, wenn $B$ offen ist.
\item $\mathring{B}$ besteht aus der Menge der inneren Punkte von $B$.
\end{enumerate}
\end{enumerate}
\end{satz}
\begin{beweis}
(a) und (b) sind jeweils trivial. Wir beweisen 1(a). Angenommen, $x \in \overline{A}$ aber ist kein Berührpunkt von $A$. Dann existiert ein $U \in \Uf(x)$, sodass $U \cap A = \emptyset$ gilt. Daraus folgt, dass $A \sub X \exc U$ gilt, woraus $A \sub X \exc O$ abgeschlossen folgt. Weiterhin existiert ein $O \in \Tc$ mit $x \in O \sub U$, also $X \exc U \sub X \exc O$. Dann ist aber $\overline{A} \sub X \exc O$, also $x \notin \overline{A}$.\\
Jetzt nehmen an, dass $x$ Berührpunkt ist, aber $x \notin  \overline{A}$. Also folgt aus $x \in X \exc \overline{A}$ offen, dass $V:= X \exc \overline{A} \in \Uf (x)$, aber $V \cap A = \emptyset$, also ist $x$ kein Berührpunkt im Widerspruch zur Annahme.
\end{beweis}
\begin{bemerkung}
Folgendes gilt allgemein:
\begin{itemize}
\item Ist $A \sub B$, so auch $\mathring{A} \sub \mathring{B}$ und $\overline{A} \sub \overline{B}$.
\item $\overline{A \cup B} = \overline{A} \cup \overline{B}$
\item $\mathring{A \cap B} = \mathring{A} \cap \mathring{B}$
\end{itemize}
\end{bemerkung}
\begin{definition}{Dichheit}{dicht}
Sei $(X, \Tc)$ ein top. Raum. $A \sub X$ heißt \textbf{dicht}, falls $\overline{A} = X$. $A\sub X$ heißt hingegen \textbf{nirgends dicht}, falls $\mathring{\overline{A}}= \emptyset$.
\end{definition}
\begin{beispiele}
\begin{enumerate}
\item $\Q \sub \R$ ist dicht.
\item $(a,b) \sub \R \sub \R^2$, $(a,b)$ nirgends dicht in $\R$.
\end{enumerate}
\end{beispiele}