\documentclass[10pt]{extarticle}

\usepackage[german]{babel}
\usepackage{graphicx}
\usepackage{framed}
\usepackage[normalem]{ulem}
\usepackage{indentfirst}
\usepackage{amsmath,amsthm,amssymb,amsfonts}
\usepackage{mathtools} % Wraparound amsmath. For fancy math typesetting
\usepackage[nointegrals]{wasysym} % nointegrals prevents wasysym from overwriting integral symbols from LaTeX and amsmath
\usepackage{bbm} % For extended bold and blackboard bold characters
\usepackage[italicdiff]{physics} % italicdiff causes derivatives to be rendered with italic d's instead of upright d's
\usepackage[T1]{fontenc}
\usepackage{xparse}
\usepackage{xstring}
\usepackage{enumerate}
\usepackage{float}
\usepackage{stmaryrd}
\usepackage{grffile}
\usepackage{cancel}
\usepackage{hyperref}
\hypersetup{
    colorlinks=true,
    linkcolor=blue,
    filecolor=magenta,      
    urlcolor=cyan,
    pdftitle={Overleaf Example},
    pdfpagemode=FullScreen,
    }

\urlstyle{same}
\numberwithin{equation}{subsection}
\usepackage{pifont} % For unusual symbols
\usepackage{mathdots} % For unusual combinations of dots
\usepackage{wrapfig}
\usepackage{lmodern,mathrsfs}
\usepackage[inline,shortlabels]{enumitem}
\setlist{topsep=2pt,itemsep=2pt,parsep=0pt,partopsep=0pt}
\usepackage[table,dvipsnames]{xcolor}
\usepackage[utf8]{inputenc}
\usepackage{csquotes} % Must be loaded AFTER inputenc
\usepackage[a4paper,top=0.5in,bottom=0.2in,left=0.5in,right=0.5in,footskip=0.3in,includefoot]{geometry}
\usepackage[most]{tcolorbox}
\usepackage{tikz,tikz-3dplot,tikz-cd,tkz-tab,tkz-euclide,pgf,pgfplots}
\usetikzlibrary{babel}
\pgfplotsset{compat=newest}
% \usepackage{comment} % For commenting large blocks of text and math efficiently
% \usepackage{fancyvrb} % For custom verbatim environments
\usepackage{multicol}
\usepackage[bottom,multiple]{footmisc} % Ensures footnotes are at the bottom of the page, and separates footnotes by a comma if they are adjacent
\usepackage[backend=bibtex,style=numeric]{biblatex}
\renewcommand*{\finalnamedelim}{\addcomma\addspace} % Forces authors' names to be separated by comma, instead of "and"
\addbibresource{bibliography}
\usepackage[nameinlink]{cleveref} % nameinlink ensures that the entire element is clickable in the pdf, not just the number

\newcommand{\remind}[1]{\textcolor{red}{\textbf{#1}}} % To remind me of unfinished work to fix later
\newcommand{\hide}[1]{} % To hide large blocks of code without using % symbols
\newcommand{\hodge}{{\star}}
% Same as \href, but the text appears in typewriter font and in a custom color
\newcommand{\Href}[3][red!50!black]{\href{#2}{\textcolor{#1}{\texttt{#3}}}}

\newcommand{\ep}{\varepsilon}
\newcommand{\vp}{\varphi}
\newcommand{\lam}{\lambda}
\newcommand{\Lam}{\Lambda}
\DeclareDocumentCommand\ip{ l m }{\braces#1{\langle}{\rangle}{#2}} % Inner product ⟨x,y⟩ (but only one argument is taken, so \ip{x,y} renders as ⟨x,y⟩)
\DeclareDocumentCommand\floor{ l m }{\braces#1{\lfloor}{\rfloor}{#2}} % Floor function ⌊x⌋
\DeclareDocumentCommand\ceil{ l m }{\braces#1{\lceil}{\rceil}{#2}} % Ceiling function ⌈x⌉

% Shortcuts for blackboard bold letters, e.g. \A outputs \mathbb{A}
\def\do#1{\csdef{#1}{\mathbb{#1}}}
\docsvlist{A,B,C,D,E,F,G,I,J,K,M,N,Q,R,T,U,V,W,X,Y,Z}
% \H is already defined as a 1-argument command, it places a double acute accent (hungarumlaut) on a character, e.g. \H{o} yields ő
% \L is already defined as the uppercase Ł (L with stroke)
% \O is already defined as the uppercase Ø (O with stroke)
% \P is already defined as the pilcrow ¶ (paragraph mark)
% \S is already defined as the section sign §

% Shortcuts for calligraphic letters, e.g. \As outputs \mathcal{A}
\def\do#1{\csdef{#1c}{\mathcal{#1}}}
\docsvlist{A,B,C,D,E,F,G,H,I,J,K,L,M,N,O,P,Q,R,S,T,U,V,W,X,Y,Z}

% Shortcuts for letters with a bar on top, e.g. \Abar outputs \overline{A}
\def\do#1{\csdef{#1bar}{\overline{#1}}}
\docsvlist{a,b,c,d,e,f,g,i,j,k,l,m,n,o,p,q,r,s,t,u,v,w,x,y,z,A,B,C,D,E,F,G,H,I,J,K,L,M,N,O,P,Q,R,S,T,U,V,W,X,Y,Z}
% \hbar is already defined as the symbol ℏ (reduced Planck constant)

% Shortcuts for boldface letters, e.g. \Ab outputs \textbf{A}
\def\do#1{\csdef{#1b}{\textbf{#1}}}
\docsvlist{a,b,c,d,e,f,g,h,i,j,k,l,m,n,o,q,r,t,u,w,x,y,z,A,B,C,D,E,F,G,H,I,J,K,L,M,N,O,P,Q,R,S,T,U,V,W,X,Y,Z}
% \pb is already defined (by the physics package) as a 2-argument command, denoting the anticommutator or Poisson bracket, e.g. \pb{A,B} yields {A,B}
% \sb is already defined in the LaTeX kernel. This is a fundamental LaTeX command, DO NOT overwrite it!
% \vb is already defined (by the physics package) as a 1-argument command, for boldface text, e.g. \vb{A} yields \textbf{A}

% Shortcuts for letters with a tilde on top, e.g. \Atil outputs \widetilde{A}
\def\do#1{\csdef{#1til}{\widetilde{#1}}}
\docsvlist{a,b,c,d,e,f,g,h,i,j,k,l,m,n,o,p,q,r,s,t,u,v,w,x,y,z,A,B,C,D,E,F,G,H,I,J,K,L,M,N,O,P,Q,R,S,T,U,V,W,X,Y,Z}

\def\do#1{\csdef{#1f}{\mathfrak{#1}}}
\docsvlist{A,B,C,D,E,F,G,H,I,J,K,L,M,N,O,P,Q,R,S,T,U,V,W,X,Y,Z}

\newcommand{\quotient}[2]{{\raisebox{.0em}{$#1$}/\raisebox{-.2em}{$#2$}}}
\newcommand{\tm}{^{\mathsf{T}}}     % Transpose
\newcommand{\hm}{^{\mathsf{H}}}     % Conjugate transpose (Hermitian conjugate)
\newcommand{\itm}{^{-\mathsf{T}}}   % Inverse transpose
\newcommand{\ihm}{^{-\mathsf{H}}}   % Inverse conjugate transpose (Inverse Hermitian conjugate)
\newcommand{\ex}{\textbf{e}_x}
\newcommand{\ey}{\textbf{e}_y}
\newcommand{\ez}{\textbf{e}_z}
\newcommand{\Aint}{A^\circ}
\newcommand{\Bint}{B^\circ}
\newcommand{\limk}{\lim_{k\to\infty}}
\newcommand{\limm}{\lim_{m\to\infty}}
\newcommand{\limn}{\lim_{n\to\infty}}
\newcommand{\limx}[1][a]{\lim_{x\to#1}}
\newcommand{\limz}[1][{z_0}]{\lim_{z\to#1}}
\newcommand{\liminfm}{\liminf_{m\to\infty}}
\newcommand{\limsupm}{\limsup_{m\to\infty}}
\newcommand{\liminfn}{\liminf_{n\to\infty}}
\newcommand{\limsupn}{\limsup_{n\to\infty}}
\newcommand{\sumkn}{\sum_{k=1}^n}
\newcommand{\sumk}[1][1]{\sum_{k=#1}^\infty}
\newcommand{\summ}[1][1]{\sum_{m=#1}^\infty}
\newcommand{\sumn}[1][1]{\sum_{n=#1}^\infty}
\newcommand{\emp}{\varnothing}
\newcommand{\exc}{\backslash}
\newcommand{\sub}{\subseteq}
\newcommand{\sups}{\supseteq}
\newcommand{\capp}{\bigcap}
\newcommand{\cupp}{\bigcup}
\newcommand{\kupp}{\bigsqcup}
\newcommand{\cappkn}{\bigcap_{k=1}^n}
\newcommand{\cuppkn}{\bigcup_{k=1}^n}
\newcommand{\kuppkn}{\bigsqcup_{k=1}^n}
\newcommand{\cappk}[1][1]{\bigcap_{k=#1}^\infty}
\newcommand{\cuppk}[1][1]{\bigcup_{k=#1}^\infty}
\newcommand{\cappm}[1][1]{\bigcap_{m=#1}^\infty}
\newcommand{\cuppm}[1][1]{\bigcup_{m=#1}^\infty}
\newcommand{\cappn}[1][1]{\bigcap_{n=#1}^\infty}
\newcommand{\cuppn}[1][1]{\bigcup_{n=#1}^\infty}
\newcommand{\kuppk}[1][1]{\bigsqcup_{k=#1}^\infty}
\newcommand{\kuppm}[1][1]{\bigsqcup_{m=#1}^\infty}
\newcommand{\kuppn}[1][1]{\bigsqcup_{n=#1}^\infty}
\newcommand{\cappa}{\bigcap_{\alpha\in I}}
\newcommand{\cuppa}{\bigcup_{\alpha\in I}}
\newcommand{\kuppa}{\bigsqcup_{\alpha\in I}}
\newcommand{\dx}{\,dx}
\newcommand{\dy}{\,dy}
\newcommand{\dt}{\,dt}
\newcommand{\dmu}{\,d\mu}
\newcommand{\dnu}{\,d\nu}
\DeclareMathOperator{\glb}{\text{glb}}
\DeclareMathOperator{\lub}{\text{lub}}
\newcommand{\xh}{\widehat{x}}
\newcommand{\yh}{\widehat{y}}
\newcommand{\zh}{\widehat{z}}
\newcommand{\<}{\langle}
\newcommand{\diff}{\mathcal{D}}
\newcommand{\sph}{\mathbb{S}}
\renewcommand{\>}{\rangle}
\newcommand{\graph}{\text{graph}}
\newcommand{\id}{\text{id}}
\newcommand{\iprod}{\mathbin{\lrcorner}}
\newcommand{\diffm}{\text{Diff}}
\DeclareMathOperator{\Ric}{Ric}
\DeclareMathOperator{\ric}{ric}
\newcommand{\so}[1]{\text{SO}(#1)}
\newcommand{\ggt}{\text{ggT}}
\newcommand{\der}[1]{\text{Der}_{#1}}
\newcommand{\gl}[2]{\text{GL}(#1, #2)}
\newcommand{\inj}{\text{inj}}
%Define behavior of the command with one parameter
\newcommand{\cinfa}[1]{\text{C}^\infty (#1)}
%Define behavior of the command with two parameter
\newcommand{\cinfb}[2]{\text{C}^\infty (#1, #2)}
\newcommand{\fracpart}[1]{\frac{\partial}{\partial #1_i}}
\NewDocumentCommand\cinf{ m g }{
  \IfNoValueTF{#2}{\cinfa{#1}}{\cinfb{#1}{#2}}
}
\newcommand{\supp}{$\text{supp}\ $}
\newcommand{\sgn}{\text{sgn}}
\newcommand{\diag}{\text{diag}}
% Shortcuts for inverse hyperbolic functions (and other operators with the same structure)
\def\do#1{\csdef{#1}{\trigbraces{\operatorname{#1}}}}
\docsvlist{
    asinh,acosh,atanh,acoth,asech,acsch,
    arsinh,arcosh,artanh,arcoth,arsech,arcsch,
    arcsinh,arccosh,arctanh,arccoth,arcsech,arccsch,
    sen,tg,cth,senh,tgh,ctgh,
    Re,Im,arg,Arg,im,ker
}

% \spn has to be defined separately as the syntax "spn" is different from the output "span"
% \span is already defined in the LaTeX kernel. This is a fundamental LaTeX command, DO NOT overwrite it!
\newcommand{\spn}{\trigbraces{\operatorname{span}}}

\makeatletter
% Redefining the commands \iff (given by LaTeX), \implies and \impliedby (given by amsmath)
% Math mode is automatically enforced, starred version makes the arrows shorter
\renewcommand{\impliedby}{\@ifstar{\ensuremath{\Longleftarrow}}{\ensuremath{\Leftarrow}}} % Corresponding Unicode character: U+21D0 ⇐
\renewcommand{\implies}{\@ifstar{\ensuremath{\Longrightarrow}}{\ensuremath{\Rightarrow}}} % Corresponding Unicode character: U+21D2 ⇒
\renewcommand{\iff}{\@ifstar{\ensuremath{\Longleftrightarrow}}{\ensuremath{\Leftrightarrow}}} % Corresponding Unicode character: U+21D4 ⇔
\makeatother

\newtheoremstyle{mystyle}{}{}{}{}{\sffamily\bfseries}{.}{ }{}
\makeatletter
\newenvironment{beweis}[1][\proofname] {\par\pushQED{\qed}{\normalfont\sffamily\bfseries\topsep6\p@\@plus6\p@\relax #1\@addpunct{.} }}{\popQED\endtrivlist\@endpefalse}
\makeatother
\theoremstyle{mystyle}{\newtheorem*{bemerkung}{Bemerkung}}
\theoremstyle{mystyle}{\newtheorem*{bemerkungen}{Bemerkungen}}
\theoremstyle{mystyle}{\newtheorem*{beispiel}{Beispiel}}
\theoremstyle{mystyle}{\newtheorem*{beispiele}{Beispiele}}
\theoremstyle{definition}{\newtheorem*{übung}{Übung}}

% Warning environment
\newtheoremstyle{warn}{}{}{}{}{\normalfont}{}{ }{}
\theoremstyle{warn}
\newtheorem*{warning}{\warningsign{0.2}\relax}

% Symbol for the warning environment, designed to be easily scalable
\newcommand{\warningsign}[1]{
    \tikz[scale=#1,every node/.style={transform shape}]{
        \draw[-,line width={#1*0.8mm},red,fill=yellow,rounded corners={#1*2.5mm}] (0,0)--(1,{-sqrt(3)})--(-1,{-sqrt(3)})--cycle;
        \node at (0,-1) {\fontsize{48}{60}\selectfont\bfseries!};
}}

% verbbox environment, for showing verbatim text next to code output (for package documentation and user learning purposes)
\NewTCBListing{verbbox}{ !O{} }{boxrule=1pt,sidebyside,skin=bicolor,colback=gray!10,colbacklower=white,valign=center,top=2pt,bottom=2pt,left=2pt,right=2pt,#1} % Last argument allows more tcolorbox options to be added




\makeatletter
% \fsize stores the current font size but is expandable (and can be called later without using \makeatletter and \makeatother)
\def\fsize{\dimexpr\f@size pt\relax}
\makeatother

\makeatletter
% Adapted from https://tex.stackexchange.com/a/19700
\def\my@vector #1,#2\@eolst{
    \ifx\relax#2\relax
        #1
    \else
        #1\my@delim
        \my@vector #2\@eolst
    \fi}
\newcommand\vcstring[2][\\]{% Converts comma-separated string to #1-separated string
    \def\my@delim{#1}
        \my@vector #2,\relax\noexpand\@eolst}
\newcommand\cvc[2][p]{% Converts comma-separated string to column vector, optional argument defines matrix brackets
    \def\my@delim{\\}
        \begin{#1matrix} % Empty argument also possible
            \my@vector #2,\relax\noexpand\@eolst
        \end{#1matrix}}
\newcommand\rvc[2][p]{% Converts comma-separated string to row vector, optional argument defines matrix brackets
    \def\my@delim{&}
        \begin{#1matrix} % Empty argument also possible
            \my@vector #2,\relax\noexpand\@eolst
        \end{#1matrix}}
% Matrix environment with variable number of arguments. Adapted from https://davidyat.es/2016/07/27/writing-a-latex-macro-that-takes-a-variable-number-of-arguments/
\newcommand{\mat}[2][p]{
    \def\matrixenvironment{#1matrix} % Specifying the matrix brackets, this has to be done beforehand as '#1' changes under \passtonextarg
    \def\my@delim{&}
        \begin{\matrixenvironment} % Begin matrix environment
            \my@vector #2,\relax\noexpand\@eolst
            \@ifnextchar\bgroup{\passtonextarg}{\end{\matrixenvironment}}% % Pass to next argument (if any), otherwise end matrix environment
}
\newcommand{\passtonextarg}[1]{\\ \my@vector #1,\relax\noexpand\@eolst
    \@ifnextchar\bgroup{\passtonextarg}{\end{\matrixenvironment}}% Passing to next argument
}
\makeatother

\definecolor{tcol_DEF}{HTML}{E40125} % Color for Definition
\definecolor{tcol_PRP}{HTML}{EB8407} % Color for Proposition
\definecolor{tcol_LEM}{HTML}{05C4D9} % Color for Lemma
\definecolor{tcol_THM}{HTML}{1346E4} % Color for Theorem
\definecolor{tcol_COR}{HTML}{7904C2} % Color for Corollary
\definecolor{tcol_REM}{HTML}{18B640} % Color for Remark
\definecolor{tcol_PRF}{HTML}{5A76B2} % Color for Proof
\definecolor{tcol_EXA}{HTML}{21340A} % Color for Example

\tcbset{
tbox_DEF_style/.style={enhanced jigsaw,
    colback=tcol_DEF!10,colframe=tcol_DEF!80!black,,
    fonttitle=\sffamily\bfseries,
    separator sign=.,label separator={},
    sharp corners,top=2pt,bottom=2pt,left=2pt,right=2pt,
    before skip=10pt,after skip=10pt,breakable
},
tbox_PRP_style/.style={enhanced jigsaw,
    colback=tcol_PRP!10,colframe=tcol_PRP!80!black,
    fonttitle=\sffamily\bfseries,
    attach boxed title to top left={yshift=-\tcboxedtitleheight},
    boxed title style={
        boxrule=0pt,boxsep=2.5pt,
        colback=tcol_PRP!80!black,colframe=tcol_PRP!80!black,
        sharp corners=uphill
    },
    separator sign=.,label separator={},
    top=\tcboxedtitleheight,bottom=2pt,left=2pt,right=2pt,
    before skip=10pt,after skip=10pt,drop fuzzy shadow,breakable
},
tbox_THM_style/.style={enhanced jigsaw,
    colback=tcol_THM!10,colframe=tcol_THM!80!black,
    fonttitle=\sffamily\bfseries,coltitle=black,
    attach boxed title to top left={xshift=10pt,yshift=-\tcboxedtitleheight/2},
    boxed title style={
        colback=tcol_THM!10,colframe=tcol_THM!80!black,height=16pt,bean arc
    },
    separator sign=.,label separator={},
    sharp corners,top=6pt,bottom=2pt,left=2pt,right=2pt,
    before skip=10pt,after skip=10pt,breakable
},
tbox_LEM_style/.style={enhanced jigsaw,
    colback=tcol_LEM!10,colframe=tcol_LEM!80!black,
    boxrule=0pt,
    fonttitle=\sffamily\bfseries,
    attach boxed title to top left={yshift=-\tcboxedtitleheight},
    boxed title style={
        boxrule=0pt,boxsep=2pt,
        colback=tcol_LEM!80!black,colframe=tcol_LEM!80!black,
        interior code={\fill[tcol_LEM!80!black] (interior.north west)--(interior.south west)--([xshift=-2mm]interior.south east)--([xshift=2mm]interior.north east)--cycle;
    }},
    separator sign=.,label separator={},
    frame hidden,borderline north={1pt}{0pt}{tcol_LEM!80!black},
    before upper={\hspace{\tcboxedtitlewidth}},
    sharp corners,top=2pt,bottom=2pt,left=5pt,right=5pt,
    before skip=10pt,after skip=10pt,breakable
},
tbox_COR_style/.style={enhanced jigsaw,
    colback=tcol_COR!10,colframe=tcol_COR!80!black,
    boxrule=0pt,
    fonttitle=\sffamily\bfseries,coltitle=black,
    separator sign={},label separator={},
    description font=\normalfont\sffamily,
    description delimiters={(}{)},
    attach title to upper,after title={.\ },
    frame hidden,borderline west={2pt}{0pt}{tcol_COR},
    sharp corners,top=2pt,bottom=2pt,left=5pt,right=5pt,
    before skip=10pt,after skip=10pt,breakable
},
}

\newtcbtheorem[number within=subsection,
    crefname={\color{tcol_DEF!50!black} definition}{\color{tcol_DEF!50!black} definitionen},
    Crefname={\color{tcol_DEF!50!black} Definition}{\color{tcol_DEF!50!black} Definitionen}
    ]{definition}{Definition}{tbox_DEF_style}{}
\newtcbtheorem[use counter from=definition,
    crefname={\color{tcol_PRP!50!black} satz}{\color{tcol_PRP!50!black} sätze},
    Crefname={\color{tcol_PRP!50!black} Satz}{\color{tcol_PRP!50!black} Sätze}
    ]{satz}{Satz}{tbox_PRP_style}{}
\newtcbtheorem[use counter from=definition,
    crefname={\color{tcol_THM!50!black} theorem}{\color{tcol_THM!50!black} theoreme},
    Crefname={\color{tcol_THM!50!black} Theorem}{\color{tcol_THM!50!black} Theoreme}
    ]{theorem}{Theorem}{tbox_THM_style}{}
\newtcbtheorem[use counter from=definition,
    crefname={\color{tcol_LEM!50!black} lemma}{\color{tcol_LEM!50!black} lemmata},
    Crefname={\color{tcol_LEM!50!black} Lemma}{\color{tcol_LEM!50!black} Lemmata}
    ]{lemma}{Lemma}{tbox_LEM_style}{}
\newtcbtheorem[use counter from=definition,
    crefname={\color{tcol_COR!50!black} korollar}{\color{tcol_COR!50!black} korollare},
    Crefname={\color{tcol_COR!50!black} Korollar}{\color{tcol_COR!50!black} Korollare}
    ]{korollar}{Korollar}{tbox_COR_style}{}

\makeatletter
\@namedef{tcolorboxshape@filingbox@ul}#1#2#3{
    (frame.south west)--(title.north west)--([xshift=-\dimexpr#1\relax]title.north east) to[out=0,in=180] ([xshift=\dimexpr#2\relax,yshift=\dimexpr#3\relax]title.south east)--(frame.north east)--(frame.south east)--cycle
}
\@namedef{tcolorboxshape@filingbox@uc}#1#2#3{
    (frame.south west)--(frame.north west)--([xshift=-\dimexpr#2\relax,yshift=\dimexpr#3\relax]title.south west) to[out=0,in=180] ([xshift=\dimexpr#1\relax]title.north west)--([xshift=-\dimexpr#1\relax]title.north east) to[out=0,in=180] ([xshift=\dimexpr#2\relax,yshift=\dimexpr#3\relax]title.south east)--(frame.north east)--(frame.south east)--cycle
}
\@namedef{tcolorboxshape@filingbox@ur}#1#2#3{
    (frame.south east)--(title.north east)--([xshift=\dimexpr#1\relax]title.north west) to[out=180,in=0] ([xshift=-\dimexpr#2\relax,yshift=\dimexpr#3\relax]title.south west)--(frame.north west)--(frame.south west)--cycle
}
\@namedef{tcolorboxshape@filingbox@dl}#1#2#3{
    (frame.north west)--(title.south west)--([xshift=-\dimexpr#1\relax]title.south east) to[out=0,in=180] ([xshift=\dimexpr#2\relax,yshift=-\dimexpr#3\relax]title.north east)--(frame.south east)--(frame.north east)--cycle
}
\@namedef{tcolorboxshape@filingbox@dc}#1#2#3{
    (frame.north west)--(frame.south west)--([xshift=-\dimexpr#2\relax,yshift=-\dimexpr#3\relax]title.north west) to[out=0,in=180] ([xshift=\dimexpr#1\relax]title.south west)--([xshift=-\dimexpr#1\relax]title.south east) to[out=0,in=180] ([xshift=\dimexpr#2\relax,yshift=-\dimexpr#3\relax]title.north east)--(frame.south east)--(frame.north east)--cycle
}
\@namedef{tcolorboxshape@filingbox@dr}#1#2#3{
    (frame.north east)--(title.south east)--([xshift=\dimexpr#1\relax]title.south west) to[out=180,in=0] ([xshift=-\dimexpr#2\relax,yshift=-\dimexpr#3\relax]title.north west)--(frame.south west)--(frame.north west)--cycle
}
\@namedef{tcolorboxshape@railingbox@ul}#1#2#3{
    (frame.south west)--(title.north west)--([xshift=-\dimexpr#1\relax]title.north east)--([xshift=\dimexpr#2\relax,yshift=\dimexpr#3\relax]title.south east)--(frame.north east)--(frame.south east)--cycle
}
\@namedef{tcolorboxshape@railingbox@uc}#1#2#3{
    (frame.south west)--(frame.north west)--([xshift=-\dimexpr#2\relax,yshift=\dimexpr#3\relax]title.south west)--([xshift=\dimexpr#1\relax]title.north west)--([xshift=-\dimexpr#1\relax]title.north east)--([xshift=\dimexpr#2\relax,yshift=\dimexpr#3\relax]title.south east)--(frame.north east)--(frame.south east)--cycle
}
\@namedef{tcolorboxshape@railingbox@ur}#1#2#3{
    (frame.south east)--(title.north east)--([xshift=\dimexpr#1\relax]title.north west)--([xshift=-\dimexpr#2\relax,yshift=\dimexpr#3\relax]title.south west)--(frame.north west)--(frame.south west)--cycle
}
\@namedef{tcolorboxshape@railingbox@dl}#1#2#3{
    (frame.north west)--(title.south west)--([xshift=-\dimexpr#1\relax]title.south east)--([xshift=\dimexpr#2\relax,yshift=-\dimexpr#3\relax]title.north east)--(frame.south east)--(frame.north east)--cycle
}
\@namedef{tcolorboxshape@railingbox@dc}#1#2#3{
    (frame.north west)--(frame.south west)--([xshift=-\dimexpr#2\relax,yshift=-\dimexpr#3\relax]title.north west)--([xshift=\dimexpr#1\relax]title.south west)--([xshift=-\dimexpr#1\relax]title.south east)--([xshift=\dimexpr#2\relax,yshift=-\dimexpr#3\relax]title.north east)--(frame.south east)--(frame.north east)--cycle
}
\@namedef{tcolorboxshape@railingbox@dr}#1#2#3{
    (frame.north east)--(title.south east)--([xshift=\dimexpr#1\relax]title.south west)--([xshift=-\dimexpr#2\relax,yshift=-\dimexpr#3\relax]title.north west)--(frame.south west)--(frame.north west)--cycle
}
\newcommand{\TColorBoxShape}[2]{\expandafter\ifx\csname tcolorboxshape@#1@#2\endcsname\relax
\expandafter\@gobble\else
\csname tcolorboxshape@#1@#2\expandafter\endcsname
\fi}
\makeatother

\tcbset{ % Styles for filingbox, railingbox and flagbox environments
% Adapted from https://tex.stackexchange.com/questions/587912/tcolorbox-custom-title-box-style
filingstyle/ul/.style 2 args={
    attach boxed title to top left={yshift=-2mm},
    boxed title style={empty,top=0mm,bottom=1mm,left=1mm,right=0mm},
    interior code={
        \path[fill=#1,rounded corners] \TColorBoxShape{filingbox}{ul}{9pt}{18pt}{6pt};
    },
    frame code={
        \path[draw=#2,line width=0.5mm,rounded corners] \TColorBoxShape{filingbox}{ul}{9pt}{18pt}{6pt};
    }},
filingstyle/uc/.style 2 args={
    attach boxed title to top center={yshift=-2mm},
    boxed title style={empty,top=0mm,bottom=1mm,left=0mm,right=0mm},
    interior code={
        \path[fill=#1,rounded corners] \TColorBoxShape{filingbox}{uc}{9pt}{18pt}{6pt};
    },
    frame code={
        \path[draw=#2,line width=0.5mm,rounded corners] \TColorBoxShape{filingbox}{uc}{9pt}{18pt}{6pt};
    }},
filingstyle/ur/.style 2 args={
    attach boxed title to top right={yshift=-2mm},
    boxed title style={empty,top=0mm,bottom=1mm,left=0mm,right=1mm},
    interior code={
        \path[fill=#1,rounded corners] \TColorBoxShape{filingbox}{ur}{9pt}{18pt}{6pt};
    },
    frame code={
        \path[draw=#2,line width=0.5mm,rounded corners] \TColorBoxShape{filingbox}{ur}{9pt}{18pt}{6pt};
    }},
filingstyle/dl/.style 2 args={
    attach boxed title to bottom left={yshift=2mm},
    boxed title style={empty,top=1mm,bottom=0mm,left=1mm,right=0mm},
    interior code={
        \path[fill=#1,rounded corners] \TColorBoxShape{filingbox}{dl}{9pt}{18pt}{6pt};
    },
    frame code={
        \path[draw=#2,line width=0.5mm,rounded corners] \TColorBoxShape{filingbox}{dl}{9pt}{18pt}{6pt};
    }},
filingstyle/dc/.style 2 args={
    attach boxed title to bottom center={yshift=2mm},
    boxed title style={empty,top=1mm,bottom=0mm,left=0mm,right=0mm},
    interior code={
        \path[fill=#1,rounded corners] \TColorBoxShape{filingbox}{dc}{9pt}{18pt}{6pt};
    },
    frame code={
        \path[draw=#2,line width=0.5mm,rounded corners] \TColorBoxShape{filingbox}{dc}{9pt}{18pt}{6pt};
    }},
filingstyle/dr/.style 2 args={
    attach boxed title to bottom right={yshift=2mm},
    boxed title style={empty,top=1mm,bottom=0mm,left=0mm,right=1mm},
    interior code={
        \path[fill=#1,rounded corners] \TColorBoxShape{filingbox}{dr}{9pt}{18pt}{6pt};
    },
    frame code={
        \path[draw=#2,line width=0.5mm,rounded corners] \TColorBoxShape{filingbox}{dr}{9pt}{18pt}{6pt};
    }},
railingstyle/ul/.style 2 args={
    attach boxed title to top left={yshift=-2mm},
    boxed title style={empty,top=0mm,bottom=1mm,left=1mm,right=0mm},
    interior code={
        \path[fill=#1] \TColorBoxShape{railingbox}{ul}{3pt}{12pt}{6pt};
    },
    frame code={
        \path[draw=#2,line width=0.5mm] \TColorBoxShape{railingbox}{ul}{3pt}{12pt}{6pt};
    }},
railingstyle/uc/.style 2 args={
    attach boxed title to top center={yshift=-2mm},
    boxed title style={empty,top=0mm,bottom=1mm,left=0mm,right=0mm},
    interior code={
        \path[fill=#1] \TColorBoxShape{railingbox}{uc}{3pt}{12pt}{6pt};
    },
    frame code={
        \path[draw=#2,line width=0.5mm] \TColorBoxShape{railingbox}{uc}{3pt}{12pt}{6pt};
    }},
railingstyle/ur/.style 2 args={
    attach boxed title to top right={yshift=-2mm},
    boxed title style={empty,top=0mm,bottom=1mm,left=0mm,right=1mm},
    interior code={
        \path[fill=#1] \TColorBoxShape{railingbox}{ur}{3pt}{12pt}{6pt};
    },
    frame code={
        \path[draw=#2,line width=0.5mm] \TColorBoxShape{railingbox}{ur}{3pt}{12pt}{6pt};
    }},
railingstyle/dl/.style 2 args={
    attach boxed title to bottom left={yshift=2mm},
    boxed title style={empty,top=1mm,bottom=0mm,left=1mm,right=0mm},
    interior code={
        \path[fill=#1] \TColorBoxShape{railingbox}{dl}{3pt}{12pt}{6pt};
    },
    frame code={
        \path[draw=#2,line width=0.5mm] \TColorBoxShape{railingbox}{dl}{3pt}{12pt}{6pt};
    }},
railingstyle/dc/.style 2 args={
    attach boxed title to bottom center={yshift=2mm},
    boxed title style={empty,top=1mm,bottom=0mm,left=0mm,right=0mm},
    interior code={
        \path[fill=#1] \TColorBoxShape{railingbox}{dc}{3pt}{12pt}{6pt};
    },
    frame code={
        \path[draw=#2,line width=0.5mm] \TColorBoxShape{railingbox}{dc}{3pt}{12pt}{6pt};
    }},
railingstyle/dr/.style 2 args={
    attach boxed title to bottom right={yshift=2mm},
    boxed title style={empty,top=1mm,bottom=0mm,left=0mm,right=1mm},
    interior code={
        \path[fill=#1] \TColorBoxShape{railingbox}{dr}{3pt}{12pt}{6pt};
    },
    frame code={
        \path[draw=#2,line width=0.5mm] \TColorBoxShape{railingbox}{dr}{3pt}{12pt}{6pt};
    }},
flagstyle/ul/.style 2 args={
    interior hidden,frame hidden,colbacktitle=#1,
    borderline west={1pt}{0pt}{#2},
    attach boxed title to top left={yshift=-8pt,yshifttext=-8pt},
    boxed title style={boxsep=3pt,boxrule=1pt,colframe=#2,sharp corners,left=4pt,right=4pt},
    bottom=0mm
    },
flagstyle/ur/.style 2 args={
    interior hidden,frame hidden,colbacktitle=#1,
    borderline east={1pt}{0pt}{#2},
    attach boxed title to top right={yshift=-8pt,yshifttext=-8pt},
    boxed title style={boxsep=3pt,boxrule=1pt,colframe=#2,sharp corners,left=4pt,right=4pt},
    bottom=0mm
    },
flagstyle/dl/.style 2 args={
    interior hidden,frame hidden,colbacktitle=#1,
    borderline west={1pt}{0pt}{#2},
    attach boxed title to bottom left={yshift=8pt,yshifttext=8pt},
    boxed title style={boxsep=3pt,boxrule=1pt,colframe=#2,sharp corners,left=4pt,right=4pt},
    top=0mm
    },
flagstyle/dr/.style 2 args={
    interior hidden,frame hidden,colbacktitle=#1,
    borderline east={1pt}{0pt}{#2},
    attach boxed title to bottom right={yshift=8pt,yshifttext=8pt},
    boxed title style={boxsep=3pt,boxrule=1pt,colframe=#2,sharp corners,left=4pt,right=4pt},
    top=0mm
    }
}

% Box in the shape of a filing divider, position of tab can be ul (up left), uc (up center), ur (up right), dl (down left), dc (down center) or dr (down right). Default is ul (upper left)
\NewTColorBox{filingbox}{ D(){ul} O{black} m O{} }{enhanced,
    top=1mm,bottom=1mm,left=1mm,right=1mm,
    title={#3},
    fonttitle=\sffamily\bfseries,
    coltitle=black,
    filingstyle/#1={#2!10}{#2},
    #4
}

% Box in the shape of a railing bar, position of tab can be ul (up left), uc (up center), ur (up right), dl (down left), dc (down center) or dr (down right). Default is ul (upper left)
\NewTColorBox{railingbox}{ D(){ul} O{black} m O{} }{enhanced,
    top=1mm,bottom=1mm,left=1mm,right=1mm,
    title={#3},
    fonttitle=\sffamily\bfseries,
    coltitle=black,
    railingstyle/#1={#2!10}{#2},
    #4
}

% Box in the shape of a flag, position of tab can be ul (up left), ur (up right), dl (down left) or dr (down right). Default is ul (upper left)
\NewTColorBox{flagbox}{ D(){ul} O{black} m O{} }{enhanced,breakable,
    top=1mm,bottom=1mm,left=1mm,right=1mm,
    title={#3},
    fonttitle=\sffamily\bfseries,
    coltitle=black,
    flagstyle/#1={#2!10}{#2},
    #4
}

\makeatletter
\newcommand*{\CreateSmartLargeOperator}[2]{
% Adapted from https://tex.stackexchange.com/questions/61598/new-command-with-cases-conditionals-if-thens/61600
    % Plain operator (no customization)
    \csdef{LargeOperator@#1@}{\csdef{LargeOperator@#1@Symbol}{\csuse{#1}}}
    % Operator with limits above and below symbol
    \csdef{LargeOperator@#1@l}{\csdef{LargeOperator@#1@Symbol}{\csuse{#1}\limits}}
    % Operato with limits beside symbol
    \csdef{LargeOperator@#1@n}{\csdef{LargeOperator@#1@Symbol}{\csuse{#1}\nolimits}}
    % Inline style operator
    \csdef{LargeOperator@#1@i}{\csdef{LargeOperator@#1@Symbol}{\textstyle\csuse{#1}}}
    % Display style operator
    \csdef{LargeOperator@#1@d}{\csdef{LargeOperator@#1@Symbol}{\displaystyle\csuse{#1}}}
    % Inline style operator with limits above and below symbol
    \csdef{LargeOperator@#1@il}{\csdef{LargeOperator@#1@Symbol}{\textstyle\csuse{#1}\limits}}
    % Inline style operator with limits beside symbol
    \csdef{LargeOperator@#1@in}{\csdef{LargeOperator@#1@Symbol}{\textstyle\csuse{#1}\nolimits}}
    % Display style operator with limits above and below symbol
    \csdef{LargeOperator@#1@dl}{\csdef{LargeOperator@#1@Symbol}{\displaystyle\csuse{#1}\limits}}
    % Display style operator with limits beside symbol
    \csdef{LargeOperator@#1@dn}{\csdef{LargeOperator@#1@Symbol}{\displaystyle\csuse{#1}\nolimits}}

% NOTE: In the command below, ##1 denotes the operator. It is NOT to be used as an argument!
\def\LargeOperatorSpecs@i##1,##2,##3,##4,##5,##6,##7\@nil{
% If no arguments, operate over n from 1 to infinity
    \ifx$##2$\csuse{LargeOperator@##1@Symbol}_{n=1}^{\infty}\else
    % If one argument, operate over n from ##2 to infinity
        \ifx$##3$\csuse{LargeOperator@##1@Symbol}_{n=##2}^{\infty}\else
        % If two arguments, operate over n from ##2 to ##3
            \ifx$##4$\csuse{LargeOperator@##1@Symbol}_{n=##2}^{##3}\else
            % If three arguments, operate over ##2 from ##3 to ##4
                \ifx$##5$\csuse{LargeOperator@##1@Symbol}_{##2=##3}^{##4}\else
                % If four arguments, operate over ##2 and ##3 from ##4 to ##5
                    \ifx$##6$\csuse{LargeOperator@##1@Symbol}_{##2,##3=##4}^{##5}\else
                    % If five arguments, operate over ##2, ##3 and ##4 from ##5 to ##6
                        \csuse{LargeOperator@##1@Symbol}_{##2,##3,##4=##5}^{##6}
                    \fi
                \fi
            \fi
        \fi
    \fi
}

% Flexible "smart" large operator macro with comma-separated arguments and optional argument for formatting. Default is over n from 1 to infinity. Adapted from https://tex.stackexchange.com/a/15722
\expandafter\DeclareDocumentCommand\csname#2\endcsname{ O{} m }{ % New operator macro
\bgroup % Group created to keep operator style (e.g. \limits) local
    \expandafter\ifx\csname LargeOperator@#1@##1\endcsname\relax
    \expandafter\@gobble\else
    \csname LargeOperator@#1@##1\expandafter\endcsname
    \fi
    \expandafter\LargeOperatorSpecs@i#1,##2,,,,,\@nil% % #1 stands in for the first "argument" of \LargeOperatorSpecs@i (the operator), the actual arguments are from ##2 onward
\egroup}
}
\makeatother

% Create the smart large operator #2 based on the large operator #1. For example, \CreateSmartLargeOperator{sum}{Sum} will define \Sum as the smart large operator based on \sum
% Equivalent Unicode characters are given here (but they are NOT the same as the operators)
\CreateSmartLargeOperator{sum}{Sum}             % Large: U+2211 ∑ (no small version)
\CreateSmartLargeOperator{prod}{Prod}           % Small: U+2293 ⊓, Large: U+220F ∏
\CreateSmartLargeOperator{coprod}{Coprod}       % Small: U+2294 ⊔, Large: U+2210 ∐
\CreateSmartLargeOperator{bigcap}{Capp}         % Small: U+2229 ∩, Large: U+22C2 ⋂
\CreateSmartLargeOperator{bigcup}{Cupp}         % Small: U+222A ∪, Large: U+22C3 ⋃
\CreateSmartLargeOperator{bigsqcup}{Kupp}       % Small: U+2294 ⊔, Large: U+2210 ∐
\CreateSmartLargeOperator{bigodot}{Odot}        % Small: U+2299 ⊙ (no large version)
\CreateSmartLargeOperator{bigoplus}{Oplus}      % Small: U+2295 ⊕ (no large version)
\CreateSmartLargeOperator{bigotimes}{Otimes}    % Small: U+2297 ⊗ (no large version)
\CreateSmartLargeOperator{biguplus}{Uplus}      % Small: U+228E ⊎ (no large version)
\CreateSmartLargeOperator{bigwedge}{Wedge}      % Small: U+2227 ∧, Large: U+22C0 ⋀
\CreateSmartLargeOperator{bigvee}{Vee}          % Small: U+2228 ∨, Large: U+22C1 ⋁

\tcolorboxenvironment{beweis}{boxrule=0pt,boxsep=0pt,blanker,
    borderline west={2pt}{0pt}{tcol_PRF},left=8pt,right=8pt,sharp corners,
    before skip=10pt,after skip=10pt,breakable
}
\tcolorboxenvironment{anmerkung}{boxrule=0pt,boxsep=0pt,blanker,
    borderline west={2pt}{0pt}{tcol_REM},left=8pt,right=8pt,
    before skip=10pt,after skip=10pt,breakable
}
\tcolorboxenvironment{anmerkungen}{boxrule=0pt,boxsep=0pt,blanker,
    borderline west={2pt}{0pt}{tcol_REM},left=8pt,right=8pt,
    before skip=10pt,after skip=10pt,breakable
}
\tcolorboxenvironment{beispiel}{boxrule=0pt,boxsep=0pt,blanker,
    borderline west={2pt}{0pt}{tcol_EXA},left=8pt,right=8pt,sharp corners,
    before skip=10pt,after skip=10pt,breakable
}
\tcolorboxenvironment{beispiele}{boxrule=0pt,boxsep=0pt,blanker,
    borderline west={2pt}{0pt}{tcol_EXA},left=8pt,right=8pt,sharp corners,
    before skip=10pt,after skip=10pt,breakable
}

% align and align* environments with inline size
\newenvironment{talign}{\let\displaystyle\textstyle\align}{\endalign}
\newenvironment{talign*}{\let\displaystyle\textstyle\csname align*\endcsname}{\endalign}

\usepackage[explicit]{titlesec}
% Setting the format for sections, subsections and subsubsections
\titleformat{\section}{\fontsize{24}{30}\sffamily\bfseries}{\thesection}{20pt}{#1}
\titleformat{\subsection}{\fontsize{16}{18}\sffamily\bfseries}{\thesubsection}{12pt}{#1}
\titleformat{\subsubsection}{\fontsize{10}{12}\sffamily\large\bfseries}{\thesubsubsection}{8pt}{#1}
% Setting the spacing for sections, subsections and subsubsections
% First argument is the left indent, second argument is the spacing above, third argument is the spacing below
\titlespacing*{\section}{0pt}{5pt}{5pt}
\titlespacing*{\subsection}{0pt}{5pt}{5pt}
\titlespacing*{\subsubsection}{0pt}{5pt}{5pt}

\newcommand{\Disp}{\displaystyle}
\newcommand{\qe}{\hfill\(\bigtriangledown\)}
\DeclareMathAlphabet\mathbfcal{OMS}{cmsy}{b}{n}
\setlength{\parindent}{0.2in}
\setlength{\parskip}{0pt}
\setlength{\columnseprule}{0pt}

\makeatletter
% Modify spacing above and below display equations
\g@addto@macro\normalsize{
    \setlength\abovedisplayskip{3pt}
    \setlength\belowdisplayskip{3pt}
    \setlength\abovedisplayshortskip{0pt}
    \setlength\belowdisplayshortskip{0pt}
}
\makeatother

\makeatletter
% Redefining the title block
\renewcommand\maketitle{
\null % \vspace does not work with nothing above it, so \null is added
\vspace{5mm}
\begingroup % Creating a group to ensure col_stripes is only defined locally, i.e. only for the title
\definecolor{col_stripes}{HTML}{1B0982} % Color of the stripes above and below the title components
    \begin{tcolorbox}[enhanced,blanker,
    borderline horizontal={2pt}{0pt}{col_stripes},
    borderline horizontal={1pt}{-3.5pt}{col_stripes},
    borderline horizontal={2pt}{-8pt}{col_stripes},
    fontupper=\fontfamily{bch},
    halign=flush center,top=10mm,bottom=10mm,after skip=20mm,
    ]
        {\fontsize{24}{28}\bfseries\selectfont\@title}\\
            \vspace{6mm}
        {\fontsize{20}{24}\selectfont\@author}\\
            \vspace{6mm}
        {\fontsize{16}{20}\selectfont\@date}
    \end{tcolorbox}
\endgroup}
% Adapted from https://tex.stackexchange.com/questions/483953/how-to-add-new-macros-like-author-without-editing-latex-ltx?noredirect=1&lq=1
\makeatother

\title{Topologie (Bachelor)}
\author{zur Vorlesung im WiSe24/25}
\date{\today} % Replace with \today to show the current date

\begin{document}

\maketitle

\definecolor{tcol_CNT1}{HTML}{72E094} % First color for Contents
\definecolor{tcol_CNT2}{HTML}{24E2D6} % Second color for Contents
\definecolor{tcol_CNV1}{HTML}{8E44AD} % First color for Conventions
\definecolor{tcol_CNV2}{HTML}{A10B49} % First color for Conventions

\begin{tcolorbox}[enhanced,
    title=Inhaltsverzeichnis,
    fonttitle=\fontsize{20}{24}\sffamily\bfseries\selectfont,
    coltitle=black,
    fontupper=\sffamily,
    interior style={left color=tcol_CNT1!80,right color=tcol_CNT2!80},
    frame style={left color=tcol_CNT1!60!black,right color=tcol_CNT2!60!black},
    attach boxed title to top center={yshift=10pt},
    boxed title style={frame hidden,
        interior style={left color=tcol_CNT1,right color=tcol_CNT2},
        frame style={left color=tcol_CNT1!60!black,right color=tcol_CNT2!60!black},
        height=24pt,bean arc,drop fuzzy shadow
    },
    top=2mm,bottom=2mm,left=2mm,right=2mm,
    before skip=20mm,after skip=20mm,
    drop fuzzy shadow,breakable]
%
\makeatletter
\@starttoc{toc}
\makeatother
\end{tcolorbox}

\begin{tcolorbox}[enhanced,
    frame hidden,
    title=Konventionen,
    fonttitle=\large\sffamily\bfseries\selectfont,
    interior code={
        \shade[top color=tcol_CNV2!50,bottom color=white] ([yshift=2mm]interior.north west) arc(-180:-90:2mm)--(interior.north east)--(interior.south east)--(interior.south west)--cycle;
        },
    overlay={
        \draw[tcol_CNV1!50!black,line width=0.5mm] ([xshift=2mm]frame.north west)--(frame.north east);
    },
    boxrule=0pt,left=2pt,right=2pt,
    sharp corners=north,
    attach boxed title to top left,
    boxed title style={interior hidden,
    left=1mm,right=1mm,
    frame code={
        \path[draw=tcol_CNV1!50!black,line width=0.5mm,fill=tcol_CNV1,rounded corners=2mm] ([xshift=2mm]frame.south east)--(frame.south east)--(frame.north east)--([xshift=0.25mm]frame.north west)--([xshift=0.25mm]frame.south west)--cycle;}
    },
    top=2mm,bottom=2mm,left=2mm,right=2mm,
    before skip=10mm,after skip=10mm]
%
\begin{itemize}
\item TBD
\end{itemize}
\end{tcolorbox}
Dies ist ein inoffizielles Skript zur Vorlesung Topologie im Wintersemester 24/25. Fehler und Verbesserungsvorschläge immer gerne an \url{rasmus.raschke@uni-hamburg.de}.
\newpage
\sloppy
\section{Mengentheoretische Topologie}
\label{PST}
\subsection{Metrische Räume}
\label{subsec:metrischeraeume}

\begin{definition}{Metrischer Raum}{metrischerraum}
Ein \textbf{metrischer Raum} ist ein Paar $(X, d)$, bestehend aus einer Menge $X$ und einer Abstandsfunktion \begin{equation}
d: X \times X \to \R,
\end{equation}
genannt \textbf{Metrik}, die die folgenden Axiome erfüllt:
\begin{enumerate}[({M}1)]
\item \textit{Positivität}: $\forall x,y \in X: d(x,y)>0$
\item \textit{Symmetrie}: $\forall x,y \in X: d(x,y)=d(y,x)$
\item \textit{Dreiecksungleichung}: $\forall x,y,z \in X: d(x,z) \leq d(x,y) + d(y,z)$
\end{enumerate}
\end{definition}

\begin{beispiele}
\begin{enumerate}
\item Im $\R^n$ ist die \textbf{Standardmetrik} oder \textbf{euklidische Metrik} für $x,y \in \R$ gegeben durch 
\begin{equation}
d_2(x,y) := \sqrt{\sum_{i=1}^n (x_i-y_i)^2}.
\end{equation}
\item Auf $(\R^n, d_n)$ ist eine Metrik durch \begin{equation}
d_n(x,y) := \sum_{i=1}^n |x_i - y_i|
\end{equation}
gegeben.
\item Die \textbf{Maximumsnorm} $(\R^n, d_\infty)$ ist gegeben durch 
\begin{equation}
d_\infty (x,y) = \max_{i \in \{1,\dots, n\}} |x_i - y_i|.
\end{equation}
\item Eine weitere Metrik auf $\R^n$ ist gegeben durch 
\begin{equation}
d_{\sqrt{\cdot}} (x,y) = \sqrt{d_2(x,y)}.
\end{equation}
Diese Metrik kommt nicht von einer Norm.
\item Die \textbf{diskrete Metrik} auf einer beliebigen Menge $X$ ist gegeben durch 
\begin{equation}
d(x,y) := \begin{cases} 0, \, x = y \\ 1, \, x \neq y \end{cases}.
\end{equation}
\item Auf $X= \mathcal{C} ([0,1], \R)$ ist für $f,g \in X$ durch das Integral eine Metrik 
\begin{equation}
d(f,g) := \int_0^1 |f(x) - g(x)| dx
\end{equation}
definiert.
\end{enumerate}
\end{beispiele}
\begin{bemerkungen}
\begin{enumerate}
\item Wenn $(X,d)$ ein metrischer Raum ist, so ist $Y \sub X$ als $(Y,d|_{Y \times Y})$ auch ein metrischer Raum.
\item Wenn $(X_1, d_1)$ und $(X_2, d_2)$ metrische Räume sind, so ist $(X_1 \times X_2, d_1 \times d_2)$ wieder ein metrischer Raum.
\item Vorsicht: Für eine Familie $(X_i, d_i)_{i \in I}$ ist der Sachverhalt komplizierter.
\end{enumerate}
\end{bemerkungen}
\begin{definition}{$\epsilon$-Ball}{ball}
Sei $(X,d)$ ein metrischer Raum, $x \in X$ und $\epsilon > 0$. Dann ist der $\epsilon$\textbf{-Ball} mit $x$ im Zentrum definiert als 
\begin{equation}
B_\epsilon (x) := \{ y \in X \, | \, d(x,y) < \epsilon \}.
\end{equation}
\end{definition}
\begin{definition}{Umgebung}{umgebung}
Sei $(X,d)$ ein metrischer Raum. Eine Menge $U \sub X$ heißt \textbf{Umgebung} von $x \in X$, falls ein $\epsilon > 0$ mit $B_\epsilon (x) \sub U$ existiert.
\end{definition}
\begin{definition}{Offen und Abgeschlossen}{offenabgeschlossen}
Sei $(X,d)$ ein metrischer Raum. Eine Teilmenge $O \sub X$ heißt \textbf{offen}, falls für alle $x \in O$ ein $\epsilon > 0$ existiert, sodass $B_\epsilon (x) \sub O$ gilt. $O$ ist also eine Umgebung all seiner Elemente.\\
Eine Menge $A \sub X$ heißt \textbf{abgeschlossen}, falls $X \exc A$ offen ist.
\end{definition}
\begin{bemerkungen}
\begin{enumerate}
\item Sei $\epsilon > 0$ und $(X,d)$ ein metrischer Raum. Dann ist $B_\epsilon (x) \sub X$ offen und eine Umgebung von $x$.
\item ÜA: Sei $(X,d)$ ein metrischer Raum und $x \in X$. Dann ist $\{x\}$ abgeschlossen.
\end{enumerate}
\end{bemerkungen}
\begin{satz}{Umgebungseigenschaften metrischer Räume}{umgebungseigenschaften}
Sei $(X,d)$ ein metrischer Raum. Dann gilt:
\begin{enumerate}[({U}1)]
\item Jede Umgebung von $x \in X$ enthält $x$ und $X$ ist eine Umgebung von $x$.
\item Ist $U \sub X$ eine Umgebung von $X$ und $U \sub V \sub X$, so ist $V$ auch eine Umgebung von $x$.
\item Wenn $U_1$ und $U_2$ Umgebungen von $x$ sind, so auch $U_1 \cap U_2$.
\item Ist $U \sub X$ eine Umgebung von $x$, so existiert eine weitere Teilmenge $V \sub X$, sodass $U$ eine Umgebung von allen $y \in V$ ist.
\end{enumerate}
\end{satz}
\begin{beweis}
\begin{enumerate}
\item Trivial.
\item Trivial.
\item Nach Voraussetzung existiert für $x \in U_1 \cap U_2$ ein $\epsilon_1 > 0$, sodass $B_{\epsilon_1} (x) \sub U_1$ und ein $\epsilon_2 > 0$, sodass $B_{\epsilon_2} (x) \sub U_2$. Definiere $\epsilon := \min (\epsilon_1, \epsilon_2)$. Dann gilt $B_\epsilon (X) \sub U_1$ und $B_\epsilon (x) \sub U_2$, also $B_\epsilon (x) \sub U_1 \cap U_2$.
\item Nach Voraussetzung existiert ein $\epsilon > 0$ mit $B_\epsilon (x) \sub U$. Dann ist die Behauptung durch $V:=B_\epsilon(x)$ erfüllt.
\end{enumerate}
\end{beweis}
\begin{satz}{Eigenschaften offener Mengen}{eigenschaftoffen}
Sei $(X,d)$ ein metrischer Raum. Dann gilt:
\begin{enumerate}
\item $\emptyset$ und $X$ sind offen.
\item Sind $O_1, O_2 \sub X$ offen, so auch $O_1 \cap O_2$.
\item Ist $(O_i)_{i \in I}$ eine Familie offener Teilmengen $O_i \sub X$, so ist $\cup_i O_i$ auch offen.
\end{enumerate}
\end{satz}
\begin{beweis}
\begin{enumerate}
\item Trivial.
\item Mit $\min (\epsilon_1, \epsilon_2)$ analog zum obigen Beweis.
\item Sei $x \in \cup_i O_i$. Dann existiert ein $i \in I$ mit $x \in O_i$, sodass ein $\epsilon > 0$ existiert mit $B_\epsilon(x) \sub O_i \sub \cup_i O_i$.
\end{enumerate}
\end{beweis}
\begin{satz}{Eigenschaften abgeschlossener Mengen}{eigenschaftenabgeschlossen}
Sei $(X,d)$ ein metrischer Raum. Dann gilt:
\begin{enumerate}[({A}1)]
\item $\emptyset$ und $X$ sind abgeschlossen.
\item Wenn $A_1, A_2 \sub X$ abgeschlossene Teilmengen sind, so ist auch $A_1 \cup A_2$ abgeschlossen.
\item Seien $(A_i)_{i \in I}$ abgeschlossene Teilmengen von $X$. Dann ist $\cup_i A_i$ wieder abgeschlossen.
\end{enumerate}
\end{satz}
\begin{beweis}
\begin{enumerate}
\item Da $\emptyset = X \exc X$ und $X = X \exc \emptyset$ gilt, sind $X$ und $\emptyset$ gemäß Satz \ref{eigenschaftoffen} offen.
\item Sei $A_1 = X \exc O_1$ und $A_2 = X \exc O_2$ mit $O_1, O_2 \sub X$ offen. Gemäß Satz \ref{eigenschaftoffen} (2.) folgt 
\begin{equation}
X \exc (A_1 \cup A_2) = (X \exc A_1) \cap (X \exc A_2) = O_1 \cap O_2,
\end{equation}
wobei $O_1 \cap O_2$ wieder offen ist.
\item Wir betrachten offene Teilmengen $O_i := X \exc A_i$. Gemäß Satz \ref{eigenschaftoffen} ist 
\begin{equation}
X \exc \bigcap_{i\in I} A_i = \bigcup_{i \in I} X \exc A_i = \bigcup_{i \in I} O_i 
\end{equation}
offen.
\end{enumerate}
\end{beweis}
\begin{definition}{stetige Abbildung}{metstetabb}
Seien $(X,d)$ und $(Y,d')$ metrische Räume und $f: X \to Y$. Dann heißt $f$ \textbf{stetig in} $x_0 \in X$, wenn für alle $\epsilon > 0$ ein $\delta > 0$ existiert, sodass 
\begin{equation}
d(x_0,x) < \delta \implies d'(f(x_0), f(x)) < \epsilon
\end{equation}
gilt. $f$ heißt \textbf{stetig}, falls dies für alle $x_0 \in X$ erfüllt ist.
\end{definition}

\begin{satz}{Äquivalente Formulierung der Stetigkeit}{umgebungstetig}
Seien $(X,d)$ und $(Y,d')$ metrische Räume und $f: X \to Y$. Dann sind äquivalent:
\begin{enumerate}
\item $f$ ist stetig.
\item $V$ ist Umgebung von $f(x)$ $\implies$ $f^{-1}(V)$ ist eine Umgebung von $x$.
\item $O \in Y$ ist offen $\implies$ $f^{-1}(O)$ ist offen in $X$.
\item $A \in Y$ ist abgeschlossen $\implies$ $f^{-1}(A)$ ist abgeschlossen in $X$.
\end{enumerate}
\end{satz}
\begin{beweis}
(1. $\implies$ 2.): Sei $V$ eine Umgebung von $f(x)$. Per Definition existiert ein $\epsilon > 0$, sodass $f(x) \in B_\epsilon(f(x)) \sub V$ gilt. Gemäß Annahme existiert ein $\delta > 0$ mit $f(B_\delta(x))\sub B_\epsilon(f(x))$. Daraus folgt, dass 
\begin{equation}
B_\delta (x) \sub f^{-1}f(B_\delta(x)) \sub f^{-1}(B_\epsilon(f(x))) \sub f^{-1}(V).
\end{equation}
Also ist $f^{-1}(V)$ eine Umgebung von $x$.\\
(2. $\implies$ 3.): $O$ ist Umgebung all seiner Elemente.\\
(3. $\implies$ 4.): $Y \exc A$ ist offen in $Y$, d.h. $f^{-1}(Y \exc A) = f^{-1}(Y) \exc f^{-1}(A) = X \exc f^{-1}(A)$ ist offen in $X$. Also ist $f^{-1}(A)$ abgeschlossen.\\
(4. $\implies$ 1.): $Y \exc B_\epsilon(f(x))$ ist abgeschlossen impliziert, dass $f^{-1} (Y \exc B_\epsilon(f(x)))$ auch abgeschlossen ist. Damit folgt, dass $X \exc f^{-1}(B_\epsilon(f(x)))$ abgeschlossen und damit $f^{-1}(B_\epsilon(f(x)))$ offen ist. Für $x \in f^{-1}(B_\epsilon(f(x)))$ existiert ein $\delta > 0$ mit $B_\delta(x) \sub f^{-1}(B_\epsilon(f(x)))$, also auch $f(B_\delta(x)) \sub B_\epsilon(f(x))$.
\end{beweis}

\begin{definition}{Äquivalenz von Metriken}{aequivalentmetrik}
Seien $d_1$ und $d_2$ Metriken auf einer Menge $X$.
\begin{enumerate}
\item Gibt es $\alpha, \beta > 0$ mit
\begin{equation}
\alpha d_1(x,y) \leq d_2(x,y) \leq \beta d_1(x,y)
\end{equation}
für alle $x,y \in X$, so heißen $d_1$ und $d_2$ \textbf{stark äquivalent}.
\item $d_1$ und $d_2$ heißen \textbf{äquivalent}, falls es für jedes $x \in X$ und alle $\epsilon > 0$ ein $\delta > 0$ gibt, sodass
\begin{enumerate}[(i)]
\item $d_1(x,y) < \delta \implies d_2(x,y) < \epsilon$
\item $d_2(x,y) < \delta \implies d_1(x,y) < \epsilon$
\end{enumerate}
gilt.
\end{enumerate}
\end{definition}
\begin{bemerkungen}
\begin{enumerate}
\item Genau dann, wenn $d_1$ und $d_2$ äquivalente Metriken sind, sind $\id_X: (X, d_1) \to (X, d_2)$ und $id_X: (X,d_2) \to (X, d_1)$ stetig.
\item $d_1$ und $d_2$ stark äquivalent impliziert, dass $\id_X$ gleichmäßig stetig ist.
\item Äquivalente Metriken ergeben die gleichen offenen (und abgeschlossenen) Mengen.
\end{enumerate}
\begin{beispiele}
\begin{enumerate}
\item Die $d_1$-, $d_2$- und $d_\infty$-Metrik auf dem $\R^n$ sind stark äquivalent.
\item Sei $d_0(x,y) = |x^3 - y^3|$ und $d_2(x,y)$ die euklidische Metrik. Die Identität $\id_\R (\R, d_2) \to (\R, d_0)$ ist stetig, aber nicht gleichmäßig stetig.
\item Sei $X$ eine beliebige Menge mit einer beliebigen Metrik $d$. Dann ist $d$ äquivalent zu 
\begin{equation}
d'(x,y) := \frac{d(x,y)}{1+d(x,y)} < 1
\end{equation}
für alle $x,y \in X$. Also ist \textit{jede Metrik äquivalent zu einer beschränkten Metrik}.
\end{enumerate}
\end{beispiele}
\end{bemerkungen}
\subsection{Topologische Räume}
\label{subsec:topologischeraeume}
Der Begriff des topologischen Raums wird durch Abstraktion der Eigenschaften offener Mengen und stetiger Abbildungen in metrischen Räumen konstruiert.
\begin{definition}{Topologischer Raum}{topologischerraum}
Ein \textbf{topologischer Raum} ist ein Paar $(X, \Tc)$, bestehend aus einer Menge $X$ und einer Familie $\Tc$ von Teilmengen von $X$, sodass folgende Axiome erfüllt sind:
\begin{enumerate}[({O}1)]
\item $\emptyset, X \in \Tc$
\item $O_1, O_2 \in \Tc \implies O_1 \cap O_2 \in \Tc$
\item Für eine Familie $(O_i)_{i \in I}$ mit $O_i \in \Tc$ für alle $i \in I$ folgt $\cup_{i \in I} O_i \in \Tc$.
\end{enumerate}
$\Tc$ heißt \textbf{Topologie} auf $X$ und alle $O \in \Tc$ heißen \textbf{offene Mengen}.
\end{definition}
\begin{bemerkung}
Äquivalent dazu ist: Eine Topologie $\Tc \sub \Pc(X)$ ist abgeschlossen unter endlichen Schnitten und beliebigen Vereinigungen. 
\end{bemerkung}
\begin{beispiele}
\begin{enumerate}
\item Metrische Räume $(X,d)$ sind auch topologische Räume mit offenen Mengen gegeben durch $d$.
\item Auf einer beliebigen Menge $X$ kann die \textbf{diskrete Topologie} $\Tc := \Pc(X)$ definiert werden, in der alle Teilmengen von $X$ offen sind.
\item Auch kann auf beliebigem $X$ die \textbf{indiskrete Topologie} oder \textbf{Klumpentopologie} durch $\Tc := \{\emptyset, X\}$ definiert werden.
\item Auf beliebigem $X$ existiert die \textbf{koendliche Topologie}: $O\sub X$ ist offen genau dann, wenn $X \exc O$ endlich ist oder $O = \emptyset$ gilt.
\end{enumerate}
\end{beispiele}
\begin{definition}{Topologische Grundbegriffe}{topgrundbegriffe}
Sei $(X, \Tc)$ ein topologischer Raum.
\begin{enumerate}
\item $A \sub X$ heißt \textbf{abgeschlossen}, falls $X \exc A \in \Tc$.
\item Sei $x \in U \sub X$. Dann heißt $U$ \textbf{Umgebung von} $x$, falls ein $O \in \Tc$ existiert, sodass $x \in O \sub U$ gilt.
\item Die Menge aller Umgebungen von $X$ wird mit $\Uf (x)$ bezeichnet und heißt \textbf{Umgebungssystem von} $x$. 
\item Ein Punkt $x \in X$ heißt \textbf{Berührpunkt} von $B \sub X$, falls für alle $U \in \Uf(x)$ gilt: $U \cap B \neq \emptyset$.
\item Die \textbf{abgeschlossene Hülle} von $B \sub X$ ist definiert als 
\begin{equation}
\overline{B} := \bigcap_{B \sub C, C \, \text{abg.}} C.
\end{equation}
\item Ein Punkt $x \in X$ heißt \textbf{innerer Punkt} von $B \sub X$, falls es ein $U \in \Uf(x)$ gibt, sodass $x \in U \sub B$ gilt.
\item Für $B \sub X$ ist 
\begin{equation}
\mathring{B} := \bigcup_{O \sub B, O \, \text{offen}} O
\end{equation}
der \textbf{offene Kern} von $B$.
\item Der \textbf{Rand von} $A \sub X$ ist definiert als
\begin{equation}
\partial A:= \{x \in X|\forall U \in \Uf(x): U \cap A \neq \emptyset \neq U \cap (X \exc A)\}.
\end{equation}
\end{enumerate}
\end{definition}
\begin{satz}{Eigenschaften bestimmter Mengen}{eigenschaftmengen}
Sei $(X, \Tc)$ ein top. Raum. Dann gilt:
\begin{enumerate}
\item Die abgeschlossenen Mengen von $X$ erfüllen (A1)-(A3).
\item Die Umgebungen erfüllen (U1)-(U4).
\end{enumerate}
\end{satz}
\begin{bemerkung}
Eine Topologie kann äquivalent durch die Auflistung abgeschlossener Mengen definiert werden, wenn diese (A1)-(A3) erfüllen.
\end{bemerkung}
\begin{definition}{Stetigkeit in top. Räumen}{stetigtopo}
Seien $(X, \Tc)$ und $(Y, \Tc')$ top. Räume und $f: X \to Y$.
\begin{enumerate}[(i)]
\item $f$ heißt \textbf{stetig in} $x \in X$, wenn für alle $U \in \Uf(f(x))$ auch $f^{-1}(U) \in \Uf(x)$ gilt.
\item $f$ heißt \textbf{stetig}, falls für alle $O \in \Tc'$ gilt, dass $f^{-1} (O) \in \Tc$.
\end{enumerate}
\end{definition}
\begin{satz}{Eigenschaften von Abschluss und Innerem}{abschlussundinneres}
Sei $(X,\Tc)$ ein top. Raum mit $A \in \Tc$. Dann gilt
\begin{enumerate}
\item \begin{enumerate}
\item $\overline{A}$ ist abgeschlossen und $A \sub \overline{A}$.
\item $A = \overline{A}$ gilt genau dann, wenn $A$ abgeschlossen ist.
\item $\overline{A}$ besteht aus der Menge der Berührpunkte von $A$.
\end{enumerate}
\item \begin{enumerate}
\item $\mathring{B}$ ist offen und $\mathring{B} \sub B$.
\item $B = \mathring{B}$ genau dann, wenn $B$ offen ist.
\item $\mathring{B}$ besteht aus der Menge der inneren Punkte von $B$.
\end{enumerate}
\end{enumerate}
\end{satz}
\begin{beweis}
(a) und (b) sind jeweils trivial. Wir beweisen 1(c). Angenommen, $x \in \overline{A}$ aber ist kein Berührpunkt von $A$. Dann existiert ein $U \in \Uf(x)$, sodass $U \cap A = \emptyset$ gilt. Daraus folgt, dass $A \sub X \exc U$ gilt, woraus $A \sub X \exc O$ abgeschlossen folgt. Weiterhin existiert ein $O \in \Tc$ mit $x \in O \sub U$, also $X \exc U \sub X \exc O$. Dann ist aber $\overline{A} \sub X \exc O$, also $x \notin \overline{A}$.\\
Jetzt nehmen an, dass $x$ Berührpunkt ist, aber $x \notin  \overline{A}$. Also folgt aus $x \in X \exc \overline{A}$ offen, dass $V:= X \exc \overline{A} \in \Uf (x)$, aber $V \cap A = \emptyset$, also ist $x$ kein Berührpunkt im Widerspruch zur Annahme.\\
2 ist dual.
\end{beweis}
\begin{bemerkung}
Folgendes gilt allgemein:
\begin{itemize}
\item Ist $A \sub B$, so auch $\mathring{A} \sub \mathring{B}$ und $\overline{A} \sub \overline{B}$.
\item $\overline{A \cup B} = \overline{A} \cup \overline{B}$
\item $\mathring{A \cap B} = \mathring{A} \cap \mathring{B}$
\end{itemize}
\end{bemerkung}
\begin{definition}{Dichheit}{dicht}
Sei $(X, \Tc)$ ein top. Raum. $A \sub X$ heißt \textbf{dicht}, falls $\overline{A} = X$. $A\sub X$ heißt hingegen \textbf{nirgends dicht}, falls $\mathring{\overline{A}}= \emptyset$.
\end{definition}
\begin{beispiele}
\begin{enumerate}
\item $\Q \sub \R$ ist dicht.
\item $[a,b] \sub \R \sub \R^2$ liegt nirgends dicht in $\R^2$.
\end{enumerate}
\end{beispiele}
\begin{definition}{Konvergenz in top. Räumen}{topkov}
Sei $(X, \Tc)$ ein top. Raum. Eine Folge $(x_i)_{i \in N}$ mit $x_i \in X$ \textbf{konvergiert} gegen $x \in X$, falls gilt:
Für alle $U \in \mathfrak{U}(x)$ existiert ein $N \in \N$, sodass $x_n \in U$ für alle $n \geq N$ gilt.
\end{definition}
Das heißt, fast alle $x_n$ müssen in $U$ liegen. Allgemein liefert dies deutlich pathologischere Beispiele als Konvergenz in metrischen Räumen.
\begin{beispiel}
Betrachte $(X, \{\emptyset, X\})$ mit $|X| \geq 2$. Dann liegt nur $U=X$ in $\mathfrak{U}(x)$, also konvergiert jede Folge gegen jedes $y \in X$.
\end{beispiel}
\subsection{Basen, Subbasen und Umgebungsbasen}
\label{subsec:basen}
Unser Ziel ist jetzt, auch nicht-endliche Topologien angeben zu können.
\begin{definition}{Basis und Subbasis}{basissubbasis}
Sei $(X, \Tc)$ ein top. Raum.
\begin{enumerate}[(a)]
\item Eine Familie $\mathcal{B}$ heißt \textbf{Basis} der Topologie $\Tc$, falls alle $O \in \Tc$ als Vereinigung von $B_i \in \Bc$ geschrieben werden können:
\begin{equation}
\forall O \in \Tc: O = \bigcup_{j \in J} B_j. 
\end{equation}
\item Eine Familie $\mathcal{S} \sub \Tc$ heißt \textbf{Subbasis} der Topologie $\Tc$, falls jedes $O \in \Tc$ eine beliebige Vereinigung endlicher Schnitte von $S_i \in \mathcal{S}$ ist.
\end{enumerate}
\end{definition}
\begin{beispiele}
\begin{enumerate}
\item Sei $(X,d)$ ein metrischer Raum. Dann ist 
\begin{equation}
\Bc = \{B_\epsilon(x) \, | \, x \in X, \epsilon > 0\}
\end{equation}
eine Basis für $X$.
\item Die diskrete Topologie $(X, \Pc(X))$ hat $\Bc = \{ \{x\} | x \in X\}$ als Basis. 
\item Die indiskrete Topologie $(X, \{X, \emptyset \})$ hat als Basis $\Bc = \{X\}$.
\item Jedes System beliebiger Teilmengen von $X$ gibt eine Subbasis einer Topologie. Sei z.B. $\mathcal{S} := \{S\}_{i\in i}$ gewünscht. Dann konstruieren wir 
\begin{equation}
\Bc = \{S_{i_1} \cap \dots \cap S_{i_n} |n \in \N, \, S_{i_k} \in \cS\}
\end{equation} 
als Basis und definieren eine Topologie
\begin{equation}
\Tc = \left\{ \bigcup_{j \in J} (S_{i_1} \cap \cdots \cap S_{i_n})_j \right\}.
\end{equation}
\end{enumerate}
\end{beispiele}
Das Gute an (Sub-)Basen ist, dass sie uns Arbeit ersparen:
\begin{satz}{Stetigkeit durch Basen}{basenstetigkeit}
Eine Abbildung $f: (X, \Tc) \to (Y, \Tc')$ zwischen top. Räumen ist stetig, falls:
\begin{enumerate}[(a)]
\item Ist $\Bc'$ eine Basis von $\Tc'$, so  ist $f^{-1}(B_i') \in \Tc$ für alle $B_i' \in \Bc'$.
\item Ist $\mathcal{S}'$ eine Subbasis von $\Tc'$; so ist $f^{-1} (S_i') \in \Tc$ für alle $S_i'\in \Sc'$.
\end{enumerate}
\end{satz}
\begin{beweis}
Folgt aus dem Verhalten von Urbildern bzgl. $\cup$ und $\cap$.
\end{beweis}
\begin{definition}{Umgebungsbasis}{umgebungsbasis}
Ein Mengensystem $\Bf (x) \sub \Uf (x)$ heißt \textbf{Umgebungsbasis} von $x$, falls für alle $U \in \Uf (x)$ ein $B \in \Bf (x)$ existiert, sodass $B \sub U$ gilt.
\end{definition}
\begin{beispiel}
Sei $(X,d)$ ein metrischer Raum mit $x \in X$. Dann ist $\Bf (x) = \{B_{\frac{1}{n}}(x) \}$ eine Umgebungsbasis von $x$.
\end{beispiel}
\begin{satz}{Stetigkeit durch Umgebungsbasen}{stetigumgebungsbasen}
Eine Abbildung $f: (X, \Tc) \to (Y, \Tc')$ zwischen top. Räumen ist stetig, falls gilt:
Ist für $f(x) \in Y$ $\Bf (f(x))$ eine Umgebungsbasis von $f(x)$, so ist $f^{-1}(B) \in \Uf (x)$ für alle $B \in \Bf (f(x))$.
\end{satz}
\begin{definition}{Abzählbarkeitsaxiome}{abzaehlbarkeitsaxiome}
Sei $(X,\Tc)$ ein top. Raum. Die Abzählbarkeitsaxiome für $X$ sind gegeben durch:
\begin{enumerate}[({AZ}1)]
\item Jedes $x \in X$ besitzt eine abzählbare Umgebungsbasis.
\item $\Tc$ besitzt eine abzählbare Basis.
\end{enumerate}
\end{definition}
\begin{beispiele}
\begin{enumerate}
\item Sei $(X,d)$ ein metrischer Raum mit $x \in X$. Die Umgebungsbasis $\Bf (x) = \{B_{\frac{1}{n}}(x)|n \in \N\}$ ist abzählbar, also erfüllt $X$ (AZ1).
\item Betrachte $(\R^n, d)$ mit $d \in  \{d_1,d_2,d_\infty\}$. Dann erfüllt $\R^n$ (AZ2) mit $\mathfrak{B} = \{B_{\frac{1}{m}}(x)| m\in \N, x \in \Q^n \}$.
\end{enumerate}
\end{beispiele}
\begin{bemerkungen}
\begin{enumerate}
\item Es gilt (AZ2) $\implies$ (AZ1): Ist $\Bc = \{B_i\}_{i \in \N}$ eine abzählbare Basis von $\Tc$, so ist 
\begin{equation}
\Bf (x) := \{B \in \Bc | x \in B \}
\end{equation}
eine abzählbare Umgebungsbasis.
\item Es gilt (AZ1) $\cancel{\implies}$ (AZ2): Ist z.B. $X$ überabzählbar mit der diskreten Topologie $(X, \Pc(X))$. Dann ist für alle $x \in X$ $\{x\}$ eine Umgebungsbasis, aber nicht abzählbar.
\end{enumerate}
\end{bemerkungen}
\begin{satz}{Topologisches Folgenkriterium}{topfolgenkrit}
Sei $(X, \Tc)$ ein top. Raum, der (AZ1) erfüllt. Dann gilt:
\begin{enumerate}[(a)]
\item $x \in \overline{A}$ für $A \sub X$ genau dann, wenn eine Folge $(a_n)$ in $A$ existiert, die gegen $x$ konvergiert.
\item $f: X \to Y$ ist stetig in $x \in X$ genau dann, wenn aus $x_n \to x$ auch $f(x_n) \to f(x)$ folgt.
\end{enumerate}
\end{satz}
\begin{beweis}
\begin{enumerate}[(a)]
\item ($\Leftarrow$) gilt immer, der Beweis funktioniert wie in Analysis.\\
($\Rightarrow$): Sei $x \in \overline{A}$ und $\{U_i\}_{i \in \N}$ eine abzählbare Umgebungsbasis von $x$. Wir definieren iterativ $V_1 := U_1$, $V_n := V_{n-1} \cap U_n$. Es gilt $V_i \sub \Uf (x)$ für alle $i$. Also ist $V_i \cap A \neq \emptyset$ für alle $i$. Wir wählen jeweils ein $a_i \in V_i \cap A$ und behaupten $a_n \to x$. Ist $W \in \Uf (x)$ beliebig, so existiert ein $i$, sodass $U_i \in W$. Also ist $V_i \sub U_i \sub W$ und $V_l \in W$ für alle $l \geq i$. Damit folgt $a_l \in W$ für alle $l \geq i$, also $a_n \to x$.
\item ($\Leftarrow$) funktioniert auch wie in Analysis.
($\Rightarrow$): $B \sub Y$ sei abgeschlossen in $Y$. Wähle $x \in \overline{f^{-1}(B)}$. Ziel ist, zu zeigen, dass aus $x \in f^{-1}(B)$ folgt, dass $f^{-1}(B) = \overline{f^{-1}(B)}$ gilt, und daraus wiederum $f^{-1}(B)$ abgeschlossen folgt.\\
Aus (a) folgt, dass eine Folge $(x_n)_{n \in \N}$ in $f^{-1}(B)$ mit $x_n \to x$ existiert. dann ist $f(x_n) \to f(x)$, aber $f(x_n) \in B$ für alle $n$. Damit gilt $f(x) \in B = \overline{B}$, also auch $x \in f^{-1}(B)$.
\end{enumerate}
\end{beweis}
\begin{satz}{Vorstufe des Urysohnschen Einbettungssatzes}{urysohnvorstufe}
Sei $(X, \Tc)$ ein top. Raum, der (AZ2) erfüllt. Dann gibt es eine abzählbare, dichte Teilmenge in $X$.
\end{satz}
\begin{beweis}
$(\Bc_n)_{n \in \N}$ seinen die Basismengen von $\Tc$. Wähle jeweils ein $P_n \in \Bc_n$. Wir behaupten, dass $P := \{P_n | n \in \N \}$ dicht für alle $n$ ist. Sei dafür $x$ beliebig. Für alle $U \in \Uf(x)$ existiert ein $B_i \in \Bc_i$, sodass $x \in B_i \sub U$ gilt. Für alle $x \in X$ existiert also ein $i$, sodass $B_i \sub U$, also $P_i \sub U$ und $P \cap U \neq \emptyset$.
\end{beweis}
\subsection{Vergleich von Topologien}
\label{subsec:vergleichvontopo}
\begin{definition}{feiner und gröber}{feingrob}
Seien $\Tc_1$ und $\Tc_2$ Topologien auf $X$. Dann heißt $\Tc_1$ \textbf{feiner} als $\Tc_2$ und $\Tc_2$ \textbf{gröber} als $\Tc_1$, wenn $\Tc_2 \sub \Tc_1$ gilt, also jedes $O \in \Tc_2$ auch in $\Tc_1$ enthalten ist.
\end{definition}
\begin{bemerkungen}
\begin{enumerate}
\item $\Tc_1$ feiner als $\Tc_2$ gilt genau dann, wenn 
\begin{equation}
\id_X: (X, \Tc_1) \to (X, \Tc_2)
\end{equation}
stetig ist.
\item Allgemein haben es Abbildungen aus $(X, \Tc_1)$ leichter, stetig zu sein, als Abbildungen aus $(X, \Tc_2)$, da für $f: X \to Y$ stetig gelten muss, dass $f^{-1}(O') \in \Tc_1 \supseteq \Tc_2$.
\item Für Abbildungen nach $X$ ist es genau umgekehrt.
\item Es gibt weniger konvergente Folgen in $(X,\Tc_1)$ als in $(X, \Tc_2)$.
\item Die indiskrete Topologie ist die feinste Topologie auf $X$, die diskrete Topologie die gröbste.
\item Topologien auf $X$ bilden eine partiell geordnete Menge.
\end{enumerate}
\end{bemerkungen}
\begin{definition}{Homöomorphismus}{homoeomorphismus}
Sei $f: (X, \Tc) \to (Y, \Tc')$ eine Abbildung zwischen topologischen Räumen. Dann heißt $f$ \textbf{Homöomorphismus}, falls $f$ stetig und bijektiv mit stetiger Umkehrabbildung ist.\\
Existiert ein Homöomorphismus zwischen $X$ und $Y$, so heißen die Räume \textbf{homöomorph}, in Zeichen $X \cong Y$.
\end{definition}
\begin{beispiele}
\begin{enumerate}
\item Die Abbildung
\begin{align}
f: [0,1) &\to \sph^1 = \{x \in \R^2 | \| x\| = 1 \} \sub \R^2\\
t &\mapsto \exp(2 \pi it)
\end{align}
ist stetig und bijektiv, aber $f^{-1}$ ist nicht stetig.
\item Wir betrachten die abgeschlossene Kreisscheibe
\begin{equation}
\D^2 := \{x \in \R^2 | \|x\| \leq 1\}
\end{equation}
ist homöomorph zum Quadrat $[-1,1] \times [-1,1]$. Beispielsweise kann $f: [-1,1]^2 \to \D^2$ durch Reskalieren und $g: \D^2 \to [-1,1]$ durch Aufblasen erreicht werden.
\item Es existiert ein Homöomorphismus 
\begin{align}
f: (-1,1) \sub \R &\to \R\\
x &\mapsto \frac{x}{1-|x|}.
\end{align}
Dieser lässt sich auf $(a,b)$ verallgemeinern, solange $a<b$ gilt.
\item Die \textbf{stereographische Projektion} ist ein Homöomorphismus
\begin{equation}
\varphi: \sph^2 = \{x \in \R^3 \, | \, \| x\|=1 \} \exc \{0,0,1\} \to \C,
\end{equation}
gegeben durch $\varphi(x_1,x_2,x_3) = \left( \frac{x_1}{1-x_3}, \frac{x_2}{1-x_3}\right)$.
\end{enumerate}
\end{beispiele}
\begin{definition}{Abgeschlossene und offene Abbildungen}{abgoffabb}
Eine stetige Abbildung $f: X \to Y$ heißt
\begin{enumerate}
\item \textbf{abgeschlossen}, falls für abgeschlossenes $A \sub X$ auch $f(A) \sub Y$ abgeschlossen ist.
\item \textbf{offen}, falls für offenes $O \sub X$ auch $f(O) \sub Y$ offen ist.
\end{enumerate}
\end{definition}
\begin{satz}{Homöomorphismen sind stetig, bijektiv und offen}{hoemoffenstetbij}
Sei $f: X \to Y$ stetig, bijektiv und offen (oder abgeschlossen). Dann ist $f$ ein Homöomorphismus.
\end{satz}
\begin{beweis}
Sei $O' \sub Y$ offen und 
\begin{equation}
f^{-1}: Y \to X
\end{equation}
die Umkehrabbildung. Dann ist $(f^{-1})^{-1}(O) = f(O)$ offen.
\end{beweis}
\subsection{Unterräume}
\label{subsec:unterraeume}
\begin{definition}{Unterraumtopologie}{unterraumtopo}
Sei $(X,\Tc)$ ein top. Raum und $Y \sub X$. Dann ist die \textbf{Unterraumtopologie} gegeben durch
\begin{equation}
\Tc_{X|Y} := \{O \cap Y|  O \in \Tc\}.
\end{equation}
\end{definition}
\begin{bemerkung}
Das bedeutet, dass $B \sub Y$ genau dann offen ist, wenn ein $O \in \Tc$ existiert, sodass $B = O \cap Y$ gilt. Außerdem ist $A \sub Y$ genau dann abgeschlossen, wenn ein abgeschlossenes $A' \sub X$ mit $A = A' \cap Y$ existiert.
\end{bemerkung}
\begin{satz}{Stetigkeit und Inklusion}{stetinkl}
Seien $(X,\Tc)$ und $(Z,\Tc')$ top. Räume mit $Y \sub X$.
\begin{enumerate}[(a)]
\item Die \textbf{Inklusionsabbildung}
\begin{equation}
\begin{split}
\iota: Y &\mapsto X\\
y &\mapsto \iota(y):=y
\end{split}
\end{equation}
ist stetig.
\item Eine Abbildung $f: Z \to Y$ ist genau dann stetig, wenn $\iota \circ f$ stetig ist.
\end{enumerate}
Dieser Sachverhalt wird durch folgendes kommutierendes Diagramm ausgedrückt:
\begin{center}
\begin{tikzcd}
Z \arrow[r,"f"] \arrow[dr, "\iota \circ f"'] &Y \arrow[d, hook, "\iota"]\\
&X
\end{tikzcd}
\end{center}
\end{satz}
\begin{beweis}
\begin{enumerate}[(a)]
\item Sei $O \sub X$ offen, dann ist $\iota^{-1}(O)=O \cap Y$ offen nach Definition von $\Tc_{X|Y}$.
\item $(\Rightarrow)$: Seien $f$ und $\iota$ stetig. Dann ist die Verkettung $\iota \circ f$ auch stetig.\\
$(\Leftarrow)$: Sei $B \sub Y$ offen, dann ist zu zeigen, dass $f^{-1}(B)$ auch offen ist. Es gilt $B \in \Tc_{X|Y}$ genau dann, wenn ein $O \sub \Tc$ mit $B = O \cap Y$ existiert. Daraus folgt 
\begin{equation}
f^{-1}(B) = f^{-1}(O \cap Y) = f^{-1} (\iota^{-1}(O)) = (\iota \circ f)^{-1} (O),
\end{equation}
was offen nach Annahme ist.
\end{enumerate}
\end{beweis}
\begin{bemerkung}
Sei $(X, \Tc)$ ein top. Raum mit $Z \sub Y \sub X$. Dann gilt $\Tc_{X|Z} = \Tc_{(X|Y)|Z}$.
\end{bemerkung}
\begin{definition}{Einbettung}{einbettung}
Seien $(X', \Tc')$ und $(X,\Tc)$ top. Räume. Eine Abbildung $f: X' \to X$ heißt \textbf{Einbettung}, falls $f$ ein Homöomorphismus auf $\im (f) = f(X')$ ist. Dabei trägt $f(X') \sub X$ die Unterraumtopologie. 
\end{definition}
\begin{beispiele}
\begin{enumerate}
\item Die Funktion
\begin{equation}
\begin{split}
f: [0,1) &\to \R^2 \\
t &\mapsto \exp(2 \pi it)
\end{split}
\end{equation}
ist keine Einbettung, weil $f: [0,1) \to f([0,1)) = \sph^1$ kein Homöomorphismus ist.
\item Sei $\phi: \R \to \R$ eine beliebige, stetige Funktion. Dann ist der \textbf{Graph} von $\phi$, gegeben durch
\begin{equation}
\begin{split}
\Gamma_\phi: \R &\to \R \times \R \\
t &\mapsto (t, \phi(t))
\end{split}
\end{equation}
eine Einbettung.
\item Einbettungen $\sph^1 \to \R^3$ heißen \textbf{Knoten}.
\end{enumerate}
\end{beispiele}
\begin{warning}
Ist $f: X \to W$ stetig und $Y \sub X$ ein Unterraum mit der Topologie $\Tc_{X|Y}$, dann ist $f|_Y = f \circ \iota$ stetig. Die Umkehrung gilt jedoch nicht! Ist z.B. 
\begin{equation}
\begin{split}
f: \R &\to [0,1]\\
t &\mapsto f(t):=\begin{cases} 1, \, t \in \R \exc \Q \\0, \, t \in \Q, \end{cases}
\end{split}
\end{equation}
so ist $f|_{\Q}$ konstant, also auch stetig, aber $f$ nicht.
\end{warning}
\subsection{Trennungsaxiome}
\label{subsec:trennungsaxiome}
\begin{definition}{Trennungsaxiome}{trennungsaxiome}
Die Trennungsaxiome für topologische Räume $(X,\Tc)$ sind gegeben durch:
\begin{enumerate}[({T}1)]
\item Für alle $x,y \in X$, $x \neq y$ existieren Umgebungen $U \sub \Uf(x)$ und $V \sub \Uf(y)$ mit $x \neq V$ und $y \neq U$.
\item Für alle $x,y \in X$ mit $x \neq y$ existieren Umgebungen $U \sub \Uf(x)$ und $V \sub \Uf(y)$ mit $U \cap V = \emptyset$.
\item Für $x \in X$ und abgeschlossenes $x \notin A \sub X$ existieren Umgebungen $U \in \Uf(x)$ und $V \in \Uf(A)$, sodass $U \cap V = \emptyset$.
\item Für abgeschlossene $A,B \sub X$ mit $A \cap B = \emptyset$ existieren Umgebungen $U \in \Uf(A)$ und $V \in \Uf(B)$, sodass $U \cap V = \emptyset$.
\end{enumerate}
(T2) heißt auch \textbf{Hausdorffeigenschaft}. Räume, die (T2) erfüllen, heißen \textbf{Hausdorffräume}.
\end{definition}
\begin{definition}{Reguläre und normale Räume}{regulaernormal}
Sei $(X, \Tc)$ ein top. Raum.
\begin{enumerate}[(a)]
\item Erfüllt $X$ (T2) und (T3), so heißt $X$ \textbf{regulär}.
\item Erfüllt $X$ (T2) und (T4), so heißt $X$ \textbf{normal}.
\end{enumerate}
\end{definition}
\begin{bemerkungen}
\begin{enumerate}
\item Ist ein Raum hausdorffsch, so sind Grenzwerte eindeutig. Also ist z.B. $(X, \{\emptyset, X\})$ ist für $|X| \geq 2$ nicht hausdorffsch.
\item Es gilt (T2) $\implies$ (T1).
\item Jedoch gilt
\begin{equation}
\text{(T4)} \centernot{\implies} \text{(T3)} \centernot{\implies} \text{(T2)},
\end{equation}
betrachte dafür z.B. den obigen nicht-Hausdorffraum. Dieser erfüllt (T3), da es keine abgeschlossenen $A \sub X$ mit $x \notin A$ gibt. Auch (T4) ist erfüllt.
\item (ÜA): Genau dann, wenn $X$ (T1) erfüllt, ist $\{x\}$ abgeschlossen für alle $x \in X$.
\end{enumerate}
\end{bemerkungen}
\begin{satz}{Normalität impliziert Regularität}{normimplreg}
Sei $(X, \Tc)$ ein normaler top. Raum. Dann ist $X$ auch regulär.
\end{satz}
\begin{beweis}
$X$ erfüllt (T2) und (T4), und (T2) impliziert (T1). Alos ist $\{x\}$ abgeschlossen. Ist $A \sub X$ abgeschlossen und $x \notin A$, so sind $\{x\}$ und $A$ disjunkte und abgeschlossene Teilmengen. Also existieren Umgebungen $U \in \Uf(\{x\}) = \Uf(x)$ und $V \in \Uf(A)$ mit $U \cap V = \emptyset$.
\end{beweis}
\begin{satz}{Hausdorffeigenschaft entspricht abgeschlossener Diagonale}{abgdiaghausdorff}
Sei $(X, \Tc)$ ein top. Raum und $\Delta:= \{(x,x)|x \in X\} \sub X \times X$ die \textbf{Diagonale} mit der Topologie $\Tc_{X \times X} := \{O_1 \times O_2 | O_i \in \Tc\}$. $X$ ist genau dann hausdorffsch, falls $\Delta \sub X \times X$ abgeschlossen ist.
\end{satz}
\begin{beweis}
Seien $x, y \in X$ mit $x \neq y$. Das ist äquivalent zu $(x,y) \in (X \times X) \exc \Delta$. Wir zeigen, dass die Hausdorffeigenschaft äquivalent dazu ist, dass das Komplement $(X \times X) \exc \Delta$ offen ist. Äquivalent zu $\Delta$ abgeschlossen ist $(X \times X) \exc \Delta$ offen. Das gilt genau dann, wenn für alle $(x,y) \in (X \times X) \exc \Delta$ ein offenes $O \in X \times X$ existiert mit $(x,y)\in O$. O.B.d.A. sei $O = O_1 \times O_2$, also $x \in O_1$ und $y \in O_2$ mit $O_1 \cap O_2 \neq \emptyset$. Das gilt genau dann, wenn $X$ hausdorffsch ist.
\end{beweis}
\begin{satz}{(T3) und (T4) über Umgebungsbasen}{trennungundbasis}
Sei $(X,\Tc)$ ein top. Raum.
\begin{enumerate}[(a)]
\item $X$ erfüllt (T3) genau dann, wenn für alle $x \in X$ die abgeschlossenen Umgebungen von $x$ eine Umgebungsbasis sind, also für alle $U \in \Uf(x)$ ein abgeschlossenes $B \in \Uf(x) \sub X$  existiert.
\item $X$ erfüllt genau dann (T4), wenn für alle abgeschlossenen $A \sub X$ gilt, dass die abgeschlossenen Umgebungen von $A$ eine Umgebungsbasis von $A$ bilden, also für alle $U \in \Uf(A)$ ein abgeschlossenes $B \sub \Uf(A)$ existiert, sodass $A \sub B \sub C$ gilt.
\end{enumerate}
\end{satz}
\begin{bemerkung}
Sei $(X, \Tc)$ ein (T4)-Raum und $A \sub X$ abgeschlossen. Dann existiert für alle $W \in \Uf(A)$ ein offenes $U \in \Uf(A)$, sodass $A \sub U \sub \overline{U} \sub \overline{B}=B$.
\end{bemerkung}
\begin{beweis}
Die Rückrichtung ist eine ÜA, wir zeigen die Hinrichtung:
\begin{enumerate}[(a)]
\item (T3) gelte. Seien $x \in X$ und $W \in \Uf(x)$ gegeben. Daraus folgt, dass ein $O \in \Tc$ mit $x \in O \sub W$ existiert. $X \exc O$ ist abgeschlossen und $x \notin X \exc O$. Mit (T3) folgt dann, dass $U \in \Uf(x)$ und $V \in \Uf(X \exc O)$ mit $U \cap V = \emptyset$ existieren. Aus $V \in \Uf(X \exc O)$ folgt, dass ein $O' \in \Tc$ mit $(X \exc O) \sub O' \sub V$ existiert. Wir behaupten, dass $X \exc O' =: B$ die gewünschten Eigenschaften hat.
\begin{enumerate}[(i)]
\item $x \in X \exc O'$: Wir haben $x \in U$ mit $U \cap V = \emptyset$, also ist $x \in X \exc V \sub X \exc O'$.
\item Es gilt $U \sub (X \exc V) \sub (X \exc O')$ mit $U \in \Uf(x)$. Obermengen von Umgebungen sind selbst wieder Umgebungen.
\item $X \exc O' \sub X \exc (X \exc O) = O \sub W$.
\end{enumerate}
\item Analog zu (a), wir ersetzen $x$ durch $A$.
\end{enumerate}
\end{beweis}
\begin{satz}{Vererbung der Trennungsaxiome auf Unterräume}{vererbungtrennung}
Sei $(X, \Tc)$ ein top. Raum.
\begin{enumerate}[(a)]
\item Ist $X$ hausdorffsch (oder regulär) und $Y \sub X$, so ist $(Y, \Tc_{X|Y})$ ebenfalls hausdorffsch (oder regulär).
\item Ist $X$ normal und $Y \sub X$ abgeschlossen, so ist auch $(Y, \Tc_{X|Y})$ normal.
\end{enumerate}
\end{satz}
\begin{bemerkung}
Wo liegt das Problem bei (b)? Seien $A,B \sub Y$ mit $A \cap B = \emptyset$. Dann ist $A= A' \cap Y$ und $B = B' \cap Y$. Ist $Y \sub X$ nicht abgeschlossen, so müssen $A$ und $B$ in $X$ nicht abgeschlossen sind.
\end{bemerkung}
\begin{beweis}
\begin{enumerate}[(a)]
\item Wir zeigen zuerst hausdorffsch. Seien $x_1,x_2 \in Y$ mit $x_1 \neq x_2$. Dann ist \\
Nun noch Regularität. Sei $x \in Y$ und $A' \sub Y$ abgeschlossen mit $x \neq A'$. Dann existiert ein abgeschlossenes $A$ in $X$ mit $A' = A \cap Y$. Aus $A$ abgeschlossen und $x \in X$ folgt, dass $x \neq A$ ist. Da $X$ regulär ist, existieren $U \in \Uf^X(x)$ und $V \in \Uf^X(A)$ mit $U \cap V = \emptyset$. Setze $U':= U \cap Y$ und $V' = V\cap Y$. Dann ist $U' \in \Uf^Y(x)$ und $V' \in \Uf^Y(A')$ mit $U' \cap V' = \emptyset$.
\item Sei $Y \sub X$ abgeschlossen und $A', B' \sub Y$ abgeschlossen mit $A' \cap B' = \emptyset$. Da $X$ normal ist, existieren abgeschlossene $A,B \sub X$ mit $A' = A \cap Y$ und $B' = B \cap Y$. Da $Y$ abgeschlossen ist, sind $A'$ und $B'$ auch abgeschlossen in $X$. Wegen (T3) sind $A'$ und $B'$ trennbar mit $U \in \Uf^X(A')$ und $V \in \Uf^X(B')$, sodass $U \cap V = \emptyset$. Dann sind die Mengen trennbar in $Y$ mit $U \cap Y$ und $V \cap Y$.
\end{enumerate}
\end{beweis}
\begin{theorem}{Fortsetzungssatz von Tietze}{tietze}
Sei $X$ normal und $A \sub X$ abgeschlossen. Betrachte eine \textcolor{red}{streng monotone} Funktion
\begin{equation}
f: A \to \R.
\end{equation}
Dann existiert eine stetige Fortsetzung
\begin{equation}
F: X \to \R
\end{equation}
mit $F|_A = f$.
\end{theorem}
Dieses Theorem wollen wir nicht beweisen.
\begin{bemerkung}
Der Fortsetzungssatz kann verstärkt werden: Ist $f$ nicht konstant, so gilt $\sup F = \sup f$ und $\inf F = \inf f$ mit $\inf f < F(x) < \sup f$ für alle $x \notin A$.
\end{bemerkung}
\begin{theorem}{Lemma von Urysohn}{urysohnlemma}
Sei $(X,\Tc)$ ein top. Raum. Dann sind folgende Aussagen äquivalent:
\begin{enumerate}[(a)]
\item $X$ erfüllt (T4).
\item Für alle abgeschlossenen Teilmengen $A \sub X$ und alle offenen Teilmengen $U$ mit $A \sub U \sub X$ gilt: Es existiert eine \textbf{Urysohn-Funktion}:
\begin{equation}
\begin{split}
f: X &\to [0,1] \sub \R \\
x &\mapsto f(x):=\begin{cases} 0 \ \text{für} \ x \in A \\ 1 \ \text{für} \ x \in X \exc U ,\end{cases}
\end{split}
\end{equation}
die stetig auf $U \exc A$ ist.
\end{enumerate}
\end{theorem}
\begin{beweis}
\begin{enumerate}
\item (b $\implies$ a): Sei $f$ eine Urysohn-Funktion und $A,B \sub X$ abgeschlossen. Dann ist $U := X \exc B$ offen und $A \sub U \sub X$. dann trennen $U_1 := f^{-1}[0,\frac{1}{2})$ und $U_2 := f^{-1} (\frac{1}{2}, 1]$ die Mengen $A$ und $B$.
\item (a $\implies$ b): Dies beweisen wir konstruktiv. Sei $A \sub U \sub X$ gegeben, $A$ abgeschlossen, $U$ offen, und $X$ erfülle (T4). Setze $U_0 := A$ und $U_1 := U$. Gemäß Satz \ref{trennungundbasis} existiert offenes $U_{\frac{1}{2}}$ mit $A =U_0 \sub U_\frac{1}{2} \sub \overline{U}_\frac{1}{2} \sub U_1 = U$. Nun konstruieren wir eine $2$-adische Schachtelung: Der nächste Schritt ist
\begin{equation}
A = U_0 \sub U_\frac{1}{4} \sub \overline{U}_\frac{1}{4} \sub U_\frac{1}{2} \sub \overline{U}_\frac{1}{2} \sub U_\frac{3}{4} \sub \overline{U}_\frac{3}{4} \sub U_1 = U.
\end{equation}
Der $n+1$-te Schritt ist dann gegeben durch 
\begin{equation}
A=U_0 \sub \cdots \sub U_\frac{2k-1}{2^{n+1}} \sub \overline{U}_\frac{2k-1}{2^{n+1}} \sub U_\frac{k}{2^n} \sub \overline{U}_\frac{k}{2n} \sub U_\frac{k+1}{2^n} \sub \cdots \sub U_1 = U.
\end{equation}
Wir setzen $D:= \left\{\frac{i}{2^n} | n, i \in \N, 0 \leq i \leq 2^n \right\}$. Sei $r,s \in D$ mit $r<s$, also $\overline{U}_r \sub U_s$. Für $t \in [0,1]$ sei $U_t := \bigcup_{r \leq t, r \in D} U_r$. Dann erhalten wir eine Urysohn-Funktion durch
\begin{equation}
f(x):= 
\begin{cases}
\inf_{x \in U_t} \{0 \leq t \leq 1\} \ &\text{für} \ x \in U\\
1 \ &\text{für} \ x \notin U
\end{cases}
\end{equation}
gegeben. Für $x \in A = U_0$ ist $f(x)=0$. Ist $f(x)<t$, so existiert $r \in D$ mit $r<t$, sodass $x \in U_r$. Ist $f(x)>t$, existiert $r' \in D$ mit $r'>t$ und $x \notin \overline{U}_r$. Also sind 
\begin{equation}
f^{-1}([0,t)) = \bigcup_{r \in D, r <t} U_r
\end{equation}
und
\begin{equation}
f^{-1}((t,1]) = \bigcup_{r' \in D, r' > t} (X \exc \overline{U}_{r'})
\end{equation}
offen. Damit ist 
\begin{equation}
test
\end{equation}
eine Subbasis von $[0,1] \sub \R$.
\end{enumerate}
\end{beweis}
\subsection{Initialtopologie und Produkte}
\label{subsec:initialtopoprodukte}
Wir wollen die Unterraumtopologie verallgemeinern:
\begin{definition}{Initialtopologie}{initialtopo}
Gegeben sei eine Menge $X$ mit einer Familie topologischer Räume $(X_i, \Tc_i)_{i \in I}$ und Abbildungen $f_i: X \to X_i$ für alle $i \in I$. Eine Topologie $\Tc$ auf $X$ heißt \textbf{Initialtopologie}, falls folgendes gilt:\\
Ist $Y$ ein beliebiger top. Raum und $g: Y \to (X, \Tc)$ eine Abbildung. Dann ist $g$ stetig genau dann, wenn $f_i \circ g$ stetig für alle $i \in I$ ist. Dies ist äquivalent zu folgendem kommutierenden Diagramm:
\begin{center}
\begin{tikzcd}
Y \arrow[r,"g"] \arrow[dr, "f_i \circ g"'] &X \arrow[d, "f_i"]\\
&X_i
\end{tikzcd}
\end{center}
\end{definition}
\begin{satz}{Eindeutigkeit und Feinheit der Initialtopologie}{eindeutfeintopo}
Sei $X$ und $(f_i: X \to X_i)_{i \in I}$ gegeben. 
\begin{enumerate}
\item Dann gibt es auf $X$ \textcolor{red}{genau eine} Initialtopologie bezüglich der $(f_i)_{i \in I}$.
\item Die Initialtopologie ist die gröbste Topologie auf $X$, sodass alle $(f_i)_{i\in I}$ stetig sind.
\item Eine Subbasis ist $\Sc := \bigcup_{i \in I} \{f_i^{-1} (O_i) | O_i \in \Tc_i\}$.
\end{enumerate}
\end{satz}
\begin{beweis}
\begin{itemize}
\item $f_i$ ist stetig, falls die Initaltopologie $\Tc$ auf $X$ existiert. Betrachte folgendes kommutierendes Diagramm:
\begin{center}
\begin{tikzcd}
(X,\Tc) \arrow[r,"g=\id_X"] \arrow[dr, "f_i"'] &(X,\Tc) \arrow[d, "f_i"]\\
&(X_i, \Tc_i)
\end{tikzcd}
\end{center}
Die Abbildung $\id_X: (X,\Tc) \to (X, \Tc)$ ist immer stetig. Dann folgt, dass alle $f_i$ stetig sind, denn $f_i = f_i \circ \id_X$.
\item Eindeutigkeit: Angenommen, es gibt zwei $\Tc_1$ und $\Tc_2$ auf $X$. Betrachte
\begin{center}
\begin{tikzcd}
(X,\Tc_1) \arrow[r,"\id_X"] \arrow[dr, "f_i \circ g"'] &(X,\Tc_2) \arrow[d, "f_i"]\\
&X_i,
\end{tikzcd}
\end{center}
also ist $\Tc_1 = \Tc_2$.
\item Subbasis: Da $f_i$ stetig ist, muss $f^{-1}(O_i)$ offen sein in $(X, \Tc)$ mit der Initialtopologie $\Tc$. Also ist $\Sc \sub \Tc$. $\Tc$ war die gröbste Topologie, also $\langle \Sc \rangle = \Tc$. Zu zeigen ist nun, dass $(X,\Tc)$ mit $\Tc = \langle \Sc \rangle$ die universelle Eigenschaft erfüllt, also 
\begin{center}
\begin{tikzcd}
Y \arrow[r,"g"] \arrow[dr, "f_i \circ g"'] &(X, \Tc) \arrow[d, "f_i"]\\
&X_i
\end{tikzcd}
\end{center}
ist genau dann stetig, wenn $f_i \circ g$ stetig für alle $i \in I$ ist.
\item Klar ist, dass aus der Stetigkeit von $g$ folgt, dass $f_i \circ g$ stetig ist, weil alle $f_i$ stetig sind. 
\item Wir nehmen an, dass $f_i \circ g$ stetig ist für alle $i \in I$. $g$ ist stetig genau dann, wenn $g^{-1}(O)$ offen in $Y$ für  alle $O \in \Sc$ ist. Es genügt $O=f_i^{-1}(O_i)$ für beliebiges $i \in I$ zu zeigen. Es gilt:
\begin{equation}
g^{-1}(f_i^{-1}(O_i)) = (g^{-1} \circ f_i^{-1})(O_i) = \underbrace{(f_i \circ g)^{-1}}_\text{stetig} \underbrace{(O_i)}_{\in \Tc_i}
\end{equation}
ist offen.
\end{itemize}
\end{beweis}
Wir wollen nun Produkte von Mengen betrachten. Sei $(X_i)_{i \in I}$ eine Familie von Mengen. Betrachte das kartesische Produkt
\begin{equation}
X = \prod_{i \in I} X_i
\end{equation}
der Mengen $X_i$. Sei $x = (x_i)_{i \in I}$ mit $x_i \in X_i$. Dann betrachten wir die \textbf{Projektionsabbildung}
\begin{equation}
\begin{split}
\pi_j: X &\to X_j\\
(x_i)_{i \in I} &\mapsto x_j
\end{split}
\end{equation}
$X = \prod X_i$ hat die universelle Eigenschaft:
Die Abbildung $g: Z \to \prod X_i$ ist bijektiv mit Umkehrung $g_i: Z \to X_i$ für alle $i \in I$. Man setzt dafür $g_i \mapsto g(z) = (g_i(z))_{i \in I}$. Ziel ist, eine Topologie auf $X = \prod X_i$ zu definieren, falls alle $X_i$ top. Räume sind.
\begin{definition}{Produkte}{produkte}
Sei $(X_i, \Tc_i)_{i \in I}$ eine Familie topologischer Räume und $\prod_{i \in I} X_i =: X$ ihr Produkt (als Mengen). Dann heißt die Initialtopologie bezüglich der Projektion
\begin{equation}
\left( \pi_i: \prod_{j \in I} X_j \to X_i \right)_{i \in I}
\end{equation}
\textbf{Produkttopologie} auf $X$.
\end{definition}
\begin{satz}{Basis der Produkttopologie}{Produktbasis}
Die Produkttopologie auf $X = \prod_{i \in I} X_i$ hat die Basis 
\begin{equation}
\Bc := \left\{\prod_{i \in I} O_i | O_i \in \Tc_i, O_i = X_i \ \text{für fast alle} \ i \in I \right\}.
\end{equation}
\end{satz}
\begin{beweis}
Betrachte die Projektion
\begin{equation}
\pi_i: \prod_{j \in I} X_j \to X_i.
\end{equation}
Für $O_i \in \Tc_i$ ist $\pi^{-1}_i (O_i) = \prod_{j \in I} O_j$ mit 
\begin{equation}
O_j = \begin{cases} O_i \ &\text{für} \ j = 1\\ X_j \ &\text{für} \ j \neq i.\end{cases}
\end{equation}
Endliche Schnitte davon geben die Elemente der Basis und das ist $\Bc$.
\end{beweis}
\begin{satz}
Zu jeder Familie stetiger Abbildungen $(f_i: T \to X_i)_{i \in I}$ gibt es genau eine stetige Funtkion $f: T \to \prod X_i$ mit $\pi_i \circ f = f_i$ für alle $i \in I$, also kommutiert das folgende Diagramm:
\begin{center}
\begin{tikzcd}
T \arrow[r,"???"] \arrow[dr, "f_i"'] &\prod_{j \in I}X_j \arrow[d, "\pi_i"]\\
&X_i
\end{tikzcd}
\end{center}
\end{satz}
\begin{beweis}
\begin{enumerate}[(i)]
\item $f$ existiert als Abbildung von Mengen wegen der universellen Eigenschaft des kartesischen Produktes von Mengen.
\item Dieses $f$ ist automatisch stetig wegen der universellen Eigenschaft der Initialtopologie: $f_i$ sind alle stetig, $\pi_i$ sind stetig.
\end{enumerate}
\end{beweis}
\section{Algebraische Topologie}
\label{algtopo}
\subsection{...}
\label{subsec:Umfn}

\end{document}
