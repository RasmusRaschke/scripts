%%%Import packages here%%%
\usepackage[german]{babel}
\usepackage{graphicx}
\usepackage{framed}
\usepackage[normalem]{ulem}
\usepackage{indentfirst}
\usepackage{amsmath,amsthm,amssymb,amsfonts}
\usepackage{mathtools} % Wraparound amsmath. For fancy math typesetting
\usepackage[nointegrals]{wasysym} % nointegrals prevents wasysym from overwriting integral symbols from LaTeX and amsmath
\usepackage{bbm} % For extended bold and blackboard bold characters
\usepackage[italicdiff]{physics} % italicdiff causes derivatives to be rendered with italic d's instead of upright d's
\usepackage[T1]{fontenc}
\usepackage{xparse}
\usepackage{xstring}
\usepackage{float}
\usepackage{stmaryrd}
\usepackage{enumerate}
\usepackage{grffile}
\usepackage{cancel}
\usepackage{centernot}
\usepackage{hyperref}
\hypersetup{
    colorlinks=true,
    linkcolor=blue,
    filecolor=magenta,      
    urlcolor=cyan,
    pdftitle={script},
    }

\urlstyle{same}
%Incscape
\usepackage{import}
\usepackage{xifthen}
\usepackage{pdfpages}
\usepackage{transparent}

\numberwithin{equation}{subsection}
\usepackage{pifont} % For unusual symbols
\usepackage{mathdots} % For unusual combinations of dots
\usepackage{wrapfig}
\usepackage{lmodern,mathrsfs}
\usepackage[inline,shortlabels]{enumitem}
\setlist{topsep=2pt,itemsep=2pt,parsep=0pt,partopsep=0pt}
%\usepackage[table,dvipsnames]{xcolor}
\usepackage[utf8]{inputenc}
\usepackage{csquotes} % Must be loaded AFTER inputenc
\usepackage[a4paper,top=0.5in,bottom=0.2in,left=0.5in,right=0.5in,footskip=0.3in,includefoot]{geometry}
\usepackage[most]{tcolorbox}
\usepackage{tikz,tikz-3dplot,tikz-cd,tkz-tab,tkz-euclide,pgf,pgfplots}
\usetikzlibrary{babel}
\pgfplotsset{compat=newest}
% \usepackage{comment} % For commenting large blocks of text and math efficiently
% \usepackage{fancyvrb} % For custom verbatim environments
\usepackage{multicol}
\usepackage[bottom,multiple]{footmisc} % Ensures footnotes are at the bottom of the page, and separates footnotes by a comma if they are adjacent
\usepackage[backend=bibtex,style=numeric]{biblatex}
\renewcommand*{\finalnamedelim}{\addcomma\addspace} % Forces authors' names to be separated by comma, instead of "and"
\addbibresource{bibliography}
\usepackage[nameinlink]{cleveref} % nameinlink ensures that the entire element is clickable in the pdf, not just the number

\newcommand{\remind}[1]{\textcolor{red}{\textbf{#1}}} % To remind me of unfinished work to fix later
\newcommand{\hide}[1]{} % To hide large blocks of code without using % symbols
% Same as \href, but the text appears in typewriter font and in a custom color
\newcommand{\Href}[3][red!50!black]{\href{#2}{\textcolor{#1}{\texttt{#3}}}}

\newcommand{\ep}{\varepsilon}
\newcommand{\vp}{\varphi}
\newcommand{\lam}{\lambda}
\newcommand{\Lam}{\Lambda}
\DeclareDocumentCommand\ip{ l m }{\braces#1{\langle}{\rangle}{#2}} % Inner product ⟨x,y⟩ (but only one argument is taken, so \ip{x,y} renders as ⟨x,y⟩)
\DeclareDocumentCommand\floor{ l m }{\braces#1{\lfloor}{\rfloor}{#2}} % Floor function ⌊x⌋
\DeclareDocumentCommand\ceil{ l m }{\braces#1{\lceil}{\rceil}{#2}} % Ceiling function ⌈x⌉

%import incscape figures easily
\graphicspath{{./figures}}
\newcommand{\incfig}[1]{%
    \def\svgwidth{0.2\columnwidth}
    \input{./figures/#1.pdf_tex}
}

% Shortcuts for blackboard bold letters, e.g. \A outputs \mathbb{A}
\newcommand{\A}{\mathbb{A}}
		\newcommand{\B}{\mathbb{B}}
		\newcommand{\C}{\mathbb{C}}
		\newcommand{\D}{\mathbb{D}}
		\newcommand{\E}{\mathbb{E}}
		\newcommand{\F}{\mathbb{F}}
		\newcommand{\G}{\mathbb{G}}
		\renewcommand{\H}{\mathbb{H}}
		\newcommand{\I}{\mathbb{I}}
		\newcommand{\J}{\mathbb{J}}
		\newcommand{\K}{\mathbb{K}}
		\renewcommand{\L}{\mathbb{L}}
		\newcommand{\M}{\mathbb{M}}
		\newcommand{\N}{\mathbb{N}}
		\renewcommand{\O}{\mathbb{O}}
		\renewcommand{\P}{\mathbb{P}}
		\newcommand{\Q}{\mathbb{Q}}
		\newcommand{\R}{\mathbb{R}}
		\renewcommand{\S}{\mathbb{S}}
		\newcommand{\T}{\mathbb{T}}
		\newcommand{\U}{\mathbb{U}}
		\newcommand{\V}{\mathbb{V}}
		\newcommand{\W}{\mathbb{W}}
		\newcommand{\X}{\mathbb{X}}
		\newcommand{\Y}{\mathbb{Y}}
		\newcommand{\Z}{\mathbb{Z}}

\def\do#1{\csdef{c#1}{\mathcal{#1}}}
\docsvlist{A,B,C,D,E,F,G,H,I,J,K,L,M,N,O,P,Q,R,S,T,U,V,W,X,Y,Z}

\def\do#1{\csdef{#1c}{\mathcal{#1}}}
\docsvlist{A,B,C,D,E,F,G,H,I,J,K,L,M,N,O,P,Q,R,S,T,U,V,W,X,Y,Z}

% Shortcuts for suetterlin letters
\def\do#1{\csdef{#1f}{\mathfrak{#1}}}
\docsvlist{A,B,C,D,E,F,G,H,I,J,K,L,M,N,O,P,Q,R,S,T,U,V,W,X,Y,Z}
		\newcommand{\fa}{\mathfrak{a}}
		\newcommand{\fb}{\mathfrak{b}}
		\newcommand{\fc}{\mathfrak{c}}
		\newcommand{\fd}{\mathfrak{d}}
		\newcommand{\fe}{\mathfrak{e}}
		\newcommand{\ff}{\mathfrak{f}}
		\newcommand{\fg}{\mathfrak{g}}
		\newcommand{\fh}{\mathfrak{h}}
		\newcommand{\ffi}{\mathfrak{i}}
		\newcommand{\fj}{\mathfrak{j}}
		\newcommand{\fk}{\mathfrak{k}}
		\newcommand{\fl}{\mathfrak{l}}
		\newcommand{\fm}{\mathfrak{m}}
		\newcommand{\fn}{\mathfrak{n}}
		\newcommand{\fo}{\mathfrak{o}}
		\newcommand{\fp}{\mathfrak{p}}
		\newcommand{\fq}{\mathfrak{q}}
		\newcommand{\fr}{\mathfrak{r}}
		\newcommand{\fs}{\mathfrak{s}}
		\newcommand{\ft}{\mathfrak{t}}
		\newcommand{\fu}{\mathfrak{u}}
		\newcommand{\fv}{\mathfrak{v}}
		\newcommand{\fw}{\mathfrak{w}}
		\newcommand{\fx}{\mathfrak{x}}
		\newcommand{\fy}{\mathfrak{y}}
		\newcommand{\fz}{\mathfrak{z}}
		
% Shortcuts for stylized letters
\def\do#1{\csdef{s#1}{\mathscr{#1}}}
\docsvlist{A,B,C,D,E,F,G,H,I,J,K,L,M,N,O,P,Q,R,S,T,U,V,W,X,Y,Z}
\def\do#1{\csdef{#1s}{\mathscr{#1}}}
\docsvlist{A,B,C,D,E,F,G,H,I,J,K,L,M,N,O,P,Q,R,S,T,U,V,W,X,Y,Z}


% Shortcuts for letters with a bar on top, e.g. \Abar outputs \overline{A}
\def\do#1{\csdef{bar#1}{\overline{#1}}}
\docsvlist{a,b,c,d,e,f,g,i,j,k,l,m,n,o,p,q,r,s,t,u,v,w,x,y,z,A,B,C,D,E,F,G,H,I,J,K,L,M,N,O,P,Q,R,S,T,U,V,W,X,Y,Z}
% \hbar is already defined as the symbol ℏ (reduced Planck constant)

% Shortcuts for boldface letters, e.g. \Ab outputs \textbf{A}
\def\do#1{\csdef{b#1}{\textbf{#1}}}
\docsvlist{a,b,c,d,e,f,g,h,i,j,k,l,m,n,o,q,r,t,u,w,x,y,z,A,B,C,D,E,F,G,H,I,J,K,L,M,N,O,P,Q,R,S,T,U,V,W,X,Y,Z}
% \pb is already defined (by the physics package) as a 2-argument command, denoting the anticommutator or Poisson bracket, e.g. \pb{A,B} yields {A,B}
% \sb is already defined in the LaTeX kernel. This is a fundamental LaTeX command, DO NOT overwrite it!
% \vb is already defined (by the physics package) as a 1-argument command, for boldface text, e.g. \vb{A} yields \textbf{A}

% Shortcuts for letters with a tilde on top, e.g. \Atil outputs \widetilde{A}
\def\do#1{\csdef{til#1}{\widetilde{#1}}}
\docsvlist{a,b,c,d,e,f,g,h,i,j,k,l,m,n,o,p,q,r,s,t,u,v,w,x,y,z,A,B,C,D,E,F,G,H,I,J,K,L,M,N,O,P,Q,R,S,T,U,V,W,X,Y,Z}

%%%Command section%%%

%General
\newcommand{\emp}{\emptyset}
\newcommand{\exc}{\backslash}
\newcommand{\sub}{\subseteq}
\newcommand{\sups}{\supseteq}
\newcommand{\capp}{\bigcap}
\newcommand{\cupp}{\bigcup}
\newcommand{\kupp}{\bigsqcup}
\newcommand{\cappkn}{\bigcap_{k=1}^n}
\newcommand{\cuppkn}{\bigcup_{k=1}^n}
\newcommand{\kuppkn}{\bigsqcup_{k=1}^n}
\newcommand{\cappa}{\bigcap_{\alpha\in I}}
\newcommand{\cuppa}{\bigcup_{\alpha\in I}}
\newcommand{\kuppa}{\bigsqcup_{\alpha\in I}}
\renewcommand{\>}{\rangle}
\newcommand{\<}{\langle}
\newcommand{\id}{\text{id}}

%Analysis
\newcommand{\limk}{\lim_{k\to\infty}}
\newcommand{\limm}{\lim_{m\to\infty}}
\newcommand{\limn}{\lim_{n\to\infty}}
\newcommand{\limx}[1][a]{\lim_{x\to#1}}
\newcommand{\limz}[1][{z_0}]{\lim_{z\to#1}}
\newcommand{\liminfm}{\liminf_{m\to\infty}}
\newcommand{\limsupm}{\limsup_{m\to\infty}}
\newcommand{\liminfn}{\liminf_{n\to\infty}}
\newcommand{\limsupn}{\limsup_{n\to\infty}}
\newcommand{\sumkn}{\sum_{k=1}^n}
\newcommand{\sumk}[1][1]{\sum_{k=#1}^\infty}
\newcommand{\summ}[1][1]{\sum_{m=#1}^\infty}
\newcommand{\sumn}[1][1]{\sum_{n=#1}^\infty}
\newcommand{\dx}{\,dx}
\newcommand{\dy}{\,dy}
\newcommand{\dz}{\,dz}
\newcommand{\dt}{\,dt}
\newcommand{\dmu}{\,d\mu}
\newcommand{\dnu}{\,d\nu}
\newcommand{\graph}{\text{graph}}

%Algebra
\newcommand{\quotient}[2]{{\raisebox{.0em}{$#1$}/\raisebox{-.2em}{$#2$}}} %Quotient Space
\newcommand{\invquotient}[2]{{\raisebox{-.2em}{$#1$}\textbackslash\raisebox{.0em}{$#2$}}} %Left Quotient Space
\DeclareMathOperator{\Hom}{\text{Hom}}
\DeclareMathOperator{\ord}{\text{ord}}
\DeclareMathOperator{\Ann}{\text{Ann}}
\newcommand{\spn}{\trigbraces{\operatorname{span}}}
\newcommand{\acts}{\curvearrowright}
\newcommand{\lnormal}{\trianglelefteq}
\newcommand{\rnormal}{\trianglerighteq}
\newcommand{\Sym}{\text{Sym}}
\newcommand{\Symso}{\text{Sym}^\text{SO}}
%Topology

%Differential Geometry
\newcommand{\iprod}{\mathbin{\lrcorner}}
\newcommand{\diffm}{\text{Diff}}
\DeclareMathOperator{\Ric}{Ric}
\DeclareMathOperator{\ric}{ric}
\newcommand{\inj}{\text{inj}}
\DeclareMathOperator{\gst}{g_{st}}
\newcommand{\hodge}{{\star}}

% Shortcuts for inverse hyperbolic functions (and other operators with the same structure)
\def\do#1{\csdef{#1}{\trigbraces{\operatorname{#1}}}}
\docsvlist{
    asinh,acosh,atanh,acoth,asech,acsch,
    arsinh,arcosh,artanh,arcoth,arsech,arcsch,
    arcsinh,arccosh,arctanh,arccoth,arcsech,arccsch,
    sen,tg,cth,senh,tgh,ctgh,
    Re,Im,arg,Arg,im,ker
}


%Maintenance
\newcommand{\sph}{\mathbb{S}} 
\newcommand{\diff}{\mathcal{D}}
\newcommand{\so}[1]{\text{SO}(#1)}
\newcommand{\der}[1]{\text{Der}_{#1}}
\newcommand{\gl}[2]{\text{GL}(#1, #2)}
%Define behavior of the command with one parameter
\newcommand{\cinfa}[1]{\text{C}^\infty (#1)}
%Define behavior of the command with two parameters
\newcommand{\cinfb}[2]{\text{C}^\infty (#1, #2)}
\newcommand{\fracpart}[1]{\frac{\partial}{\partial #1_i}}
\NewDocumentCommand\cinf{ m g }{
  \IfNoValueTF{#2}{\cinfa{#1}}{\cinfb{#1}{#2}}
}
\newcommand{\supp}{$\text{supp}\ $}
\newcommand{\sgn}{\text{sgn}}
\newcommand{\diag}{\text{diag}}

\makeatletter
% Redefining the commands \iff (given by LaTeX), \implies and \impliedby (given by amsmath)
% Math mode is automatically enforced, starred version makes the arrows shorter
\renewcommand{\impliedby}{\@ifstar{\ensuremath{\Longleftarrow}}{\ensuremath{\Leftarrow}}} % Corresponding Unicode character: U+21D0 ⇐
\renewcommand{\implies}{\@ifstar{\ensuremath{\Longrightarrow}}{\ensuremath{\Rightarrow}}} % Corresponding Unicode character: U+21D2 ⇒
\renewcommand{\iff}{\@ifstar{\ensuremath{\Longleftrightarrow}}{\ensuremath{\Leftrightarrow}}} % Corresponding Unicode character: U+21D4 ⇔
\makeatother


%%%Define Boxes%%%

\newtheoremstyle{mystyle}{}{}{}{}{\sffamily\bfseries}{.}{ }{}
\makeatletter
\newenvironment{beweis}[1][\proofname] {\par\pushQED{\qed}{\normalfont\sffamily\bfseries\topsep6\p@\@plus6\p@\relax #1\@addpunct{.} }}{\popQED\endtrivlist\@endpefalse}
\makeatother
\theoremstyle{mystyle}{\newtheorem*{bemerkung}{Bemerkung}}
\theoremstyle{mystyle}{\newtheorem*{bemerkungen}{Bemerkungen}}
\theoremstyle{mystyle}{\newtheorem*{beispiel}{Beispiel}}
\theoremstyle{mystyle}{\newtheorem*{beispiele}{Beispiele}}
\theoremstyle{definition}{\newtheorem*{übung}{Übung}}
\theoremstyle{definition}{\newtheorem*{lösung}{Lösung}}

% Warning environment
\newtheoremstyle{warn}{}{}{}{}{\normalfont}{}{ }{}
\theoremstyle{warn}
\newtheorem*{warning}{\warningsign{0.2}\relax}

% Symbol for the warning environment, designed to be easily scalable
\newcommand{\warningsign}[1]{
    \tikz[scale=#1,every node/.style={transform shape}]{
        \draw[-,line width={#1*0.8mm},red,fill=yellow,rounded corners={#1*2.5mm}] (0,0)--(1,{-sqrt(3)})--(-1,{-sqrt(3)})--cycle;
        \node at (0,-1) {\fontsize{48}{60}\selectfont\bfseries!};
}}

% verbbox environment, for showing verbatim text next to code output (for package documentation and user learning purposes)
\NewTCBListing{verbbox}{ !O{} }{boxrule=1pt,sidebyside,skin=bicolor,colback=gray!10,colbacklower=white,valign=center,top=2pt,bottom=2pt,left=2pt,right=2pt,#1} % Last argument allows more tcolorbox options to be added




\makeatletter
% \fsize stores the current font size but is expandable (and can be called later without using \makeatletter and \makeatother)
\def\figsize{\dimexpr\f@size pt\relax}
\makeatother

%%%Matrix Shortcut%%%

\makeatletter
% Adapted from https://tex.stackexchange.com/a/19700
\def\my@vector #1,#2\@eolst{
    \ifx\relax#2\relax
        #1
    \else
        #1\my@delim
        \my@vector #2\@eolst
    \fi}
\newcommand\vcstring[2][\\]{% Converts comma-separated string to #1-separated string
    \def\my@delim{#1}
        \my@vector #2,\relax\noexpand\@eolst}
\newcommand\cvc[2][p]{% Converts comma-separated string to column vector, optional argument defines matrix brackets
    \def\my@delim{\\}
        \begin{#1matrix} % Empty argument also possible
            \my@vector #2,\relax\noexpand\@eolst
        \end{#1matrix}}
\newcommand\rvc[2][p]{% Converts comma-separated string to row vector, optional argument defines matrix brackets
    \def\my@delim{&}
        \begin{#1matrix} % Empty argument also possible
            \my@vector #2,\relax\noexpand\@eolst
        \end{#1matrix}}
% Matrix environment with variable number of arguments. Adapted from https://davidyat.es/2016/07/27/writing-a-latex-macro-that-takes-a-variable-number-of-arguments/
\newcommand{\mat}[2][p]{
    \def\matrixenvironment{#1matrix} % Specifying the matrix brackets, this has to be done beforehand as '#1' changes under \passtonextarg
    \def\my@delim{&}
        \begin{\matrixenvironment} % Begin matrix environment
            \my@vector #2,\relax\noexpand\@eolst
            \@ifnextchar\bgroup{\passtonextarg}{\end{\matrixenvironment}}% % Pass to next argument (if any), otherwise end matrix environment
}
\newcommand{\passtonextarg}[1]{\\ \my@vector #1,\relax\noexpand\@eolst
    \@ifnextchar\bgroup{\passtonextarg}{\end{\matrixenvironment}}% Passing to next argument
}
\makeatother

\definecolor{tcol_DEF}{HTML}{E40125} % Color for Definition
\definecolor{tcol_PRP}{HTML}{EB8407} % Color for Proposition
\definecolor{tcol_LEM}{HTML}{05C4D9} % Color for Lemma
\definecolor{tcol_THM}{HTML}{1346E4} % Color for Theorem
\definecolor{tcol_COR}{HTML}{7904C2} % Color for Corollary
\definecolor{tcol_REM}{HTML}{18B640} % Color for Remark
\definecolor{tcol_PRF}{HTML}{5A76B2} % Color for Proof
\definecolor{tcol_EXA}{HTML}{21340A} % Color for Example

\tcbset{
tbox_DEF_style/.style={enhanced jigsaw,
    colback=tcol_DEF!10,colframe=tcol_DEF!80!black,,
    fonttitle=\sffamily\bfseries,
    separator sign=.,label separator={},
    sharp corners,top=2pt,bottom=2pt,left=2pt,right=2pt,
    before skip=10pt,after skip=10pt,breakable
},
tbox_PRP_style/.style={enhanced jigsaw,
    colback=tcol_PRP!10,colframe=tcol_PRP!80!black,
    fonttitle=\sffamily\bfseries,
    attach boxed title to top left={yshift=-\tcboxedtitleheight},
    boxed title style={
        boxrule=0pt,boxsep=2.5pt,
        colback=tcol_PRP!80!black,colframe=tcol_PRP!80!black,
        sharp corners=uphill
    },
    separator sign=.,label separator={},
    top=\tcboxedtitleheight,bottom=2pt,left=2pt,right=2pt,
    before skip=10pt,after skip=10pt,drop fuzzy shadow,breakable
},
tbox_THM_style/.style={enhanced jigsaw,
    colback=tcol_THM!10,colframe=tcol_THM!80!black,
    fonttitle=\sffamily\bfseries,coltitle=black,
    attach boxed title to top left={xshift=10pt,yshift=-\tcboxedtitleheight/2},
    boxed title style={
        colback=tcol_THM!10,colframe=tcol_THM!80!black,height=16pt,bean arc
    },
    separator sign=.,label separator={},
    sharp corners,top=6pt,bottom=2pt,left=2pt,right=2pt,
    before skip=10pt,after skip=10pt,breakable
},
tbox_LEM_style/.style={enhanced jigsaw,
    colback=tcol_LEM!10,colframe=tcol_LEM!80!black,
    boxrule=0pt,
    fonttitle=\sffamily\bfseries,
    attach boxed title to top left={yshift=-\tcboxedtitleheight},
    boxed title style={
        boxrule=0pt,boxsep=2pt,
        colback=tcol_LEM!80!black,colframe=tcol_LEM!80!black,
        interior code={\fill[tcol_LEM!80!black] (interior.north west)--(interior.south west)--([xshift=-2mm]interior.south east)--([xshift=2mm]interior.north east)--cycle;
    }},
    separator sign=.,label separator={},
    frame hidden,borderline north={1pt}{0pt}{tcol_LEM!80!black},
    before upper={\hspace{\tcboxedtitlewidth}},
    sharp corners,top=2pt,bottom=2pt,left=5pt,right=5pt,
    before skip=10pt,after skip=10pt,breakable
},
tbox_COR_style/.style={enhanced jigsaw,
    colback=tcol_COR!10,colframe=tcol_COR!80!black,
    boxrule=0pt,
    fonttitle=\sffamily\bfseries,coltitle=black,
    separator sign={},label separator={},
    description font=\normalfont\sffamily,
    description delimiters={(}{)},
    attach title to upper,after title={.\ },
    frame hidden,borderline west={2pt}{0pt}{tcol_COR},
    sharp corners,top=2pt,bottom=2pt,left=5pt,right=5pt,
    before skip=10pt,after skip=10pt,breakable
},
}

\newtcbtheorem[number within=subsection,
    crefname={\color{tcol_DEF!50!black} definition}{\color{tcol_DEF!50!black} definitionen},
    Crefname={\color{tcol_DEF!50!black} Definition}{\color{tcol_DEF!50!black} Definitionen}
    ]{definition}{Definition}{tbox_DEF_style}{}
\newtcbtheorem[use counter from=definition,
    crefname={\color{tcol_PRP!50!black} satz}{\color{tcol_PRP!50!black} sätze},
    Crefname={\color{tcol_PRP!50!black} Satz}{\color{tcol_PRP!50!black} Sätze}
    ]{satz}{Satz}{tbox_PRP_style}{}
\newtcbtheorem[use counter from=definition,
    crefname={\color{tcol_THM!50!black} theorem}{\color{tcol_THM!50!black} theoreme},
    Crefname={\color{tcol_THM!50!black} Theorem}{\color{tcol_THM!50!black} Theoreme}
    ]{theorem}{Theorem}{tbox_THM_style}{}
\newtcbtheorem[use counter from=definition,
    crefname={\color{tcol_LEM!50!black} lemma}{\color{tcol_LEM!50!black} lemmata},
    Crefname={\color{tcol_LEM!50!black} Lemma}{\color{tcol_LEM!50!black} Lemmata}
    ]{lemma}{Lemma}{tbox_LEM_style}{}
\newtcbtheorem[use counter from=definition,
    crefname={\color{tcol_COR!50!black} korollar}{\color{tcol_COR!50!black} korollare},
    Crefname={\color{tcol_COR!50!black} Korollar}{\color{tcol_COR!50!black} Korollare}
    ]{korollar}{Korollar}{tbox_COR_style}{}

\makeatletter
\@namedef{tcolorboxshape@filingbox@ul}#1#2#3{
    (frame.south west)--(title.north west)--([xshift=-\dimexpr#1\relax]title.north east) to[out=0,in=180] ([xshift=\dimexpr#2\relax,yshift=\dimexpr#3\relax]title.south east)--(frame.north east)--(frame.south east)--cycle
}
\@namedef{tcolorboxshape@filingbox@uc}#1#2#3{
    (frame.south west)--(frame.north west)--([xshift=-\dimexpr#2\relax,yshift=\dimexpr#3\relax]title.south west) to[out=0,in=180] ([xshift=\dimexpr#1\relax]title.north west)--([xshift=-\dimexpr#1\relax]title.north east) to[out=0,in=180] ([xshift=\dimexpr#2\relax,yshift=\dimexpr#3\relax]title.south east)--(frame.north east)--(frame.south east)--cycle
}
\@namedef{tcolorboxshape@filingbox@ur}#1#2#3{
    (frame.south east)--(title.north east)--([xshift=\dimexpr#1\relax]title.north west) to[out=180,in=0] ([xshift=-\dimexpr#2\relax,yshift=\dimexpr#3\relax]title.south west)--(frame.north west)--(frame.south west)--cycle
}
\@namedef{tcolorboxshape@filingbox@dl}#1#2#3{
    (frame.north west)--(title.south west)--([xshift=-\dimexpr#1\relax]title.south east) to[out=0,in=180] ([xshift=\dimexpr#2\relax,yshift=-\dimexpr#3\relax]title.north east)--(frame.south east)--(frame.north east)--cycle
}
\@namedef{tcolorboxshape@filingbox@dc}#1#2#3{
    (frame.north west)--(frame.south west)--([xshift=-\dimexpr#2\relax,yshift=-\dimexpr#3\relax]title.north west) to[out=0,in=180] ([xshift=\dimexpr#1\relax]title.south west)--([xshift=-\dimexpr#1\relax]title.south east) to[out=0,in=180] ([xshift=\dimexpr#2\relax,yshift=-\dimexpr#3\relax]title.north east)--(frame.south east)--(frame.north east)--cycle
}
\@namedef{tcolorboxshape@filingbox@dr}#1#2#3{
    (frame.north east)--(title.south east)--([xshift=\dimexpr#1\relax]title.south west) to[out=180,in=0] ([xshift=-\dimexpr#2\relax,yshift=-\dimexpr#3\relax]title.north west)--(frame.south west)--(frame.north west)--cycle
}
\@namedef{tcolorboxshape@railingbox@ul}#1#2#3{
    (frame.south west)--(title.north west)--([xshift=-\dimexpr#1\relax]title.north east)--([xshift=\dimexpr#2\relax,yshift=\dimexpr#3\relax]title.south east)--(frame.north east)--(frame.south east)--cycle
}
\@namedef{tcolorboxshape@railingbox@uc}#1#2#3{
    (frame.south west)--(frame.north west)--([xshift=-\dimexpr#2\relax,yshift=\dimexpr#3\relax]title.south west)--([xshift=\dimexpr#1\relax]title.north west)--([xshift=-\dimexpr#1\relax]title.north east)--([xshift=\dimexpr#2\relax,yshift=\dimexpr#3\relax]title.south east)--(frame.north east)--(frame.south east)--cycle
}
\@namedef{tcolorboxshape@railingbox@ur}#1#2#3{
    (frame.south east)--(title.north east)--([xshift=\dimexpr#1\relax]title.north west)--([xshift=-\dimexpr#2\relax,yshift=\dimexpr#3\relax]title.south west)--(frame.north west)--(frame.south west)--cycle
}
\@namedef{tcolorboxshape@railingbox@dl}#1#2#3{
    (frame.north west)--(title.south west)--([xshift=-\dimexpr#1\relax]title.south east)--([xshift=\dimexpr#2\relax,yshift=-\dimexpr#3\relax]title.north east)--(frame.south east)--(frame.north east)--cycle
}
\@namedef{tcolorboxshape@railingbox@dc}#1#2#3{
    (frame.north west)--(frame.south west)--([xshift=-\dimexpr#2\relax,yshift=-\dimexpr#3\relax]title.north west)--([xshift=\dimexpr#1\relax]title.south west)--([xshift=-\dimexpr#1\relax]title.south east)--([xshift=\dimexpr#2\relax,yshift=-\dimexpr#3\relax]title.north east)--(frame.south east)--(frame.north east)--cycle
}
\@namedef{tcolorboxshape@railingbox@dr}#1#2#3{
    (frame.north east)--(title.south east)--([xshift=\dimexpr#1\relax]title.south west)--([xshift=-\dimexpr#2\relax,yshift=-\dimexpr#3\relax]title.north west)--(frame.south west)--(frame.north west)--cycle
}
\newcommand{\TColorBoxShape}[2]{\expandafter\ifx\csname tcolorboxshape@#1@#2\endcsname\relax
\expandafter\@gobble\else
\csname tcolorboxshape@#1@#2\expandafter\endcsname
\fi}
\makeatother

\tcbset{ % Styles for filingbox, railingbox and flagbox environments
% Adapted from https://tex.stackexchange.com/questions/587912/tcolorbox-custom-title-box-style
filingstyle/ul/.style 2 args={
    attach boxed title to top left={yshift=-2mm},
    boxed title style={empty,top=0mm,bottom=1mm,left=1mm,right=0mm},
    interior code={
        \path[fill=#1,rounded corners] \TColorBoxShape{filingbox}{ul}{9pt}{18pt}{6pt};
    },
    frame code={
        \path[draw=#2,line width=0.5mm,rounded corners] \TColorBoxShape{filingbox}{ul}{9pt}{18pt}{6pt};
    }},
filingstyle/uc/.style 2 args={
    attach boxed title to top center={yshift=-2mm},
    boxed title style={empty,top=0mm,bottom=1mm,left=0mm,right=0mm},
    interior code={
        \path[fill=#1,rounded corners] \TColorBoxShape{filingbox}{uc}{9pt}{18pt}{6pt};
    },
    frame code={
        \path[draw=#2,line width=0.5mm,rounded corners] \TColorBoxShape{filingbox}{uc}{9pt}{18pt}{6pt};
    }},
filingstyle/ur/.style 2 args={
    attach boxed title to top right={yshift=-2mm},
    boxed title style={empty,top=0mm,bottom=1mm,left=0mm,right=1mm},
    interior code={
        \path[fill=#1,rounded corners] \TColorBoxShape{filingbox}{ur}{9pt}{18pt}{6pt};
    },
    frame code={
        \path[draw=#2,line width=0.5mm,rounded corners] \TColorBoxShape{filingbox}{ur}{9pt}{18pt}{6pt};
    }},
filingstyle/dl/.style 2 args={
    attach boxed title to bottom left={yshift=2mm},
    boxed title style={empty,top=1mm,bottom=0mm,left=1mm,right=0mm},
    interior code={
        \path[fill=#1,rounded corners] \TColorBoxShape{filingbox}{dl}{9pt}{18pt}{6pt};
    },
    frame code={
        \path[draw=#2,line width=0.5mm,rounded corners] \TColorBoxShape{filingbox}{dl}{9pt}{18pt}{6pt};
    }},
filingstyle/dc/.style 2 args={
    attach boxed title to bottom center={yshift=2mm},
    boxed title style={empty,top=1mm,bottom=0mm,left=0mm,right=0mm},
    interior code={
        \path[fill=#1,rounded corners] \TColorBoxShape{filingbox}{dc}{9pt}{18pt}{6pt};
    },
    frame code={
        \path[draw=#2,line width=0.5mm,rounded corners] \TColorBoxShape{filingbox}{dc}{9pt}{18pt}{6pt};
    }},
filingstyle/dr/.style 2 args={
    attach boxed title to bottom right={yshift=2mm},
    boxed title style={empty,top=1mm,bottom=0mm,left=0mm,right=1mm},
    interior code={
        \path[fill=#1,rounded corners] \TColorBoxShape{filingbox}{dr}{9pt}{18pt}{6pt};
    },
    frame code={
        \path[draw=#2,line width=0.5mm,rounded corners] \TColorBoxShape{filingbox}{dr}{9pt}{18pt}{6pt};
    }},
railingstyle/ul/.style 2 args={
    attach boxed title to top left={yshift=-2mm},
    boxed title style={empty,top=0mm,bottom=1mm,left=1mm,right=0mm},
    interior code={
        \path[fill=#1] \TColorBoxShape{railingbox}{ul}{3pt}{12pt}{6pt};
    },
    frame code={
        \path[draw=#2,line width=0.5mm] \TColorBoxShape{railingbox}{ul}{3pt}{12pt}{6pt};
    }},
railingstyle/uc/.style 2 args={
    attach boxed title to top center={yshift=-2mm},
    boxed title style={empty,top=0mm,bottom=1mm,left=0mm,right=0mm},
    interior code={
        \path[fill=#1] \TColorBoxShape{railingbox}{uc}{3pt}{12pt}{6pt};
    },
    frame code={
        \path[draw=#2,line width=0.5mm] \TColorBoxShape{railingbox}{uc}{3pt}{12pt}{6pt};
    }},
railingstyle/ur/.style 2 args={
    attach boxed title to top right={yshift=-2mm},
    boxed title style={empty,top=0mm,bottom=1mm,left=0mm,right=1mm},
    interior code={
        \path[fill=#1] \TColorBoxShape{railingbox}{ur}{3pt}{12pt}{6pt};
    },
    frame code={
        \path[draw=#2,line width=0.5mm] \TColorBoxShape{railingbox}{ur}{3pt}{12pt}{6pt};
    }},
railingstyle/dl/.style 2 args={
    attach boxed title to bottom left={yshift=2mm},
    boxed title style={empty,top=1mm,bottom=0mm,left=1mm,right=0mm},
    interior code={
        \path[fill=#1] \TColorBoxShape{railingbox}{dl}{3pt}{12pt}{6pt};
    },
    frame code={
        \path[draw=#2,line width=0.5mm] \TColorBoxShape{railingbox}{dl}{3pt}{12pt}{6pt};
    }},
railingstyle/dc/.style 2 args={
    attach boxed title to bottom center={yshift=2mm},
    boxed title style={empty,top=1mm,bottom=0mm,left=0mm,right=0mm},
    interior code={
        \path[fill=#1] \TColorBoxShape{railingbox}{dc}{3pt}{12pt}{6pt};
    },
    frame code={
        \path[draw=#2,line width=0.5mm] \TColorBoxShape{railingbox}{dc}{3pt}{12pt}{6pt};
    }},
railingstyle/dr/.style 2 args={
    attach boxed title to bottom right={yshift=2mm},
    boxed title style={empty,top=1mm,bottom=0mm,left=0mm,right=1mm},
    interior code={
        \path[fill=#1] \TColorBoxShape{railingbox}{dr}{3pt}{12pt}{6pt};
    },
    frame code={
        \path[draw=#2,line width=0.5mm] \TColorBoxShape{railingbox}{dr}{3pt}{12pt}{6pt};
    }},
flagstyle/ul/.style 2 args={
    interior hidden,frame hidden,colbacktitle=#1,
    borderline west={1pt}{0pt}{#2},
    attach boxed title to top left={yshift=-8pt,yshifttext=-8pt},
    boxed title style={boxsep=3pt,boxrule=1pt,colframe=#2,sharp corners,left=4pt,right=4pt},
    bottom=0mm
    },
flagstyle/ur/.style 2 args={
    interior hidden,frame hidden,colbacktitle=#1,
    borderline east={1pt}{0pt}{#2},
    attach boxed title to top right={yshift=-8pt,yshifttext=-8pt},
    boxed title style={boxsep=3pt,boxrule=1pt,colframe=#2,sharp corners,left=4pt,right=4pt},
    bottom=0mm
    },
flagstyle/dl/.style 2 args={
    interior hidden,frame hidden,colbacktitle=#1,
    borderline west={1pt}{0pt}{#2},
    attach boxed title to bottom left={yshift=8pt,yshifttext=8pt},
    boxed title style={boxsep=3pt,boxrule=1pt,colframe=#2,sharp corners,left=4pt,right=4pt},
    top=0mm
    },
flagstyle/dr/.style 2 args={
    interior hidden,frame hidden,colbacktitle=#1,
    borderline east={1pt}{0pt}{#2},
    attach boxed title to bottom right={yshift=8pt,yshifttext=8pt},
    boxed title style={boxsep=3pt,boxrule=1pt,colframe=#2,sharp corners,left=4pt,right=4pt},
    top=0mm
    }
}

% Box in the shape of a filing divider, position of tab can be ul (up left), uc (up center), ur (up right), dl (down left), dc (down center) or dr (down right). Default is ul (upper left)
\NewTColorBox{filingbox}{ D(){ul} O{black} m O{} }{enhanced,
    top=1mm,bottom=1mm,left=1mm,right=1mm,
    title={#3},
    fonttitle=\sffamily\bfseries,
    coltitle=black,
    filingstyle/#1={#2!10}{#2},
    #4
}

% Box in the shape of a railing bar, position of tab can be ul (up left), uc (up center), ur (up right), dl (down left), dc (down center) or dr (down right). Default is ul (upper left)
\NewTColorBox{railingbox}{ D(){ul} O{black} m O{} }{enhanced,
    top=1mm,bottom=1mm,left=1mm,right=1mm,
    title={#3},
    fonttitle=\sffamily\bfseries,
    coltitle=black,
    railingstyle/#1={#2!10}{#2},
    #4
}

% Box in the shape of a flag, position of tab can be ul (up left), ur (up right), dl (down left) or dr (down right). Default is ul (upper left)
\NewTColorBox{flagbox}{ D(){ul} O{black} m O{} }{enhanced,breakable,
    top=1mm,bottom=1mm,left=1mm,right=1mm,
    title={#3},
    fonttitle=\sffamily\bfseries,
    coltitle=black,
    flagstyle/#1={#2!10}{#2},
    #4
}

\makeatletter
\newcommand*{\CreateSmartLargeOperator}[2]{
% Adapted from https://tex.stackexchange.com/questions/61598/new-command-with-cases-conditionals-if-thens/61600
    % Plain operator (no customization)
    \csdef{LargeOperator@#1@}{\csdef{LargeOperator@#1@Symbol}{\csuse{#1}}}
    % Operator with limits above and below symbol
    \csdef{LargeOperator@#1@l}{\csdef{LargeOperator@#1@Symbol}{\csuse{#1}\limits}}
    % Operato with limits beside symbol
    \csdef{LargeOperator@#1@n}{\csdef{LargeOperator@#1@Symbol}{\csuse{#1}\nolimits}}
    % Inline style operator
    \csdef{LargeOperator@#1@i}{\csdef{LargeOperator@#1@Symbol}{\textstyle\csuse{#1}}}
    % Display style operator
    \csdef{LargeOperator@#1@d}{\csdef{LargeOperator@#1@Symbol}{\displaystyle\csuse{#1}}}
    % Inline style operator with limits above and below symbol
    \csdef{LargeOperator@#1@il}{\csdef{LargeOperator@#1@Symbol}{\textstyle\csuse{#1}\limits}}
    % Inline style operator with limits beside symbol
    \csdef{LargeOperator@#1@in}{\csdef{LargeOperator@#1@Symbol}{\textstyle\csuse{#1}\nolimits}}
    % Display style operator with limits above and below symbol
    \csdef{LargeOperator@#1@dl}{\csdef{LargeOperator@#1@Symbol}{\displaystyle\csuse{#1}\limits}}
    % Display style operator with limits beside symbol
    \csdef{LargeOperator@#1@dn}{\csdef{LargeOperator@#1@Symbol}{\displaystyle\csuse{#1}\nolimits}}

% NOTE: In the command below, ##1 denotes the operator. It is NOT to be used as an argument!
\def\LargeOperatorSpecs@i##1,##2,##3,##4,##5,##6,##7\@nil{
% If no arguments, operate over n from 1 to infinity
    \ifx$##2$\csuse{LargeOperator@##1@Symbol}_{n=1}^{\infty}\else
    % If one argument, operate over n from ##2 to infinity
        \ifx$##3$\csuse{LargeOperator@##1@Symbol}_{n=##2}^{\infty}\else
        % If two arguments, operate over n from ##2 to ##3
            \ifx$##4$\csuse{LargeOperator@##1@Symbol}_{n=##2}^{##3}\else
            % If three arguments, operate over ##2 from ##3 to ##4
                \ifx$##5$\csuse{LargeOperator@##1@Symbol}_{##2=##3}^{##4}\else
                % If four arguments, operate over ##2 and ##3 from ##4 to ##5
                    \ifx$##6$\csuse{LargeOperator@##1@Symbol}_{##2,##3=##4}^{##5}\else
                    % If five arguments, operate over ##2, ##3 and ##4 from ##5 to ##6
                        \csuse{LargeOperator@##1@Symbol}_{##2,##3,##4=##5}^{##6}
                    \fi
                \fi
            \fi
        \fi
    \fi
}

% Flexible "smart" large operator macro with comma-separated arguments and optional argument for formatting. Default is over n from 1 to infinity. Adapted from https://tex.stackexchange.com/a/15722
\expandafter\DeclareDocumentCommand\csname#2\endcsname{ O{} m }{ % New operator macro
\bgroup % Group created to keep operator style (e.g. \limits) local
    \expandafter\ifx\csname LargeOperator@#1@##1\endcsname\relax
    \expandafter\@gobble\else
    \csname LargeOperator@#1@##1\expandafter\endcsname
    \fi
    \expandafter\LargeOperatorSpecs@i#1,##2,,,,,\@nil% % #1 stands in for the first "argument" of \LargeOperatorSpecs@i (the operator), the actual arguments are from ##2 onward
\egroup}
}
\makeatother

% Create the smart large operator #2 based on the large operator #1. For example, \CreateSmartLargeOperator{sum}{Sum} will define \Sum as the smart large operator based on \sum
% Equivalent Unicode characters are given here (but they are NOT the same as the operators)
\CreateSmartLargeOperator{sum}{Sum}             % Large: U+2211 ∑ (no small version)
\CreateSmartLargeOperator{prod}{Prod}           % Small: U+2293 ⊓, Large: U+220F ∏
\CreateSmartLargeOperator{coprod}{Coprod}       % Small: U+2294 ⊔, Large: U+2210 ∐
\CreateSmartLargeOperator{bigcap}{Capp}         % Small: U+2229 ∩, Large: U+22C2 ⋂
\CreateSmartLargeOperator{bigcup}{Cupp}         % Small: U+222A ∪, Large: U+22C3 ⋃
\CreateSmartLargeOperator{bigsqcup}{Kupp}       % Small: U+2294 ⊔, Large: U+2210 ∐
\CreateSmartLargeOperator{bigodot}{Odot}        % Small: U+2299 ⊙ (no large version)
\CreateSmartLargeOperator{bigoplus}{Oplus}      % Small: U+2295 ⊕ (no large version)
\CreateSmartLargeOperator{bigotimes}{Otimes}    % Small: U+2297 ⊗ (no large version)
\CreateSmartLargeOperator{biguplus}{Uplus}      % Small: U+228E ⊎ (no large version)
\CreateSmartLargeOperator{bigwedge}{Wedge}      % Small: U+2227 ∧, Large: U+22C0 ⋀
\CreateSmartLargeOperator{bigvee}{Vee}          % Small: U+2228 ∨, Large: U+22C1 ⋁

\tcolorboxenvironment{beweis}{boxrule=0pt,boxsep=0pt,blanker,
    borderline west={2pt}{0pt}{tcol_PRF},left=8pt,right=8pt,sharp corners,
    before skip=10pt,after skip=10pt,breakable
}
\tcolorboxenvironment{anmerkung}{boxrule=0pt,boxsep=0pt,blanker,
    borderline west={2pt}{0pt}{tcol_REM},left=8pt,right=8pt,
    before skip=10pt,after skip=10pt,breakable
}
\tcolorboxenvironment{anmerkungen}{boxrule=0pt,boxsep=0pt,blanker,
    borderline west={2pt}{0pt}{tcol_REM},left=8pt,right=8pt,
    before skip=10pt,after skip=10pt,breakable
}
\tcolorboxenvironment{beispiel}{boxrule=0pt,boxsep=0pt,blanker,
    borderline west={2pt}{0pt}{tcol_EXA},left=8pt,right=8pt,sharp corners,
    before skip=10pt,after skip=10pt,breakable
}
\tcolorboxenvironment{beispiele}{boxrule=0pt,boxsep=0pt,blanker,
    borderline west={2pt}{0pt}{tcol_EXA},left=8pt,right=8pt,sharp corners,
    before skip=10pt,after skip=10pt,breakable
}

% align and align* environments with inline size
\newenvironment{talign}{\let\displaystyle\textstyle\align}{\endalign}
\newenvironment{talign*}{\let\displaystyle\textstyle\csname align*\endcsname}{\endalign}

\usepackage[explicit]{titlesec}
% Setting the format for sections, subsections and subsubsections
\titleformat{\section}{\fontsize{24}{30}\sffamily\bfseries}{\thesection}{20pt}{#1}
\titleformat{\subsection}{\fontsize{16}{18}\sffamily\bfseries}{\thesubsection}{12pt}{#1}
\titleformat{\subsubsection}{\fontsize{10}{12}\sffamily\large\bfseries}{\thesubsubsection}{8pt}{#1}
% Setting the spacing for sections, subsections and subsubsections
% First argument is the left indent, second argument is the spacing above, third argument is the spacing below
\titlespacing*{\section}{0pt}{5pt}{5pt}
\titlespacing*{\subsection}{0pt}{5pt}{5pt}
\titlespacing*{\subsubsection}{0pt}{5pt}{5pt}

\newcommand{\Disp}{\displaystyle}
\newcommand{\qe}{\hfill\(\bigtriangledown\)}
\DeclareMathAlphabet\mathbfcal{OMS}{cmsy}{b}{n}
\setlength{\parindent}{0.2in}
\setlength{\parskip}{0pt}
\setlength{\columnseprule}{0pt}

\makeatletter
% Modify spacing above and below display equations
\g@addto@macro\normalsize{
    \setlength\abovedisplayskip{3pt}
    \setlength\belowdisplayskip{3pt}
    \setlength\abovedisplayshortskip{0pt}
    \setlength\belowdisplayshortskip{0pt}
}
\makeatother

\makeatletter
% Redefining the title block
\renewcommand\maketitle{
\null % \vspace does not work with nothing above it, so \null is added
\vspace{5mm}
\begingroup % Creating a group to ensure col_stripes is only defined locally, i.e. only for the title
\definecolor{col_stripes}{HTML}{1B0982} % Color of the stripes above and below the title components
    \begin{tcolorbox}[enhanced,blanker,
    borderline horizontal={2pt}{0pt}{col_stripes},
    borderline horizontal={1pt}{-3.5pt}{col_stripes},
    borderline horizontal={2pt}{-8pt}{col_stripes},
    fontupper=\fontfamily{bch},
    halign=flush center,top=10mm,bottom=10mm,after skip=20mm,
    ]
        {\fontsize{24}{28}\bfseries\selectfont\@title}\\
            \vspace{6mm}
        {\fontsize{20}{24}\selectfont\@author}\\
            \vspace{6mm}
        {\fontsize{16}{20}\selectfont\@date}
    \end{tcolorbox}
\endgroup}
% Adapted from https://tex.stackexchange.com/questions/483953/how-to-add-new-macros-like-author-without-editing-latex-ltx?noredirect=1&lq=1
\makeatother
