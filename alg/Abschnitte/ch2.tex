\section{Ringe}
\label{sec:ringe}
\subsection{Ringe, Ideale und Homomorphismen}
\label{subsec:ringeidealehomo}
\begin{definition}{Ring}{ring}
Ein \textbf{Ring} ist ein Tripel $(R, +, \cdot)$, bestehend aus einer Menge $R$ und Abbildungen
\begin{equation}
+: R \times R \to R
\end{equation}
und
\begin{equation}
\cdot: R \times R \to R,
\end{equation}
sodass gilt:
\begin{enumerate}[({R}1)]
\item Das Paar $(R,+)$ ist eine abelsche Gruppe.
\item Für $a,b \in R$ gilt:
\begin{equation}
\begin{split}
a \cdot (b \cdot c) = (a \cdot b) \cdot c \\
(a+b) \cdot c = a\cdot c + b \cdot c \\
a \cdot (b+c) = a \cdot b + a \cdot c
\end{split}
\end{equation}
\item Es existiert ein Einselement $1 \in R$, sodass für alle $a \in R$ gilt:
\begin{equation}
1 \cdot a = a \cdot 1 = a.
\end{equation}
\end{enumerate}
Gilt zusätzlich 
\begin{equation}
a \cdot b = b \cdot a,
\end{equation}
dann heißt der Ring \textbf{kommutativ}.
\end{definition}
\begin{bemerkung}
Manche Autoren definieren Ringe, ohne ein Einselement zu fordern. Dann werden Ringe mit Einselement als \textbf{unitale Ringe} bezeichnet.
\end{bemerkung}
\begin{beispiele}
\begin{enumerate}
\item Der \textbf{Nullring} $\{0\}$ ist ein Ring, da wir insbesondere nicht fordern, dass $0 \neq 1$ gelten muss. Jedoch ist der Nullring (bis auf Umbenennung der Elemente) der einzige Ring mit dieser Eigenschaft.
\item $(\Z, +, \cdot)$ bildet einen kommutativen Ring.
\item Jeder Körper bildet einen kommutativen Ring.
\item Für $\alpha \in \C$ ist die Menge 
\begin{equation}
\Z [\alpha] := \left\{ \sum_{k=0}^w \lambda_k \alpha^k \, | \, k \in \N, \lambda_k \in \Z \right\} \sub \C
\end{equation}
abgeschlossen unter Addition und Multiplikation. Die Ringaxiome werden von $(\C, +, \cdot)$ vererbt, also bildet $\Z[\alpha]$ einen kommutativen Ring, den \textbf{Polynomring über} $\Z$. Besonders wichtig für die Zahlentheorie is der Fall, wenn $\alpha$ eine \textit{ganze algebraische Zahl} ist, also $\alpha$ als Nullstelle eines monischen Polynoms
\begin{equation}
x^n + a_{n-1}x^{n-1} + \cdots + a_1x + a_0
\end{equation}
mit Koeffizienten $a_i \in \Z$ auftritt. Zum Beispiel ist die imaginäre Zahl $i \in \C$ als Nullstelle des Polynoms $x^2+1$ eine ganze algebraische Zahl. Der Ring
\begin{equation}
\Z[i] = \{ a+bi | a,b \in \Z \}
\end{equation}
heißt \textbf{Gaußscher Zahlenring}.
\item Für einen Ring $R$ bildet die Menge $R[x]$ der Polynome mit Koeffizienten in $R$ einen Ring, genannt \textbf{Polynomring über} $R$.
\item Für jeden Körper $\K$ bildet die Menge $M(n,\K)$ der $\K$-wertigen $n \times n$-Matrizen einen Ring mit elementweiser Addition und Matrixmultiplikation.
\item Für Ringe $R$ und $S$ ist das Produkt $R \times S$ wieder ein Ring mit komponentenweisen Verknüpfungen. 
\end{enumerate}
\end{beispiele}
\begin{definition}{Schiefkörper}{schiefkoerper}
Sei $(R,+,\cdot)$ ein Ring. Existiert zusätzlich für alle $a \in R$ ein $a^{-1} \in R$ mit 
\begin{equation}
a \cdot a^{-1} = a^{-1} a = 1,
\end{equation}
so heißt $(R,+,\cdot)$ \textbf{Schiefkörper}.
\end{definition}
\begin{beispiel}
Die \textbf{Quaternionen} $\H$ bilden einen nicht-kommutativen Ring. Da jedes $q \in \H \exc \{0\}$ allerdings ein multiplikatives Inverses besitzt, ist $\H$ ein Schiefkörper.
\end{beispiel}
\begin{bemerkung}
Historisch wurde der Begriff des Rings und zugehörige Definitionen wie Ideale und Moduln in der algebraischen Zahlentheorie des 19. Jahrhunderts eingeführt und entwickelt (\textit{Kummer}, \textit{Noether}, \textit{Dedekind}, \textit{Hilbert}, \dots).\\
Ziel der Einführung der Ringstruktur ist die Verallgemeinerung des Primzahlbegriffs und der Primfaktorzerlegung für ganze algebraische Zahlen. Zum Beispiel zerfällt die Primzahl $2$ in ein Produkt
\begin{equation}
2 = (1+i)(1-i),
\end{equation}
welches in $\Z[i]$ die neue Primfaktorzerlegung von $2$ wird. Die Tatsache, dass $\Z[i]$ noch immer eindeutige Primfaktorzerlegung besitzt, hat direkte zahlentheoretische Konsequenzen. Ein Beispiel dafür ist der sehr elegante Beweis des folgenden Satzes.
\end{bemerkung}
\begin{satz}{Quadratsumme}{quadratsumme}
Eine ganze Zahl $n \in \Z$ ist genau dann eine Summe $a^2+b^2$ von Quadraten mit $a,b \in \Z$, falls gilt: Jeder Primfaktor $p \mid n$ mit $ p \equiv_4 3$ kommt mit gerader Vielfachheit vor. 
\end{satz}
Umgekehrt gibt es algebraische Zahlenringe ohne eindeutige Primfaktorzerlegung, z.B. $\Z [\sqrt{-5}]$:
\begin{equation}
6 = 2 \cdot 3 = (1-\sqrt{-5})(1+\sqrt{-5}).
\end{equation}
\textbf{Ab jetzt vereinbaren wir, dass alle vorkommenden Ringe kommutativ sind.}
\begin{definition}{Ideal}{ideal}
Sei $R$ ein Ring. Eine nicht-leere Teilmenge $\cI \sub R$ heißt \textbf{Ideal}, falls gilt:
\begin{enumerate}[({I}1)]
\item Für alle $x, y \in \cI$ gilt: $x+y \in \cI$.
\item Für alle $r \in R$ und $x \in \cI$ gilt: $r \cdot x \in \cI$.
\end{enumerate}
\end{definition}
\begin{beispiel}Hauptideale\\
Sei $R$ ein Ring und $x \in R$. Dann bildet 
\begin{equation}
(x) := Rx = \{rx | r \in R \} \sub R
\end{equation}
ein Ideal, das von $x$ erzeugte \textbf{Hauptideal}. Allgemeiner ist für jede Teilmenge $M \sub R$ 
\begin{equation}
(M) := \bigcap_{M \sub \cI \sub R} \cI \sub R
\end{equation}
das von $M$ \textbf{erzeugte Ideal}.
\end{beispiel}
\begin{bemerkung}
Jeder vom Nullring verschiedene Ring $R$ besitzt mindestens zwei unterschiedliche Ideale, nämlich $\{0\}$ und $R$ selbst.
\end{bemerkung}
\begin{satz}{Ring $=$ Körper?}{ringkörper}
Ein Ring $R$ ist genau dann ein Körper, wenn $R$ genau zwei verschiedene Ideale besitzt.
\end{satz}
\begin{beweis}
Sei $\K$ ein Körper und $\{0\} \subset \cI \sub \K$ ein Ideal. Wähle $0 \neq x \in \cI$. Dann existiert ein $r \in R$ mit $rx = 1 \in \cI$. Also gilt für alle $s \in \K$: $s \cdot 1 = s \in \cI$, und somit $\cI = \K$.\\
Angenommen, $R$ hat genau zwei Ideale. Sei $0 \neq x \in R$, dann ist $\{0\} \subset (x) \sub R$ ein Ideal. Daraus folgt aber, dass $(x)=R \ni 1$, also existiert ein $r \in R$ mit $rx = 1$.
\end{beweis}
\begin{definition}{Ringhomomorphismus}{ringhomomorphismen}
Eine Abbildung $\phi: R \to S$ zwischen Ringen $R$ und $S$ heißt \textbf{(Ring-)Homomorphismus}, falls gilt:
\begin{enumerate}[({H}1)]
\item Für alle $r_1, r_2 \in R$ gilt:
\begin{align}
\phi(r_1+r_2) &= \phi(r_1)+\phi(r_2)\\
\phi(r_1 \cdot r_2) &= \phi(r_1) \cdot \phi(r_2).
\end{align}
\item Für $1 \in R$, $1' \in S$ gilt:
\begin{equation}
\phi(1) = 1'.
\end{equation}
\end{enumerate}
\end{definition}
\begin{beispiel}
Sei $R$ ein Ring, $R[x]$ der zugehörige Polynomring über $R$ und $\alpha \in R$. Dann ist die \textbf{Auswertungsabbildung}
\begin{equation}
\begin{split}
\text{ev}_\alpha: R[x] &\to R\\
f(x) &\mapsto f(\alpha)
\end{split}
\end{equation}
ein Homomorphismus, genannt \textbf{Einsetzungshomomorphismus}.
\end{beispiel}
\begin{definition}{Unterring}{unterring}
Sei $(R,+,\cdot)$ ein Ring. Eine Teilmenge $U \sub R$ heißt \textbf{Unterring} von $R$, falls $(U, +|_U, \cdot|_U)$ wieder ein Ring ist.
\end{definition}
\begin{satz}{Bild und Kern von Homomorphismen}{ringbildkern}
Sei $\phi: R \to S$ ein Ringhomomorphismus. Dann gelten folgende Aussagen:
\begin{enumerate}[(i)]
\item $\im (\phi) \sub S$ ist ein Unterring.
\item $\ker (\phi) \sub R$ ist ein Ideal.
\end{enumerate}
\end{satz}
\begin{beweis}
Wir beweisen (ii): Seien $x,y \in \ker (\phi)$ und $r \in R$. Dann gilt:
\begin{enumerate}
\item $\phi(x+y) = \cancel{\phi(x)} + \cancel{\phi(y)} = 0$.
\item $\phi(rx) = \phi(r) \cancel{\phi(x)} = 0$.
\end{enumerate}
\end{beweis}
\begin{übung}
Beweise (i).
\end{übung}
\begin{korollar}{Aus \ref{ringbildkern}}{ausringbildkern}
Sei $\K$ ein Körper, $S$ ein Ring und $\phi: \K \to S$ ein Ringhomomorphismus. Dann ist $\phi$ entweder injektiv, oder $S=\{0\}$.
\end{korollar}
\begin{beweis}
Die einzigen Ideale von $\K$ sind $\K$ und $\{0\}$. Für $\ker (\phi) = \{0\}$ ist $\phi$ injektiv. Für $\ker (\phi) = \K$ ist $\phi \equiv 0$. Also folgt $1 = \phi(1) = 0$ und damit $S=\{0\}$, da der Nullring als einziger diese Eigenschaft aufweist.
\end{beweis}
\begin{definition}{Restklassen}{restklassen}
Sei $R$ ein Ring und $\mathcal{I} \sub R$ ein Ideal. Dann definiert
\begin{equation}
r_1 \sim r_2 : \iff r_1 - r_2 \in \mathcal{I}
\end{equation}
eine Äquivalenzrelation auf $R$. Die Äquivalenzklasse eines $r \in R$ hat die Gestalt
\begin{equation}
[r] = \{r+x|x \in \Ic \},
\end{equation}
genannt \textbf{Restklassen modulo} $\Ic$. Die Menge aller Restklassen modulo $\Ic$ wird mit $\quotient{R}{\Ic}$ bezeichnet.
\end{definition}
\begin{satz}{Quotientenringe}{quotientenringe}
Sei $R$ ein Ring und $\mathcal{I} \sub R$ ein Ideal.
\begin{enumerate}[(i)]
\item Die Verknüpfungen 
\begin{equation}
\begin{split}
+: \quotient{R}{\mathcal{I}} \times \quotient{R}{\mathcal{I}} &\to \quotient{R}{\mathcal{I}}\\
([r_1], [r_2]) \mapsto [r_1 + r_2]
\end{split}
\end{equation}
und
\begin{equation}
\begin{split}
\cdot: \quotient{R}{\mathcal{I}} \times \quotient{R}{\mathcal{I}} &\to \quotient{R}{\mathcal{I}}\\
([r_1], [r_2]) \mapsto [r_1 \cdot r_2]
\end{split}
\end{equation}
definieren eine Ringstruktur auf $\quotient{R}{\mathcal{I}}$, genannt \textbf{Quotientenring} von $R$ modulo $\mathcal{I}$.
\item Die Abbildung 
\begin{equation}
\begin{split}
\pi: R &\to \quotient{R}{\mathcal{I}}\\
r &\mapsto [r]
\end{split}
\end{equation}
ist ein Ringhomomorphismus mit $\ker (\pi) = \mathcal{I}$.
\end{enumerate}
\end{satz}
\begin{übung}
Man beweise diesen Satz analog zum Fall für Gruppen.
\end{übung}
\begin{beispiel}
Sei $n \in \N, n \geq 2$. Dann ist $\quotient{\Z}{(n)}$ der Restklassenring modulo $n$. Für $n=p$ prim ist $\Z$
\end{beispiel}
\begin{theorem}{Isomorphiesatz}{ringisomorphiesatz}
Sei $\phi: R \to S$ ein Ringhomomorphismus. Dann ist die induzierte Abbildung
\begin{equation}
\begin{split}
\overline{\phi}: \quotient{R}{\ker (\phi)} &\to \im (\phi)\\
[r] &\mapsto \phi(r)
\end{split}
\end{equation}
ein Ringisomorphismus.
\end{theorem}
\begin{übung}
Man beweise diesen Satz analog zum Fall für Gruppen.
\end{übung}

\subsection{Primelemente und Primideale}
\label{subsec:primelementeideale}
\begin{definition}{Teiler}{teiler}
Sei $R$ ein Ring und $a,b \in R$. Wir sagen, dass $a$ $b$ \textbf{teilt}, wenn ein $r \in R$ mit $b=ar$ existiert, und schreiben $a \mid b$.\\
Für Ideale gilt
\begin{equation}
a \mid b \iff b \in (a) \iff (b) \sub (a).
\end{equation}
Damit führen wir \textit{einen} Teilbarkeitsbegriff für Ideale ein:
\begin{equation}
\Ic \mid \Jc : \iff \Jc \sub \Ic.
\end{equation}
\end{definition}
\begin{definition}{Einheit}{einheit}
Sei $R$ ein Ring. Ein $n \in R$ heißt \textbf{Einheit}, falls es ein $v\in R$ mit $nv = 1$ gibt. Die Einheiten von $R$ bilden eine abelsche Gruppe unter Multiplikation, bezeichnet mit $R^\times$.
\end{definition}
\begin{definition}{Primelement}{primelement}
Ein Element $0 \neq p \in R$ mit $p \notin R^\times$ heißt \textbf{Primelement}, falls für alle $a,b \in R$ gilt:
\begin{equation}
p \mid ab \implies p \mid a \, \text{oder} \, p \mid b.
\end{equation}
\end{definition}
\begin{beispiel}
Die Gruppe der Einheiten in $\Z$ ist $\Z^\times = \{\pm 1\}$ und die Primelemente sind die Zahlen $\pm p$ für $p \in \N$ prim.
\end{beispiel}
\begin{satz}{Gaußsche Zahlen}{gaußzahlen}
Die Einheiten im Ring $\Z[i]$ der Gaußschen Zahlen bilden die Gruppe
\begin{equation}
\Z[i]^\times = \{\pm 1, \pm i\}
\end{equation}
der vierten Einheitswurzeln.
\end{satz}
\begin{beweis}
Wir definieren für $a+bi =:\alpha \in \Z[i)$ die Norm
\begin{equation}
N(\alpha) := \alpha \overline{\alpha} = a^2 + b^2 \in \N.
\end{equation}
Die Norm ist multiplikativ:
\begin{equation}
N(\alpha \beta)= N(\alpha) N(\beta),
\end{equation}
daher gilt für $u,v \in \Z[i)$ mit $uv=1$ auch $N(u)N(v)=1$. Die einzige Gaußsche Zahl mit Norm $<1$ ist $0$. Daher muss jede Einheit Norm $1$ haben. Umgekehrt sind daher auch alle Zahlen der Norm $1$ Einheiten.
\end{beweis}
\begin{definition}{Primideal}{primideal}
Sei $R$ ein Ring. Ein Ideal $\fp \subset R$ heißt \textbf{Primideal}, falls für alle $x,y \in R$ gilt:
\begin{equation}
xy \in \fp \implies x\in \fp \, \text{oder} \, y \in \fp.
\end{equation}
\end{definition}
\begin{satz}{Kriterien für Einheit und Primideal}{kriterieneinprim}
Sei $R$ ein Ring.
\begin{enumerate}[(i)]
\item Ein Element $a \in R$ ist eine Einheit genau dann, wenn $(a) =R$.
\item Ein Element $0 \neq a \in R$ ist ein Primelement genau dann, wenn $(a) \sub R$ ein Primideal ist.
\end{enumerate}
\end{satz}
\begin{übung}
Man beweise den Satz direkt aus den Definitionen.
\end{übung}
\begin{übung}
Sei $R$ ein Ring. Für Ideale $\Ic, \Jc \sub R$ definieren wir das \textbf{Produktideal} durch
\begin{equation}
\Ic \cdot \Jc := \left(\{xy| x \in \Ic, y \in \Jc\}\right).
\end{equation}
Dann ist $\fp \sub R$ ein Primideal genau dann, wenn für alle Ideale $\Ic, \Jc \sub R$ gilt:
\begin{equation}
\fp \mid \Ic \cdot \Jc \implies \fp \mid \Ic \, \text{oder} \, \fp \mid \Jc,
\end{equation}
das heißt: $\Ic \cdot \Jc \sub \fp  \implies \Ic \sub \fp \, \text{oder} \, \Jc \sub \fp$.\\
Ein (Gegen-)Beispiel ist 
\begin{equation}
\Ic + \Jc := \{x+y| x \in \Ic, y \in \Jc\} = (\Ic \cup \Jc) \neq \Ic \cdot \Jc.
\end{equation}
\end{übung}
\begin{definition}{Integritätsbereich und Nullteilerfreiheit}{nullteilerfreiheit}
Ein Ring $R$ heißt \textbf{nullteilerfrei} oder \textbf{Integritätsbereich}, falls $R\neq \{0\}$ und für alle $r,s \in R$ gilt:
\begin{equation}
rs = 0 \implies r=0 \ \text{oder} \ s=0.
\end{equation}
\end{definition}
\begin{satz}{Nullteilerfreiheit und Primideale}{primnullteilerfrei}
Sei $R$ ein Ring und $\Ic \sub R$ ein Ideal. Dann sind äquivalent:
\begin{enumerate}[(i)]
\item $\quotient{R}{I}$ ist nullteilerfrei.
\item $\Ic \sub R$ ist ein Primideal.
\end{enumerate}
\end{satz}
\begin{übung}
Beweise diesen Satz direkt aus den Definitionen.
\end{übung}
\begin{definition}{Irreduzibilität}{irreduzibilität}
Sei $R$ ein Ring. Ein Element $0 \neq r \in R$ mit $r \notin R^\times$ heißt \textbf{unzerlegbar} oder \textbf{irreduzibel}, falls für alle $a,b \in R$ mit $r=ab$ gilt, dass $a \in R^\times$ oder $b \in R^\times$.
\end{definition}
\begin{beispiel}
Jedes Primelement $p \in R$ eines nullteilerfreien Rings $R$ ist irreduzibel. Denn aus $p = ab$ folgt $p \mid a$ oder $p \mid b$, also o.B.d.A. $p \mid a$. Dann existiert $t \in R$, sodass $a = tp$. Dann gilt aber $p=ptb \implies 1 = tb$ wegen der Nullteilerfreiheit, damit also $b \in R^\times$.
\end{beispiel}
\begin{satz}{Irreduzible Elemente erzeugen maximale Hauptideale}{irreduzibelmaximal}
Sei $R$ ein nullteilerfreier Ring. Dann sind für ein $0 \neq r \in R$ äquivalent:
\begin{enumerate}[(i)]
\item $r$ ist irreduzibel.
\item $(r)$ ist maximal unter den echten\footnote{Ein Hauptideal $\Ic$ heißt echt, wenn $\Ic \subset R$.} Hauptidealen von $R$.
\end{enumerate}
\end{satz}
\begin{beweis}
(i) $\implies$ (ii): Sei $r$ irreduzibel und $(r) \sub (s) \sub R$. Dann gibt es also $a \in R$ mit $r = as$. Da $r$ irreduzibel ist, ist entweder $a \in R^\times$ oder $b \in R^\times$. Für $a \in R^\times$ folgt direkt $(r)=(s)$, für $s \in R^\times$ folgt hingegen $(s)=R$. In beiden Fällen ist $(r)$ maximal unter den echten Hauptidealen.\\
(ii) $\implies$ (i): Sei $(r)$ maximal unter den Hauptidealen und $r=ab$ eine Zerlegung von $r$. Dann gilt, dass $(r)\sub (b) \sub R$, also entweder $(r) = (b)$ oder $(b)=R$. Im ersten Fall existiert ein $s \in R$ mit $b=rs$, also $r = asr$. Da $R$ nullteilerfrei ist, folgt $as=1$, also $a \in R^\times$. Im zweiten Fall ist schon $b \in R^\times$, also ist $r$ immer irreduzibel.
\end{beweis}
\begin{definition}{Maximale Ideale}{maximaleideale}
Sei $R$ ein Ring und $\mathfrak{m} \subset R$ ein Ideal. Dieses heißt \textbf{maximal}, wenn es maximal unter allen echten Idealen von $R$ ist, wenn also gilt:
\begin{equation}
\forall \Jc \subset R, \Jc \, \text{Ideal} \, : \mathfrak{m} \sub \Jc \implies \mathfrak{m} = \Jc
\end{equation}
\end{definition}
\begin{satz}{Maximale Ideale erzeugen Körper}{maxidealkörper}
Sei $R$ ein Ring und $\Ic \sub R$ ein Ideal. Dann sind äquivalent:
\begin{enumerate}[(i)]
\item $\Ic \sub R$ ist maximal.
\item $\quotient{R}{\Ic}$ ist ein Körper.
\end{enumerate}
\end{satz}
\begin{beweis}
(i) $\implies$ (ii): Sei $\Ic \subset R$ maximal. Sei $[0] \neq [x] \in \quotient{R}{\Ic}$. Dann gilt: $x \notin \Ic \implies \Ic \subset (I \cup [x]) \sub R$ ist ein Ideal. Da $\Ic$ maximal ist, gilt also $(\Ic \cup [x]) = R$. Dann existiert ein $y \in \Ic$ und ein $r \in R$ mit $y +rx=1$. Also $[r] \cdot [x] = [1]$, womit ein multiplikatives Inverses existiert, also ist $\quotient{R}{\cI}$ ein Körper.\\
(ii) $\implies$ (i): Sei $\quotient{R}{\Ic}$ ein Körper und $\Ic \subset \Jc \sub R$ ein Ideal. Sei $x \in \Jc \setminus \Ic$, dann gilt also $0 \neq [x] \in \quotient{R}{\Ic}$. Also hat $[x]$ ein Inverses $r \in R$ mit $[x] \cdot [r] = [1]$. Daraus folgt, dass $1 = xr +y$, also $1 \in \Jc$ und damit $\Jc = R$.
\end{beweis}
\begin{korollar}{Aus Satz \ref{maxidealkörper}}{ausmaxidealkörper}
Jedes maximale Ideal ist ein Primideal.
\end{korollar}
\begin{beispiele}
\begin{enumerate}
\item Sei $\K$ ein Körper und $\K[x,y] = (\K[x])[y]$ der Polynomring in Variablen $x$ und $y$. Die Gruppe der Einheiten ist $\K[x,y]^\times = \K^\times$. Das Ideal $(y)$ ist gerade der Kern des Evaluationshomomorphismus
\begin{equation}
\begin{split}
\K[x,y] &\to \K[x]\\
f(x,y) &\mapsto f(x,0).
\end{split}
\end{equation}
Nach dem Isomorphiesatz \ref{ringisomorphiesatz} gilt 
\begin{equation}
\quotient{K[x,y]}{(y)} \cong \K[x],
\end{equation}
und weiterhin, da $\K[x]$ nullteilerfrei ist, dass $(y)$ ein Primideal, also $y$ ein Primelement und damit irreduzibel ist. Also ist $(y)$ maximal unter den Hauptidealen gemäß Satz \ref{irreduzibelmaximal}. Da $\K[x]$ kein Körper ist, kann $(y)$ kein maximales Ideal sein. Dies sehen wir auch explizit sofort ein: Sei z.B.
\begin{equation}
(y) \subset (x,y) \subset \K[x,y].
\end{equation}
Weiterhin ist $(x,y)$ der Kern von
\begin{equation}
\begin{split}
\K[x,y] &\to \K\\
f(x,y) &\mapsto f(0,0).
\end{split}
\end{equation}
Also ist $\K \cong \quotient{\K[x,y]}{(x,y)}$, woraus folgt, dass $(x,y)$ maximal ist.
\item Für $\K = \overline{\K}$ gilt der \textbf{Hilbertsche Nullstellensatz}: Die maximalen Ideale in $\K[x,y]$ sind genau die Ideale der Form $(x-a,y-b)$ mit $a,b \in \K$.
\end{enumerate}
\end{beispiele}
\begin{definition}{Hauptidealring}{hauptidealring}
Ein nullteilerfreier Ring $R$ heißt \textbf{Hauptidealring}, falls jedes Ideal ein Hauptideal ist.
\end{definition}
\begin{satz}{}{}
Sei $R$ ein Hauptidealring. Dann sind für $r \in R$ äquivalent:
\begin{enumerate}[(i)]
\item $r$ ist irreduzibel.
\item $r$ ist Primelement.
\end{enumerate}
\end{satz}
\begin{beweis}
(ii) $\implies$ (i): Schon bewiesen.\\
(i) $\implies$ (ii): Sei $r$ irreduzibel. Dann ist $(r)$ maximal unter den Hauptidealen, also ist $(r)$ maximal und damit ein Primideal. Damit muss $r$ ein Primelement sein.
\end{beweis}
\begin{definition}{Euklidische Ringe}{euklidischeringe}
Ein Integritätsbereich $R$ heißt \textbf{euklidischer Ring}, falls eine Abbildung
\begin{equation}
\lambda: R \exc \{0\} \to \N
\end{equation}
mit folgender Eigenschaft existiert: Für alle $a,b \in R$ mit $b \neq 0$ existieren Elemente $q, r \in R$, sodass $a = q \cdot b + r$, wobei entweder $r=0$ oder $\lambda (r) < \lambda (b)$ gilt.
\end{definition}
\begin{satz}{Euklidische Ringe sind HIR}{euklidisthir}
Jeder euklidische Ring ist ein Hauptidealring.
\end{satz}
\begin{beweis}
Sei $\{0\} \neq \Ic \sub R$ ein Ideal. Wähle $0 \neq x \in \Ic$ mit $\lambda (x)$ minimal (Wohlordnung von $\N$). Für $y \in \Ic$ beliebig existieren $q,r \in R$ mit $y = qx +r$, sodass entweder $r=0$ oder $\lambda(r) < \lambda (x)$ gilt. Der letztere Fall ist unmöglich, da $\lambda(x)$ minimal ist. Also gilt $r=0$, woraus $\Ic = (x)$ folgt.
\end{beweis}
\begin{beispiele}
Folgende Ringe sind euklidisch:
\begin{enumerate}
\item Der Ring $(\Z, +, \cdot)$ mit $\lambda (n) = |n|$.
\item Der Polynomring $\K[x]$ mit $\lambda(f) = \deg (f)$, wobei $\K$ ein Körper sein muss.
\item Die gaußschen Zahlen $\Z[i]$ mit $\lambda(\alpha) = N(\alpha) = a^2 +b^2$.
\end{enumerate}
\begin{beweis}
1. und 2. sind bereits aus der Schule als Division mit Rest und Polynomdivision bekannt. Zeigen wir also 3.: Zu gegebenen $\alpha, \beta \in \Z[i]$, mit $\alpha, \beta \neq 0$ müssen wir $q,r \in \Z[i]$ finden, sodass $\alpha = q \beta +r$, wobei entweder $r=0$ oder $N(r) < N(\beta)$ gilt. Fasse $\Z[i] \sub \C$ auf und betrachte die komplexe Zahl $z:= \frac{\alpha}{\beta} \in \C$. Falls $z \in \Z[i]$ gilt, setze $q = z$ und $r=0$. Andernfalls gilt $r = \alpha - q \beta$ und wir suchen $r$ mit $N(r) < N(\beta)$. Das gilt genau dann, wenn $\frac{N(r)}{N(\beta)} = N\left(\frac{r}{\beta}\right) < 1$, was äquivalent zu $| \underbrace{\frac{\alpha}{\beta}}_{=z \in \C} -\underbrace{q}_{\in \Z[i]} | < 1 \, (\ast)$ ist. Dieses $q\in \Z[i]$ gilt es, zu finden. Die Zahl $z$ liegt in einer der quadratischen Maschen, die vom Gitter $\Z[i]$ aufgespannt werden. Einer der Eckpunkte $q$ dieses Quadrats erfüllt immer die Ungleichung $(\ast)$, da $\max d(q,z) = \frac{\sqrt{2}}{2} < 1$.
\end{beweis}
\end{beispiele}
\begin{definition}{assoziiert}{assoziiert}
Sei $R$ ein Integritätsbereich. Zwei Elemente $a,b \in R$ heißen \textbf{assoziiert}, geschrieben $a \sim b$, wenn ein $n \in \R^\times$ mit $a = nb$ existiert.
\end{definition}
\begin{satz}{Assoziiertheit = gleiche Hauptideale}{asshauptideal}
Es gilt $a \sim b$ genau dann, wenn $(a) = (b)$.
\end{satz}
\begin{beweis}
Wir zeigen $\Leftarrow$: Es existiert ein $r \in R$ mit $b =ra$ und ein $s \in R$ mit $a = sb$. Dann gilt $b=rsb \iff (1-rs) b = 0$, wegen der Nullteilerfreiheit also $rs=1$.
\end{beweis}
\begin{definition}{faktoriell}{faktoriell}
Ein Integritätsbereich $R$ heißt \textbf{faktoriell}, falls für jede Nichteinheit $a \in R$ mit $a \neq 0 $ gilt:
\begin{enumerate}[(i)]
\item Es gibt eine Faktorzerlegung 
\begin{equation}
\label{eq:faktoren}
a = p_1p_2 \cdots p_m
\end{equation}
in $m$ irreduzible Faktoren.
\item Falls $a = p_1p_2 \cdots p_m$ und $a = q_1 q_2 \cdots q_n$ Faktorzerlegungen in irreduzible Faktoren sind, dann gilt:
\begin{enumerate}[(a)]
\item $m = n$
\item Nach geeigneter Umbennenung der Faktoren gilt für alle $1 \leq i \leq m$: $p_i \sim q_i$.
\end{enumerate}
\end{enumerate}
\end{definition}
\begin{satz}{Irreduzible Elemente sind Primelemente in Faktorringen}{irredprimfaktor}
Sei $R$ ein Integritätsbereich, in dem jede Nichteinheit $a \neq 0$ eine Faktorzerlegung in Form von Gleichung $\ref{eq:faktoren}$ besitzt. Dann sind äquivalent:
\begin{enumerate}[(a)]
\item $R$ ist faktoriell.
\item Jedes irreduzible Element ist ein Primelement.
\end{enumerate}
\end{satz}
\begin{beweis}
(a) $\implies$ (b): Sei $R$ faktoriell und $r\in R$ irreduzibel. Seien $a,b \in R$ mit $r \mid ab$. Per Definition gibt es ein $s \in R$ mit $sr = ab$. Zerlege alle Teile der Gleichung in irreduzible Elemente, also
\begin{equation}
s_1s_2 \cdots s_t r = a_1 a_2  \cdots a_m b_1 b_2 \cdots b_n.
\end{equation}
Also ist $r \sim a_i$ für ein $1 \leq i \leq m$ oder $r \sim b_j$ für $1 \leq j \leq n$. Je nachdem gilt also $r \mid a$ oder $r \mid b$.\\
(b) $\implies$ (a): Sei $a \in R$ mit $a \neq 0$ und $a \notin R^\times$. Betrachte zwei Zerlegungen
\begin{equation}
a = p_1 p_2 \cdots p_m = q_1 q_2 \cdots q_n.
\end{equation}
in irreduzible Elemente. O.B.d.A. sei $m \leq n$. Es gilt $p_1 \mid q_1 q_2 \cdots q_n$ und, da $p_i$ ein Primelement ist, auch, dass ein $j$ mit $p_i \mid q_j$ existiert. O.B.d.A. sei $p_i \mid q_i$. Da aber $q_i$ irreduzibel ist, existiert ein $n \in R^\times$ mit $q_1 = np_1$. Durch Kürzen mit $p_1$ erhalten wir 
\begin{equation}
p_2 p_3 \cdots p_m = \tilde{q}_2 q_3 \cdots q_n
\end{equation}
mit $q_2 \sim \tilde{q}_2$. Durch Induktion über $m$ reduzieren wir auf $m=1$. Dann gilt:
\begin{equation}
p_1 = q_1 q_2 \cdots q_n.
\end{equation}
Da aber $p_1$ irreduzibel ist, muss $n=1$ und $p_1 = q_1$ gelten.
\end{beweis}
\begin{satz}{Hauptidealringe sind faktoriell}{hirfaktoriell}
Sei $R$ ein Hauptidealring. Dann ist $R$ faktoriell.
\end{satz}
\begin{beweis}
Nach obigen Überlegungen genügt es, die Existenz von Zerlegungen wie in Gleichung \ref{eq:faktoren} nachzuweisen.\\
Angenommen, es gibt $0 \neq a$ mit $a \notin R^\times$, welches keine Faktorzerlegung \ref{eq:faktoren} zulässt. Dann ist $a$ also reduzibel (nicht irreduzibel), es gibt also eine nichttriviale Zerlegung 
\begin{equation}
a = x_1y_1
\end{equation}
mit $0 \neq x_1, y_1$ und $x_1, y_1 \notin R^\times$. Dann hat entweder $x_1$ oder $y_1$ keine Zerlegung gemäß \ref{eq:faktoren}. Sei o.B.d.A. $x_1$ nichttrivial zerlegbar in 
\begin{equation}
x_1 = x_2 y_2.
\end{equation}
Rekursive Fortsetzung dieses Prozesses liefert eine Folge
\begin{equation}
(x_1) \subset (x_2) \subset (x_3) \subset \cdots \subset R
\end{equation}
von echten Inklusionen. Die Vereinigung
\begin{equation}
\Uc = \bigcup_{i \geq 0} (x_i) \sub R
\end{equation}
ist ein Ideal. Da $R$ ein Hauptidealring ist, existiert ein $b \in R$ mit $(b) = \Uc$. Dann existiert ein $i \geq 0$ mit $b \in (x_i)$, also $(b) = (x_i)$. Dies impliziert aber auch, dass für alle $j > i$ gilt, dass $(b) = (x_j)$, denn $(x_i) \sub (x_j) \sub (b)$. Dies ist ein Widerspruch zu den echten Inklusionen.
\end{beweis}
\begin{übung}
Verifiziere:
\begin{itemize}
\item Die rekursive Fortsetzung des Prozesses liefert tatsächlich eine wohldefinierte Folge.
\item Die Vereinigung ist tatsächlich ein Ideal (Das gilt nicht allgemein!).
\end{itemize}
\end{übung}
\begin{bemerkung}
Das Argument aus dem Beweis von Satz \ref{hirfaktoriell} zeigt, dass jeder Hauptidealring die folgende Eigenschaft hat:\\
Für jede aufsteigende Kette 
\begin{equation}
\Ic_1 \sub \Ic_2 \sub \cdots \sub R
\end{equation}
von Idealen $\Ic_i$ in $R$ gibt es ein $k_0 \in \N$, sodass für alle $k \geq k_0$ gilt, dass $\Ic_{k_0} = \Ic_k$. Man sagt, dass jede aufsteigende Kette von Idealen \textbf{stationär} wird. Ringe mit dieser Eigenschaft heißen \textbf{Noethersche Ringe}.
\end{bemerkung}
Zum Abschluss der Diskussion zur verallgemeinerten Primfaktorzerlegung betrachten wir die Klassifikation der Primideale in $\Z$.
\begin{definition}{Spektrum}{spektrum}
Die Menge der Primideale $\fp$ eines Ringes $R$ heißt Spektrum $\Pf$ von $R$.
\end{definition}
\begin{bemerkung}
Für $R=\Z$ ist $\Pf = \{(0), (p)\}$, wobei $p$ eine Primzahl ist.
\end{bemerkung}
\begin{lemma}{Primzahlen in $\Z$}{primz}
Sei $p$ eine ungerade Primzahl. Dann sind äquivalent:
\begin{enumerate}[(a)]
\item Es gibt Zahlen $a,b \in \Z$ mit $a^2 + b^2 = p$.
\item Es gilt $p \equiv_4 1$.
\end{enumerate}
\end{lemma}
\begin{beweis}
Sei zunächst $p$ eine ungerade Primzahl mit $p = a^2 + b^2$. Die Quadrate in $\quotient{\Z}{(4)}$ sind $0^2 = 0$, $1^2 = 1$, $2^2 = 0$ und $3^2 = 1$. Daraus folgt (b).\\
Sei nun $p = 1 + 4n$ mit $n \in \N$. Wir rechnen zunächst in $\quotient{\Z}{(p)}$:
\begin{equation}
[(p-1)!] = [1] [2] \cdots [p-1] = [-1]
\end{equation}
Dies gilt, da der Restklassenring $\quotient{\Z}{(p)}$ ein Körper ist, also jedes von $0$ verschiedene Element ein Inverses und die einzigen selbstinversen Elemente sind $1$ und $-1$, die Nullstellen des Polynoms $x^2-1=0$. Das heißt also, dass $[2] [p-2] = [1]$, $[3][p-3]=1$ usw., sodass die Gleichung erfüllt ist.\\
Wir rechnen weiter:
\begin{equation}
[-1] = [(p-1)!] = [1][2] \cdots [2n][p-2n] \cdots [p-1]  = [(2n)!]^2.
\end{equation}
Also ist $x =(2n)!$ eine Lösung der Kongruenz $x^2 \equiv_4 -1$. Die Primzahl $p$ ist also ein Teiler von $x^2 +1 = (x+i)(x-i)$, aber $p$ teilt weder $(x+i)$ noch $(x-i)$. Daher ist $p$ kein Primelement in den Gaußschen Zahlen. Da $\Z[i]$ ein Hauptidealring und damit faktoriell ist, gibt es also eine Zerlegung
\begin{equation}
p = \alpha \beta
\end{equation}
in $\Z[i]$ mit $\alpha, \beta \notin \Z[i]^\times$. Wir gehen über zu Normen:
\begin{equation}
p^2 = N(\alpha) \cdot N(\beta).
\end{equation}
Da $\alpha$ und $\beta$ Nichteinheiten sind, ist $N(\alpha), N(\beta) \neq 1$. Also muss $N(\alpha) = p = N(\beta)$ gelten. Ist $\alpha = a+ib$, so ist $p = a^2 + b^2$.
\end{beweis}
\begin{theorem}{Das Spektrum der gaußschen Zahlen}{spektgauss}
Das Spektrum $\Pf(\Z[i])$ besteht aus:
\begin{enumerate}[(i)]
\item $(1+i)$
\item $\{(a+bi) | a^2+b^2=p, p \, \text{prim}, p \equiv_4 1, a>|b| \}$
\item $\{(p)| p \, \text{prim}, p \equiv_4 3 \}$.
\end{enumerate} 
\end{theorem}
\begin{beweis}
Die Erzeuger $x$ der Ideale aus (i) und (ii) sind Primelemente: Für eine Zerlegung $x=\alpha \beta$ gilt $N(\alpha)N(\beta) = p$, wobei $p$ prim ist. Also ist $N(\alpha)=1$ oder $N(\beta)=1$, und damit entweder $\alpha$ oder $\beta$ eine Einheit.\\
Eine Primzahl $p$ aus (iii) muss irreduzibel sein, da für eine nicht-triviale Zerlegung $p=\alpha \beta$, analog zu Beweis \ref{primz}, gelten würde, dass $N(\alpha) = N(\beta) = p = a^2+b^2$. Dies ist aber ein Widerspruch, da $p \equiv_4 3$.\\
Es bleibt nur noch zu zeigen, dass diese Liste vollständig ist. Dafür zeigen wir, dass jedes Primelement $\pi \in \Z[i]$ assoziiert zu (genau) einem der Erzeuger aus (i), (ii) oder (iii) ist. Zunächst folgt aus der Primfaktorzerlegung 
\begin{equation}
N(\pi) = \pi \cdot \overline{\pi} = p_1p_2 \cdots p_n
\end{equation}
in $\N$, dass eine Zahl $i \in \N$ mit $1 \leq i \leq n$ existiert, sodass $\pi \mid p_i$. Setze $p:=p_i$. Dann teilt $N(\pi)$ also auch $N(p)=p^2$. Also ist entweder $N(\pi)=p$ oder $N(\pi)=p^2$. Gilt $N(\pi)=p$, so ist $\pi$ assoziiert zu einem der Erzeuger aus (i) oder (ii). Ist andererseits $N(\pi)=p^2$, so gilt, dass $N\left(\frac{p}{\pi}\right) = 1$, also ist $\pi \sim p$. In diesem Fall muss gelten, dass $p \equiv_4 3$, denn andernfalls wäre $\pi = \epsilon (a+ib)(a-ib)$ mit $\epsilon \in \Z[i]^\times = \{\pm 1, \pm i\}$ reduzibel, im Widerspruch zu Lemma \ref{primz}.
\end{beweis}
\begin{beispiel}
Aus Theorem \ref{spektgauss} erschließt sich also ein präzises Verständnis des Zerlegungsverhaltens von Primidealen beim Übergang von $\Z$ nach $\Z[i]$:
\begin{enumerate}[(a)]
\item Das Primideal $(2)$ zerfällt in $(1+i)^2$.
\item Die Primideale $(p)$ mit $p \equiv_4 1$ zerfallen in ein Produkt $(p)=(a+ib)(a-ib)$ von komplex konjugierten Idealen.
\item Die Primideale $(p)$ mit $p \equiv_4 3$ bleiben prim.
\end{enumerate}
\end{beispiel}
\begin{definition}{Dedekindring}{dedekindring}
Ein Integritätsbereich $R$ heißt \textbf{Dedekindring}, falls $R$ kein Körper ist und für jedes Paar von Idealen $\Ic, \Jc \sub R$ die folgenden Bedingungen äquivalent sind:
\begin{enumerate}[(i)]
\item $\Ic \mid \Jc$
\item Es existiert ein Ideal $\Kc \sub R$ mit $\Jc = \Ic \cdot \Kc$.
\end{enumerate}
\end{definition}
\begin{bemerkung}
Oftmals ist die folgende, äquivalente Definition nützlicher zur Überprüfung:\\
Ein Integritätsbereich $R$ ist ein Dedekindring, wenn:
\begin{enumerate}[(a)]
\item $R$ ist noetherisch.
\item $R$ ist \textit{ganz abgeschlossen} in seinem \textit{Divisionskörper} 
\begin{equation}
(R \setminus \{0\})^{-1}R := \left\{ \left. \frac{r}{s} \right| r \in R, s \in R \setminus \{0\}\right\}
\end{equation}
\item Die Primideale $\fp \neq (0)$ sind gerade die maximalen Ideale.
\end{enumerate}
\end{bemerkung}
Wir geben noch einen Satz ohne Beweis an:
\begin{satz}{Reduzibilität in Dedekindringen}{dedekindred}
Sei $R$ ein Dedekindring und sei $\Ic \subset R$ ein echtes Ideal mit $\Ic \neq (0)$. Dann gibt es eine bis aus Permutation der Faktoren eindeutige Zerlegung
\begin{equation}
\Ic = \fp_1 \fp_2 \cdots \fp_n
\end{equation}
mit $\fp_i \sub R$ Primideal.
\end{satz}
\begin{übung}
Zeige, dass sich die Uneindeutigkeit der Zerlegung
\begin{equation}
6 = 2 \cdot 3 = (1+\sqrt{-5})(1-\sqrt{-5})
\end{equation}
in $\Z[\sqrt{-5}]$ nach Übergang zu Idealen
\begin{equation}
\begin{split}
\fp_1 &= (2, 1+\sqrt{-5})\\
\fp_2 &= (2, 1-\sqrt{-5})\\
\fp_3 &= (3, 1+\sqrt{-5})\\
\fp_4 &= (3, 1-\sqrt{-5})
\end{split}
\end{equation}
gilt:
\begin{equation}
\begin{split}
(2) &= \fp_1 \fp_2\\
(3) &= \fp_3 \fp_4\\
(1+\sqrt{-5}) &= \fp_1 \fp_3\\
(1-\sqrt{-5}) &= \fp_2 \fp_4,
\end{split}
\end{equation}
also ist $(6) = \fp_1 \fp_2 \fp_3 \fp_4$ eindeutig.
\end{übung}

\subsection{Faktorisierung in Polynomringen}
\label{sec:galois}

\begin{beispiele}
Sei $\K$ ein Körper. Dann ist $\K[x]$ euklidisch, also insbesondere ein Hauptidealring. Damit ist $\K[x]$ faktoriell, $\K[x]^\times = \K \exc \{0\} \cdots$. Was sind die irreduziblen Polynome?
\begin{enumerate}
\item Sei $\K = \C$. Dann gilt der Fundamentalsatz der Algebra, sodass jedes Polynom $f(x) \in \C[x]$ in Linearfaktoren zerfällt:
\begin{equation}
f(x) = c \prod_{i=1}^n (x-\lambda_i).
\end{equation}
Die irreduziblen Polynome sind also $(x-\lambda_i)$ mit $\lambda_i \in \C$.
\item Für $\K = \R$ haben wir:
\begin{itemize}
\item $(x-\lambda)$, $\lambda \in \R$ ist irreduzibel.
\item $x^2+ax+b$ mit $a,b \in \R$ ist irreduzibel, falls für die Diskriminante $a^2-4b<0$ gilt. Ist $\deg(f)\geq 3$, ist $f$ nicht irreduzibel, da in $\C[x]$ gilt:
\begin{equation}
f(x) = r(x-\lambda_1)(x-\lambda_2) \cdots (x-\lambda_n)= r(x-\overline{\lambda}_1) \cdots (x -\overline{\lambda}_n)
\end{equation}
da für $f \in \R[x]$ die Konjugation der Identität entspricht. Also folgt $\lambda_i = \overline{\lambda}_1$. Ist $\lambda_1 = \overline{\lambda}_1$, hat $f$ eine reelle Nullstelle $\lambda = \lambda_1$, also $f=(x-\lambda)g$. Ist hingegen $\lambda_i = \overline{lambda}_i =: \lambda$ mit $i \neq 1$, so ist 
\begin{equation}
f = (x-\lambda)(x-\overline{\lambda})g = \underbrace{(x^2 - (\lambda + \overline{\lambda})x + \lambda \overline{\lambda})}_{\in \R[x]}g.
\end{equation} 
\end{itemize}
\item $\K = \Q$ wollen wir sodann näher untersuchen.
\item $\K = \F_2$. Ist $x^2++x+1$ irreduzibel? Ausprobieren aller Elemente von $\F_2$ zeigt, dass das Polynom irreduzibel ist. Daraus folgt auch, dass das Polynom in $\Z[x]$ irreduzibel ist! Auf dieser Idee wollen wir fortan aufbauen.
\end{enumerate}
\end{beispiele}
Unser Ziel ist nun, den Zusammenhang der Irreduzibilität in $\Z[x]$ mit der in $\Q[x]$ zu verstehen.
\begin{definition}{Primitivität}{primitiv}
Wir nennen ein Polynom 
\begin{equation}
f(x) = a_nx^n+a_{n-1}x^{n-1} + \cdots + a_0 \in \Z[x]
\end{equation}
\textbf{primitiv}, falls:
\begin{enumerate}[(i)]
\item $a_n > 0$ 
\item $\text{ggT} (a_0, a_1, \dots, a_n) = \{\pm 1\}$, also $(a_0, a_1, \dots, a_n) = (1)$.
\end{enumerate}
\end{definition}