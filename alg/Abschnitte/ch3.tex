\section{Faktorisierung in Polynomringen}
\label{sec:galois}

\begin{beispiele}
Sei $\K$ ein Körper. Dann ist $\K[x]$ euklidisch, also insbesondere ein Hauptidealring. Damit ist $\K[x]$ faktoriell, $\K[x]^\times = \K \exc \{0\} \cdots$. Was sind die irreduziblen Polynome?
\begin{enumerate}
\item Sei $\K = \C$. Dann gilt der Fundamentalsatz der Algebra, sodass jedes Polynom $f(x) \in \C[x]$ in Linearfaktoren zerfällt:
\begin{equation}
f(x) = c \prod_{i=1}^n (x-\lambda_i).
\end{equation}
Die irreduziblen Polynome sind also $(x-\lambda_i)$ mit $\lambda_i \in \C$.
\item Für $\K = \R$ haben wir:
\begin{itemize}
\item $(x-\lambda)$, $\lambda \in \R$ ist irreduzibel.
\item $x^2+ax+b$ mit $a,b \in \R$ ist irreduzibel, falls für die Diskriminante $a^2-4b<0$ gilt. Ist $\deg(f)\geq 3$, ist $f$ nicht irreduzibel, da in $\C[x]$ gilt:
\begin{equation}
f(x) = r(x-\lambda_1)(x-\lambda_2) \cdots (x-\lambda_n)= r(x-\overline{\lambda}_1) \cdots (x -\overline{\lambda}_n)
\end{equation}
da für $f \in \R[x]$ die Konjugation der Identität entspricht. Also folgt $\lambda_i = \overline{\lambda}_1$. Ist $\lambda_1 = \overline{\lambda}_1$, hat $f$ eine reelle Nullstelle $\lambda = \lambda_1$, also $f=(x-\lambda)g$. Ist hingegen $\lambda_i = \overline{lambda}_i =: \lambda$ mit $i \neq 1$, so ist 
\begin{equation}
f = (x-\lambda)(x-\overline{\lambda})g = \underbrace{(x^2 - (\lambda + \overline{\lambda})x + \lambda \overline{\lambda})}_{\in \R[x]}g.
\end{equation} 
\end{itemize}
\item $\K = \Q$ wollen wir sodann näher untersuchen.
\item $\K = \F_2$. Ist $x^2++x+1$ irreduzibel? Ausprobieren aller Elemente von $\F_2$ zeigt, dass das Polynom irreduzibel ist. Daraus folgt auch, dass das Polynom in $\Z[x]$ irreduzibel ist! Auf dieser Idee wollen wir fortan aufbauen.
\end{enumerate}
\end{beispiele}
Unser Ziel ist nun, den Zusammenhang der Irreduzibilität in $\Z[x]$ mit der in $\Q[x]$ zu verstehen.
\begin{definition}{Primitivität}{primitiv}
Wir nennen ein Polynom 
\begin{equation}
f(x) = a_nx^n+a_{n-1}x^{n-1} + \cdots + a_0 \in \Z[x]
\end{equation}
\textbf{primitiv}, falls:
\begin{enumerate}[(i)]
\item $a_n > 0$ 
\item $\text{ggT} (a_0, a_1, \dots, a_n) = \{\pm 1\}$, also $(a_0, a_1, \dots, a_n) = (1)$.
\end{enumerate}
\end{definition}