\section{Galoistheorie}
\label{sec:galoistheorie}

\begin{satz}{•}{•}
Seien $\L \mid \K$ und $\E \mid \K$ Körpererweiterungen. seien weiterhin $\sigma_1, \sigma_2, \dots, \sigma_n$ paarweise verschiedene Ringhomomorphismen $\sigma_i: \L \to \E$ mit $\sigma_i|_{\K}=\id_\K$. Dann gilt:
\begin{equation}
n \leq [\L : \K].
\end{equation}
\end{satz}
\begin{beweis}
Falls $[\L : \K] = \infty$ gilt, ist die Aussage trivial. Sei also $[\L : \K] = m < \infty$ und sei $(e_1, e_2, \dots, e_m)$ eine $\K$-Basis von $\L$. Das LGS
\begin{equation}
\E^{m \times n} \ni \mat{\sigma_1(l_1), \cdots, \sigma_n(l_1)}{\sigma_1(l_2), \cdots, \sigma_n(l_2)}{\vdots, \vdots, \vdots}{\sigma_1(l_m), \cdots, \sigma_n(l_m)} \cvc{x_1, x_2, \vdots, x_n} = \cvc{0,0,\vdots,0}.
\end{equation}
Falls $n>m$ gilt, existiert eine nicht-triviale Lösung $(\lambda_1, \dots, \lambda_n) \in \E^n$, also 
\begin{equation}
\sum_{i=1}^n \lambda_i \sigma_i = 0.
\end{equation}
Dies ist ein Widerspruch zu Lemma \textbf{Dedekind}.
\end{beweis}
\begin{korollar}{Aus }{•}
Satz 3.2 folgt direkt.
\end{korollar}
\subsection{Galoiserweiterungen}
\label{subsec:galoiserweiterungen}

\begin{definition}{Galoiserweiterung}{galoiserweiterung}
Eine endliche Körpererweiterung $\L \mid \K$ heißt \textbf{Galoiserweiterung}, falls 
\begin{equation}
|\Gal (\L \mid \K)| = [\L : \K].
\end{equation}
\end{definition}
\begin{definition}{Fixkörper}{fixkörper}
Sei $\L$ ein Körper und $G \leq \Aut(\L)$. Der Unterkörper
\begin{equation}
\L^G := \{ x \in \L | \forall \sigma \in G: \sigma(x) = x\} \sub \L
\end{equation}
heißt \textbf{Fixkörper} von $G$.
\end{definition}
\begin{übung}
Zeige, dass $\L^G$ tatsächlich ein Unterkörper ist.
\end{übung}
\begin{satz}{Fixkörper ist Grundkörper}{•}
Sei $\L \mid \K$ eine Galoiserweiterung mit $G = \Gal(\L \mid \K)$. Dann gilt $\L^G = \K$.
\end{satz}
\begin{beweis}
Direkt aus den Definitionen folgt, dass $\Gal(\L \mid \K) = \Gal(\L \mid \L^G)$. Weiterhin gilt $\K \sub \L^G \sub \L$. Da $\L \mid \K$ galoissch ist, gilt $|G|=[\L : \K]$. Aus Satz \textbf{3.2} folgt, dass
\begin{equation}
[\L :\K] = |G| = |\Gal(\L \mid \L^G)| \leq [\L : \L^G]
\end{equation}
gilt, also
\begin{equation}
[\L:\L^G] = [\L : \K] \implies [\L^G : K] = 1 \implies \L^G = \K.
\end{equation}
\end{beweis}
\begin{übung}
Begründe, warum die letzte Implikation gelten muss.
\end{übung}
\begin{satz}{Satz von Artin}{artin}
Sei $\L$ ein Körper und $G \leq \Aut (\L)$. Dann gilt
\begin{equation}
[\L: \L^G] = |G|.
\end{equation}
Falls $G$ endlich ist, ist damit insbesondere die Körpererweiterung $\L \mid \L^G$ galoissch mit Galoisgruppe $G$.
\end{satz}
\begin{beweis}
Sei $\K := \L^G$. Wegen $G \leq \Gal(\L \mid \K)$ gilt
\begin{equation}
|G| \leq |\Gal(\L : \K)| \leq [\L : \K].
\end{equation}
Wir zeigen nun $[\L:\K] \leq |G|$. Ist $|G|=\infty$, ist die Aussage trivial, also sei $|G|=n<\infty$. Wir zeigen, dass je $n+1$ Elemente $x_1, x_2, \dots, x_{n+1} \in \L$ linear abhängig über $\K$ sind. Betrachte dazu für $G=\{\sigma_1, \sigma_2, \dots, \sigma_n\}$ das $\L$-lineare Gleichungssystem
\begin{equation}
\sum_{j=1}^{n+1} y_j \sigma_i(x_j) = 0
\end{equation}
für $1 \leq i \leq n$. Dieses LGS hat $n+1$ Variablen und $n$ Gleichungen, also existiert eine nicht-triviale Lösung $(\lambda_1, \lambda_2, \dots, \lambda_{n+1}) \in \L^{n+1}$. Sei o.B.d.A. $\lambda_1 \neq 0$. Für $1 \leq i \leq n$ wenden wir $\sigma_i^{-1}$ auf die $i$-te Gleichung an und erhalten
\begin{equation}
\sum_{j=1}^{n+1} \sigma_i^{-1}(\lambda_j) \cdot x_j = 0
\end{equation}
für $1 \leq i \leq n$. Summiere über alle Gleichungen:
\begin{equation}
\sum_{j=1}^{n+1} \underbrace{\alpha_j}_{\in \K} x_j = 0
\end{equation}
mit
\begin{equation}
\alpha_j = \sum_{i=1}^n \sigma_i^{-1}(\lambda_j)=\sum_{\sigma \in G} \sigma(\lambda_j) \in \L^G,
\end{equation}
da für die Gruppe $G$ Inversenbildung $g \mapsto g^{-1}$ eine Bijektion ist und die Summe invariant unter Wirkung von $G$ ist. Es bleibt noch, zu zeigen, dass die Koeffizienten $\alpha$ nicht-trivial sind. Wegen des Dedekind-Lemmae gilt $\sum_{\sigma \in G} \sigma \neq 0$, es gibt also $l \in \L$ mit $\sum_{\sigma \in G} \sigma (l)\neq 0$. Ersetze den Lösungsvektor $(\lambda_1, \lambda_2, \dots, \lambda_{n+1})$ durch die Reskalierung $l\cdot \lambda_1^{-1} \in \L$, also:
\begin{equation}
\left( (l\lambda_1^{-1}) \lambda_1, (l\lambda_1^{-1}) \lambda_2, \dots, (l\lambda_1^{-1}) \lambda_{n+1} \right) \in \L^{n+1}.
\end{equation}
Daraus folgt, dass
\begin{equation}
\alpha_1 = \sum_{\sigma \in G} \sigma(l) \neq 0 ,
\end{equation}
also liegt eine nicht-triviale $\K$-Linearkombination vor.
\end{beweis}
\begin{korollar}{Aus Satz \ref{artin}}{ausartin}
Sei $\L \mid \K$ eine endliche Körpererweiterung. Dann gilt:
\begin{equation}
(|\Gal(\L \mid \K)|) \mid ([\L : \K])
\end{equation}
\end{korollar}
\begin{beweis}
Aus dem Satz von Artin folgt, dass
\begin{equation}
|\Gal(\L \mid \K)| = |\Gal(\L \mid \L^G)| = [\L : \L^G] \mid [\L : \K]
\end{equation}
\end{beweis}

\subsection{Galoiskorrespondenz}
\label{subsec:galoiskorrespondenz}

Sei $\L \mid \K$ eine Körpererweiterung mit Galoisgruppe $\Gal(\L\mid \K) =: G$. Betrachte die Mengen
\begin{equation}
\{H | H \leq G\}
\end{equation}
und
\begin{equation}
\{\M | \K \sub \M \sub \L\}.
\end{equation}
Die Körper $\K$ und $\M$ heißen Unterkörper, $\M$ heißt Zwischenkörper von $\K, \L$. Zwischen diesen Mengen existieren Abbildungen: \dots

\begin{theorem}{Hauptsatz der Galois-Theorie}{hauptsatzgalois}
Die Abbildungen 
\begin{equation}
f: \{H | H \leq G\} \to \{\M | \K \sub \M \sub \L\}
\end{equation}
und
\begin{equation}
g: \{\M | \K \sub \M \sub \L\} \to \{H | H \leq G\}
\end{equation}
sind zueinander inverse Bijektionen.
\end{theorem}
\begin{beispiel}
Sei $\L \mid \K$ eine Galoiserweiterung mit $\Gal(\L \mid \K) \cong \Sf_3$. Diese Gruppe verstehen wir bereits: (Diagramm einfügen)
\end{beispiel}
\begin{beweis}
$g \circ f = \id$ ist klar, da
\begin{equation}
H = \Gal(\L \mid \L^H)
\end{equation}
nach dem Satz von Artin.
\end{beweis}
