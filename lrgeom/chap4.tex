\chapter{Comparision Geometry}
\label{chap:comparision}
\vspace*{-0.9cm}

% \startcontents[chapters]
% \printcontents[chapters]{}{4}{}
\section{Hessian Operators and the Riccati Equation}
\label{sec:hessricc}
\begin{definition}[Hessian]
    \marginnote{In local coordinates $(x^i)$, the Hessian has the (familiar) form \[
            \Hess_u = (\partial_j \partial_i u - \Gamma_{ij}^k \partial_k u) \, dx^i \otimes dx^j
    .\] }
    Let $(M,g)$ be a SRMF and $u: M \to \mathbb{R}$ smooth. The \textbf{covariant Hessian} is the $(0,2)$-tensor \[
        \Hess_u(X,Y):=(\nabla^2 u)(X,Y) = \nabla_X(\nabla_Y u) - \nabla_{\nabla_X Y}u
    .\]  The \textbf{Hessian operator} is the $(1,1)$-tensor \[
    \mathcal{H}_u := \Hess_u^\sharp
    .\] 
\end{definition}
\begin{remark}
   The Hessian is symmetric if $\nabla$ is the Levi-Civita-Connection (or, more generally, if and only if $\nabla$ is torsion-free):
   \begin{align*}
       \Hess_u(X,Y) &= Y(X(u)) - du(\nabla_YX) = (XY-[X,Y])u - du(\nabla_YX) \\
                    &= X(Y(u))-du(\nabla_YX+[X,Y])\\
                    &= X(Y(u))- du(\nabla_X Y-T(X,Y))\\
                    &= \Hess_u(Y,X) + T(X,Y)(u)
   \end{align*}
   Unraveling the definition of $\mathcal{H}_u$, we get \[
   g(\mathcal{H}_u(X), Y) = \Hess_u(X,Y) = \Hess_u(Y,X)=g(X,\mathcal{H}_u(Y))
   ,\] and hence $\mathcal{H}_u$ is self-adjoint.
\end{remark}
With the radial distance function defined above, we obtain the Hessian operator $\mathcal{H}_r$ associated to that function.
\begin{notation}
    Given a normal neighbourhood $U$, we denote the geodesic spheres of radius $r_0$ and proper time $\tau_0$, respectively, by \[
        S_{r_0}:=\left\{x \in U \mid r(x)=r_0\right\} 
    \] and \[
    H_{\tau_0} := \left\{x \in I_U^+(p) \mid \tau (x)=\tau_0 \right\} 
    .\] 
        The Gauß' Lemma immediately tells us that $T_q\mathbb{S}_{r(q)}=\partial_r|_q^\perp \subseteq T_qM$ and similarly for $\tau$. Denote the projection on the radial tangent by \[
            \pi_q^r: T_qM \to T_qS_{r(q)}
        .\] 
\end{notation}
\begin{lemma}
    Let $(M,g)$ be a Riemannian or time-oriented Lorentzian manifold, $(U,x^i)$ be a normal chart around $p \in M$ and let $r$ be the radial distance function. Then we have for all $q \in U \setminus \{p\}$:
    \begin{enumerate}
        \item $\mathcal{H}_r(\partial_r|_q)=0$ and $\mathcal{H}_\tau(\partial_\tau|_q)=0$
        \item For all $X \in T_qM$ with $X \perp \partial_r|_q$: \[
        \mathcal{H}_r(X)=\pi_q(\nabla_X(\grad r))
        ,\] and similarly for $X \perp \partial_\tau|_q$:
        \[
        \mathcal{H}_\tau(X)=\pi_q(\nabla_X(\grad \tau)) = \pi_q^\mathbb{H} ( \nabla_X \grad(\tau))
    ,\] where $\pi_q^\mathbb{H}: T_qM \to \partial_\tau|_q^\perp \cong T\mathbb{H}^{n-1}$ denotes the orthogonal projection \[
    \pi_q^\mathbb{H}(Y) = Y + g(Y, \partial_\tau)\partial_\tau
    .\] 
    \end{enumerate}
\end{lemma}
\begin{proof}
    We proof the Lorentzian case:
    \begin{enumerate}
        \item Let $Y \in T_qM$ be arbitrary. We have:
            \begin{align*}
                g(\mathcal{H}_\tau(\partial_\tau|_q), Y) &= g(\partial_\tau|_q, \mathcal{H}_\tau(Y)) = g(\partial_\tau|_q, - \nabla_Y \grad(\tau)) \\
                                                         &= g(\nabla_Y\partial_\tau, \partial_\tau|_q) = \frac{1}{2} Y_q(g(\partial_\tau, \partial_\tau)) =0,
            \end{align*}
            where we used symmetry of $g$ and the fact that $- \grad (\tau)=\partial_\tau$ since the metric is $-d\tau^2+\tilde{g}$.
        \item This follows since the previous remark showed $\mathcal{H}_\tau(X)= \nabla_X \grad (\tau)$ and $\mathcal{H}_\tau (X) \perp \partial_\tau$ follows from the upper equation.
    \end{enumerate}
\end{proof}
\begin{remark}
    \marginnote{
        Note that $g|_{TH_{\tau_0} \times TH_{\tau_0}}$ is Riemannian, so the restriction is self-adjoint for a positive-definite inner product even in the Lorentzian case.
    }
    For any $\tau_0$ such that $H_{\tau_0} \neq \emptyset,$ $\mathcal{H}_\tau$ restricts to a self-adjoint linear operator \[
    \mathcal{H}_\tau: TH_{\tau_0} \to TH_{\tau_0}
\] and is given by the Weingarten map $\mathcal{H}_\tau(X) = -\nabla_X N$ where $N=\partial_\tau$ is the future-pointing unit normal to $H_{\tau_0}$.
\end{remark}
\begin{prop}[The Hessian and Jacobi Fields]
    \label{prop:hessjacobi}
    Let $(\mathcal{L},g)$ be a LMF, $p \in \mathcal{L}$ and $U \subseteq \mathcal{L}$ be a normal neighbourhood around $p$. Set $\tau := \tau_U(p, \cdot)$. Let $\gamma: [0,b] \to U$ be a future-directed timelike unit-speed geodesic starting at $p$ and let $J \in \mathfrak{J}^\perp(\gamma)$ such that $J(0)=0$.\marginnote{In the Riemannian case, we have \[
    \nabla_{\frac{d}{d t}} J(t)=\mathcal{H}_r(J(t))
    .\] } Then \[
        \nabla_{\frac{d}{dt}}J(t)=-\mathcal{H}_\tau(J(t))
    \] for all $\tau_0 \in (0,b]$.
\end{prop}
\begin{proof}
    Let $v:=\dot{\gamma}(0)$ and $w:= \nabla_{\frac{d}{dt}}J(0)$. By equation \ref{eq:explicitpointjacobi}, we have a variation through geodesics \[
        \Gamma_s(t)=\exp_ {\sigma(s)}(tU(s))
        \]  where $\sigma(s)$ is any curve with $\sigma(0)=p$ and $\dot{\sigma}(0)=J(0)$. The vector field $U$ satisfies $U(0)=0$ and $\nabla_{\frac{d}{ds}}U(0)=w$ and the variational field of $\Gamma$ is $\left. \frac{\partial}{\partial s}\right|_{s=0} = J$. Since $J(0)=0$, choose $\sigma(s)\equiv p$. Since $w \perp v$ ($J$ is normal), we have \marginnote{Note that $g_p \cong \eta_p$.} \[
    v \in H := \left\{X \in T_p \mathcal{L} \mid g_p(X,X)=-1 \right\} 
.\] Hence, $w \in T_vH$. Since $U: I \to T_pM = T_{\sigma (s)}M$ with $\nabla_{\frac{d}{ds}}U(0) \in T_vH$ and $U(0)=v \in H$, we can choose $U$ such that $U(s) \in H$ for all $s$. Therefore, each curve $t \mapsto \Gamma_s(t)$ is a future-directed timelike unit-speed geodesic starting at $p$, implying 
\begin{equation}
    \label{eq:helper1}
\partial_t \Gamma_{s_0}(t_0)=\partial_\tau|_{\Gamma (s_0,t_0)}
\end{equation}
Now we compute:
\[
    \nabla_{\frac{d}{dt}}J = \nabla_{\frac{d}{dt}} \left. \frac{\partial}{\partial s}\right|_{s=0} \Gamma = \nabla_{\frac{d}{ds}} \left( \frac{\partial}{\partial t} \Gamma \right)(0,t)=\nabla_{\frac{d}{ds}} \left( \partial_\tau \circ \Gamma \right)(0,t)
    ,\] where we used equation \ref{eq:helper1}. Since $\partial_\tau$ is a smooth vector field on $I_U^+(p)$ (which is an open neighbourhood of $\Gamma_0(\tau_0)$ for any $\tau_0 \in (0,b]$). This implies \[
    \nabla_{\frac{d}{ds}} \left( \partial_\tau \circ \Gamma \right)(0,t) = \nabla_{\partial_s \Gamma|_{s=0}} \partial_\tau = \nabla_J \partial_\tau = - \nabla_J \grad (\tau) = \mathcal{H}_\tau (J)
\] using $J \perp \dot{\gamma} = \partial_\tau$.  
\end{proof} 
Recall the previously defined function \ref{eq:curvfunc}.
\begin{prop}[Sectional Curvature and Hessian]
    Let $(\mathcal{L}, g)$ be a LMF, $p \in \mathcal{L}$ and $U$ be a normal neighbourhood of $p$. Then $(\mathcal{L}, g)$ has constant sectional curvature $c$ on $I^+_U(p)$ if and only if \[
        \mathcal{H}_\tau = - \frac{s_{-c}' \circ \tau}{s_{-c} \circ \tau} \cdot \pi^H
    \] on $I^+_U(p)$.
\end{prop}
\begin{proof}
    $(\Rightarrow):$ Let $q \in I_U^+(p)$, $\gamma: [0,b] \to M$ be the future-directed timelike unit-speed geodesic from $p$ to $q$, and let $\{E_1=\dot{\gamma}, E_2, \dots, E_n\}$ be an ONF along $\gamma$. Proposition \ref{prop:jacobiinconst} tells us that \[
        J_i := s_{-c}E_i, \, i\geq 2
    \] are normal Jacobi fields along $\gamma$ with $J_i(0)=0$. Compute \[
    s_{-c}' E_i = \nabla_{\frac{d}{dt}} J_i = - \mathcal{H}_\tau (J_i) = -s_{-c} \mathcal{H}_\tau(E_i)
    ,\] where we used proposition \ref{prop:hessjacobi} evaluating at $t=b$ and $s_{-c}(b)^{-1}$ noting $s_{-c}(b)\neq 0$ because $\gamma(b)=q \in U$, which is normal. Hence $q$ is not conjugate to $p$, implying $J_i(b)\neq 0$. This yields \[
\mathcal{H}_\tau (E_i(b))=- \frac{s_{-c}'(b)}{s_{-c}(b)} E_i (b)
,\] but $E_i(b)=\pi^H(E_i(b))$ because it is already orthogonal to $\dot{\gamma}(b)$. For $i=1$, we have \[
0=\mathcal{H}_\tau (E_1(b))=\pi^H (E_1(b))
\] because $E_1(b)=\dot{\gamma}(b)$.\par
$(\Leftarrow):$ Let $q \in I_U^+(p)$ and $\gamma: [0,b) \to U$ be the corresponding geodesic. From theorem \ref{thm:constcurvmet}, we know that \[
    \phi: I_U^+(p) \to \mathbb{R}_+ \times \mathbb{H}^{n-1}
    \] given by $\phi(q)=(\tau(q), \xi(q))$ is a diffeomorpism onto its image. If we can show that it is an isometry between $g$ and $g_c := -d\tau^2 + s_{-c}(\tau)^2g_{\mathbb{H}^{n-1}}$, then $g$ has constant sectional curvature. Following the proof of theorem \ref{thm:constcurvmet}, we immediately get 1 and 2. In 3, the constant curvature assumption only entered via Jacobi fields, more precisely via proposition \ref{prop:jacobiinconst} yielding: Any $J \in \mathfrak{J}^\perp(\gamma)$ with $J(0)=0$ is given by \[
J(t)=ks_{-c}(t)E(t)
\] for some $k \in \mathbb{R}$ and $E \perp \dot{\gamma}$ as well as parallel. It only remains to show this in our setting. Let $J \in \mathfrak{J}^\perp(\gamma)$. We have \[
\nabla_{\frac{d}{dt}}J=\frac{s_{-c}'}{s_{-c}}J
\] on $(0,b]$. By the product rule, $\frac{1}{s_{-c}}J$ is parallel along $\gamma$ on $(0,b]$. Define $\tilde{E}(t):=\frac{1}{s_{-c}}J$ and let $E= \frac{\tilde{E}}{\|\tilde{E}\|}$. Then $E$ is a parallel unit VF orthogonal $\dot{\gamma}$ and $J=\|\tilde{E}\|  s_{-c} E$. Set $k:=\|\tilde{E}\|$, which is constant since $\tilde{E}$ was parallel. It suffices to have this for $t \in (0,b]$ since this still gives $k=\|\nabla_{\frac{d}{dt}} J(0)\|$ by $k$ constant.
\end{proof}
\lecture{09.01.26}
Consider a LMF $(\mathcal{L}, g)$ with a normal neighbourhood $U$. The radial distance function $\tau := \tau_U(p, \cdot)$ is smooth on $I_U^+(p)$. Let $\gamma: [0,b]\to U$ be a timelike future-directed unit-speed geodesic starting at $p$. In this case, the Hessian operator restricts to \[
\mathcal{H}_\tau: \mathfrak{X}(\gamma) \to \mathfrak{X}(\gamma)
,\] which is a $(1,1)$-tensor field along $\gamma$ and $\mathcal{C}^\infty (I)$-linear. Therefore, we also get another $\mathcal{C}^\infty (I)$-linear $(1,1)$-tensor field 
\begin{equation}
    \label{eq:nablahess}
    \nabla_{\frac{d}{dt}}\mathcal{H}_\tau (X):=\nabla_{\frac{d}{dt}} \left( \mathcal{H}_\tau(X)\right) - \mathcal{H}_\tau \left( \nabla_{\frac{d}{dt}} X\right).
\end{equation}
\begin{definition}[Tidal Force Operator]
Let $(M,g)$ be a RMF or a \red{LMF} and let $\gamma: [0,b]\to M$ be a \red{(timelike)} geodesic.\marginnote{One sees that this is indeed $\mathcal{C}^\infty (I)$-linear, self-adjoint and $\tr \R_{\dot{\gamma}}=-\Ric (\dot{\gamma},\dot{\gamma})$. Comparing with the Jacobi equation also shows that \[
        \R_{\dot{\gamma}}(J)=\nabla_{\frac{d}{dt}}^2 J
.\] } The \textbf{tidal force operator} is defined by
    \begin{align*}
        \R_{\dot{\gamma}}: \mathfrak{X}(\gamma) &\to \mathfrak{X}(\gamma)\\
        X &\mapsto \R_{\dot{\gamma}}(X):=- \R(X,\dot{\gamma})\dot{\gamma}.
    \end{align*}
\end{definition}
\begin{remark}
    The tidal force operator restricts to a well-defined $\mathcal{C}^\infty (I)$-linear operator $\mathfrak{X}(\gamma)^\perp \to \mathfrak{X}(\gamma)^\perp$ and the trace of that restriction still yields $-\Ric(\dot{\gamma}, \dot{\gamma})$ since $R_{\dot{\gamma}}(\dot{\gamma})=0$. This restriction is still self-adjoint with respect to $g|_{\mathfrak{X}(\gamma)^\perp \times \mathfrak{X}(\gamma)^\perp}$. This metric restriction is Riemannian even if the original $g$ is Lorentzian.
\end{remark}
\begin{theorem}[Riccati Equation]
    \label{thm:riccatieqn}
    Let $(M,g)$ be a RMF or a \red{LMF}, $U$ be a normal neighbourhood of some $p \in M$ and $\gamma: [0,b] \to U$ be a \red{timelike future-directed} radial unit-speed geodesic starting at $p$. Let $r$ \red{($\tau$)} be the radial distance function on $U$ \red{($I^+_U(p)$)}. Then, the Hessian operator satisfies the \textbf{Riccati equation} (REQ):\marginnote{One can compare this with the classical scalar Riccati equation: Let $f: I \to \mathbb{R}$ and $\sigma: I \to \mathbb{R}$. Then the classical RQN is given by \[
    f' + f^2 + \sigma = 0
    .\] }
\begin{equation}
    \label{eq:riccatiriem}
    \nabla_{\frac{d}{dt}} \mathcal{H}_r + \mathcal{H}_r^2 - \R_{\dot{\gamma}} = 0
\end{equation}
\begin{equation}
    \label{eq:riccatilor}
\red{\nabla_{\frac{d}{dt}} \widehat{\mathcal{H}} + \widehat{\mathcal{H}}^2 - \R_{\dot{\gamma}}=0},
\end{equation}
\marginnote{Note that this can only apply on the half-open interval $(0,b]$ since $r$ and $\tau$ are undefined at $0$.}
with $\widehat{\mathcal{H}}:=-\mathcal{H}_\tau$ in the Lorentzian case, on $\gamma|_{(0,b]}.$
\end{theorem}
\begin{proof}
    We prove the Lorentzian case, the Riemannian case can be found in \cite{LeeRM}. Fix $\tau_0 \in (0,b]$. It suffices to check the (REQ) for $\partial_\tau|_{\tau_0}$ and $X \in T_{\gamma(\tau_0)}H_{\tau_0}$, i.e. $X \perp \partial_\tau|_{\tau_0}$.
    For $\partial_\tau$, we already know $\mathcal{H}_\tau (\partial_\tau)=0$, so $\mathcal{H}^2_\tau(\partial_\tau)=0$. Furthermore, $\partial_\tau|_{\tau_0}=\dot{\gamma}(\tau_0)$, and hence $\R_{\dot{\gamma}}(\partial_\tau)=0$. Lastly, we have \[
        \nabla_{\frac{d}{dt}}\mathcal{H}_\tau (\partial_\tau) = \nabla_{\frac{d}{dt}} \left( \eqnmark[blue]{node2}{\cancel{\mathcal{H}_\tau (\partial_\tau)}} \right) - \mathcal{H}_\tau \left( \eqnmark[blue]{node1}{\cancel{\nabla_{\frac{d}{dt}} \dot{\gamma}}} \right) = 0
    ,\] 
    \annotate[yshift=0.5em]{above, label above}{node1}{geodesic}
    \annotate[yshift=0.5em]{above, label above}{node2}{vanishes on $(0,b]$}
    and thus the REQ holds in this case.
    Now, let $X \perp \dot{\gamma}(\tau_0)$. By \ref{cor:jacobibvp}, we find $J \in \mathfrak{J}(\gamma)^\perp$ with $J(\tau_0)=X$ and $J(0)=0$. This yields:
    \begin{align*}
        \nabla_{\frac{d}{dt}}\mathcal{H}_\tau (X) &= \left.\left( \nabla_{\frac{d}{dt}}\left(\mathcal{H}_\tau (J) \right) - \mathcal{H}_\tau (\nabla_{\frac{d}{dt}} J)  \right)\right|_{t=\tau_0} \\
                                                  &\eqnmark[blue]{node3}{=} \left.\left( \nabla_{\frac{d}{dt}}\left(\red{-} \nabla_{\frac{d}{dt}}J \right) - \mathcal{H}_\tau (\red{-} \mathcal{H}_\tau(J))  \right)\right|_{t=\tau_0} \\
                                                  &= - \nabla_{\frac{d}{dt}}^2 J(\tau_0) + \mathcal{H}^2_\tau (X) = -\R_{\dot{\gamma}}(J(\tau_0)) + \mathcal{H}_\tau^2(X).
    \end{align*}
    \annotate[yshift=-0.7em]{left, label below}{node3}{Proposition \ref{prop:hessjacobi}}
\end{proof}
\begin{theorem}[Matrix Riccati Comparison]
    \label{thm:matrixriccaticomp}
    Let $\Sym_\mathbb{R}(n) \subseteq \Mat_\mathbb{R}(n)$ be the space of symmetric real $n \times n$-matrices identified with self-adjoint endomorphisms of Euclidean $\mathbb{R}^n$. Let $H,\widetilde{H}: (a,b] \to \Sym_\mathbb{R}(n)$ be smooth such that the \textbf{matrix Riccati equations} \[
    H' + H^2 + S = 0
    \] and\marginnote{We use the notion $$A \geq B : \iff A - B \geq 0 \iff \langle (A-B)x,x  \rangle $$ for all $x \in \mathbb{R}^n$. This means all eigenvalues of $A-B$ are non-negative.}
    \[
        \widetilde{H}' + \widetilde{H}^2 + \widetilde{S} = 0
    \] are satisfied for some continuous $S, \widetilde{S}: [a,b] \to \Sym_\mathbb{R}(n)$ which are smooth on $(a,b]$. If \[
    \widetilde{S}\geq S
\]  on $[a,b]$ and \[
\lim_{t \searrow 0} \widetilde{H}(t)-H(t)\leq 0
\] (meaning the limit also exists in the first place), then \[
\widetilde{H} \leq H
\] on $(a,b]$.
\end{theorem}
\begin{proof}
    See \cite{LeeRM}.
\end{proof}
\begin{theorem}[Riccati Manifold Comparison]
    \label{thm:riccatimfncomp}
    Let $(M,g)$ be a RMF or a \red{LMF} and $\gamma:[a,b] \to M$ be a \red{timelike} unit-speed geodesic segment. Suppose \[
        \eta, \widetilde{\eta}: \mathfrak{X}(\gamma|_{(a,b]})^\perp \to \mathfrak{X}(\gamma|_{(a,b]})
    \] are $\mathcal{C}^\infty (I)$-linear, self-adjoint with respect to the restricted metric and satisfy the RQNs \[
    \nabla_{\frac{d}{dt}} \eta + \eta^2 + \sigma = 0
    \] and \[
    \nabla_{\frac{d}{dt}} \widetilde{\eta} + \widetilde{\eta}^2 + \widetilde{\sigma} =0
    \] for some self-adjoint, $\mathcal{C}^\infty (I)$-linear $\sigma, \widetilde{\sigma}: \mathfrak{X}(\gamma)^\perp \to \mathfrak{X}(\gamma)^\perp$. \marginnote{This roughly means that a comparision of RQNs and initial values leads to a comparision of solutions.}If \[
\forall t \in [a,b]: \widetilde{\sigma}(t)\geq \sigma(t)
\]  and \[
\lim_{t \searrow a} \widetilde{\eta}(t)-\eta(t) \leq 0
\] holds, then \[
\forall t \in [a,b]: \widetilde{\eta}(t)\leq \eta(t)
.\] 
\end{theorem}
\begin{proof}
    The goal is to utilize theorem \ref{thm:matrixriccaticomp}.
    To this end, take a parallel ONF $(E_i) \cup \{\dot{\gamma}\}$ with $1 \leq i \leq n-1$ along $\gamma$. In this frame, we represent $\eta, \widetilde{\eta}, \sigma, \widetilde{\sigma}$ by symmetric, matrix valued maps \[
        H, \widetilde{H}: [a,b] \to \Sym_\mathbb{R}(n-1)
    \] and \[
    S, \widetilde{S}: [a,b] \to \Sym_\mathbb{R}(n-1)
    \] with components $\eta(t)(E_i)=H_i^j(t)E_j$ and analogously for the other operators. We have:
    \begin{align*}
        \nabla_{\frac{d}{dt}}\left(\eta(E_i)\right) &= \nabla_{\frac{d}{dt}}\eta (E_i) + \eta \left( \nabla_{\frac{d}{dt}} E_i \right) = -\eta^2(E_i)-\sigma(E_i)\\
                                                    &=-\eta(H_i^jE_j)-S_i^kE_k = -H_i^j\eta(E_j)-S_i^kE_k \\
                                                    &=-H_i^j H_j^k E_k - S_i^k E_k
    \end{align*}
    Without expanding, we obtain:
    \begin{align*}
        \nabla_{\frac{d}{dt}}\left( \eta(E_i)\right) &= \nabla_{\frac{d}{dt}}(H_i^k E_k)\\
                                                     &=(H^k_i)'E_k+H^k_i \cancel{\nabla_{\frac{d}{dt}}E_k} = (H^k_i)'E_k.
    \end{align*}
    Therefore, \[
    H' + H^2 + S =0
\] holds, with similar calculations estabilishing the same for $\widetilde{H}$ and $\widetilde{S}$. Clearly, $\widetilde{\sigma} \geq \sigma$ is equivalent to $\widetilde{S} \geq S$ and similar for the other condition. Theorem \ref{thm:matrixriccaticomp} is thus applicable and we conclude \[
\widetilde{H} \leq H \iff \widetilde{\eta} \leq  \eta
.\] 
\end{proof}
\begin{remark}
    In the Riemannian case, we can do the same for \[
        \eta, \widetilde{\eta}: \mathfrak{X}(\gamma) \to \mathfrak{X}(\gamma)
    .\] However, we mainly consider $\eta = \mathcal{H}_r$ and $\widetilde{\eta} \propto \pi^S$. Splitting $\mathfrak{X}(\gamma) \cong \spn\{\dot{\gamma}\} \oplus \mathfrak{X}(\gamma)^\perp$, then we obtain the following matrix forms of $\mathcal{H}_r$ and $\pi^S$:
    \[
    \begin{pmatrix}
        0 & 0\\
        0 & \mathcal{H}_r|_{\mathfrak{X}(\gamma)^\perp}
    \end{pmatrix}
    \] 
    and \[
    \begin{pmatrix}
        0 & 0\\
        0 & \id
    \end{pmatrix}
    .\] Therefore, the restriction to $\mathfrak{X}(\gamma)^\perp$ does not discard any information.
\end{remark}
\section{Hessian Comparison}
\label{sec:hessiancomp}
In this section, we attempt to compare $\mathcal{H}_r$ with $\frac{s_c' \circ r}{s_c \circ r} \pi^S$ and $\mathcal{H}_\tau$ with $- \frac{s_{-c}'\circ \tau}{s_{-c}\circ \tau}\pi^H$.
In the following, any inequalities for $\mathcal{H}_\tau$ in the Lorentzian case are to be understood for the restriction \[
    \mathcal{H}_\tau|_{\mathfrak{X}(\gamma)^\perp}: \mathfrak{X}(\gamma)^\perp \to \mathfrak{X}(\gamma)^\perp
.\] 
\begin{theorem}[Hessian Comparison]
    Let $(M,g)$ be a RMF, $p \in M$, $U$ be a normal neighbourhood of $p$ and $r: U \to [0,\infty)$ be the radial distance function at $p$. Let $c \in \mathbb{R}$.
    \begin{enumerate}[(a)]
        \item If all sectional curvatures of $M$ are less or equal $c$, then \[
        \mathcal{H}_r(q)\geq \frac{s_c'(r(q))}{s_c(r(q))} \pi_q^S
        \] for \[
        \begin{cases}
            \forall q \in U \setminus \{p\} &c \leq 0\\
            \forall q \in U \setminus \{p\}: r(q)<\pi R &c>0
        \end{cases}
    \] with $R:=\frac{1}{\sqrt{c}}$.
\item If all sectional curvatures of $M$ are greater or equal $c$, then \[
\mathcal{H}_r|_q \leq \frac{s_c'(r(q))}{s_c(r(q))}\pi^S_q
\] for all $q \in U \setminus \{p\}$.
    \end{enumerate}
\end{theorem}
\lecture{13.01.26}
\begin{TODO}
    Add first part of lecture.
\end{TODO}
\begin{theorem}[Lorentzian Jacobi Field Comparison]
    Let $(M,g)$ be a LMF, $\gamma: [0,b] \to M$ be a unit-speed timelike geodesic, and $J \in \mathfrak{J}(\gamma)^\perp$ with $J(0)=0$.
    \begin{enumerate}
        \item If all timelike sectional curvatures are non-positive, then \[
                \|J(t)\|_g \leq s_{-c}(t) \cdot \|\nabla_{\frac{d}{d t}} J(0)\|_g 
            \] for all $t \in [0,b_1]$ where $$b_1 := \sup \{t \in [0,b] \mid \gamma(t) \text{ is not conjugate to } \gamma(0) \text{ along } \gamma\}.$$
        \item If all timelike sectional curvatures are non-negative, then \[
                \|J(t)\| \geq s_{-c}(t) \| \nabla_{\frac{d}{d t}} J(0)\|
            \] for all $t \in [0,b_2]$ where \[
            b_2 := 
            \begin{cases}
                b &c\leq 0\\
                \min \{b , R \pi \} &c>0
            \end{cases}
            .\] 
    \end{enumerate}
\end{theorem}
\begin{proof}
    Let $J$ be such a Jacobi field which is not the zero field.
    \begin{enumerate}[1. {Step:}]
        \item We show monotonicity. Let $b_0 := \min\{b_0,b_1\}$ and define \[
        f: (0,b_0) \to \mathbb{R}
    \] by $$f(t):=\log (s_{-c}(t)^{-1} \|J(t)\|)=\log(\|J(t)\|)-\log(s_{-c}(t)).$$ Define $\widetilde{f}(t):=s_{-c}(t)^{-1}\|J(t)\|$, which is strictly positive on $(0,b_0)$ since $s_{-c}(t)$ and $J(t)$ are not zero by choice of $b_0$ and spacelike character of $J(t)$ (which is normal). Note that $f$ is non-increasing resp. non-decreasing if and only if $\widetilde{f}$ is by monotonicity of $\log$.
    Now compute \[
        \dot{f}(t)=\frac{1}{\|J\|^2} g(\nabla_{\frac{d}{dt}}J,J )- \frac{\dot{s}_{-c}}{s_{-c}}
    .\] Assume for now that $\gamma: [0,b_0] \to M$ is contained in a normal neighbourhood of $p:=\gamma(0)$. Using $\widetilde{\mathcal{H}}(J)=\nabla_{\frac{d}{dt}} J$ (prop. 3.5) and Hessian comparison, we get \[
\dot{f}(t)=\frac{1}{\|J\|^2} (g(\widetile{\mathcal{H}}(J), J) - g(\underbrace{\frac{\dot{s}_{-c}}{s_{-c}}J}_{=\widehat{\mathcal{H}}^c(J)},J)) = 
    \begin{cases}
        \leq 0 &\text{Case 1}\\
        \geq 0 &\text{Case 2}
    \end{cases}
    .\] 
\item Now we treat the initial condition. We want to show that \[
        \lim_{t \searrow 0} \frac{\|J(t)\|}{s_{-c}(t)}=\|\nabla_{\frac{d}{dt}}J(0)\| 
,\] which is equivalent to \[
\lim_{t \searrow 0} \frac{\|J\|}{s_{-c}(t)} = \|\nabla_{\frac{d}{dt}}J(0)\| >0 
.\] The last inequality holds as $J$ is normal. Applying l'Hospital's rule twice yields 
\begin{align*}
    \lim_{t \searrow 0} \frac{1}{s_{-c}^2} \|J\| &= \lim_{t \searrow 0} \frac{2g(\nabla_{\frac{d}{dt}}J,J )}{2 s_{-c} \dot{s}_{-c}} \\
                                                 &= \lim_{t \searrow 0} \frac{\| \nabla_{\frac{d}{dt}} J \| +  }{\dot{s}}
\end{align*}
    \end{enumerate}
    
\end{proof}
