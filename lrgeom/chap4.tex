\chapter{Comparision Geometry}
\label{chap:comparision}
\vspace*{-0.9cm}

% \startcontents[chapters]
% \printcontents[chapters]{}{4}{}

As a preliminary, we construct a special function on normal neighbourhoods: 
\marginnote{
    Given a time-oriented Lorentzian manifold, we define the radial distance function by the proper time $\tau: I^U(p) \to \mathbb{R}$ with $\tau(q):=\tau_U(p,q)>0$.
}
Given a Riemannian manifold $(M,g)$ and a normal chart $(U, x^i)$ around some $p \in M$. Define the \textbf{radial distance function} $r: U \to \mathbb{R}$ by $r(q)=d(p,q) >0$. If the coodinates are centered at $p$, we have the explicit form \[
    r(x)=\sqrt{(x^1)^2 + \cdots + (x^n)^2}
,\] and on $U \setminus \{p\}$ we get the \textbf{radial vector field} \[
\partial_r = \frac{x^i}{r(x)} \partial_i
.\]  This is smooth on $U \setminus \{p\}$ and independent of choice of charts.
\begin{definition}[Hessian]
    \marginnote{In local coordinates $(x^i)$, the Hessian has the (familiar) form \[
            \Hess_u = (\partial_j \partial_i u - \Gamma_{ij}^k \partial_k u) \, dx^i \otimes dx^j
    .\] }
    Let $(M,g)$ be a SRMF and $u: M \to \mathbb{R}$ smooth. The \textbf{covariant Hessian} is the $(0,2)$-tensor \[
        \Hess_u(X,Y):=(\nabla^2 u)(X,Y) = \nabla_X(\nabla_Y u) - \nabla_{\nabla_X Y}u
    .\]  The \textbf{Hessian operator} is the $(1,1)$-tensor \[
    \mathcal{H}_u := \Hess_u^\sharp
    .\] 
\end{definition}
\begin{remark}
   The Hessian is symmetric if $\nabla$ is the Levi-Civita-Connection (or, more generally, if and only if $\nabla$ is torsion-free):
   \begin{align*}
       \Hess_u(X,Y) &= Y(X(u)) - du(\nabla_YX) = (XY-[X,Y])u - du(\nabla_YX) \\
                    &= X(Y(u))-du(\nabla_YX+[X,Y])\\
                    &= X(Y(u))- du(\nabla_X Y-T(X,Y))\\
                    &= \Hess_u(Y,X) + T(X,Y)(u)
   \end{align*}
   Unraveling the definition of $\mathcal{H}_u$, we get \[
   g(\mathcal{H}_u(X), Y) = \Hess_u(X,Y) = \Hess_u(Y,X)=g(X,\mathcal{H}_u(Y))
   ,\] and hence $\mathcal{H}_u$ is self-adjoint.
\end{remark}
With the radial distance function defined above, we obtain the Hessian operator $\mathcal{H}_r$ associated to that function.
\begin{notation}
    Given a normal neighbourhood $U$, we denote the geodesic spheres of radius $r_0$ and proper time $\tau_0$, respectively, by \[
        \mathbb{S}_{r_0}:=\left\{x \in U \mid r(x)=r_0\right\} 
    \] and \[
    \mathbb{S}_{\tau_0} := \left\{x \in U \mid \tau (x)=\tau_0 \right\} 
    .\] 
        The Gauß' Lemma immediately tells us that $T_q\mathbb{S}_{r(q)}=\partial_r|_q^\perp \subseteq T_qM$ and similarly for $\tau$. Denote the projection on the radial tangent by \[
            \pi_q^r: T_qM \to T_q\mathbb{S}_{r(q)}
        .\] 
\end{notation}
\begin{lemma}
    Let $(M,g)$ be a Riemannian or time-oriented Lorentzian manifold, $(U,x^i)$ be a normal chart around $p \in M$ and let $r$ be the radial distance function. Then we have for all $q \in U \setminus \{p\}$:
    \begin{enumerate}
        \item $\mathcal{H}_r(\partial_r|_q)=0$ and $\mathcal{H}_\tau(\partial_\tau|_q)=0$
        \item For all $X \in T_qM$ with $X \perp \partial_r|_q$: \[
        \mathcal{H}_r(X)=\pi_q(\nabla_X(\grad r))
        ,\] and similarly for $X \perp \partial_\tau|_q$:
        \[
        \mathcal{H}_\tau(X)=\pi_q(\nabla_X(\grad \tau)) = \pi_q^\mathbb{H} ( \nabla_X \grad(\tau))
    ,\] where $\pi_q^\mathbb{H}: T_qM \to \partial_\tau|_q^\perp \cong T\mathbb{H}^{n-1}$ denotes the orthogonal projection \[
    \pi_q^\mathbb{H}(Y) = Y + g(Y, \partial_\tau)\partial_\tau
    .\] 
    \end{enumerate}
\end{lemma}
\begin{proof}
    We proof the Lorentzian case:
    \begin{enumerate}
        \item Let $Y \in T_qM$ be arbitrary. We have:
            \begin{align*}
                g(\mathcal{H}_\tau(\partial_\tau|_q), Y) &= g(\partial_\tau|_q, \mathcal{H}_\tau(Y)) = g(\partial_\tau|_q, - \nabla_Y \grad(\tau)) \\
                                                         &= g(\nabla_Y\partial_\tau, \partial_\tau|_q) = \frac{1}{2} Y_q(g(\partial_\tau, \partial_\tau)) =0,
            \end{align*}
            where we used symmetry of $g$ and the fact that $- \grad (\tau)=\partial_\tau$ since the metric is $-d\tau^2+\tilde{g}$.
        \item This follows since the previous remark showed $\mathcal{H}_\tau(X)= \nabla_X \grad (\tau)$ and $\mathcal{H}_\tau (X) \perp \partial_\tau$ follows from the upper equation.
    \end{enumerate}
\end{proof}
\begin{remark}
    \marginnote{
        Note that $g|_{TH_{\tau_0} \times TH_{\tau_0}}$ is Riemannian, so the restriction is self-adjoint for a positive-definite inner product even in the Lorentzian case.
    }
    For any $\tau_0$ such that \[
        H_{\tau_0} := \left\{q \in I^+_U(p)  \mid \tau_U(p,q)=\tau_0\right\} \neq \emptyset,
    \] $\mathcal{H}_\tau$ restricts to a self-adjoint linear operator \[
    \mathcal{H}_\tau: TH_{\tau_0} \to TH_{\tau_0}
\] and is given by the Weingarten map $\mathcal{H}_\tau(X) = -\nabla_X N$ where $N=\partial_\tau$ is the future-pointing unit normal to $H_{\tau_0}$.
\end{remark}
\begin{prop}[The Hessian and Jacobi Fields]
    Let $(\mathcal{L},g)$ be a LMF, $p \in \mathcal{L}$ and $U \subseteq \mathcal{L}$ be a normal neighbourhood around $p$. Set $\tau := \tau_U(p, \cdot)$. Let $\gamma: [0,b] \to U$ be a future-directed timelike unit-speed geodesic starting at $p$ and let $J \in \mathfrak{J}^\perp(\gamma)$ such that $J(0)=0$. Then \[
        \nabla_{\frac{d}{dt}}J(\tau_0)=-\mathcal{H}_\tau(J(\tau_0))
    \] for all $\tau_0 \in (0,b]$.
\end{prop}
\begin{proof}
    Let $v:=\dot{\gamma}(0)$ and $w:= \nabla_{\frac{d}{dt}}J(0)$. By equation \ref{eq:explicitpointjacobi}, we have a variation through geodesics \[
        \Gamma_s(t)=\exp_ {\sigma(s)}(tU(s))
        \]  where $\sigma(s)$ is any curve with $\sigma(0)=p$ and $\dot{\sigma}(0)=J(0)$. The vector field $U$ satisfies $U(0)=0$ and $\nabla_{\frac{d}{ds}}U(0)=w$ and the variational field of $\Gamma$ is $\left. \frac{\partial}{\partial s}\right|_{s=0} = J$. Since $J(0)=0$, choose $\sigma(s)\equiv p$. Since $w \perp v$ ($J$ is normal), we have \marginnote{Note that $g_p \cong \eta_p$.} \[
    v \in H := \left\{X \in T_p \mathcal{L} \mid g_p(X,X)=-1 \right\} 
.\] Hence, $w \in T_vH$. Since $U: I \to T_pM = T_{\sigma (s)}M$ with $\nabla_{\frac{d}{ds}}U(0) \in T_vH$ and $U(0)=v \in H$, we can choose $U$ such that $U(s) \in H$ for all $s$. Therefore, each curve $t \mapsto \Gamma_s(t)$ is a future-directed timelike unit-speed geodesic starting at $p$, implying 
\begin{equation}
    \label{eq:helper1}
\partial_t \Gamma_{s_0}(t_0)=\partial_\tau|_{\Gamma (s_0,t_0)}
\end{equation}
Now we compute:
\[
    \nabla_{\frac{d}{dt}}J = \nabla_{\frac{d}{dt}} \left. \frac{\partial}{\partial s}\right|_{s=0} \Gamma = \nabla_{\frac{d}{ds}} \left( \frac{\partial}{\partial t} \Gamma \right)(0,t)=\nabla_{\frac{d}{ds}} \left( \partial_\tau \circ \Gamma \right)(0,t)
    ,\] where we used equation \ref{eq:helper1}. Since $\partial_\tau$ is a smooth vector field on $I_U^+(p)$ (which is an open neighbourhood of $\Gamma_0(\tau_0)$ for any $\tau_0 \in (0,b]$). This implies \[
    \nabla_{\frac{d}{ds}} \left( \partial_\tau \circ \Gamma \right)(0,t) = \nabla_{\partial_s \Gamma|_{s=0}} \partial_\tau = \nabla_J \partial_\tau = - \nabla_J \grad (\tau) = \mathcal{H}_\tau (J)
\] using $J \perp \dot{\gamma} = \partial_\tau$.  
\end{proof} 
Recall the previously defined function 
\begin{equation}
    \label{eq:concurfunc}
    s_c(t):= 
    \begin{cases}
        R \sinh ( \frac{t}{R}) &c<0, \, R:= \frac{1}{\sqrt{c}}\\
        t &c=0\\
        R \sin (\frac{t}{R}) &c>0, \, R:=\frac{1}{\sqrt{c}}.
    \end{cases}
\end{equation}
\begin{prop}[Sectional Curvature and Hessian]
    Let $(\mathcal{L}, g)$ be a LMF, $p \in \mathcal{L}$ and $U$ be a normal neighbourhood of $p$. Then $(\mathcal{L}, g)$ has constant sectional curvature $c$ on $I^+_U(p)$ if and only if \[
        \mathcal{H}_\tau = - \frac{s_{-c}' \circ \tau}{s_{-c} \circ \tau} \cdot \pi^H
    \] on $I^+_U(p)$.
\end{prop}
\begin{proof}
    $(\Rightarrow):$ Let $q \in I_U^+(p)$, $\gamma: [0,b] \to M$ be the future-directed timelike unit-speed geodesic from $p$ to $q$, and let $\{E_1=\dot{\gamma}, E_2, \dots, E_n\}$ be an ONF along $\gamma$. The proposition about Jacobi fields in constant curvature spaces tells us that \[
        J_i := s_{-c}E_i \, i\geq 2
    \] are normal Jacobi fields along $\gamma$ with $J_i(0)=0$. Compute \[
    s_{-c}' E_i = \nabla_{\frac{d}{dt}} J_i = - \mathcal{H}_\tau (J_i) = -s_{-c} \mathcal{H}_\tau(E_i)
,\] where we used proposition $3.5$. Evaluation at $t=b$ and $s_{-c}(b)^{-1}$ noting $s_{-c}(b)\neq 0$ because $\gamma(b)=q \in U$, which is normal. Hence $q$ is not conjugate to $p$ implying $J_i(b)\neq 0$. This yields \[
\mathcal{H}_\tau (E_i(b))=- \frac{s_{-c}'(b)}{s_{-c}(b)} E_i (b)
,\] but $E_i(b)=\pi^H(E_i(b))$ because it is already orthogonal to $\dot{\gamma}(b)$. For $i=1$, we have \[
0=\mathcal{H}_\tau (E_1(b))=\pi^H (E_1(b))
\] because $E_1(b)=\dot{\gamma}(b)$.\par
$(\Leftarrow):$ Let $q \in I_U^+(p)$ and $\gamma: [0,b) \to U$ be the corresponding geodesic. From theorem 2.23, we know that \[
    \phi: I_U^+(p) \to \mathbb{R}_+ \times \mathbb{H}^{n-1}
\] given by $\phi(q)=(\tau(q), \xi(q))$ is a diffeomorpism onto its image. If we can show that it is an isometry between $g$ and $g_c := -d\tau^2 + s_{-c}(\tau)^2g_{\mathbb{H}^{n-1}}$, then $g$ has constant sectional curvature. Following the proof of 2.23, we immediately get 1 and 2a. In 2b, the constant curvature assumption only entered via Jacobi fields, more precisely via 2.20 yielding: Any $J \in \mathfrak{J}^\perp(\gamma)$ with $J(0)=0$ is given by \[
J(t)=ks_{-c}(t)E(t)
\] for some $k \in \mathbb{R}$ and $E \perp \dot{\gamma}$ as well as parallel. It only remains to show this in our setting. Let $J \in \mathfrak{J}^\perp(\gamma)$. We have \[
\nabla_{\frac{d}{dt}}J=\frac{s_{-c}'}{s_{-c}}J
\] on $(0,b]$. By the product rule, $\frac{1}{s_{-c}}J$ is parallel along $\gamma$ on $(0,b]$. Define $\tilde{E}(t):=\frac{1}{s_{-c}}J$ and let $E= \frac{\tilde{E}}{\|\tilde{E}\|}$. Then $E$ is a parallel unit VF orthogonal $\dot{\gamma}$ and $J=\|\tilde{E}\|  s_{-c} E$. Set $k:=\|\tilde{E}\|$, which is constant since $\tilde{E$ was parallel. It suffices to have this for $t \in (0,b]$ since this still gives $k=\|\nabla_{\frac{d}{dt}} J(0)\|$ by $k$ constant.
\end{proof}
