\chapter{Jacobi Fields}
\label{chap:jacobi}
\vspace*{-0.9cm}

% \startcontents[chapters]
% \printcontents[chapters]{}{3}{}

\section{The Jacobi Equation}
\label{sec:jacobieq}
In this section we focus on semi-Riemannian manifolds.
\begin{definition}[Variation through Geodesics]
    Let $(M,g)$ be semi-Riemannian and $\gamma: I \to M$ be a geodesic. If $K$ is another interval and \[
    \Gamma: K \times I \to M
    \] is a variation of $\gamma$ such that $\gamma_s: I \to M$ is a geodesic for all $s \in K$, we call $\Gamma$ a \textbf{variation through geodesics}. 
\end{definition}

Variation through geodesics give rise to several particular vector fields: The \emph{variational field} of $\Gamma$ is a vector field $J \in \mathfrak{X}(\Gamma)$ defined as \[
    J(t):=\left.\frac{\partial}{\partial s}\right|_{s=0} \Gamma (s,t)
.\] Furthermore, we define two accessory vector fields 
\[    T(s,t) := \frac{\partial \Gamma}{\partial t} \]
\[S(s,t)  := \frac{\partial \Gamma}{\partial s} \]
which will be useful.
\begin{lemma}[Symmetry Lemma]
    Let $\Gamma: K \times I \to M$ be a smooth family of curves. Then we have \[
        \nabla_{\frac{d}{ds}} T = \nabla_{\frac{d}{dt}}S
    .\] 
\end{lemma}
\begin{proof}
    Take local coordinates $(x^i)$ on some coordinate neighbourhood and write $\Gamma(s,t)=(\gamma^1(s,t), \dots, \gamma^n(s,t))$. We have $S = \frac{\partial \gamma^k}{\partial s} \partial_k$ and $T= \frac{\partial \gamma^k}{\partial t} \partial_k$. Calculating the left side directly yields:
    \begin{align*}
\nabla_{\frac{d}{ds}} T &= \nabla_{\frac{d}{ds}}\left( \frac{\partial \gamma^k}{\partial s} \partial_k \right) = \frac{\partial^2 \gamma^k}{\partial t \partial s} \partial_k + \frac{\partial \gamma^i}{\partial t} \frac{\partial \gamma^j}{\partial s} \nabla_{\partial_j}\partial_i \\
                        &= \left(  \frac{\partial^2 \gamma^k}{\partial t \partial s} + \frac{\partial \gamma^i}{\partial t} \frac{\partial \gamma^j}{\partial s} \Gamma_{ji}^k \right) \partial_k
    \end{align*}
    Exchanging $i \leftrightarrow j$ and using the symmetry $\Gamma_{ij}^k=\Gamma_{ji}^k$ yields the desired identity.
\end{proof}
\begin{lemma}[Curvature Lemma]
    Let $(M,g)$ be semi-Riemannian and $\Gamma: K \times I \to M$ be a smooth family of curves. Then for any $V \in \mathfrak{X}(\Gamma)$, we have \[
        \nabla_{\frac{d}{ds}} \nabla_{\frac{d}{dt}} V - \nabla_{\frac{d}{dt}} \nabla_{\frac{d}{ds}} = R(S, T)V
    .\] 
\end{lemma}
\begin{proof}
    Take local coodinates $(x^i)$ and write $\Gamma(s,t)=(\gamma^1(s,t), \dots, \gamma^n(s,t))$ as well as $V=V^i\partial_i$. We calculate the two left derivatives explicitly:
    \[
        \nabla_{\frac{d}{dt}}V=\frac{\partial V^i}{dt}\partial_i + V^i \nabla_{\frac{d}{dt}} \partial_i
    \] and \[
    \nabla_{\frac{d}{ds}} \nabla_{\frac{d}{dt}} V= \left( \frac{\partial^2 V^i}{\partial s \partial t} + \frac{\partial V^i}{\partial t} \nabla_{\frac{d}{ds}} + \frac{\partial V^i}{\partial s} \nabla_{\frac{d}{dt}} + V^i \nabla_{\frac{d}{ds}} \nabla_{\frac{d}{dt}} \right) \partial_i
    .\] 
Exchanging $s \leftrightarrow t$ yields $\nabla_{\frac{d}{dt}} \nabla_{\frac{d}{ds}} V$ and we see immediately that after subtraction, only the rightmost term remains: \[
\left( \nabla_{\frac{d}{ds}} \nabla_{\frac{d}{dt}} - \nabla_{\frac{d}{dt}} \nabla_{\frac{d}{ds}}\right) V = V^i \left(\nabla_{\frac{d}{ds}} \nabla_{\frac{d}{dt}} - \nabla_{\frac{d}{dt}} \nabla_{\frac{d}{ds}} \right) \partial_i
.\] 
Extending $\partial_i$ and the covariant derivative, we can calculate:
\[
    \nabla_{\frac{d}{ds}} \nabla_{\frac{d}{dt}} \partial_i = \nabla_{\frac{d}{ds}} \left(  \frac{\partial \gamma^j}{\partial t} \nabla_{\partial_j}\partial_i\right)= \frac{\partial^2 \gamma^j}{\partial s \partial t} \nabla_{\partial_j} \partial_i + \frac{\partial \gamma^j}{\partial t} \frac{\partial \gamma^k}{\partial s} \nabla_{\partial_k} \nabla_{\partial_j}\partial_i
\] and analogously \[
    \nabla_{\frac{d}{dt}} \nabla_{\frac{d}{ds}} \partial_i = \frac{\partial^2 \gamma^j}{\partial t \partial s} \nabla_{\partial_j} \partial_i + \frac{\partial \gamma^j}{\partial s} \frac{\partial \gamma^k}{\partial t} \nabla_{\partial_k} \nabla_{\partial_j}\partial_i
.\] 
Exchanging $j \leftrightarrow k$ and subtracting cancels out the left term and we obtain: \[
    \nabla_{\frac{d}{ds}} \nabla_{\frac{d}{dt}} \partial_i -     \nabla_{\frac{d}{dt}} \nabla_{\frac{d}{ds}} \partial_i \partial_i = \frac{\partial \gamma^j}{\partial t} \frac{\partial \gamma^k}{\partial s} (\nabla_{\partial_k} \nabla_{\partial_j}-\nabla_{\partial_j}\nabla_{\partial_k}) \partial_i = R(S,T)\partial_i
.\] 
\end{proof}
We are now ready to consider the main theorem of this section:
\begin{theorem}[Jacobi Equation]
    Let $(M,g)$ be semi-Riemannian, $\gamma: I \to M$ be a geodesic and $\Gamma: K \times I \to M$ be a variation of $\gamma$ through geodesics with variational field $J$. Then $J$ satisfies the \textbf{Jacobi Equation}: 
    \begin{equation}
        \nabla_{\frac{d}{dt}}^2 J + R(J,\dot{\gamma})\dot{\gamma}=0.
    \end{equation}
\end{theorem}
\begin{proof}
    The fact that $\Gamma$ is a variation through geodesics tells us that $\nabla_{\frac{d}{dt}}T \equiv 0$. Dervating once more yields $\nabla_{\frac{d}{ds}}\nabla_{\frac{d}{dt}}\equiv 0$. By the curvature and the symmetry lemma, we have \[
        \nabla_{\frac{d}{ds}} \nabla_{\frac{d}{dt}} T = \nabla_{\frac{d}{dt}} \nabla_{\frac{d}{ds}} T + R(S,T)T = \nabla_{\frac{d}{dt}}^2 S + R(S,T)T
    .\] 
At $s=0$, we have $S(0,t)=\partial_s \Gamma_s(t)= J(t)$ and $T(0,t)=\partial_t \Gamma_0 (t)=\dot{\gamma}(t)$ as claimed.
\end{proof}
\begin{prop}[Existence and Uniqueness]
Let $(M,g)$ be a SRMF, $\gamma: I \to M$ be a geodesic, $t_0 \in I$ and $p:=\gamma(t_0)$. For all $v,w \in T_pM$, there is a unique Jacobi field $J$ along $\gamma$ satisfying the initial data \[
        J(t_0) = v, \, \nabla_{\frac{d}{dt}} J(t_0)=w
    .\] 
\end{prop}
\begin{proof}
    Take $J \in \mathfrak{X}(\gamma)$ and choose a parallel ONF $(E_i)$. We write $v = v^iE_i(t_0)$, $w = w^iE_i(t_0)$, $\dot{\gamma}(t)=\gamma^i(t)E_i(t)$ and $J=J^i(t)E_i(t)$. The Jacobi equation holds if and only if the following equation is satisfied:
    \marginnote{Note that all terms of the form $\nabla_{\frac{d}{dt}} E_i$ in the left term vanish as the ONF is parallel w.r.t. $\nabla$.}
    \begin{align*}
        0 &= \nabla_{\frac{d}{dt}}^2 J + R(J, \dot{\gamma})\dot{\gamma} = \nabla_{\frac{d}{dt}}^2 (J^iE_i)  + R(J^j E_j, \dot{\gamma}^kE_k)\dot{\gamma}^lE_l\\
          &= \ddot{J}^iE_i + J^j \dot{\gamma}^k \dot{\gamma}^l R(E_j,E_k)E_l 
    \end{align*}
    This yields a second-order system of $n$ equations \[
        \frac{d^2 J^i(t)}{dt^2} = - \dot{J}^j  (t) \dot{\gamma}^k(t)\dot{\gamma}^l(t) R^i_{jkl}
    .\] Substituting $W^i(t):=\dot{J}^i(t)$, this reduces to $2n$ first-order equations. The existence and uniqueness theorem of ODEs on manifolds thus guarantees that the claim holds.
\end{proof}
\begin{definition}[Jacobi Field]
    Let $(M,g)$ be a SRMF and $\gamma: I \to M$ be a geodesic. We call a vector field $J \in \mathfrak{X}(\gamma)$ a \textbf{Jacobi field} if it satiesfies the Jacobi equation. The space $\mathfrak{J}(\gamma) \subseteq \mathfrak{X}(\gamma)$ denotes the space of all Jacobi fields along $\gamma$.
\end{definition}
\begin{corollary}
   Let $(M,g)$ be an $n$-dimensional SRMF and $\gamma: I \to M$ be any geodesic. Then $\mathfrak{J}(\gamma)$ is a $2n$-dimensional linear subspace of $\mathfrak{X}(\gamma)$.
\end{corollary}
\begin{proof}
    Linearity of the Jacobi equation guarantees that $\mathfrak{J}(\gamma)$ is linear. Fixing some $p=\gamma(t_0)$, we have an isomorphism $\mathfrak{J}(\gamma) \cong T_p M \oplus T_p M$ given by the previous preposition as $J \mapsto (J(t_0), \nabla_{\frac{d}{dt}} J(t_0))$.
\end{proof}
\begin{prop}
    Let $(M,g)$ be a SRMF and $\gamma: I \to M$ be a geodesic. If either 
    \begin{enumerate}[(i)]
        \item $M$ is complete, or
        \item $I$ is compact
    \end{enumerate}
    then every Jacobi field along $\gamma$ is the variation field of some variation of $\gamma$ though geodesics.
\end{prop}
\begin{proof}
    \marginnote{The idea is the following: We define a small curve through a starting point on $\gamma$ and use the assumptions to guarantee that (after eventually contracting the domain) $\exp$ can be used to define a variation through geodesics. After that, we use uniqueness of Jacobi fields to show that the variational vector field of this constructed variation has to agree with $J$.}
    Let $J \in \mathfrak{J}(\gamma)$. By translating, we can assume $0 \in I$ and set $p:=\gamma(0)$, $v:=\dot{\gamma}(0)$. This means we can write $\gamma(t)=\exp_p(tv)$ for all $t \in I$. By construction of the tangent space, there is a small open interval $(-\epsilon, \epsilon)$ and a smooth curve $\sigma: (- \epsilon, \epsilon) \to M$ satisfying \[
        \sigma(0)=p, \, \dot{\sigma}(0)=J(0)
    .\] Choose a vector field $V \in \mathfrak{X}(\sigma)$ with data \[
    V(0)=w \, \nabla_{\frac{d}{ds}}V(0)=\nabla_{\frac{d}{dt}}J(0)
    .\] We want to define a variation through geodesics by 
    \begin{equation}
    \label{eq:jacobivar}
        \Gamma(s,t):=\exp_{\sigma(s)}(tV(s)).
    \end{equation}
    If $M$ is geodesically complete, we can always define $\Gamma$ on $(-\epsilon, \epsilon) \times I$. If $I$ is compact, we can use that $\mathcal{E}_p$ is open and contains the compactum $\{(p,tv)\in TM \mid t\in I\}$. Therefore, we find some $\delta >0$ such that $\Gamma$ is definable on $(-\delta, \delta) \times M$. \\
Evaluating at $s=0$ yields \[
    \Gamma_0(t)= \exp_{\sigma(0)}(tV(0))=\exp_p(tv)=\gamma(t),
\] which tells us that $\Gamma$ is indeed a variation of $\gamma$: By definition of $\exp$, it is also a variation thorugh geodesics with variational field $W(t):=\frac{\partial \Gamma}{\partial s}(0,t) \in \mathfrak{J}(\gamma)$. \\
Now we match initial data:
\begin{enumerate}
    \item $W(0)=\frac{\partial}{\partial s} \Gamma(0,0) = \dot{\sigma}(0)=J(0)$
    \item We have $\frac{\partial}{\partial t}\Gamma_s(0) = V(s)\exp(0)=V(s)$. By applying the symmetry lemma, we obtain: \[
            \nabla_{\frac{d}{dt}}W(0)=\nabla_{\frac{d}{dt}} \frac{\partial}{\partial s} \Gamma(0,0) = \nabla_{\frac{d}{ds}} \frac{\partial}{\partial t}\Gamma(0,0) = \nabla_{\frac{d}{ds}}V(0)=\nabla_{\frac{d}{dt}}J(0)
    .\] Since $J$ and $W$ have the same initial data, we can conclude by uniqueness that $J \equiv W$.
\end{enumerate}
\end{proof}

\section{Jacobi Fields vanishing at a point}
\label{sec:jacobipoint}
We turn our attention to Jacobi fields which vanish at some point $p \in M$.
\begin{lemma}
    Let $(M,g)$ be a SRMF, $I \ni 0$ an interval, $\gamma: I \to M$ be a geodesic, and $J \in \mathfrak{J}(\gamma)$ such that $J(0)=0$. 
    If $M$ is geodesically complete or $I$ is compact, $J$ is the variation field of the geodesic variation
    \begin{equation}
    \label{eq:explicitpointjacobi}
    \Gamma (s,t):=\exp_p(t(v+sw))
\end{equation}
 with $p = \gamma(0)$, $v = \dot{\gamma}(0)$, and $w=\nabla_{\frac{d}{dt}} J(0)$.
\end{lemma}
\begin{proof}
    This follows directly from equation \ref{eq:jacobivar} by taking $\sigma(s)\equiv p$, and $V(s)=v+sw$.
\end{proof}
\begin{prop}[Jacobi Fields Vanishing at a Point]
    Let $(M,g)$ be a SRMF, $p \in M$, $\gamma: I \to M$ be a geodesic with $0 \in I$ and $\gamma(0)=p$.
    \begin{enumerate}
        \item For every $w \in T_pM$, the Jacobi field with initial data $J(0)=0$ and $\nabla_{\frac{d}{dt}}J (0)=w$ is given by
        \begin{equation}
            \label{eq:jacobipoint}
            J(t)=(\exp_p)_{\ast, tv} (tw)
        \end{equation}
        with $v = \dot{\gamma}(0)$ and $tw \in T_{tv}(T_pM) \cong T_pM$.
    \item If $(x^i)$ are normal coordinates on a normal neighbourhood completely containing $\im \gamma$ with $w = w^i \partial_i|_0$, $J$ admits the form
        \begin{equation}
            J(t)=t w^i\partial_i|_{\gamma(t)}.
        \end{equation}
    \end{enumerate}
\end{prop}
\begin{proof}
    \begin{enumerate}
        \item Restricting to a compact subinterval of $I$, we can use equation \ref{eq:explicitpointjacobi} and calculate the pushforward: \[
            \frac{\partial}{\partial s} \Gamma(0,t)=\left( \exp_p(t(v+sw)) \right)_{\ast, s=0} = (\exp_p)_{\ast, tv} \circ (tw)=(\exp_p)_{\ast, tv} (tw)
    .\] As all $t \in I$ are contained in such compact subintervals, this holds for all $t$.
    \item Given normal coordinates $(U,x^i)$, the exponential map is the identity in coordinates, so we get $\Gamma(s,t)=t(v^i+sw^i)\partial_i$. We can calculate directly:
        \[
            \frac{\partial}{\partial s} \Gamma(0,t) = tw^i \partial_i|_{\gamma(t)}
        .\] 
    \end{enumerate}
\end{proof}
This makes it possible to reach all vectors in a normal neighbourhood with Jacobi fields:
\begin{corollary}
    Let $(M,g)$ be a SRMF, $p \in M$ and $U$ be a normal neighbourhood centered at $p$. For any $q \in U \setminus \{p\}$, every vector $v \in T_qM$ is the value of a Jacobi field $J$ vanishing at $p$ along a radial geodesic.
\end{corollary}
\begin{proof}
    Take normal coordinates $(U, x^i)$ and $q  = (q^1, \dots, q^n) \in U \setminus \{p\}$ as well as $w = w^i \partial_i|_q \in T_qM$. The radial geodesic $\gamma(t):=(tq^1, \dots, tq^n)$ has endpoints $\gamma(0)=p$ and $\gamma(1)=q$. By the previous preposition, \[
        J(t):=tw^i\partial_i |_{\gamma(t)}
    \] is a Jacobi field satisfying $J(0)=0$ adn $J(1)=w$. 
\end{proof}

\section{Normal and Tangential Jacobi Fields}
\label{sec:normaltangential}
