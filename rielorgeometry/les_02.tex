\chapter{Riemannian Geometry}
In this chapter, we are concerned with Riemannian manifolds as metric spaces. The main goal is to prove the theorem of Hopf-Rinow.
\begin{definition}[Regular Curve]
   A piecewise $\mathcal{C}^1$-curve \[
       \gamma: [a,b] \to M
   \] is called \textbf{regular} if 
   \[
       \forall s \in [a,b]:\, \dot{\gamma}(s)\neq 0
   \] and \[
   \dot{\gamma}_{\pm}(t_i) \neq 0
   \] at all $\mathcal{C}^1$-break-points.
\end{definition} 
\begin{definition}[Arc-length]
    Let $(M,g)$ be a semi-Riemannian manifold and $\gamma: [a,b]\to M$ a (piecewise) $\mathcal{C}^1$-curve. The \textbf{arc-length} is defined to be the functional
    \[
        L[\gamma] = \int_a^b \sqrt{| g(\dot{\gamma}(t), \dot{\gamma}(t))|} \, dt
    .\] 
\end{definition}
\begin{remark}
   \begin{enumerate}
      \item In the Riemannian case, the $| \cdot |$ is redundant.
        \item In semi-Riemannian geometry, there are curves with $L[\gamma]=0$.
        \item The arc-length functional is invariant under length parametrization.
        \item If $\gamma$ is regular, there exists a strictly monotonous reparametrization \[
            \phi: [\tilde{a}, \tilde{b}] \to [a,b] \] such that $\tau := \gamma \circ \phi$ satisfies $g(\dot{\tau}, \dot{\tau})=1$. This is a reparametrization by arc-length: \[
            L[\tau_{[\tilde{a},s]}]=s-\tilde{a}
        \] for all $s \in [\tilde{a}, \tilde{b}]$.
   \end{enumerate} 
\end{remark}

\begin{theorem}[Gauß' Lemma]
   The exponential map is a radial isometry: For any $p \in M$, $x \in \mathcal{D}_p$ and $v,w \in T_x(T_pM) \cong T_pM$ with $v = \alpha x$ for some $\alpha \in \mathbb{R}$, the equations \[
       g_{\exp_p(x)}(d(\exp_p(v))_x, d(\exp_p(w))_x)=g_p(v,w)
   \]  and \[
   \dot{\gamma}(t)=\frac{d}{dt}\exp_p(tv)
   \] hold.
\end{theorem}
