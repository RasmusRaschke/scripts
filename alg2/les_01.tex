\chapter{Introduction}
\section{Ring Theory}
\subsection{\textcolor{blue}{Lecture 15.10.25}}
\subsection{\textcolor{blue}{Lecture 17.10.25}}
\subsection*{Important Ring Homomorphisms}
\begin{theorem}[Initial Ring]
    The ring of integers $\mathbb{Z}$ is initial in $\mathtt{Ring}$, i.e. for every unital ring $R$, there is a unique ring homomorphism $f: \mathbb{Z} \to \mathbb{R}$ and $f$ is determined by $f(1)=1_R$.
\end{theorem}
The last statement works by using the homomorphism property \[
f(\sum 1)=\sum f(1)
.\] 
\begin{theorem}[Terminal Ring]
    The \emph{null ring} is terminal in $\mathtt{Ring}$, i.e. for every ring there is a unique ring homomorphism $f: R \to \{0\}$.
\end{theorem}
\begin{eg}
    Let $(A,+)$ be an abelian group and denote by $\End(A)$ the endomorphisms $A \to A$. Given any $f,g \in End(A)$, we define \[
        (f+g)(x):=f(x)+g(x)
    \] and \[
    (f \cdot g)(x):=f(g(x))
\] for any $x \in A$. This makes $\End(A)$ an abelian group. The identity map $1 \in \End(A)$ turns $\End(A)$ into a ring. 
\end{eg}
\begin{ex}
   What happens if $A$ is not abelian?
\end{ex}
There are several standard constructions of rings:
\begin{definition}[Opposite Ring]
    Let $(R,+,\cdot)$ be a ring. The \textbf{opposite ring} $R^\text{op}$ is the same abelian group $(R,+)$ together with the inverted multiplication \[
        (r,s) \mapsto s \cdot r
    .\] 
\end{definition}
\begin{definition}[Polynomial Ring]
    Given any ring $R$, define the \textbf{polynomial ring} of polynomials in $x$ with coefficients in $R$ by \[
        R[x]:= \left\{ \sum_i a_ix^i \mid a_i \in R,\, a_i=0 \text{ for }i \text{ suff. large}\right\} 
    .\]  Addition, multiplication and identity are inherited from $R$.
\end{definition}
We construct higher polynomial rings $R[x_1, \dots, x_n]:=R[x_1, \dots, x_{n-1}][x_n]$ inductively. For $p(x)\in \mathbb{F}[x]$, the degree is the highest non-zero power of $x$ appearing in $p(x)$. We have \[
\deg(p(x)\cdot q(x))=\deg(p(x))+\deg(q(x))
.\]  This is not well-defined unless $R$ is an integral domain: $\mathbb{R}[x]$ to $\mathbb{Z}/6\mathbb{Z}[x]$ shows this.
\begin{eg}
    The ring of \emph{Laurent polynomials} is given by $R[x,x^{-1}]$.
\end{eg}
\begin{eg}
    The \textbf{ring of power series} in $x$ is given by \[
        R\llbracket x \rrbracket:=\left\{ \sum_{i \geq 0}a_ix^i \mid a_i \in R \right\}
    ,\] so we allow infinite sums. If one considers $1-x \in \mathbb{R}[x]$, it does not have an inverse in $\mathbb{R}\llbracket x \rrbracket]]$. However, in $\mathbb{R}\llbracket x \rrbracket$ one has the (formal) geometric series \[
    \frac{1}{1-x} = \sum_{i \geq 0} x^i
    \] as an inverse.
\end{eg}
\begin{definition}[Principal Ideal]
    A (left/right/two-sided) \textbf{principal ideal} of a ring $R$ is a subset $Ra / aR / RaR$ for some $a \in R$ defined by \[
        Ra:= \{ ra \mid r \in R \}
    .\] 
\end{definition}
\begin{ex}
    Principal ideals are ideals.
\end{ex}
\begin{remark}
   If $R$ is commutative, all these notions collapse to one and one writes $\langle a \rangle$ for the ideal generated by $a$. 
\end{remark}
\begin{eg}
We already know many principal ideals, e.g. $\langle 2 \rangle$ in $2\mathbb{Z}$ or $\langle n \rangle$ in $n \mathbb{Z}$. In $\mathbb{Z}$, every ideal is principal. For any ring, $\langle 0 \rangle$ and $\langle 1 \rangle$ are principal ideals. In polynomial rings, we always have principal ideals in the form of powers of $x$, e.g. $\langle x \rangle$, $\langle x^2 \rangle$, or $\langle x^2+1 \rangle$.  In $R[x,y]$, $\langle x,y \rangle$ is a principal ideal.
\end{eg}
\section{Modules}
The idea is to generalize the idea of vector spaces, which are over fields, to something defined over rings.
\begin{definition}[Module]
    A \textbf{left $R$-module} (module over $R$) is an abelian group $M$ together with a map 
    \begin{align*}
        R \times M &\to M \\
        (r,m) &\mapsto r \cdot m
\end{align*}
satisfying
\begin{enumerate}
    \item $r(m+n)=rm+rn$
    \item $(r+s)m=rm+sm$
    \item $(rs)m=r(sm)$
    \item $1_R \cdot m = m$
\end{enumerate}
Right modules are defined analogously.
\end{definition}
\begin{ex}
    There are several statements easy to prove:
    \begin{itemize}
        \item $\forall m \in M: 0 \cdot m = 0_M$    
        \item $(-1) \cdot m = -m$
    \end{itemize}
    
\end{ex}
\begin{theorem}[Abelian groups as module]
    Every abelian group is a $\mathbb{Z}$-module in exactly one way.
\end{theorem}
\begin{proof}
    $\mathbb{Z}$ is initial, so there is a unique homomorphism $\mathbb{Z} \to R$ for all unital $R$.
\end{proof}
This shows that abelian groups are nothing but $\mathbb{Z}$-modules (or, abstractly, $\mathbb{Z}$-vector spaces). $\End(\mathtt{AGrp})$ is a ring and we have an action of $\mathbb{Z}$ on any abelian groups by endomorphisms.
\begin{eg}
    Every ring $R$ is a (left) $R$-module over itself. Furthermore, every (left) ideal $\mathcal{I} \subseteq R$ is a (left) $R$-module.  Of course, there is also the trivial module $M=\{0\}$.
\end{eg}
If $\mathcal{I} \subseteq R$ is a left ideal, $R / I$ is not a ring.
\begin{ex}
   If $\mathbb{I} \subseteq R$ is a left ideal, $R / I$ is a left module. 
\end{ex}
\subsection*{Submodules}
\begin{definition}[Submodule]
    A \textbf{submodule} $N$ of a left $R$-module $M$ is a subgroup preserved by the action of $R$, i.e. \[
    \forall r \in R \, \forall n \in N: \, rn \in N
    .\] 
\end{definition}
\begin{note}
    The (left) ideals of $R$ are the left submodules of $R$ viewing $R$ as a module over itself.
\end{note}
\begin{definition}[Simple Module]
    A module $M$ is \textbf{simple} if its only submodules are $M$ and $\{0\}$.
\end{definition}
\subsection*{Module Homomorphisms}
\begin{definition}[Module Homomorphism]
    An $R$\textbf{-module homomorphism} is a homomorphism of abelian groups compatible with the $R$-module structure:
    If $M,N$ are $R$-modules and $\phi: M \to N$ is a homomorphism, then
    \begin{enumerate}
        \item $\forall m_1,m_2 \in M: \, \phi(m_1+m_2)=\phi(m_1)+\phi(m_2)$
        \item $\forall r \in R, \, \forall m \in M: \ \phi(rm)=r\phi(m)$.
    \end{enumerate}
\end{definition}
\begin{theorem}[Kernel and Image are Subs]
   Let $\phi$ be an $R$-mod homomorphism. Both $\ker \phi$ and $\im \phi$ are submodules. 
\end{theorem}
\subsection{\textcolor{blue}{Lecture 22.10.25}}
\begin{definition}[Center]
    Let $R$ be a ring. The \textbf{center} of $R$ is defined as \[
    Z(R):=\left\{x \in R \mid \forall r \in R: xr = rx \right\} 
    .\] 
\end{definition}
\begin{ex}
    Let $M$ be an $R$-module and $r \in Z(R)$, then \[
    rM := \left\{r \cdot m \mid m \in M\right\}
    \] is a submodule. If $\mathcal{I} \subseteq R$ is any left ideal of $R$, then $\mathcal{I}M$ is a submodule of $M$.
\end{ex}
\begin{prop}[Submodules are normal]
   Let $N \subseteq M$ be a submodule. Then, $N$ is a normal subgrup of $M$ viewed as abelian groups.
\end{prop}
\begin{remark}
    This tells us that $M / N$ is an abelian group. We want to give it some $R$-mod structure as follows: Consider the canonical projection $\pi: M \to M / N$ with $\pi(m)=m+N$. We have \[
    r \cdot (m+N)=r \cdot \pi(m) = \pi(r \cdot m)=r \cdot m + N
    ,\] hence we define $r \cdot (m+N)=r \cdot m +N$. This is closed under addition.
\end{remark}
\begin{prop}[Quotient Submodule]
    Let $M$ be an $R$-module and $N \subseteq M$ be a submodule. Then, $M / N$ is also an $R$-module.
\end{prop}
\begin{prop}[Quotient Ideal]
    Suppose $\mathcal{I} \subseteq R$ is a two-sided ideal. Then, $\mathcal{I}$, $R$ and $R / \mathcal{I}$ are all $R$-modules.
\end{prop}
\begin{theorem}[Universal Property of Quotient Modules]
   Let $M$ be an $R$-module and $N \subseteq M$ be a submodule. Then for every $R$-module homomorphism \[
   \phi: M \to P
   \]  such that $N \subseteq \ker \phi$, there exists a unique $R$-mod homomorphism $\widetilde{\phi} $ that makes the following diagram commute:
   \begin{figure}[H]
       \centering
   \begin{tikzcd}
       M \ar[r, "\pi"] \ar[d, "\phi"']& M / N \ar[dl, "\exists ! \widetilde{\phi}", dashed] \\
       P
   \end{tikzcd}
\end{figure}
\end{theorem}
\begin{proof}
    Define \[
    \widetilde{\phi}: M / N \to P
    \]  by $\widetilde{\phi} (m+N):=\phi(m)$. We have to check that it is a $R$-mod homomorphism, well-defined and unique.
\end{proof}
\begin{theorem}[Homomorphism Theorem for Rings]
    Every $R$-module homomorphism $\phi: M \to P$ can be decomposed as 
    \begin{figure}[H]
        \centering
        \begin{tikzcd}
            M \ar[r,"\phi"] \ar[d, "\pi"]& P \\
            M / \ker(\phi) \ar[r, "\overline{\phi}"] & \im(\phi) \ar[u, hook, "\iota"]
        \end{tikzcd}
    \end{figure}
    where $\overline{\phi}$ is an isomorphism induced by the universal property.
\end{theorem}
\begin{proof}
    $\widetilde{\phi}$ with $\im(\phi)$ as target and $\ker(\phi)$ as the quotient, show it is iso.
\end{proof}
\begin{corollary}
    Suppose $\phi: M \to P$ is a surjective $R$-module homomorphism. Then \[
    P \cong M / \ker(\phi)
    .\] 
\end{corollary}
\subsection*{Left-Right Confusion}
We denote left $R$-modules by $_R M$ and right modules by $M_R$.
\begin{remark}
    Every right $R$-module $M_R$ can be considered as a left $R^\text{op}$-module $_{R^\text{op}}M$ by the opposite multiplication \[
        \mu^\text{op}: R^\text{op} \times M \to M
    \] with $\mu^\text{op}(r,m)=m \cdot r.$ Equivalently, $_R M \cong M_{R^\text{op}}$.
\end{remark}
\begin{lemma}
    Let $R$ be a commutative ring. Then every left module is natrually a right module, and vice versa.
\end{lemma}
\begin{definition}[Bimodule]
    Let $R,S$ be not necessarily distinct rings. An $R-S$ bimodule $_R M_S$ is an abelian group $M$ that is a left $R$-module and a right $S$ module such that \[
    \forall r \in R, s \in S, m \in M: (r\cdot m) \cdot s=r\cdot (m \cdot s)
    .\] 
\end{definition}
\begin{definition}[Generated Submodule]
    Let $M$ be an $R$-module and $A \subseteq M$ be a subset. Then \[
        \langle A \rangle := \left\{\sum_{i \in I} r_i a_i \mid r_i \in R, a_i \in A, \, \text{only finitely many } a_i r_i \neq 0 \right\} 
    \] denotes the submodule generated by $A$.
\end{definition}
\begin{remark}
    We also have \[
        \langle A \rangle = \bigcap_{U_i \subseteq M} U_i
    .\] where each $U_i$ is a submodule containing $A$, so $\langle A \rangle$ is the smallest submodule containing $A$.
\end{remark}
\begin{definition}[Generators and Cyclicity]
    Let $M$ be an $R$-module and $A \subseteq M$.
    \begin{itemize}
        \item If $M = \langle A \rangle$, $A$ is the \textbf{generating set} of $M$. 
        \item If $A$ generates $M$ and is finite, $M$ is called \textbf{finitely generated}.
        \item A module $M$ is \textbf{cyclic} if it admits a generating set with a single element.
    \end{itemize}
\end{definition}
\begin{ex}
   Show that the cyclic groups are all cyclic $\mathbb{Z}$-modules. 
\end{ex}
\begin{definition}[Annihilator]
   Let $M$ be an $R$-module. The \textbf{annihilator} of a subset $U \subseteq M$ is given by \[
   \Ann_R(U):=\left\{r \in R \mid \forall u \in U: r \cdot u =0\right\} 
   .\] 
\end{definition}
If $M$ is a left $R$-module, the annihilator of some $U \subseteq M$ is a left ideal of $R$. For a single $x \in M$, we write \[
\Ann_R(x):=\left\{r \in R \mid r \cdot x = 0\right\} 
.\] 
\begin{corollary}
    There is a isomorphism of left $R$-modules \[
    R / \Ann(x) \to Rx
    .\] 
\end{corollary}
\begin{prop}
    If $U \subseteq M$ is a submodule, then $\Ann(U)$ is a two-sided ideal of $R$.
\end{prop}
\subsection*{Algebras}
\begin{definition}[Associative Algebra]
    Let $R$ be a commutative ring. An \textbf{associative $R$-algebra} is an $R$-module $A$ with the structure of an associative \emph{but not necessarily unital} ring, such that ring addition agrees with module addition \[
        \underbrace{a_1 + a_2}_\text{algebra} := \underbrace{a_1 + a_2}_\text{module}
    \] and satisfies \[
    \lambda(m\cdot n)=(\lambda m) \cdot n = m \cdot (\lambda n)
    \] for $\lambda \in R$ and $m,n \in A$. If there is a unit, we call $A$ \textbf{unital}.
\end{definition}
\begin{definition}[Group Ring]
    Let $G$ be a group and $K$ be a commutative ring. The \textbf{group ring} $K[G]$ is the abelian group of maps \[
    f: G \to K
    \] that vanish on all but finitly many elements of $G$.
\end{definition}
\begin{note}
    Elements of $K[G]$ can be expressed uniquely as linear combinations \[
        f = \sum_{g \in G} f_g \delta_g
    ,\] where $f_g \in K$ and $\delta_g$ is the map $g \mapsto 1 \in K$. This is often written as $f=\sum_g f(g)g$ for $f(g)\in K$. The multiplication is given by convolution:
    \[
        \left( \sum_g a_g g\right)\ast \left( \sum_h b_h h\right) = \sum_{x \in G} \left( \sum_{g,h \in G, g \cdot h =x} a_g b_h \right)x
    .\] We obtain the identity $\delta_g \ast \delta_h = \delta_{gh}$.
\end{note}
\begin{ex}
    Let $G = \mathbb{Z}_3$ represented by $\langle a \mid a^3=1 \rangle$. Choose $K = \mathbb{C}$. $\mathbb{C}[\mathbb{Z}_3]$ has elements \[
    p = z_01 + z_1 a + z_2 a^2
    .\] Show that \[
    \mathbb{C}[\mathbb{Z}_3]=\mathbb{C}[a] / \langle a^3 -1 \rangle
    .\] 
\end{ex}
\begin{definition}[Representation]
    A \textbf{representation} of a group $G$ is a pair $(V,\rho)$ where $V$ is a $\mathbb{K}$-vector space, and $\rho$ is a group homomorphism \[
        \rho: G \to \GL(V):=\left\{\phi \in \End(V) \mid \phi \, \text{invertible}\right\} 
    .\] 
\end{definition}
\begin{remark}
   Given a $G$-representation $(V, \rho)$ then the map 
   \begin{align*}
       G \times V &\to V\\
       (g,v) &\mapsto \rho(g)v
   \end{align*} defines a module action for the ring $K[G]$.\\
   Given $(V,\rho)$, can one find a $K[G]$-module? Yes, since we can define \[
   \sum_g (\lambda_g \delta_g)v:=\sum \lambda_g \rho(g)(v)
,\] which is a $K[G]$-module structure on $V$, given a representation. We have \[
\{ G-\text{representations}\} \cong \{ K[G]-\text{modules}\}
.\] 
\end{remark}
