\chapter{Introduction}
\section{Ring Theory}
\subsection{\textcolor{blue}{Lecture 15.10.25}}
\subsection{\textcolor{blue}{Lecture 17.10.25}}
\subsection*{Important Ring Homomorphisms}
\begin{theorem}[Initial Ring]
    The ring of integers $\mathbb{Z}$ is initial in $\mathtt{Ring}$, i.e. for every unital ring $R$, there is a unique ring homomorphism $f: \mathbb{Z} \to \mathbb{R}$ and $f$ is determined by $f(1)=1_R$.
\end{theorem}
The last statement works by using the homomorphism property \[
f(\sum 1)=\sum f(1)
.\] 
\begin{theorem}[Terminal Ring]
    The \emph{null ring} is terminal in $\mathtt{Ring}$, i.e. for every ring there is a unique ring homomorphism $f: R \to \{0\}$.
\end{theorem}
\begin{eg}
    Let $(A,+)$ be an abelian group and denote by $\End(A)$ the endomorphisms $A \to A$. Given any $f,g \in End(A)$, we define \[
        (f+g)(x):=f(x)+g(x)
    \] and \[
    (f \cdot g)(x):=f(g(x))
\] for any $x \in A$. This makes $\End(A)$ an abelian group. The identity map $1 \in \End(A)$ turns $\End(A)$ into a ring. 
\end{eg}
\begin{ex}
   What happens if $A$ is not abelian?
\end{ex}
There are several standard constructions of rings:
\begin{definition}[Opposite Ring]
    Let $(R,+,\cdot)$ be a ring. The \textbf{opposite ring} $R^\text{op}$ is the same abelian group $(R,+)$ together with the inverted multiplication \[
        (r,s) \mapsto s \cdot r
    .\] 
\end{definition}
\begin{definition}[Polynomial Ring]
    Given any ring $R$, define the \textbf{polynomial ring} of polynomials in $x$ with coefficients in $R$ by \[
        R[x]:= \left\{ \sum_i a_ix^i \mid a_i \in R,\, a_i=0 \text{ for }i \text{ suff. large}\right\} 
    .\]  Addition, multiplication and identity are inherited from $R$.
\end{definition}
We construct higher polynomial rings $R[x_1, \dots, x_n]:=R[x_1, \dots, x_{n-1}][x_n]$ inductively. For $p(x)\in \mathbb{F}[x]$, the degree is the highest non-zero power of $x$ appearing in $p(x)$. We have \[
\deg(p(x)\cdot q(x))=\deg(p(x))+\deg(q(x))
.\]  This is not well-defined unless $R$ is an integral domain: $\mathbb{R}[x]$ to $\mathbb{Z}/6\mathbb{Z}[x]$ shows this.
\begin{eg}
    The ring of \emph{Laurent polynomials} is given by $R[x,x^{-1}]$.
\end{eg}
\begin{eg}
    The \textbf{ring of power series} in $x$ is given by \[
        R\llbracket x \rrbracket:=\left\{ \sum_{i \geq 0}a_ix^i \mid a_i \in R \right\}
    ,\] so we allow infinite sums. If one considers $1-x \in \mathbb{R}[x]$, it does not have an inverse in $\mathbb{R}\llbracket x \rrbracket]]$. However, in $\mathbb{R}\llbracket x \rrbracket$ one has the (formal) geometric series \[
    \frac{1}{1-x} = \sum_{i \geq 0} x^i
    \] as an inverse.
\end{eg}
\begin{definition}[Principal Ideal]
    A (left/right/two-sided) \textbf{principal ideal} of a ring $R$ is a subset $Ra / aR / RaR$ for some $a \in R$ defined by \[
        Ra:= \{ ra \mid r \in R \}
    .\] 
\end{definition}
\begin{ex}
    Principal ideals are ideals.
\end{ex}
\begin{remark}
   If $R$ is commutative, all these notions collapse to one and one writes $\langle a \rangle$ for the ideal generated by $a$. 
\end{remark}
\begin{eg}
We already know many principal ideals, e.g. $\langle 2 \rangle$ in $2\mathbb{Z}$ or $\langle n \rangle$ in $n \mathbb{Z}$. In $\mathbb{Z}$, every ideal is principal. For any ring, $\langle 0 \rangle$ and $\langle 1 \rangle$ are principal ideals. In polynomial rings, we always have principal ideals in the form of powers of $x$, e.g. $\langle x \rangle$, $\langle x^2 \rangle$, or $\langle x^2+1 \rangle$.  In $R[x,y]$, $\langle x,y \rangle$ is a principal ideal.
\end{eg}
\section{Modules}
The idea is to generalize the idea of vector spaces, which are over fields, to something defined over rings.
\begin{definition}[Module]
    A \textbf{left $R$-module} (module over $R$) is an abelian group $M$ together with a map 
    \begin{align*}
        R \times M &\to M \\
        (r,m) &\mapsto r \cdot m
\end{align*}
satisfying
\begin{enumerate}
    \item $r(m+n)=rm+rn$
    \item $(r+s)m=rm+sm$
    \item $(rs)m=r(sm)$
    \item $1_R \cdot m = m$
\end{enumerate}
Right modules are defined analogously.
\end{definition}
\begin{ex}
    There are several statements easy to prove:
    \begin{itemize}
        \item $\forall m \in M: 0 \cdot m = 0_M$    
        \item $(-1) \cdot m = -m$
    \end{itemize}
    
\end{ex}
\begin{theorem}[Abelian groups as module]
    Every abelian group is a $\mathbb{Z}$-module in exactly one way.
\end{theorem}
\begin{proof}
    $\mathbb{Z}$ is initial, so there is a unique homomorphism $\mathbb{Z} \to R$ for all unital $R$.
\end{proof}
This shows that abelian groups are nothing but $\mathbb{Z}$-modules (or, abstractly, $\mathbb{Z}$-vector spaces). $\End(\mathtt{AGrp})$ is a ring and we have an action of $\mathbb{Z}$ on any abelian groups by endomorphisms.
\begin{eg}
    Every ring $R$ is a (left) $R$-module over itself. Furthermore, every (left) ideal $\mathcal{I} \subseteq R$ is a (left) $R$-module.  Of course, there is also the trivial module $M=\{0\}$.
\end{eg}
If $\mathcal{I} \subseteq R$ is a left ideal, $R / I$ is not a ring.
\begin{ex}
   If $\mathbb{I} \subseteq R$ is a left ideal, $R / I$ is a left module. 
\end{ex}
\subsection*{Submodules}
\begin{definition}[Submodule]
    A \textbf{submodule} $N$ of a left $R$-module $M$ is a subgroup preserved by the action of $R$, i.e. \[
    \forall r \in R \, \forall n \in N: \, rn \in N
    .\] 
\end{definition}
\begin{note}
    The (left) ideals of $R$ are the left submodules of $R$ viewing $R$ as a module over itself.
\end{note}
\begin{definition}[Simple Module]
    A module $M$ is \textbf{simple} if its only submodules are $M$ and $\{0\}$.
\end{definition}
\subsection*{Module Homomorphisms}
\begin{definition}[Module Homomorphism]
    An $R$\textbf{-module homomorphism} is a homomorphism of abelian groups compatible with the $R$-module structure:
    If $M,N$ are $R$-modules and $\phi: M \to N$ is a homomorphism, then
    \begin{enumerate}
        \item $\forall m_1,m_2 \in M: \, \phi(m_1+m_2)=\phi(m_1)+\phi(m_2)$
        \item $\forall r \in R, \, \forall m \in M: \ \phi(rm)=r\phi(m)$.
    \end{enumerate}
\end{definition}
\begin{theorem}[Kernel and Image are Subs]
   Let $\phi$ be an $R$-mod homomorphism. Both $\ker \phi$ and $\im \phi$ are submodules. 
\end{theorem}
